\documentclass[arbeit=ober,oneside]{ArbeitRST}
% Paket zum Erzeugen von Blindtext (ist nur für dieses Beispieldokument sinnvoll)
\usepackage{blindtext}

\usepackage[utf8]{inputenc}
\usepackage{amsmath}
\usepackage{amsfonts}
\usepackage{amssymb}
\usepackage{graphicx}
%\usepackage{fancyhdr}
\usepackage{pgfplots}
%Absatzeinrückung ausschalten
\setlength{\parindent}{0pt}
\usepackage{tikz}
\usepackage{caption}
\usepackage[%per=slash,
decimalsymbol=comma,
loctolang={DE:ngerman,UK:english},abbreviations]{siunitx}
\usepackage{pstricks}
\usepackage{mathrsfs}
\usepackage{mathtools}
\usepackage{fancyref}
\usepackage{tabularx} 
\usepackage{siunitx}

% zwei Parameter zum Verändern des Layouts
%\parindent0em
%\parskip2ex
% Einige Einstellungen für das hyperref-Paket
% Die farbig dargestellten Links, Gleichungsnummern etc. sollten automatisch beim Ausdrucken schwarz dargestellt werden.
% An dieser Stelle können Sie die Farbgebung anpassen.
\hypersetup{
    unicode=false,          % non-Latin characters in Acrobat’s bookmarks
    pdftoolbar=true,        % show Acrobat’s toolbar?
    pdfmenubar=true,        % show Acrobat’s menu?
    pdffitwindow=false,     % window fit to page when opened
    pdfstartview={FitH},    % fits the width of the page to the window
    pdftitle={RST Vorlage},    % title
    pdfauthor={Author},     % author
    pdfsubject={Subject},   % subject of the document
    pdfcreator={Creator},   % creator of the document
    pdfproducer={Producer}, % producer of the document
    pdfkeywords={keyword1} {key2} {key3}, % list of keywords
    pdfnewwindow=true,      % links in new window
    colorlinks=true,       % false: boxed links; true: colored links
    linkcolor=blue,          % color of internal links (change box color with linkbordercolor)
    citecolor=green,        % color of links to bibliography
    filecolor=magenta,      % color of file links
    urlcolor=cyan           % color of external links
}
% Entfernt die farbigen Markierungen - bitte Druckversion mit dieser Option kompilieren
\hypersetup{hidelinks}
\begin{document}
%% Titelseite
% Name des Verfassers
\author{Martin P. Mustermann}
% Geburtsort
\geburtsort{Dresden}
% Geburtsdatum
\geburtsdatum{1. Januar 1912}
% Titel der Arbeit
\title{Steuerung und Regelung des Auslegers einer mobilen Betonpumpe}
% Untertitel
%\subtitle{Untertitel}
% Angabe der Betreuer
\betreuer{Dipl.-Ing. Carsten Knoll}
%\betreuer{Betreuer 2}
% Datum der Einreichung
\date{31. März 2015}
% Titelseite erstellen
\maketitle
%% Selbstständigkeitserklärung
% Ort der Selbstständigkeitserklärung (Standard: Dresden)
%\selbstort{Pirna}
% Datum der Selbstständigkeitserklärung (Standard: aktuelles Systemdatum)
%\selbstdatum{1. Januar 2011}
%\selbststaendigkeitserklaerung
%% Kurzfassung / Abstract
%\kurzfassung{An dieser Stelle fügen Sie bitte eine deutsche Kurzfassung ein.}{Please insert the English abstract here.}
%% Inhaltsverzeichnis

\addtocontents{toc}{\protect\thispagestyle{empty}}
\tableofcontents
\thispagestyle{empty}
\listoffigures
\thispagestyle{empty}
\listoftables
\thispagestyle{empty}
%\pagestyle{headings}
\setcounter{page}{0}
\chapter{Einleitung}
%%Einleitung
%Thema
%Erläuterung des Problems
%Aufgabe/Ziel

In dieser Seminararbeit soll eine mobile Pumpe zur Beförderung von flüssigem Beton, wie sie im Baubereich eingesetzt wird, aus regelungstechnischer Sicht untersucht werden.

%Grafik
\begin{figure}[h!]
\centering
\includegraphics[scale=0.6]{betonpumpe.png}
\caption{Mobile Betonpumpe mit ausgefahrenem Ausleger, Quelle: Liebherr}
\end{figure}

In der oberen Darstellung ist eine mobile Betonpumpe bestehend aus Ausleger und LKW-Chassis zu sehen. Der Ausleger besteht aus fünf Teilen und lässt sich über Hydraulikzylinder und eine Umlenkkinematik in verschiedene Stellungen fahren. Am Chassis ist außerdem ein  Abstützsystem angebracht, welches Kräfte, die vom Ausleger ausgehen, ableitet. Es bietet darüber hinaus auch Halt in unebenem Gelände.\\

\begin{table}[h!]
\caption{Beispielwerte der Armsegmente, welche aus gegebener Gesamtmasse und Gesamtlänge ermittelt wurden}
\label{tab:WertePumpe}
\centering
\begin{tabular}{c|c|c}
\rule[-1ex]{0pt}{2.5ex} Armsegment & Masse/kg & Länge/m \\ 
\hline \rule[-1ex]{0pt}{2.5ex} 1 & 2250 & 9 \\ 
\rule[-1ex]{0pt}{2.5ex} 2 & 1700 & 8 \\ 
\rule[-1ex]{0pt}{2.5ex} 3 & 1350 & 7 \\ 
\rule[-1ex]{0pt}{2.5ex} 4 & 900 & 7 \\ 
\rule[-1ex]{0pt}{2.5ex} 5 & 480 & 6 \\ 
\end{tabular} 
\end{table}

Tabelle \ref{tab:WertePumpe} enthält die Massen und Längen der einzelnen Segmente des Auslegers, welche aus der Gesamtmasse und Gesamtlänge abgeschätzt wurden.

\section{Probleme}
Aufgrund der Leichtbauweise und dem damit verbundenem elastischen Verhalten des Materials kommt es durch den diskontinuierlichen Pumpvorgang zum Schwingen des Auslegers. Das Problem kann mit einer aktiven oder passiven Schwingungsdämpfung minimiert werden. Die aktive Schwingungsdämpfung stellt einen geschlossenen Regelkreis dar. Die Biegung kann dabei bspw. über Dehnungsmessstreifen erfasst werden und mit einem Stellsignal auf die Hydraulikzylinder korrigiert werden. Bei der passiven Schwingungsdämpfung wird versucht, über konstruktive Maßnahmen die Schwingung zu dämpfen. Das kann z.B. durch den Einsatz von bestimmten Materialien erreicht werden. Ein anderer Ansatz ist die Verwendung von passiven Vibrations-Absorbern. 

\section{Aufgabe}
In dieser Arbeit soll lediglich die aktive Schwingungsdämpfung betrachtet werden. Es soll zum einen eine Regelung und weiterhin eine Steuerung entworfen werden, um den Aktor in verschiedene Stellungen überführen zu können. \\
Als erstes wird ein Modell, welches die Dynamik des Auslegers abbildet, gesucht. Dann soll unter Verwendung der Programmiersprache Python mit Hilfe einer Simulation der Steuerungs- und Regelungsentwurf erfolgen. Die Steuerung sorgt dafür, dass möglichst alle Gelenkwinkel der Pumpe einer vorgegeben Solltrajekorie folgen. Der Regler soll dabei mögliche Abweichungen, welche durch ein vom Entwurfsmodell abweichendes reales Streckenverhalten hervorgerufen werden können, korrigieren. 


\chapter{Modellbildung}
Im folgenden Abschnitt wird das modellierte mechanische System vorgestellt. 
Der Ausleger wird als Mehrfachpendel betrachtet. Dabei wird sich auf einen Ausleger aus vier Teilen beschränkt, da das Aufstellen der Bewegungsgleichungen später symbolisch erfolgt und bereits in diesem Fall sehr unübersichtliche Terme enstehen. Darüber hinaus wird in diesem Abschnitt erläutert, wie die Biegung der Armsegmente modelliert wird.

% Rechtfertigung der Modellbeschreibung -> Hydraulikantriebe,etc.
% genauere Erklärung der Skizze
\section{Modellierung der Biegung}
\subsection{Modellierung mit verteilten Parameter}

\begin{figure}[h!]
\centering
\def\svgscale{0.8}
\input{Biegung.pdf_tex}
\caption{Biegebalken}
\label{fig:VertPara}
\end{figure}

Die Beschreibung der Balkendynamik erfolgt im verteiltparametrischen Fall über \mbox{eine} partielle Differentialgleichung (PDGL), welche man mit Hilfe der Euler-Bernoulli-Balkentheorie herleiten kann. 

\begin{equation}
EI\dfrac{\partial^4 z(x,t)}{\partial x^4}-\mu \dfrac{\partial^2 z(x,t)}{\partial t^2} = q(x,t).
\end{equation}

Dabei ist $EI$ die Biegesteifigkeit, $q$ die Streckenlast und $\mu$ die hydrodynamische Masse.\\
Vorteile der Modellierung mit Hilfe einer PDGL ist eine relativ genaue Beschreibung der Durchbiegung. Die Nachteile sind der kompliziertere Steuerungs- und Reglerentwurf und die damit verbundene zusätzliche Einarbeitungszeit in das Thema. Außerdem ist nicht ersichtlich, ob die so entworfene Steuerung und Regelung gegenüber einem Modell mit konzentrierten Parametern einen wirklich signifikanten Performance-Gewinn bringt. Aufgrund des wesentlich komplizierteren Modells wird es wahrscheinlich auch zu längeren Simulationszeiten kommen. Darüber hinaus können die Autoren nicht abschätzen, wie aufwendig die numerische Lösung von PDGLs in Python ist, bzw. welche Bibliotheken dafür bereits existieren.


\subsection{Modellierung mit konzentrierten Parameter}
\begin{figure}[!h]
\centering
\def\svgscale{0.8}
\input{KonzParameter.pdf_tex}
\caption{Modellierung der Biegung mit konzentrierten Parametern}
\label{fig:KonzPara}
\end{figure}

Um möglichst nah an den Fall der verteilten Parameter zu kommen, müssen in einem Armsegment möglichst viele konzentrierte Feder-Dämpfer-Elemente untergebracht sein. Dabei kommt mit jedem Element eine neue verkoppelte Systemgleichung hinzu, was eine symbolische Betrachtung ab einem gewissen Punkt unüberschaubar macht, die Rechenzeit der Simulation stark erhöht und die Wahl geeigneter Parameter, sowie die Verifikation erschwert.\\
Aus Gründen der einfacheren symbolischen Handhabung und des einfacheren Steuerungs- und Reglerentwurfs wurde sich daher für die Variante mit einem konzentrierten Feder-Dämpfer-Element im Segment-Schwerpunkt entschieden.
%Skizze
\begin{figure}[h!]
\centering
\def\svgscale{0.8}
\input{Skizze.pdf_tex}
\caption{Skizze des Modells für zwei Armsegmente und Durchbiegung}
\label{fig:Skizze}
\end{figure}

% Formeln

% Wahl der Koordinaten
% Potentielle und Kinetische Koenergie
% Euler Lagrange-Formalismus
Abbildung \ref{fig:Skizze} stellt das Modell eines Verteilermasts mit zwei Armsegmenten dar. Der erste Index i beschreibt den physikalisch vorhandenen i-ten Teil des Auslegers. Der Index j steht für das j-te Element des i-ten Auslegers. Wobei j=2 die passiven Gelenke, welche die Biegung modellieren, beschreibt. Diese sind in der Skizze durch ein Federsymbol kenntlich gemacht. Die aktuierten Gelenke hingegen sind in grau dargestellt. In diese werden äußere Momente $\tau$ durch die hydraulischen Antriebe eingeprägt. Es werden Relativwinkel für die Winkel $\theta$ verwendet.\\

\paragraph{Zusammenfassung}
Für die Modellierung der Durchbiegung eines Segments wird ein zusätzliches passives Gelenk mit konzentriertem Feder- und Dämpfungsparameter verwendet.\\
Für die Geometrie werden rechteckige Hohlträger angenommen, welches für die Berechnung der Trägheitsmomente entscheidend ist.\\
Die Dynamik der Hydraulikzylinder wird vernachlässigt, um das Modell nicht noch komplizierter zu machen. In der Praxis sollten diese unbedingt berücksichtigt werden.\\
Außerdem wird angenommen, dass alle Winkel und Winkelgeschwindigkeiten über entsprechende Sensoren messbar sind.

\newpage
\section{Herleitung der Bewegungsgleichung}
Ziel ist es nun die nichtlinearen Bewegungsgleichungen mit den vorgestellten Modellannahmen zu ermitteln. Dabei wird der Euler-Lagrange-Formalismus verwendet.\\ 
Zunächst werden geeignete Koordinaten für die Beschreibung der Schwerpunktlagen der Massen gesucht. Die Schwerpunkte der Teilmassen des i-ten Armsegments werden kartesisch beschrieben durch die Vektoren

\begin{equation}
\vect{r_\mathrm{i}}=(x_\mathrm{ij},y_\mathrm{ij})^T.
\end{equation}

Die Gelenke sind über starre Stabelemente mit der Länge $a_\mathrm{ij}$ gekoppelt. Die Minimalkoordinaten des i-ten Armsegment entsprechen daher den Winkeln
\begin{equation}
\vect{\theta} = (\theta_\mathrm{11},\theta_\mathrm{12},\theta_\mathrm{21},\theta_\mathrm{22})^T.
\end{equation}

Die Euler-Lagrange-Gleichungen 2.Art lauten
\begin{equation}
\dfrac{d}{dt}\dfrac{\partial L(\vect{\theta},\dot{\vect{\theta}})}{\partial \dot{\theta}_\mathrm{ij}}-\dfrac{\partial L(\vect{\theta},\dot{\vect{\theta}})}{\partial \theta_\mathrm{ij}}=\tau_\mathrm{i}-d_\mathrm{i}.
\label{eq:lagr}
\end{equation}

Die Lagrange-Funktionen sind
\begin{equation}
L(\vect{\theta},\dot{\vect{\theta}})=T(\vect{\theta},\dot{\vect{\theta}})-V(\vect{\theta}).
\end{equation}
$T$ stellt hierbei die kinetische Koenergie und $V$ die potentielle Energie der Massen dar.
Die kinetische Koenergie berechnet sich zu 
\begin{equation}
T = \sum \left( \dfrac{1}{2}\cdot m_\mathrm{ij}\cdot(v^2_{x,ij}+v^2_\mathrm{y,ij})+\dfrac{1}{2}\cdot J_{ij}\cdot\omega_\mathrm{ij}^2 \right).
\end{equation}
Man erkennt einen translatorischen und einen rotatorischen Anteil.\\
Die potentielle Energie berechnet sich zu 

\begin{equation}
V = \sum \left( m_\mathrm{ij}\cdot g \cdot y_\mathrm{ij} + k_i \cdot \theta_\mathrm{ij} \right).
\end{equation}

Die kartesischen Schwerpunktkoordinaten lassen sich über die bekannten Armsegmentlängen $a_\mathrm{ij}$, Schwerpunktlängen $l_\mathrm{ij}$ und die Gelenkwinkel $q_\mathrm{ij}$ berechnen. Exemplarisch ergibt sich damit für die ersten zwei Masseelemente aus Abbildung \ref{fig:Skizze}:

\begin{align*}
x_\mathrm{11} &= l_\mathrm{11}\cdot \cos(\theta_\mathrm{11}) \\
y_\mathrm{11} &= l_\mathrm{11}\cdot \sin(\theta_\mathrm{11}) \\
x_\mathrm{12} &= a_\mathrm{11}\cdot \cos(\theta_\mathrm{11})+l_\mathrm{12}\cdot\cos(\theta_\mathrm{11}-\theta_\mathrm{12})\\
y_\mathrm{12} &= a_\mathrm{11}\cdot \sin(\theta_\mathrm{11})+l_\mathrm{12}\cdot\sin(\theta_\mathrm{11}-\theta_\mathrm{12})\\
 			 &\mathrel{\makebox[\widthof{=}]{\vdots}} 
\end{align*}

Man erhält dabei durch Lösung von (\ref{eq:lagr}) ein System nichtlinearer Bewegungsgleichungen, welches die allgemeine Form

\begin{equation}
\vect{M}(\vect{\theta})\cdot\ddot{\vect{\theta}}+\vect{C}(\vect{\theta},\dot{\vect{\theta}})\cdot\dot{\vect{\theta}}+\vect{K}\cdot\vect{\theta}+\vect{g}(\vect{\theta})=\vect{\tau}
\label{eq:BWGL}
\end{equation}

besitzt.\\
Dabei stellt $\vect{M}(\vect{\theta})$ die Massenmatrix, der Term $\vect{C}(\vect{\theta},\dot{\vect{\theta}})$ Zentrifugalkräfte, $\vect{K}\cdot\vect{\theta}$ die elastischen Fesselungskräfte und $\vect{g}(\vect{\theta})$ Gravitationskräfte dar.

\section{Aufstellen des Zustandsraummodells}
\label{abs:aufstellen_Zustandsmodell}
Für eine regelungstechnische Betrachtung werden die Bewegungsgleichungen nun in ein Zustandsraummodell in Regelungsnormalform überführt. Dafür werden die Zustandsvektoren


\begin{align}
\begin{aligned}
\vect{x}_\mathrm{1} = ( \theta_\mathrm{11}, ..., \theta_\mathrm{22} )^T \\
\vect{x}_\mathrm{2} = ( \dot{\theta}_\mathrm{11}, ..., \dot{\theta}_\mathrm{22} )^T 
\end{aligned}
\end{align}
sowie der Vektor für die Eingangsgrößen
\begin{align}
\vect{u} = (\tau_1, \tau_2)^T
\end{align}
eingeführt.\\
Für Simulationen muss Gleichung (\ref{eq:BWGL}) nach $\ddot{\vect{\theta}}$ umgestellt werden, da die Simulation die Bewegungsgleichungen über eine numerische Integration löst.

\begin{align*}
\vect{M} \cdot \ddot{\vect{\theta}}		 &= \vect{\tau}-\vect{C} \dot{\vect{\theta}}-\vect{K} \vect{\theta}-\vect{g} 			\\
\vect{M}^{-1}\cdot \vect{M} \cdot \ddot{\vect{\theta}} &= \vect{M}^{-1}(\vect{\tau}-\vect{C} \dot{\vect{\theta}}-\vect{K}\vect{\theta}-\vect{g})	\\
\ddot{\vect{\theta}}				 &= \vect{M}^{-1}(\vect{\tau}-\vect{C}\dot{\vect{\theta}}-\vect{K}\vect{\theta}-\vect{g})
\end{align*}

Dies entspricht nach Umschreiben mit Hilfe der eingeführten Zustandsgrößen:

\begin{align}
\begin{aligned}
\dot{\vect{x}}_\mathrm{1} & =  \vect{x}_\mathrm{2} \\
\dot{\vect{x}}_\mathrm{2} & =  \vect{M^{-1}}(\vect{u}- \vect{C}(\vect{x}_\mathrm{1},\vect{x}_\mathrm{2})\vect{x}_\mathrm{2}-\vect{g}(\vect{x}_\mathrm{1})-\vect{K}(\vect{x}_\mathrm{1},\vect{x}_\mathrm{2}))
\end{aligned}
\end{align}

\section{Ruhelagen}
\subsection{Unvollständig aktuiertes Modell}
Da Winkelgeschwindigkeiten und -beschleunigungen in der Ruhelage verschwinden, wird 
\begin{equation}\label{eq:uva_winkel}
\ddot{\vect{\theta}} = \dot{\vect{\theta}} \stackrel{!}{=} \vect{0}.
\end{equation}
gefordert.\\
Für die Winkel der aktiven Gelenke ergibt sich damit in der Ruhelage
\begin{align}\label{eq:uva_momente}
\begin{aligned}
\theta_\mathrm{11} = \theta^e_\mathrm{11}\\
\theta_\mathrm{21} = \theta^e_\mathrm{21}.
\end{aligned}
\end{align}

An dieser Stelle wird gefordert, dass die Balken starr sind. Die Annahme wird hier getroffen, um den Entwurf einer Vorsteuerung zu vereinfachen. Es wird
\begin{equation}\label{eq:uva_winkel_passiv}
\theta_\mathrm{12} = \theta_\mathrm{22} \stackrel{!}{=} 0
\end{equation}
angenommen. Es handelt sich in diesem Fall um ein vollständig aktuiertes Modell. Dieses Modell wird letztendlich für die Trajektorienplanung verwendet. Im Falle eines unteraktuierten Systems wird die Trajektorienplanung wesentlich schwerer. Die Trajektorienplanung stellt in diesem Fall ein Randwertproblem dar.

\subsection{Exakte Ruhelagen/unvollständig aktuiertes Modell}
Es gelten zunächst genauso Gleichung (\ref{eq:uva_winkel}) und (\ref{eq:uva_momente}). Allerdings gilt Gleichung (\ref{eq:uva_winkel_passiv}) nun nicht mehr. Es muss jetzt das nichtlineare Gleichungssystem  

\begin{equation}
M(\vect{\theta})\cdot\ddot{\vect{\theta}}+C(\vect{\theta},\dot{\vect{\theta}})\cdot\dot{\vect{\theta}}+K\cdot\vect{\theta}+g(\vect{\theta}) - \vect{\tau} = \vect{0} 
\end{equation}

nach $\tau^e_\mathrm{11}$, $\tau^e_\mathrm{21}$, $\theta^e_\mathrm{12}$ und $\theta^e_\mathrm{22}$ aufgelöst werden.

\section{Linearisierung}
Die Bewegungsgleichung liegt nach der Umformung von Abschnitt \ref{abs:aufstellen_Zustandsmodell} in der Form 

\begin{equation}
\ddot{\vect{\theta}} = \vect{f}(\vect{\theta},\dot{\vect{\theta}},\vect{\tau})
\end{equation}

vor. Sie wird über die Glieder erster Ordnung einer Taylorreihe um die Ruhelage herum approximiert. 
\begin{equation}
\ddot{\widetilde{\vect{\theta}}} = \left. \dfrac{\partial \vect{f}}{\partial \vect{\theta}}\right|_{AP} \widetilde{\vect{\theta}} + 
\left. \dfrac{\partial \vect{f}}{\partial \dot{\vect{\theta}}}\right|_{AP} \dot{\widetilde{\vect{\theta}}} + 
\left. \dfrac{\partial \vect{f}}{\partial \vect{\tau}}\right|_{AP} \widetilde{\vect{\tau}} 
\end{equation}

Mit Hilfe der eingeführten Zustandsvariablen erhält man das folgende System: 
\begin{equation}
\left( \begin{array}{c}
\dot{\vect{x}}_\mathrm{1} \\ \hline
\dot{\vect{x}}_\mathrm{2}
\end{array}\right) 
= 
\left( \begin{array}{c|c}
\vect{0} & \vect{I} \\ \hline
-\overline{\vect{M}}^{-1}\overline{\vect{K}} & \overline{\vect{M}}^{-1}\overline{\vect{C}}
\end{array} \right) 
\left( \begin{array}{c}
\vect{x_\mathrm{1}} \\ \hline
\vect{x_\mathrm{2}}
\end{array}\right) +
\left( \begin{array}{c}
\vect{0} \\ \hline
-\overline{{\vect{M}}}^{-1}
\end{array} \right) 
\left( \begin{array}{c}
\vect{u_1} \\ \hline
\vect{u_2}
\end{array}\right)
\end{equation}

\section{Schwierigkeiten bei der Implementierung des Zustandsraummodells}
Wie bereits erläutert, ist für die Lösung des DGL-Systems über eine numerische Integration eine Invertierung der Matrix $\vect{M}$ notwendig. Zunächst werden einige Eigenschaften der Matrix aufgeführt.\\
Sie ist symmetrisch und positiv definit. D.h.
\begin{equation*}
\vect{M}(\vect{\theta}) = \vect{M}^T(\vect{\theta})
\end{equation*}
Daraus folgt, dass 

\begin{equation*}
\det{\vect{M}(\vect{\theta})}\neq 0
\end{equation*}
und damit die Matrix invertierbar ist. Bei der symbolischen Invertierung war das Problem, dass die Matrixelemente für vier Gelenke bereits sehr lange symbolische Ausdrücke enthalten. Im folgenden sind die Anzahl an Operatoren, die in jedem Matrixelement standen, angegeben.
\begin{equation*}
\bf{COUNT\_OPERATORS(\vect{M})} = \left( \begin{array}{cccc}
195 & 161 & 114 & 59 \\
155 & 132 & 98 & 51 \\
106 & 94 & 71 & 41 \\
53 & 47 & 39 & 20 
\end{array} \right)
\end{equation*}
Nach der Linearisierung kommen noch mehr Operatoren hinzu. Daher ist eine Invertierung in symbolischer Form, aufgrund eines sehr hohen Zeitaufwandes, kaum möglich. Ein Weg die Matrix trotzdem symbolisch zu invertieren, ist die Invertierung mit Hilfe der Cholesky-Zerlegung, die im nachfolgenden Abschnitt genauer erläutert wird. \\
Schlussendlich wurde die Invertierung numerisch durchgeführt. 
\newpage
\section{Cholesky-Zerlegung}
In einer Anordnung von $n$ seriellen Manipulatoren sind $2n$-Freiheitsgerade, welche die Dimension von $\vect{M}\in\Reals^{2n\times 2n}$ beschreiben. Zur Umstellung des Gleichungssystems (\ref{eq:BWGL}) ist die Inverse der Matrix notwendig. Bei der symbolischen Implementierungen ist an dieser Stelle ein sehr hoher Rechenaufwand zu erwarten. Jede Massenmatrix eines solchen mechanischen Systems ist symmetrisch ($\vect{M}=\vect{M}^T$) und positiv definit ($\forall$ EW $>0$) \cite{janschek2009systementwurf}. Durch diese Eigenschaften lässt sich eine Cholesky-Zerlegung $\vect{M}=\vect{L}\vect{L}^T$ (mit $\vect{L}$ untere Dreiecksmatrix) durchführen \cite{schwarz2009numerische}. Wenn man sich die Multiplikation 
\begin{equation}
\begin{aligned}
	\vect{L}\vect{L}^T &= 
	\begin{pmatrix}
		l_{11} & 0      & 0	&	\cdots & 0\\
		l_{21} & l_{22} & 0 &  &\\
		l_{31} & l_{32} & l_{33} & &\\
		\vdots & & & \ddots & 0\\
		l_{2n1} & & & & l_{2n2n}
	\end{pmatrix} \cdot 
	\begin{pmatrix}
	l_{11} & l_{21}      & l_{31}	&	\cdots & l_{2n1}\\
	0 & l_{22} & l_{32} & &\\
	0 & 0 & l_{33} & &\\
	\vdots & & & \ddots& 0\\
	0 & & & & l_{2n2n} 
	\end{pmatrix}\\
	&= 
	\begin{pmatrix}
	l_{11}^2 & l_{21}l_{11}      & l_{31}l_{11}	&	\cdots\\
	l_{21}l_{11} & l_{21}^2+l_{22}^2 & l_{21}l_{31}+l_{22}l_{32} & \\
	l_{31}l_{11} & l_{31}l_{21}+l_{32}l_{22} & l_{31}^2+l_{32}^2+l_{33}^2 & \\
	\vdots & & & \ddots 
	\end{pmatrix}
\end{aligned}
\end{equation}

betrachtet, können durch einen Koeffizientenvergleich mit
\begin{equation}
\vect{M}=\begin{pmatrix}
m_{11} & m_{12}      & m_{13}	&	\cdots& m_{12n}\\
m_{21} & m_{22} & m_{23} & &\\
m_{31} & m_{32} & m_{33} & &\\
\vdots & & & \ddots & \\
m_{2n1}& & & & m_{2n2n}
\end{pmatrix}
\end{equation}

die Elemente $l_{ij}$ mit $i,j=1,2,\dots,2n$ können in folgender Weise berechnet werden

\begin{equation}
\begin{aligned}
	l_{ij}=
	\begin{cases}
	0, j>i\\
	\sqrt{m_{ii}-\sum \limits_{k=1}^{j-1}l^{2}_{ik}} , i=j\\
	\frac{1}{l_{jj}} \left( m_{ij}-\sum \limits_{k=1}^{j-1}l_{ik}l_{jk}\right) , i>j
	\end{cases}
\end{aligned}
\end{equation}

An dieser Stelle ist zu erahnen, dass die Wurzel nur in jedem Fall reell gelöst werden kann, wenn die zuvor genannten Eigenschaften der Matrix gelten.

Die für die allgemeine Invertierung einer Matrix zu berechnende Determinante ist durch diese Zerlegung stark vereinfacht worden. Bei einer Dreiecksmatrix haben nur die Diagonalelemente der Matrix einen Einfluss, da alle anderen Summanden der Regel von Sarrus $0$ ergeben. Außerdem gilt $\det(\vect{L}\vect{L}^T)=\det\vect{L}\cdot\det\vect{L}^T$. Die Multiplikation der Diagonalelemente von $\vect{L}$ und der ihrer Transponierten sind identisch, so dass $\det(\vect{L}\vect{L}^T)=(\det\vect{L})^2$ ist.


%\chapter{Einleitung}
%%Einleitung
%Thema
%Erläuterung des Problems
%Aufgabe/Ziel
In dieser Seminararbeit soll eine verfahrbare Pumpe zur Beförderung von flüssigem Beton, wie sie im Baubereich eingesetzt wird, aus regelungstechnischer Sicht untersucht werden. Eine solche Autobetonpumpe ist in der folgenden Abbildung dargestellt. 

%Grafik
\begin{figure}[h!]
\centering
\includegraphics[scale=0.75]{betonpumpe-004.jpg}
\caption[]{Autobetonpumpe mit ausgefahrem Verteilermast, Quelle: \\http://www.trans-beton.de/images/picts/betonpumpe-004.jpg}
\end{figure}

Als erstes soll ein dynamisches Modell für die Pumpe gefunden werden. Dann soll mit Hilfe von numerischer Berechnungen/Simulationen unter Verwendung der Programmiersprache Python der Steuerungsentwurf erfolgen. Alle Gelenkwinkel der Pumpe sollen dabei möglichst exakt einer vorgegeben Solltrajekorie folgen.
Beim Einsatz der Pumpe kommt es darüberhinaus zu starken Schwingungen, welche durch einen Regler verringert werden sollen.

%\begin{figure}[h!]
%\centering
%\includegraphics[scale=0.23]{betonpumpe.jpg}
%\caption[]{Betonpumpe, Quelle: \\http://i00.i.aliimg.com/photo/v1/112691528/Truck_mounted_Concrete_Boom_Pump.jpg}
%\end{figure}


\chapter{Modellbildung}
Die Anordnung soll als Mehrfachpendel modelliert werden. Die Masse der Armsegmente wird als konzentriert in den einzelnen Schwerpunkten angenommen. Die Durchbiegung eines Armsegments wird durch ein unaktuiertes Gelenk mit einer Federsteifigkeit $k$ modelliert. Auf eine verteiltparametrische Beschreibung soll verzichtet werden.
% Rechtfertigung der Modellbeschreibung -> Hydraulikantriebe,etc.
% genauere Erklärung der Skizze

%Skizze
\begin{figure}[h!]
\centering
\def\svgscale{0.8}
\input{Skizze.pdf_tex}
\caption{Skizze des Modells für zwei Armsegmente und Durchbiegung}
\label{fig:Skizze}
\end{figure}

% Formeln

% Wahl der Koordinaten
% Potentielle und Kinetische Koenergie
% Euler Lagrange-Formalismus
Abbildung \ref{fig:Skizze} stellt das Modell eines Verteilermasts mit 2 Armsegmenten dar. Der erste Index beschreibt das physikalisch vorhandene i-te Armsegment des Verteilermasts. Der zweite Index nur für die Modellierung der Biegung vorhandene Unterteilung der einzelnen Segmente. Die aktuierten Gelenke sind in grau dargestellt. In diese werden äußere Momente $F$ durch die hydraulische Antriebe eingeprägt. Die restlichen Gelenke dienen nur zur Modellierung der Biegung der Armsegmente. Alle Gelenke sind masselos. Es werden Relativwinkel zwischen den Segmenten verwendet.\\
\section{Herleitung der Bewegungsgleichung}
Ziel ist es die nichtlinearen Bewegungsgleichungen für ein Modell mit beliebig vielen Gelenkwinkel aufzustellen. Für die Herleitung der Bewegungsgleichung wird der Euler-Lagrange-Formalismus verwendet.\\ 
Zunächst werden geeignete Koordinaten für die Beschreibung der Schwerpunktlagen der Massen gesucht. Die Schwerpunkte der Teilmassen des i-ten Armsegments werden kartesisch beschrieben durch die Vektoren

\begin{equation}
\vec{x_\mathrm{i}}=(x_\mathrm{ij},y_\mathrm{ij})^T.
\end{equation}

Die Gelenke sind über starre Stabelemente mit der Länge $a_\mathrm{ij}$ gekoppelt. Die Minimalkoordinaten des i-ten Armsegment entsprechen daher den Winkeln
\begin{equation}
\vec{q_\mathrm{i}} = (q_\mathrm{i1},...,q_\mathrm{ij})^T.
\end{equation}

Die Euler-Lagrange-Gleichungen 2.Art lauten
\begin{equation}
\dfrac{d}{dt}\dfrac{\partial L(\vec{q},\dot{\vec{q}})}{q_\mathrm{ij}}-\dfrac{\partial L(\vec{q},\dot{\vec{q}})}{q_\mathrm{ij}}=F_\mathrm{i}-d_\mathrm{i}
\label{eq:lagr}
\end{equation}

mit der Lagrange-Funktion
\begin{equation}
L(\vec{q},\dot{\vec{q}})=T(\vec{q},\dot{\vec{q}})-U(\vec{q})
\end{equation}
wobei $T$ die kinetische Koenergie und $U$ die potentielle Energie der Massen darstellt.
Die kinetische Koenergie berechnet sich zu 
\begin{equation}
T = \sum \left( \dfrac{1}{2}\cdot m_\mathrm{ij}\cdot(v^2_{x,ij}+v^2_\mathrm{y,ij})+\dfrac{1}{2}\cdot J_{ij}\cdot\omega_\mathrm{ij}^2 \right).
\end{equation}
Sie enthält einen translatorischen und einen rotatorischen Anteil.\\
Die potentielle Energie berechnet sich zu 

\begin{equation}
U = \sum \left( m_\mathrm{ij}\cdot g \cdot y_\mathrm{ij} + k_i \cdot q_\mathrm{ij} \right).
\end{equation}

Die kartesischen Schwerpunktkoordinaten lassen sich über die bekannten Armsegmentlängen $a_\mathrm{ij}$, Schwerpunktlängen $l_\mathrm{ij}$ und die Gelenkwinkel $q_\mathrm{ij}$ berechnen. Exemplarisch ergibt sich damit für die ersten zwei Masseelemente aus Abbildung \ref{fig:Skizze}:

\begin{align*}
x_\mathrm{11} &= l_\mathrm{11}\cdot \cos(q_\mathrm{11}) \\
y_\mathrm{11} &= l_\mathrm{11}\cdot \sin(q_\mathrm{11}) \\
x_\mathrm{12} &= a_\mathrm{11}\cdot \cos(q_\mathrm{11})+l_\mathrm{12}\cdot\cos(q_\mathrm{11}-q_\mathrm{12})\\
y_\mathrm{12} &= a_\mathrm{11}\cdot \sin(q_\mathrm{11})+l_\mathrm{12}\cdot\sin(q_\mathrm{11}-q_\mathrm{12})\\
 			 &\mathrel{\makebox[\widthof{=}]{\vdots}} 
\end{align*}

Man erhält dabei durch Lösung von (\ref{eq:lagr}) ein System nichtlinearer Bewegungsgleichungen, welches die allgemeine Form

\begin{equation}
M(\vec{q})\cdot\ddot{\vec{q}}+C(\vec{q},\dot{\vec{q}})\cdot\dot{\vec{q}}+K\cdot\vec{q}+g(\vec{q})=\vec{F}
\label{eq:BWGL}
\end{equation}

hat. \\
Dabei stellt $M(\vec{q})$ die Massenmatrix, der Term $C(\vec{q},\dot{\vec{q}})$ Zentrifugalkräfte, $K\cdot\vec{q}$ die elastischen Fesselungskräfte und $g(\vec{q})$ Gravitationskräfte dar.\\   
Für Simulation muss (\ref{eq:BWGL}) nach $\ddot{\vec{q}}$ umgestellt werden, da diese die Bewegungsgleichungen über eine numerische Integration löst.

\begin{align*}
M \cdot \ddot{\vec{q}}		 &= \vec{F}-C\dot{\vec{q}}-K\vec{q}-g 			\\
M^{-1}\cdot M \ddot{\vec{q}} &= M^{-1}(\vec{F}-C\dot{\vec{q}}-K\vec{q}-g)	\\
\ddot{\vec{q}}				 &= M^{-1}(\vec{F}-C\dot{\vec{q}}-K\vec{q}-g)
\end{align*}

\newpage
\section{Cholesky-Zerlegung}
In einer Anordnung von $n$ seriellen Manipulatoren sind $2n$-Freiheitsgerade, welche die Dimension von $\vect{M}\in\Reals^{2n\times 2n}$ beschreiben. Zur Umstellung des Gleichungssystems (\ref{eq:BWGL}) ist die Inverse der Matrix notwendig. Bei der symbolischen Implementierungen ist an dieser Stelle ein sehr hoher Rechenaufwand zu erwarten. Jede Massenmatrix eines solchen mechanischen Systems ist symmetrisch ($\vect{M}=\vect{M}^T$) und positiv definit ($\forall$ EW $>0$) \cite{janschek2009systementwurf}. Durch diese Eigenschaften lässt sich eine Cholesky-Zerlegung $\vect{M}=\vect{L}\vect{L}^T$ (mit $\vect{L}$ untere Dreiecksmatrix) durchführen \cite{schwarz2009numerische}. Wenn man sich die Multiplikation 
\begin{equation}
\begin{aligned}
	\vect{L}\vect{L}^T &= 
	\begin{pmatrix}
		l_{11} & 0      & 0	&	\cdots & 0\\
		l_{21} & l_{22} & 0 &  &\\
		l_{31} & l_{32} & l_{33} & &\\
		\vdots & & & \ddots & 0\\
		l_{2n1} & & & & l_{2n2n}
	\end{pmatrix} \cdot 
	\begin{pmatrix}
	l_{11} & l_{21}      & l_{31}	&	\cdots & l_{2n1}\\
	0 & l_{22} & l_{32} & &\\
	0 & 0 & l_{33} & &\\
	\vdots & & & \ddots& 0\\
	0 & & & & l_{2n2n} 
	\end{pmatrix}\\
	&= 
	\begin{pmatrix}
	l_{11}^2 & l_{21}l_{11}      & l_{31}l_{11}	&	\cdots\\
	l_{21}l_{11} & l_{21}^2+l_{22}^2 & l_{21}l_{31}+l_{22}l_{32} & \\
	l_{31}l_{11} & l_{31}l_{21}+l_{32}l_{22} & l_{31}^2+l_{32}^2+l_{33}^2 & \\
	\vdots & & & \ddots 
	\end{pmatrix}
\end{aligned}
\end{equation}

betrachtet, können durch einen Koeffizientenvergleich mit
\begin{equation}
\vect{M}=\begin{pmatrix}
m_{11} & m_{12}      & m_{13}	&	\cdots& m_{12n}\\
m_{21} & m_{22} & m_{23} & &\\
m_{31} & m_{32} & m_{33} & &\\
\vdots & & & \ddots & \\
m_{2n1}& & & & m_{2n2n}
\end{pmatrix}
\end{equation}

die Elemente $l_{ij}$ mit $i,j=1,2,\dots,2n$ können in folgender Weise berechnet werden

\begin{equation}
\begin{aligned}
	l_{ij}=
	\begin{cases}
	0, j>i\\
	\sqrt{m_{ii}-\sum \limits_{k=1}^{j-1}l^{2}_{ik}} , i=j\\
	\frac{1}{l_{jj}} \left( m_{ij}-\sum \limits_{k=1}^{j-1}l_{ik}l_{jk}\right) , i>j
	\end{cases}
\end{aligned}
\end{equation}

An dieser Stelle ist zu erahnen, dass die Wurzel nur in jedem Fall reell gelöst werden kann, wenn die zuvor genannten Eigenschaften der Matrix gelten.

Die für die allgemeine Invertierung einer Matrix zu berechnende Determinante ist durch diese Zerlegung stark vereinfacht worden. Bei einer Dreiecksmatrix haben nur die Diagonalelemente der Matrix einen Einfluss, da alle anderen Summanden der Regel von Sarrus $0$ ergeben. Außerdem gilt $\det(\vect{L}\vect{L}^T)=\det\vect{L}\cdot\det\vect{L}^T$. Die Multiplikation der Diagonalelemente von $\vect{L}$ und der ihrer Transponierten sind identisch, so dass $\det(\vect{L}\vect{L}^T)=(\det\vect{L})^2$ ist.



\chapter{Steuerungsentwurf}
 
\begin{figure}[h!]
\centering
\input{Vorsteuerung.pdf_tex}
\caption{Regelkreis}
\end{figure}

Planung einer Trajektorie. Allg Darstellung als Polynom n-ten Grades

\begin{figure}[h!]
\centering
\input{Trajektorie.pdf_tex}
\caption{Trajektorie}
\end{figure}

\begin{equation}
q_{\mathrm{traj}}(t) = a_\mathrm{n}t^n+\dots+a_\mathrm{1}t+a_\mathrm{0}
\end{equation}
Randbedingungen

\begin{align*}
q(T_\mathrm{0})			&=q_\mathrm{a} & q(T_\mathrm{1})&=q_\mathrm{soll}\\	
\dot{q}(T_\mathrm{0})	&=0 		   & \dot{q}(T_\mathrm{1})&=0\\
\ddot{q}(T_\mathrm{0})	&=0 		   & \ddot{q}(T_\mathrm{1})&=0
\end{align*}

damit lässt sich ein lineares Gleichungssystem mit 6 Gleichungen aufstellen. Es ergibt sich also ein Polynom 6.Grades.\\
Mit Hilfe der ermittelten Trajektorie für die Gelenkwinkel und der ersten und zweiten Ableitung, die berechnet werden können, können mit (\ref{eq:BWGL}) die erforderlichen Momente berechnet werden, welche die Motoren aufbringen müssen.

\begin{equation}
F_\mathrm{Vorst} = M(\vec{q}_{\mathrm{traj}})\cdot\ddot{\vec{q}}_{\mathrm{traj}}+C(\vec{q}_{\mathrm{traj}},\dot{\vec{q}}_{\mathrm{traj}})\cdot\dot{\vec{q}}_{\mathrm{traj}}+K\cdot{\vec{q}}_{\mathrm{traj}}+g(\vec{q}_{\mathrm{traj}}) 
\end{equation}

Die folgende Abbildung zeigen die Simulationsergebnisse bei Verwendung einer reinen Steuerung bei vier vorhanden Gelenkwinkel, von denen zwei aktuiert sind. 
\begin{figure}[h!]
\centering
\scalebox{0.75}{%% Creator: Matplotlib, PGF backend
%%
%% To include the figure in your LaTeX document, write
%%   \input{<filename>.pgf}
%%
%% Make sure the required packages are loaded in your preamble
%%   \usepackage{pgf}
%%
%% Figures using additional raster images can only be included by \input if
%% they are in the same directory as the main LaTeX file. For loading figures
%% from other directories you can use the `import` package
%%   \usepackage{import}
%% and then include the figures with
%%   \import{<path to file>}{<filename>.pgf}
%%
%% Matplotlib used the following preamble
%%   \usepackage{fontspec}
%%   \setsansfont{DejaVu Sans}
%%   \setmonofont{DejaVu Sans Mono}
%%
\begingroup%
\makeatletter%
\begin{pgfpicture}%
\pgfpathrectangle{\pgfpointorigin}{\pgfqpoint{8.000000in}{6.000000in}}%
\pgfusepath{use as bounding box}%
\begin{pgfscope}%
\pgfsetbuttcap%
\pgfsetroundjoin%
\definecolor{currentfill}{rgb}{1.000000,1.000000,1.000000}%
\pgfsetfillcolor{currentfill}%
\pgfsetlinewidth{0.000000pt}%
\definecolor{currentstroke}{rgb}{1.000000,1.000000,1.000000}%
\pgfsetstrokecolor{currentstroke}%
\pgfsetdash{}{0pt}%
\pgfpathmoveto{\pgfqpoint{0.000000in}{0.000000in}}%
\pgfpathlineto{\pgfqpoint{8.000000in}{0.000000in}}%
\pgfpathlineto{\pgfqpoint{8.000000in}{6.000000in}}%
\pgfpathlineto{\pgfqpoint{0.000000in}{6.000000in}}%
\pgfpathclose%
\pgfusepath{fill}%
\end{pgfscope}%
\begin{pgfscope}%
\pgfsetbuttcap%
\pgfsetroundjoin%
\definecolor{currentfill}{rgb}{1.000000,1.000000,1.000000}%
\pgfsetfillcolor{currentfill}%
\pgfsetlinewidth{0.000000pt}%
\definecolor{currentstroke}{rgb}{0.000000,0.000000,0.000000}%
\pgfsetstrokecolor{currentstroke}%
\pgfsetstrokeopacity{0.000000}%
\pgfsetdash{}{0pt}%
\pgfpathmoveto{\pgfqpoint{1.000000in}{0.600000in}}%
\pgfpathlineto{\pgfqpoint{7.200000in}{0.600000in}}%
\pgfpathlineto{\pgfqpoint{7.200000in}{5.400000in}}%
\pgfpathlineto{\pgfqpoint{1.000000in}{5.400000in}}%
\pgfpathclose%
\pgfusepath{fill}%
\end{pgfscope}%
\begin{pgfscope}%
\pgfpathrectangle{\pgfqpoint{1.000000in}{0.600000in}}{\pgfqpoint{6.200000in}{4.800000in}} %
\pgfusepath{clip}%
\pgfsetrectcap%
\pgfsetroundjoin%
\pgfsetlinewidth{1.003750pt}%
\definecolor{currentstroke}{rgb}{0.000000,0.000000,1.000000}%
\pgfsetstrokecolor{currentstroke}%
\pgfsetdash{}{0pt}%
\pgfpathmoveto{\pgfqpoint{1.000000in}{1.496000in}}%
\pgfpathlineto{\pgfqpoint{1.012401in}{1.497036in}}%
\pgfpathlineto{\pgfqpoint{1.039684in}{1.501310in}}%
\pgfpathlineto{\pgfqpoint{1.055186in}{1.502128in}}%
\pgfpathlineto{\pgfqpoint{1.082468in}{1.506045in}}%
\pgfpathlineto{\pgfqpoint{1.097970in}{1.506664in}}%
\pgfpathlineto{\pgfqpoint{1.124632in}{1.510206in}}%
\pgfpathlineto{\pgfqpoint{1.140754in}{1.510634in}}%
\pgfpathlineto{\pgfqpoint{1.166797in}{1.513756in}}%
\pgfpathlineto{\pgfqpoint{1.182918in}{1.513942in}}%
\pgfpathlineto{\pgfqpoint{1.208961in}{1.516663in}}%
\pgfpathlineto{\pgfqpoint{1.224462in}{1.516612in}}%
\pgfpathlineto{\pgfqpoint{1.251125in}{1.518886in}}%
\pgfpathlineto{\pgfqpoint{1.265387in}{1.518663in}}%
\pgfpathlineto{\pgfqpoint{1.292049in}{1.520496in}}%
\pgfpathlineto{\pgfqpoint{1.305691in}{1.520112in}}%
\pgfpathlineto{\pgfqpoint{1.331733in}{1.521489in}}%
\pgfpathlineto{\pgfqpoint{1.345375in}{1.520966in}}%
\pgfpathlineto{\pgfqpoint{1.369557in}{1.521972in}}%
\pgfpathlineto{\pgfqpoint{1.385059in}{1.521321in}}%
\pgfpathlineto{\pgfqpoint{1.403040in}{1.522354in}}%
\pgfpathlineto{\pgfqpoint{1.427843in}{1.521658in}}%
\pgfpathlineto{\pgfqpoint{1.437764in}{1.521960in}}%
\pgfpathlineto{\pgfqpoint{1.460086in}{1.520690in}}%
\pgfpathlineto{\pgfqpoint{1.479308in}{1.520784in}}%
\pgfpathlineto{\pgfqpoint{1.495430in}{1.520068in}}%
\pgfpathlineto{\pgfqpoint{1.512791in}{1.520826in}}%
\pgfpathlineto{\pgfqpoint{1.530773in}{1.519604in}}%
\pgfpathlineto{\pgfqpoint{1.556816in}{1.520199in}}%
\pgfpathlineto{\pgfqpoint{1.569217in}{1.519963in}}%
\pgfpathlineto{\pgfqpoint{1.595880in}{1.521287in}}%
\pgfpathlineto{\pgfqpoint{1.608281in}{1.521175in}}%
\pgfpathlineto{\pgfqpoint{1.638664in}{1.523213in}}%
\pgfpathlineto{\pgfqpoint{1.650445in}{1.523681in}}%
\pgfpathlineto{\pgfqpoint{1.677108in}{1.526612in}}%
\pgfpathlineto{\pgfqpoint{1.691369in}{1.527137in}}%
\pgfpathlineto{\pgfqpoint{1.721752in}{1.531077in}}%
\pgfpathlineto{\pgfqpoint{1.734153in}{1.532160in}}%
\pgfpathlineto{\pgfqpoint{1.763296in}{1.537016in}}%
\pgfpathlineto{\pgfqpoint{1.776938in}{1.538636in}}%
\pgfpathlineto{\pgfqpoint{1.806081in}{1.544513in}}%
\pgfpathlineto{\pgfqpoint{1.819102in}{1.546519in}}%
\pgfpathlineto{\pgfqpoint{1.850725in}{1.553891in}}%
\pgfpathlineto{\pgfqpoint{1.862506in}{1.556405in}}%
\pgfpathlineto{\pgfqpoint{1.891029in}{1.564461in}}%
\pgfpathlineto{\pgfqpoint{1.904050in}{1.567400in}}%
\pgfpathlineto{\pgfqpoint{1.936914in}{1.577604in}}%
\pgfpathlineto{\pgfqpoint{1.948695in}{1.581197in}}%
\pgfpathlineto{\pgfqpoint{1.973497in}{1.590403in}}%
\pgfpathlineto{\pgfqpoint{1.990239in}{1.595322in}}%
\pgfpathlineto{\pgfqpoint{2.017522in}{1.606042in}}%
\pgfpathlineto{\pgfqpoint{2.032403in}{1.610968in}}%
\pgfpathlineto{\pgfqpoint{2.060306in}{1.622476in}}%
\pgfpathlineto{\pgfqpoint{2.074567in}{1.627586in}}%
\pgfpathlineto{\pgfqpoint{2.101850in}{1.639371in}}%
\pgfpathlineto{\pgfqpoint{2.116112in}{1.644595in}}%
\pgfpathlineto{\pgfqpoint{2.145255in}{1.657470in}}%
\pgfpathlineto{\pgfqpoint{2.158896in}{1.662758in}}%
\pgfpathlineto{\pgfqpoint{2.186799in}{1.675684in}}%
\pgfpathlineto{\pgfqpoint{2.201060in}{1.681330in}}%
\pgfpathlineto{\pgfqpoint{2.232063in}{1.696182in}}%
\pgfpathlineto{\pgfqpoint{2.244464in}{1.701885in}}%
\pgfpathlineto{\pgfqpoint{2.272987in}{1.717062in}}%
\pgfpathlineto{\pgfqpoint{2.284768in}{1.723314in}}%
\pgfpathlineto{\pgfqpoint{2.330033in}{1.753848in}}%
\pgfpathlineto{\pgfqpoint{2.343674in}{1.763310in}}%
\pgfpathlineto{\pgfqpoint{2.357316in}{1.771734in}}%
\pgfpathlineto{\pgfqpoint{2.380878in}{1.787139in}}%
\pgfpathlineto{\pgfqpoint{2.397000in}{1.795670in}}%
\pgfpathlineto{\pgfqpoint{2.420562in}{1.809641in}}%
\pgfpathlineto{\pgfqpoint{2.437924in}{1.817947in}}%
\pgfpathlineto{\pgfqpoint{2.463346in}{1.832336in}}%
\pgfpathlineto{\pgfqpoint{2.480088in}{1.840142in}}%
\pgfpathlineto{\pgfqpoint{2.506131in}{1.854625in}}%
\pgfpathlineto{\pgfqpoint{2.522872in}{1.862255in}}%
\pgfpathlineto{\pgfqpoint{2.547675in}{1.875632in}}%
\pgfpathlineto{\pgfqpoint{2.566897in}{1.883722in}}%
\pgfpathlineto{\pgfqpoint{2.587359in}{1.893853in}}%
\pgfpathlineto{\pgfqpoint{2.615882in}{1.904123in}}%
\pgfpathlineto{\pgfqpoint{2.627663in}{1.908378in}}%
\pgfpathlineto{\pgfqpoint{2.671067in}{1.917110in}}%
\pgfpathlineto{\pgfqpoint{2.697730in}{1.918039in}}%
\pgfpathlineto{\pgfqpoint{2.705791in}{1.917638in}}%
\pgfpathlineto{\pgfqpoint{2.732453in}{1.914183in}}%
\pgfpathlineto{\pgfqpoint{2.743614in}{1.913558in}}%
\pgfpathlineto{\pgfqpoint{2.757256in}{1.912365in}}%
\pgfpathlineto{\pgfqpoint{2.767177in}{1.913707in}}%
\pgfpathlineto{\pgfqpoint{2.783298in}{1.916192in}}%
\pgfpathlineto{\pgfqpoint{2.798800in}{1.918017in}}%
\pgfpathlineto{\pgfqpoint{2.811201in}{1.922326in}}%
\pgfpathlineto{\pgfqpoint{2.825463in}{1.926683in}}%
\pgfpathlineto{\pgfqpoint{2.842204in}{1.931398in}}%
\pgfpathlineto{\pgfqpoint{2.891189in}{1.953494in}}%
\pgfpathlineto{\pgfqpoint{2.909791in}{1.963559in}}%
\pgfpathlineto{\pgfqpoint{2.929633in}{1.972969in}}%
\pgfpathlineto{\pgfqpoint{2.955676in}{1.987782in}}%
\pgfpathlineto{\pgfqpoint{2.971177in}{1.995616in}}%
\pgfpathlineto{\pgfqpoint{2.997840in}{2.011056in}}%
\pgfpathlineto{\pgfqpoint{3.013961in}{2.019224in}}%
\pgfpathlineto{\pgfqpoint{3.039384in}{2.034029in}}%
\pgfpathlineto{\pgfqpoint{3.056126in}{2.042388in}}%
\pgfpathlineto{\pgfqpoint{3.083408in}{2.058317in}}%
\pgfpathlineto{\pgfqpoint{3.097670in}{2.065837in}}%
\pgfpathlineto{\pgfqpoint{3.172697in}{2.115837in}}%
\pgfpathlineto{\pgfqpoint{3.193779in}{2.132466in}}%
\pgfpathlineto{\pgfqpoint{3.212381in}{2.144130in}}%
\pgfpathlineto{\pgfqpoint{3.231603in}{2.157174in}}%
\pgfpathlineto{\pgfqpoint{3.252685in}{2.168814in}}%
\pgfpathlineto{\pgfqpoint{3.273147in}{2.181642in}}%
\pgfpathlineto{\pgfqpoint{3.293609in}{2.192335in}}%
\pgfpathlineto{\pgfqpoint{3.315932in}{2.206133in}}%
\pgfpathlineto{\pgfqpoint{3.335154in}{2.216065in}}%
\pgfpathlineto{\pgfqpoint{3.359336in}{2.230936in}}%
\pgfpathlineto{\pgfqpoint{3.377938in}{2.240469in}}%
\pgfpathlineto{\pgfqpoint{3.400880in}{2.254177in}}%
\pgfpathlineto{\pgfqpoint{3.422582in}{2.264456in}}%
\pgfpathlineto{\pgfqpoint{3.439944in}{2.273836in}}%
\pgfpathlineto{\pgfqpoint{3.485829in}{2.290173in}}%
\pgfpathlineto{\pgfqpoint{3.509391in}{2.294733in}}%
\pgfpathlineto{\pgfqpoint{3.519312in}{2.296445in}}%
\pgfpathlineto{\pgfqpoint{3.532953in}{2.295462in}}%
\pgfpathlineto{\pgfqpoint{3.544114in}{2.295308in}}%
\pgfpathlineto{\pgfqpoint{3.555896in}{2.294949in}}%
\pgfpathlineto{\pgfqpoint{3.575738in}{2.293543in}}%
\pgfpathlineto{\pgfqpoint{3.589999in}{2.296681in}}%
\pgfpathlineto{\pgfqpoint{3.615422in}{2.301941in}}%
\pgfpathlineto{\pgfqpoint{3.659446in}{2.322080in}}%
\pgfpathlineto{\pgfqpoint{3.682388in}{2.335976in}}%
\pgfpathlineto{\pgfqpoint{3.696650in}{2.344301in}}%
\pgfpathlineto{\pgfqpoint{3.784078in}{2.408178in}}%
\pgfpathlineto{\pgfqpoint{3.806401in}{2.427337in}}%
\pgfpathlineto{\pgfqpoint{3.819422in}{2.438782in}}%
\pgfpathlineto{\pgfqpoint{3.841124in}{2.461223in}}%
\pgfpathlineto{\pgfqpoint{3.860966in}{2.480772in}}%
\pgfpathlineto{\pgfqpoint{3.871507in}{2.490225in}}%
\pgfpathlineto{\pgfqpoint{3.930413in}{2.534488in}}%
\pgfpathlineto{\pgfqpoint{3.963896in}{2.558120in}}%
\pgfpathlineto{\pgfqpoint{3.979398in}{2.568487in}}%
\pgfpathlineto{\pgfqpoint{3.996140in}{2.579529in}}%
\pgfpathlineto{\pgfqpoint{4.042644in}{2.602231in}}%
\pgfpathlineto{\pgfqpoint{4.062486in}{2.608559in}}%
\pgfpathlineto{\pgfqpoint{4.076128in}{2.612323in}}%
\pgfpathlineto{\pgfqpoint{4.086049in}{2.612450in}}%
\pgfpathlineto{\pgfqpoint{4.100930in}{2.613279in}}%
\pgfpathlineto{\pgfqpoint{4.112711in}{2.613137in}}%
\pgfpathlineto{\pgfqpoint{4.129453in}{2.611866in}}%
\pgfpathlineto{\pgfqpoint{4.167277in}{2.618657in}}%
\pgfpathlineto{\pgfqpoint{4.206341in}{2.634881in}}%
\pgfpathlineto{\pgfqpoint{4.219982in}{2.643492in}}%
\pgfpathlineto{\pgfqpoint{4.236104in}{2.652698in}}%
\pgfpathlineto{\pgfqpoint{4.250985in}{2.661505in}}%
\pgfpathlineto{\pgfqpoint{4.286949in}{2.686318in}}%
\pgfpathlineto{\pgfqpoint{4.299970in}{2.696509in}}%
\pgfpathlineto{\pgfqpoint{4.316712in}{2.709374in}}%
\pgfpathlineto{\pgfqpoint{4.337174in}{2.723992in}}%
\pgfpathlineto{\pgfqpoint{4.363836in}{2.745808in}}%
\pgfpathlineto{\pgfqpoint{4.376238in}{2.756189in}}%
\pgfpathlineto{\pgfqpoint{4.411581in}{2.791536in}}%
\pgfpathlineto{\pgfqpoint{4.427703in}{2.809481in}}%
\pgfpathlineto{\pgfqpoint{4.440724in}{2.820247in}}%
\pgfpathlineto{\pgfqpoint{4.456226in}{2.834982in}}%
\pgfpathlineto{\pgfqpoint{4.469247in}{2.846244in}}%
\pgfpathlineto{\pgfqpoint{4.492189in}{2.864076in}}%
\pgfpathlineto{\pgfqpoint{4.510171in}{2.879300in}}%
\pgfpathlineto{\pgfqpoint{4.535594in}{2.898054in}}%
\pgfpathlineto{\pgfqpoint{4.550475in}{2.909744in}}%
\pgfpathlineto{\pgfqpoint{4.593879in}{2.935956in}}%
\pgfpathlineto{\pgfqpoint{4.627983in}{2.947838in}}%
\pgfpathlineto{\pgfqpoint{4.637904in}{2.948180in}}%
\pgfpathlineto{\pgfqpoint{4.660226in}{2.948159in}}%
\pgfpathlineto{\pgfqpoint{4.670767in}{2.947576in}}%
\pgfpathlineto{\pgfqpoint{4.679448in}{2.948171in}}%
\pgfpathlineto{\pgfqpoint{4.688749in}{2.951066in}}%
\pgfpathlineto{\pgfqpoint{4.727193in}{2.968712in}}%
\pgfpathlineto{\pgfqpoint{4.753235in}{2.986463in}}%
\pgfpathlineto{\pgfqpoint{4.765017in}{2.995292in}}%
\pgfpathlineto{\pgfqpoint{4.799740in}{3.024195in}}%
\pgfpathlineto{\pgfqpoint{4.812141in}{3.035892in}}%
\pgfpathlineto{\pgfqpoint{4.830743in}{3.053259in}}%
\pgfpathlineto{\pgfqpoint{4.846245in}{3.067333in}}%
\pgfpathlineto{\pgfqpoint{4.887169in}{3.114241in}}%
\pgfpathlineto{\pgfqpoint{4.897710in}{3.126832in}}%
\pgfpathlineto{\pgfqpoint{4.938634in}{3.166581in}}%
\pgfpathlineto{\pgfqpoint{4.965917in}{3.189894in}}%
\pgfpathlineto{\pgfqpoint{4.979558in}{3.201566in}}%
\pgfpathlineto{\pgfqpoint{5.009941in}{3.224844in}}%
\pgfpathlineto{\pgfqpoint{5.021722in}{3.233100in}}%
\pgfpathlineto{\pgfqpoint{5.063886in}{3.255607in}}%
\pgfpathlineto{\pgfqpoint{5.098610in}{3.263118in}}%
\pgfpathlineto{\pgfqpoint{5.114731in}{3.261902in}}%
\pgfpathlineto{\pgfqpoint{5.152555in}{3.269258in}}%
\pgfpathlineto{\pgfqpoint{5.186039in}{3.286205in}}%
\pgfpathlineto{\pgfqpoint{5.197820in}{3.294839in}}%
\pgfpathlineto{\pgfqpoint{5.221382in}{3.312362in}}%
\pgfpathlineto{\pgfqpoint{5.233783in}{3.322359in}}%
\pgfpathlineto{\pgfqpoint{5.267267in}{3.351840in}}%
\pgfpathlineto{\pgfqpoint{5.279668in}{3.364078in}}%
\pgfpathlineto{\pgfqpoint{5.300130in}{3.384514in}}%
\pgfpathlineto{\pgfqpoint{5.315012in}{3.400298in}}%
\pgfpathlineto{\pgfqpoint{5.348495in}{3.442299in}}%
\pgfpathlineto{\pgfqpoint{5.368337in}{3.464870in}}%
\pgfpathlineto{\pgfqpoint{5.387559in}{3.484068in}}%
\pgfpathlineto{\pgfqpoint{5.406161in}{3.502721in}}%
\pgfpathlineto{\pgfqpoint{5.430343in}{3.524426in}}%
\pgfpathlineto{\pgfqpoint{5.445845in}{3.538252in}}%
\pgfpathlineto{\pgfqpoint{5.483048in}{3.564193in}}%
\pgfpathlineto{\pgfqpoint{5.492969in}{3.568124in}}%
\pgfpathlineto{\pgfqpoint{5.522732in}{3.575639in}}%
\pgfpathlineto{\pgfqpoint{5.541334in}{3.574518in}}%
\pgfpathlineto{\pgfqpoint{5.569237in}{3.582412in}}%
\pgfpathlineto{\pgfqpoint{5.577918in}{3.587157in}}%
\pgfpathlineto{\pgfqpoint{5.615122in}{3.613880in}}%
\pgfpathlineto{\pgfqpoint{5.657906in}{3.653860in}}%
\pgfpathlineto{\pgfqpoint{5.705651in}{3.706361in}}%
\pgfpathlineto{\pgfqpoint{5.726733in}{3.733088in}}%
\pgfpathlineto{\pgfqpoint{5.755256in}{3.768612in}}%
\pgfpathlineto{\pgfqpoint{5.770757in}{3.784750in}}%
\pgfpathlineto{\pgfqpoint{5.789359in}{3.804540in}}%
\pgfpathlineto{\pgfqpoint{5.817262in}{3.830411in}}%
\pgfpathlineto{\pgfqpoint{5.829663in}{3.841739in}}%
\pgfpathlineto{\pgfqpoint{5.869347in}{3.870739in}}%
\pgfpathlineto{\pgfqpoint{5.881128in}{3.875500in}}%
\pgfpathlineto{\pgfqpoint{5.909031in}{3.884442in}}%
\pgfpathlineto{\pgfqpoint{5.935074in}{3.885828in}}%
\pgfpathlineto{\pgfqpoint{5.961116in}{3.893502in}}%
\pgfpathlineto{\pgfqpoint{5.974137in}{3.902307in}}%
\pgfpathlineto{\pgfqpoint{6.004520in}{3.924462in}}%
\pgfpathlineto{\pgfqpoint{6.058466in}{3.978852in}}%
\pgfpathlineto{\pgfqpoint{6.087609in}{4.012275in}}%
\pgfpathlineto{\pgfqpoint{6.140934in}{4.083444in}}%
\pgfpathlineto{\pgfqpoint{6.158916in}{4.104244in}}%
\pgfpathlineto{\pgfqpoint{6.172557in}{4.119129in}}%
\pgfpathlineto{\pgfqpoint{6.218442in}{4.161533in}}%
\pgfpathlineto{\pgfqpoint{6.248825in}{4.180716in}}%
\pgfpathlineto{\pgfqpoint{6.259366in}{4.182921in}}%
\pgfpathlineto{\pgfqpoint{6.276728in}{4.185040in}}%
\pgfpathlineto{\pgfqpoint{6.292849in}{4.185692in}}%
\pgfpathlineto{\pgfqpoint{6.299670in}{4.188021in}}%
\pgfpathlineto{\pgfqpoint{6.312691in}{4.195385in}}%
\pgfpathlineto{\pgfqpoint{6.339354in}{4.213408in}}%
\pgfpathlineto{\pgfqpoint{6.379658in}{4.252640in}}%
\pgfpathlineto{\pgfqpoint{6.428023in}{4.309843in}}%
\pgfpathlineto{\pgfqpoint{6.460266in}{4.355791in}}%
\pgfpathlineto{\pgfqpoint{6.470187in}{4.369244in}}%
\pgfpathlineto{\pgfqpoint{6.525993in}{4.430152in}}%
\pgfpathlineto{\pgfqpoint{6.552655in}{4.456500in}}%
\pgfpathlineto{\pgfqpoint{6.591099in}{4.482550in}}%
\pgfpathlineto{\pgfqpoint{6.618382in}{4.488966in}}%
\pgfpathlineto{\pgfqpoint{6.639464in}{4.490157in}}%
\pgfpathlineto{\pgfqpoint{6.653105in}{4.497784in}}%
\pgfpathlineto{\pgfqpoint{6.677908in}{4.513131in}}%
\pgfpathlineto{\pgfqpoint{6.720072in}{4.555071in}}%
\pgfpathlineto{\pgfqpoint{6.743634in}{4.583209in}}%
\pgfpathlineto{\pgfqpoint{6.756656in}{4.598620in}}%
\pgfpathlineto{\pgfqpoint{6.787659in}{4.645106in}}%
\pgfpathlineto{\pgfqpoint{6.806261in}{4.673857in}}%
\pgfpathlineto{\pgfqpoint{6.852765in}{4.728398in}}%
\pgfpathlineto{\pgfqpoint{6.867027in}{4.742596in}}%
\pgfpathlineto{\pgfqpoint{6.884388in}{4.760547in}}%
\pgfpathlineto{\pgfqpoint{6.893689in}{4.766104in}}%
\pgfpathlineto{\pgfqpoint{6.909191in}{4.775335in}}%
\pgfpathlineto{\pgfqpoint{6.919112in}{4.780373in}}%
\pgfpathlineto{\pgfqpoint{6.925313in}{4.780786in}}%
\pgfpathlineto{\pgfqpoint{6.940814in}{4.780942in}}%
\pgfpathlineto{\pgfqpoint{6.958796in}{4.787186in}}%
\pgfpathlineto{\pgfqpoint{6.968717in}{4.791278in}}%
\pgfpathlineto{\pgfqpoint{6.976778in}{4.797364in}}%
\pgfpathlineto{\pgfqpoint{7.013981in}{4.832062in}}%
\pgfpathlineto{\pgfqpoint{7.061106in}{4.889454in}}%
\pgfpathlineto{\pgfqpoint{7.097690in}{4.944777in}}%
\pgfpathlineto{\pgfqpoint{7.106371in}{4.957678in}}%
\pgfpathlineto{\pgfqpoint{7.149155in}{5.010810in}}%
\pgfpathlineto{\pgfqpoint{7.167757in}{5.030670in}}%
\pgfpathlineto{\pgfqpoint{7.183258in}{5.047725in}}%
\pgfpathlineto{\pgfqpoint{7.196900in}{5.057782in}}%
\pgfpathlineto{\pgfqpoint{7.200000in}{5.060078in}}%
\pgfpathlineto{\pgfqpoint{7.200000in}{5.060078in}}%
\pgfusepath{stroke}%
\end{pgfscope}%
\begin{pgfscope}%
\pgfpathrectangle{\pgfqpoint{1.000000in}{0.600000in}}{\pgfqpoint{6.200000in}{4.800000in}} %
\pgfusepath{clip}%
\pgfsetrectcap%
\pgfsetroundjoin%
\pgfsetlinewidth{1.003750pt}%
\definecolor{currentstroke}{rgb}{0.000000,0.500000,0.000000}%
\pgfsetstrokecolor{currentstroke}%
\pgfsetdash{}{0pt}%
\pgfpathmoveto{\pgfqpoint{1.000000in}{1.400000in}}%
\pgfpathlineto{\pgfqpoint{1.008681in}{1.401522in}}%
\pgfpathlineto{\pgfqpoint{1.016122in}{1.400430in}}%
\pgfpathlineto{\pgfqpoint{1.031003in}{1.397330in}}%
\pgfpathlineto{\pgfqpoint{1.038444in}{1.398663in}}%
\pgfpathlineto{\pgfqpoint{1.052085in}{1.401509in}}%
\pgfpathlineto{\pgfqpoint{1.059526in}{1.400264in}}%
\pgfpathlineto{\pgfqpoint{1.073787in}{1.397319in}}%
\pgfpathlineto{\pgfqpoint{1.081228in}{1.398661in}}%
\pgfpathlineto{\pgfqpoint{1.094249in}{1.401501in}}%
\pgfpathlineto{\pgfqpoint{1.101690in}{1.400414in}}%
\pgfpathlineto{\pgfqpoint{1.116572in}{1.397315in}}%
\pgfpathlineto{\pgfqpoint{1.124012in}{1.398709in}}%
\pgfpathlineto{\pgfqpoint{1.137034in}{1.401485in}}%
\pgfpathlineto{\pgfqpoint{1.144474in}{1.400340in}}%
\pgfpathlineto{\pgfqpoint{1.158736in}{1.397294in}}%
\pgfpathlineto{\pgfqpoint{1.166177in}{1.398657in}}%
\pgfpathlineto{\pgfqpoint{1.179198in}{1.401468in}}%
\pgfpathlineto{\pgfqpoint{1.186639in}{1.400348in}}%
\pgfpathlineto{\pgfqpoint{1.200900in}{1.397280in}}%
\pgfpathlineto{\pgfqpoint{1.208341in}{1.398723in}}%
\pgfpathlineto{\pgfqpoint{1.220742in}{1.401452in}}%
\pgfpathlineto{\pgfqpoint{1.228183in}{1.400395in}}%
\pgfpathlineto{\pgfqpoint{1.243064in}{1.397293in}}%
\pgfpathlineto{\pgfqpoint{1.250505in}{1.398971in}}%
\pgfpathlineto{\pgfqpoint{1.261666in}{1.401437in}}%
\pgfpathlineto{\pgfqpoint{1.269107in}{1.400432in}}%
\pgfpathlineto{\pgfqpoint{1.283988in}{1.397283in}}%
\pgfpathlineto{\pgfqpoint{1.291429in}{1.399092in}}%
\pgfpathlineto{\pgfqpoint{1.301970in}{1.401415in}}%
\pgfpathlineto{\pgfqpoint{1.309411in}{1.400410in}}%
\pgfpathlineto{\pgfqpoint{1.323672in}{1.397269in}}%
\pgfpathlineto{\pgfqpoint{1.331113in}{1.399161in}}%
\pgfpathlineto{\pgfqpoint{1.341034in}{1.401357in}}%
\pgfpathlineto{\pgfqpoint{1.347855in}{1.400632in}}%
\pgfpathlineto{\pgfqpoint{1.362736in}{1.397447in}}%
\pgfpathlineto{\pgfqpoint{1.385679in}{1.400663in}}%
\pgfpathlineto{\pgfqpoint{1.396840in}{1.397655in}}%
\pgfpathlineto{\pgfqpoint{1.406761in}{1.398783in}}%
\pgfpathlineto{\pgfqpoint{1.416682in}{1.401438in}}%
\pgfpathlineto{\pgfqpoint{1.428463in}{1.398831in}}%
\pgfpathlineto{\pgfqpoint{1.434663in}{1.397243in}}%
\pgfpathlineto{\pgfqpoint{1.445825in}{1.399702in}}%
\pgfpathlineto{\pgfqpoint{1.453265in}{1.401602in}}%
\pgfpathlineto{\pgfqpoint{1.466287in}{1.398234in}}%
\pgfpathlineto{\pgfqpoint{1.471247in}{1.397311in}}%
\pgfpathlineto{\pgfqpoint{1.482408in}{1.399895in}}%
\pgfpathlineto{\pgfqpoint{1.490469in}{1.401707in}}%
\pgfpathlineto{\pgfqpoint{1.502250in}{1.398646in}}%
\pgfpathlineto{\pgfqpoint{1.510931in}{1.397553in}}%
\pgfpathlineto{\pgfqpoint{1.517752in}{1.399927in}}%
\pgfpathlineto{\pgfqpoint{1.524572in}{1.401313in}}%
\pgfpathlineto{\pgfqpoint{1.533253in}{1.400640in}}%
\pgfpathlineto{\pgfqpoint{1.549375in}{1.397844in}}%
\pgfpathlineto{\pgfqpoint{1.567977in}{1.401376in}}%
\pgfpathlineto{\pgfqpoint{1.576038in}{1.399456in}}%
\pgfpathlineto{\pgfqpoint{1.585339in}{1.397462in}}%
\pgfpathlineto{\pgfqpoint{1.592159in}{1.398719in}}%
\pgfpathlineto{\pgfqpoint{1.604560in}{1.401535in}}%
\pgfpathlineto{\pgfqpoint{1.612001in}{1.400672in}}%
\pgfpathlineto{\pgfqpoint{1.628123in}{1.397651in}}%
\pgfpathlineto{\pgfqpoint{1.636184in}{1.399687in}}%
\pgfpathlineto{\pgfqpoint{1.645485in}{1.401574in}}%
\pgfpathlineto{\pgfqpoint{1.652925in}{1.400692in}}%
\pgfpathlineto{\pgfqpoint{1.669667in}{1.397734in}}%
\pgfpathlineto{\pgfqpoint{1.677728in}{1.399589in}}%
\pgfpathlineto{\pgfqpoint{1.687649in}{1.401587in}}%
\pgfpathlineto{\pgfqpoint{1.695090in}{1.400619in}}%
\pgfpathlineto{\pgfqpoint{1.711211in}{1.397792in}}%
\pgfpathlineto{\pgfqpoint{1.719272in}{1.399384in}}%
\pgfpathlineto{\pgfqpoint{1.730433in}{1.401571in}}%
\pgfpathlineto{\pgfqpoint{1.738494in}{1.400365in}}%
\pgfpathlineto{\pgfqpoint{1.752135in}{1.397798in}}%
\pgfpathlineto{\pgfqpoint{1.760196in}{1.399003in}}%
\pgfpathlineto{\pgfqpoint{1.773837in}{1.401536in}}%
\pgfpathlineto{\pgfqpoint{1.781898in}{1.400164in}}%
\pgfpathlineto{\pgfqpoint{1.794919in}{1.397791in}}%
\pgfpathlineto{\pgfqpoint{1.802980in}{1.399078in}}%
\pgfpathlineto{\pgfqpoint{1.816002in}{1.401544in}}%
\pgfpathlineto{\pgfqpoint{1.824062in}{1.400303in}}%
\pgfpathlineto{\pgfqpoint{1.837704in}{1.397749in}}%
\pgfpathlineto{\pgfqpoint{1.845765in}{1.399094in}}%
\pgfpathlineto{\pgfqpoint{1.858786in}{1.401575in}}%
\pgfpathlineto{\pgfqpoint{1.866847in}{1.400271in}}%
\pgfpathlineto{\pgfqpoint{1.880488in}{1.397705in}}%
\pgfpathlineto{\pgfqpoint{1.888549in}{1.399078in}}%
\pgfpathlineto{\pgfqpoint{1.900950in}{1.401620in}}%
\pgfpathlineto{\pgfqpoint{1.908391in}{1.400550in}}%
\pgfpathlineto{\pgfqpoint{1.923272in}{1.397677in}}%
\pgfpathlineto{\pgfqpoint{1.931333in}{1.399082in}}%
\pgfpathlineto{\pgfqpoint{1.943734in}{1.401613in}}%
\pgfpathlineto{\pgfqpoint{1.951175in}{1.400473in}}%
\pgfpathlineto{\pgfqpoint{1.965437in}{1.397637in}}%
\pgfpathlineto{\pgfqpoint{1.973497in}{1.398979in}}%
\pgfpathlineto{\pgfqpoint{1.986519in}{1.401526in}}%
\pgfpathlineto{\pgfqpoint{1.993959in}{1.400334in}}%
\pgfpathlineto{\pgfqpoint{2.007601in}{1.397588in}}%
\pgfpathlineto{\pgfqpoint{2.015662in}{1.398952in}}%
\pgfpathlineto{\pgfqpoint{2.028683in}{1.401386in}}%
\pgfpathlineto{\pgfqpoint{2.036124in}{1.400293in}}%
\pgfpathlineto{\pgfqpoint{2.050385in}{1.397567in}}%
\pgfpathlineto{\pgfqpoint{2.059066in}{1.399281in}}%
\pgfpathlineto{\pgfqpoint{2.070227in}{1.401268in}}%
\pgfpathlineto{\pgfqpoint{2.078288in}{1.400197in}}%
\pgfpathlineto{\pgfqpoint{2.092549in}{1.397535in}}%
\pgfpathlineto{\pgfqpoint{2.100610in}{1.399073in}}%
\pgfpathlineto{\pgfqpoint{2.112391in}{1.401245in}}%
\pgfpathlineto{\pgfqpoint{2.120452in}{1.400133in}}%
\pgfpathlineto{\pgfqpoint{2.134713in}{1.397526in}}%
\pgfpathlineto{\pgfqpoint{2.142774in}{1.398979in}}%
\pgfpathlineto{\pgfqpoint{2.155176in}{1.401306in}}%
\pgfpathlineto{\pgfqpoint{2.163236in}{1.400054in}}%
\pgfpathlineto{\pgfqpoint{2.176878in}{1.397530in}}%
\pgfpathlineto{\pgfqpoint{2.184938in}{1.398895in}}%
\pgfpathlineto{\pgfqpoint{2.197960in}{1.401407in}}%
\pgfpathlineto{\pgfqpoint{2.206021in}{1.400123in}}%
\pgfpathlineto{\pgfqpoint{2.219042in}{1.397563in}}%
\pgfpathlineto{\pgfqpoint{2.226483in}{1.398764in}}%
\pgfpathlineto{\pgfqpoint{2.240744in}{1.401547in}}%
\pgfpathlineto{\pgfqpoint{2.248805in}{1.400208in}}%
\pgfpathlineto{\pgfqpoint{2.261206in}{1.397724in}}%
\pgfpathlineto{\pgfqpoint{2.268027in}{1.399002in}}%
\pgfpathlineto{\pgfqpoint{2.280428in}{1.401808in}}%
\pgfpathlineto{\pgfqpoint{2.288489in}{1.400786in}}%
\pgfpathlineto{\pgfqpoint{2.303370in}{1.398275in}}%
\pgfpathlineto{\pgfqpoint{2.321972in}{1.401637in}}%
\pgfpathlineto{\pgfqpoint{2.325693in}{1.400413in}}%
\pgfpathlineto{\pgfqpoint{2.331893in}{1.397932in}}%
\pgfpathlineto{\pgfqpoint{2.344294in}{1.398542in}}%
\pgfpathlineto{\pgfqpoint{2.351115in}{1.400555in}}%
\pgfpathlineto{\pgfqpoint{2.360416in}{1.399672in}}%
\pgfpathlineto{\pgfqpoint{2.377158in}{1.397276in}}%
\pgfpathlineto{\pgfqpoint{2.393899in}{1.400886in}}%
\pgfpathlineto{\pgfqpoint{2.402580in}{1.399180in}}%
\pgfpathlineto{\pgfqpoint{2.413121in}{1.397137in}}%
\pgfpathlineto{\pgfqpoint{2.419942in}{1.398279in}}%
\pgfpathlineto{\pgfqpoint{2.433583in}{1.401299in}}%
\pgfpathlineto{\pgfqpoint{2.441024in}{1.400227in}}%
\pgfpathlineto{\pgfqpoint{2.455906in}{1.397380in}}%
\pgfpathlineto{\pgfqpoint{2.463966in}{1.398989in}}%
\pgfpathlineto{\pgfqpoint{2.475748in}{1.401438in}}%
\pgfpathlineto{\pgfqpoint{2.483188in}{1.400361in}}%
\pgfpathlineto{\pgfqpoint{2.498070in}{1.397367in}}%
\pgfpathlineto{\pgfqpoint{2.506131in}{1.398853in}}%
\pgfpathlineto{\pgfqpoint{2.518532in}{1.401377in}}%
\pgfpathlineto{\pgfqpoint{2.525973in}{1.400221in}}%
\pgfpathlineto{\pgfqpoint{2.540234in}{1.397195in}}%
\pgfpathlineto{\pgfqpoint{2.547675in}{1.398377in}}%
\pgfpathlineto{\pgfqpoint{2.561316in}{1.401153in}}%
\pgfpathlineto{\pgfqpoint{2.568757in}{1.399942in}}%
\pgfpathlineto{\pgfqpoint{2.583018in}{1.396951in}}%
\pgfpathlineto{\pgfqpoint{2.591079in}{1.398355in}}%
\pgfpathlineto{\pgfqpoint{2.603480in}{1.400847in}}%
\pgfpathlineto{\pgfqpoint{2.610921in}{1.399746in}}%
\pgfpathlineto{\pgfqpoint{2.625803in}{1.396691in}}%
\pgfpathlineto{\pgfqpoint{2.633863in}{1.398305in}}%
\pgfpathlineto{\pgfqpoint{2.645025in}{1.400491in}}%
\pgfpathlineto{\pgfqpoint{2.652465in}{1.399503in}}%
\pgfpathlineto{\pgfqpoint{2.667347in}{1.396395in}}%
\pgfpathlineto{\pgfqpoint{2.676028in}{1.398610in}}%
\pgfpathlineto{\pgfqpoint{2.685329in}{1.400064in}}%
\pgfpathlineto{\pgfqpoint{2.692149in}{1.399192in}}%
\pgfpathlineto{\pgfqpoint{2.703310in}{1.396562in}}%
\pgfpathlineto{\pgfqpoint{2.716332in}{1.400468in}}%
\pgfpathlineto{\pgfqpoint{2.720672in}{1.401709in}}%
\pgfpathlineto{\pgfqpoint{2.734933in}{1.399015in}}%
\pgfpathlineto{\pgfqpoint{2.739894in}{1.398601in}}%
\pgfpathlineto{\pgfqpoint{2.750435in}{1.401688in}}%
\pgfpathlineto{\pgfqpoint{2.759736in}{1.402500in}}%
\pgfpathlineto{\pgfqpoint{2.767177in}{1.400809in}}%
\pgfpathlineto{\pgfqpoint{2.776478in}{1.398646in}}%
\pgfpathlineto{\pgfqpoint{2.783298in}{1.399579in}}%
\pgfpathlineto{\pgfqpoint{2.797560in}{1.402392in}}%
\pgfpathlineto{\pgfqpoint{2.805621in}{1.401167in}}%
\pgfpathlineto{\pgfqpoint{2.819262in}{1.398462in}}%
\pgfpathlineto{\pgfqpoint{2.827323in}{1.399856in}}%
\pgfpathlineto{\pgfqpoint{2.839724in}{1.402166in}}%
\pgfpathlineto{\pgfqpoint{2.847785in}{1.400937in}}%
\pgfpathlineto{\pgfqpoint{2.862046in}{1.398292in}}%
\pgfpathlineto{\pgfqpoint{2.870107in}{1.399676in}}%
\pgfpathlineto{\pgfqpoint{2.882508in}{1.401874in}}%
\pgfpathlineto{\pgfqpoint{2.890569in}{1.400620in}}%
\pgfpathlineto{\pgfqpoint{2.904210in}{1.398058in}}%
\pgfpathlineto{\pgfqpoint{2.912271in}{1.399380in}}%
\pgfpathlineto{\pgfqpoint{2.924672in}{1.401574in}}%
\pgfpathlineto{\pgfqpoint{2.932733in}{1.400423in}}%
\pgfpathlineto{\pgfqpoint{2.946995in}{1.397868in}}%
\pgfpathlineto{\pgfqpoint{2.955676in}{1.399484in}}%
\pgfpathlineto{\pgfqpoint{2.966837in}{1.401397in}}%
\pgfpathlineto{\pgfqpoint{2.974897in}{1.400262in}}%
\pgfpathlineto{\pgfqpoint{2.989159in}{1.397749in}}%
\pgfpathlineto{\pgfqpoint{2.997220in}{1.399159in}}%
\pgfpathlineto{\pgfqpoint{3.009621in}{1.401384in}}%
\pgfpathlineto{\pgfqpoint{3.017682in}{1.400124in}}%
\pgfpathlineto{\pgfqpoint{3.030703in}{1.397705in}}%
\pgfpathlineto{\pgfqpoint{3.038764in}{1.398933in}}%
\pgfpathlineto{\pgfqpoint{3.052405in}{1.401480in}}%
\pgfpathlineto{\pgfqpoint{3.060466in}{1.400176in}}%
\pgfpathlineto{\pgfqpoint{3.072867in}{1.397748in}}%
\pgfpathlineto{\pgfqpoint{3.080308in}{1.398899in}}%
\pgfpathlineto{\pgfqpoint{3.094569in}{1.401653in}}%
\pgfpathlineto{\pgfqpoint{3.102630in}{1.400395in}}%
\pgfpathlineto{\pgfqpoint{3.115032in}{1.397945in}}%
\pgfpathlineto{\pgfqpoint{3.121852in}{1.399395in}}%
\pgfpathlineto{\pgfqpoint{3.132393in}{1.401845in}}%
\pgfpathlineto{\pgfqpoint{3.141074in}{1.401073in}}%
\pgfpathlineto{\pgfqpoint{3.155956in}{1.398557in}}%
\pgfpathlineto{\pgfqpoint{3.165257in}{1.400435in}}%
\pgfpathlineto{\pgfqpoint{3.173317in}{1.401176in}}%
\pgfpathlineto{\pgfqpoint{3.183238in}{1.398141in}}%
\pgfpathlineto{\pgfqpoint{3.191299in}{1.396887in}}%
\pgfpathlineto{\pgfqpoint{3.210521in}{1.400314in}}%
\pgfpathlineto{\pgfqpoint{3.219822in}{1.397902in}}%
\pgfpathlineto{\pgfqpoint{3.226643in}{1.396895in}}%
\pgfpathlineto{\pgfqpoint{3.233463in}{1.398363in}}%
\pgfpathlineto{\pgfqpoint{3.245245in}{1.401034in}}%
\pgfpathlineto{\pgfqpoint{3.253305in}{1.400091in}}%
\pgfpathlineto{\pgfqpoint{3.268807in}{1.397332in}}%
\pgfpathlineto{\pgfqpoint{3.276868in}{1.399201in}}%
\pgfpathlineto{\pgfqpoint{3.287409in}{1.401298in}}%
\pgfpathlineto{\pgfqpoint{3.294849in}{1.400318in}}%
\pgfpathlineto{\pgfqpoint{3.310351in}{1.397393in}}%
\pgfpathlineto{\pgfqpoint{3.318412in}{1.398999in}}%
\pgfpathlineto{\pgfqpoint{3.330193in}{1.401344in}}%
\pgfpathlineto{\pgfqpoint{3.337634in}{1.400230in}}%
\pgfpathlineto{\pgfqpoint{3.352515in}{1.397275in}}%
\pgfpathlineto{\pgfqpoint{3.360576in}{1.398755in}}%
\pgfpathlineto{\pgfqpoint{3.372977in}{1.401177in}}%
\pgfpathlineto{\pgfqpoint{3.380418in}{1.399979in}}%
\pgfpathlineto{\pgfqpoint{3.394679in}{1.396988in}}%
\pgfpathlineto{\pgfqpoint{3.402740in}{1.398334in}}%
\pgfpathlineto{\pgfqpoint{3.415762in}{1.400830in}}%
\pgfpathlineto{\pgfqpoint{3.423202in}{1.399567in}}%
\pgfpathlineto{\pgfqpoint{3.436844in}{1.396612in}}%
\pgfpathlineto{\pgfqpoint{3.444284in}{1.397718in}}%
\pgfpathlineto{\pgfqpoint{3.458546in}{1.400390in}}%
\pgfpathlineto{\pgfqpoint{3.465987in}{1.399013in}}%
\pgfpathlineto{\pgfqpoint{3.479008in}{1.396196in}}%
\pgfpathlineto{\pgfqpoint{3.486449in}{1.397512in}}%
\pgfpathlineto{\pgfqpoint{3.498850in}{1.399896in}}%
\pgfpathlineto{\pgfqpoint{3.505671in}{1.398916in}}%
\pgfpathlineto{\pgfqpoint{3.519312in}{1.395914in}}%
\pgfpathlineto{\pgfqpoint{3.533573in}{1.399517in}}%
\pgfpathlineto{\pgfqpoint{3.539774in}{1.400291in}}%
\pgfpathlineto{\pgfqpoint{3.554035in}{1.398114in}}%
\pgfpathlineto{\pgfqpoint{3.558376in}{1.398989in}}%
\pgfpathlineto{\pgfqpoint{3.567057in}{1.402331in}}%
\pgfpathlineto{\pgfqpoint{3.576978in}{1.402247in}}%
\pgfpathlineto{\pgfqpoint{3.594339in}{1.399476in}}%
\pgfpathlineto{\pgfqpoint{3.611701in}{1.402554in}}%
\pgfpathlineto{\pgfqpoint{3.620382in}{1.400678in}}%
\pgfpathlineto{\pgfqpoint{3.630923in}{1.398583in}}%
\pgfpathlineto{\pgfqpoint{3.638364in}{1.399748in}}%
\pgfpathlineto{\pgfqpoint{3.651385in}{1.402221in}}%
\pgfpathlineto{\pgfqpoint{3.659446in}{1.400957in}}%
\pgfpathlineto{\pgfqpoint{3.673707in}{1.398222in}}%
\pgfpathlineto{\pgfqpoint{3.681768in}{1.399631in}}%
\pgfpathlineto{\pgfqpoint{3.693549in}{1.401771in}}%
\pgfpathlineto{\pgfqpoint{3.701610in}{1.400524in}}%
\pgfpathlineto{\pgfqpoint{3.715252in}{1.397942in}}%
\pgfpathlineto{\pgfqpoint{3.723312in}{1.399182in}}%
\pgfpathlineto{\pgfqpoint{3.736334in}{1.401568in}}%
\pgfpathlineto{\pgfqpoint{3.744394in}{1.400144in}}%
\pgfpathlineto{\pgfqpoint{3.756176in}{1.398060in}}%
\pgfpathlineto{\pgfqpoint{3.764236in}{1.399088in}}%
\pgfpathlineto{\pgfqpoint{3.779118in}{1.401895in}}%
\pgfpathlineto{\pgfqpoint{3.787799in}{1.399798in}}%
\pgfpathlineto{\pgfqpoint{3.795860in}{1.398739in}}%
\pgfpathlineto{\pgfqpoint{3.805161in}{1.400069in}}%
\pgfpathlineto{\pgfqpoint{3.818182in}{1.402451in}}%
\pgfpathlineto{\pgfqpoint{3.835544in}{1.397486in}}%
\pgfpathlineto{\pgfqpoint{3.839884in}{1.399042in}}%
\pgfpathlineto{\pgfqpoint{3.844224in}{1.399602in}}%
\pgfpathlineto{\pgfqpoint{3.861586in}{1.397410in}}%
\pgfpathlineto{\pgfqpoint{3.867167in}{1.396690in}}%
\pgfpathlineto{\pgfqpoint{3.877088in}{1.398573in}}%
\pgfpathlineto{\pgfqpoint{3.890729in}{1.400544in}}%
\pgfpathlineto{\pgfqpoint{3.898790in}{1.398838in}}%
\pgfpathlineto{\pgfqpoint{3.907471in}{1.397411in}}%
\pgfpathlineto{\pgfqpoint{3.916152in}{1.398765in}}%
\pgfpathlineto{\pgfqpoint{3.929173in}{1.400801in}}%
\pgfpathlineto{\pgfqpoint{3.937854in}{1.399628in}}%
\pgfpathlineto{\pgfqpoint{3.950875in}{1.397549in}}%
\pgfpathlineto{\pgfqpoint{3.960176in}{1.399223in}}%
\pgfpathlineto{\pgfqpoint{3.970717in}{1.400607in}}%
\pgfpathlineto{\pgfqpoint{3.980018in}{1.399248in}}%
\pgfpathlineto{\pgfqpoint{3.993039in}{1.397169in}}%
\pgfpathlineto{\pgfqpoint{4.001720in}{1.398706in}}%
\pgfpathlineto{\pgfqpoint{4.012261in}{1.400190in}}%
\pgfpathlineto{\pgfqpoint{4.021562in}{1.398787in}}%
\pgfpathlineto{\pgfqpoint{4.035204in}{1.396611in}}%
\pgfpathlineto{\pgfqpoint{4.042644in}{1.398001in}}%
\pgfpathlineto{\pgfqpoint{4.052565in}{1.399820in}}%
\pgfpathlineto{\pgfqpoint{4.060626in}{1.398577in}}%
\pgfpathlineto{\pgfqpoint{4.076128in}{1.396082in}}%
\pgfpathlineto{\pgfqpoint{4.081708in}{1.397462in}}%
\pgfpathlineto{\pgfqpoint{4.089769in}{1.399736in}}%
\pgfpathlineto{\pgfqpoint{4.102790in}{1.398366in}}%
\pgfpathlineto{\pgfqpoint{4.108371in}{1.397150in}}%
\pgfpathlineto{\pgfqpoint{4.130073in}{1.402456in}}%
\pgfpathlineto{\pgfqpoint{4.142474in}{1.399841in}}%
\pgfpathlineto{\pgfqpoint{4.149915in}{1.399840in}}%
\pgfpathlineto{\pgfqpoint{4.167897in}{1.402449in}}%
\pgfpathlineto{\pgfqpoint{4.187739in}{1.399237in}}%
\pgfpathlineto{\pgfqpoint{4.197660in}{1.401364in}}%
\pgfpathlineto{\pgfqpoint{4.206341in}{1.402346in}}%
\pgfpathlineto{\pgfqpoint{4.215022in}{1.400580in}}%
\pgfpathlineto{\pgfqpoint{4.226183in}{1.398619in}}%
\pgfpathlineto{\pgfqpoint{4.234243in}{1.399591in}}%
\pgfpathlineto{\pgfqpoint{4.248505in}{1.401832in}}%
\pgfpathlineto{\pgfqpoint{4.257186in}{1.400103in}}%
\pgfpathlineto{\pgfqpoint{4.268347in}{1.398125in}}%
\pgfpathlineto{\pgfqpoint{4.276408in}{1.399191in}}%
\pgfpathlineto{\pgfqpoint{4.290669in}{1.401532in}}%
\pgfpathlineto{\pgfqpoint{4.298730in}{1.400033in}}%
\pgfpathlineto{\pgfqpoint{4.310511in}{1.397879in}}%
\pgfpathlineto{\pgfqpoint{4.318572in}{1.399117in}}%
\pgfpathlineto{\pgfqpoint{4.332213in}{1.401603in}}%
\pgfpathlineto{\pgfqpoint{4.340274in}{1.400366in}}%
\pgfpathlineto{\pgfqpoint{4.352675in}{1.398088in}}%
\pgfpathlineto{\pgfqpoint{4.360116in}{1.399681in}}%
\pgfpathlineto{\pgfqpoint{4.371277in}{1.402077in}}%
\pgfpathlineto{\pgfqpoint{4.379338in}{1.401439in}}%
\pgfpathlineto{\pgfqpoint{4.393599in}{1.399053in}}%
\pgfpathlineto{\pgfqpoint{4.402280in}{1.400590in}}%
\pgfpathlineto{\pgfqpoint{4.412201in}{1.400350in}}%
\pgfpathlineto{\pgfqpoint{4.418402in}{1.397953in}}%
\pgfpathlineto{\pgfqpoint{4.430183in}{1.396605in}}%
\pgfpathlineto{\pgfqpoint{4.445685in}{1.400326in}}%
\pgfpathlineto{\pgfqpoint{4.454985in}{1.398350in}}%
\pgfpathlineto{\pgfqpoint{4.464286in}{1.396770in}}%
\pgfpathlineto{\pgfqpoint{4.471107in}{1.398048in}}%
\pgfpathlineto{\pgfqpoint{4.484128in}{1.401120in}}%
\pgfpathlineto{\pgfqpoint{4.491569in}{1.399934in}}%
\pgfpathlineto{\pgfqpoint{4.505831in}{1.397021in}}%
\pgfpathlineto{\pgfqpoint{4.513271in}{1.398346in}}%
\pgfpathlineto{\pgfqpoint{4.525673in}{1.401023in}}%
\pgfpathlineto{\pgfqpoint{4.533113in}{1.399819in}}%
\pgfpathlineto{\pgfqpoint{4.547375in}{1.396559in}}%
\pgfpathlineto{\pgfqpoint{4.554815in}{1.397688in}}%
\pgfpathlineto{\pgfqpoint{4.568457in}{1.400318in}}%
\pgfpathlineto{\pgfqpoint{4.575898in}{1.398808in}}%
\pgfpathlineto{\pgfqpoint{4.588299in}{1.395747in}}%
\pgfpathlineto{\pgfqpoint{4.595740in}{1.396759in}}%
\pgfpathlineto{\pgfqpoint{4.610001in}{1.399403in}}%
\pgfpathlineto{\pgfqpoint{4.616822in}{1.397803in}}%
\pgfpathlineto{\pgfqpoint{4.626743in}{1.395029in}}%
\pgfpathlineto{\pgfqpoint{4.633563in}{1.396267in}}%
\pgfpathlineto{\pgfqpoint{4.650305in}{1.399722in}}%
\pgfpathlineto{\pgfqpoint{4.655266in}{1.399346in}}%
\pgfpathlineto{\pgfqpoint{4.668907in}{1.401294in}}%
\pgfpathlineto{\pgfqpoint{4.675728in}{1.403386in}}%
\pgfpathlineto{\pgfqpoint{4.683788in}{1.403371in}}%
\pgfpathlineto{\pgfqpoint{4.691849in}{1.400949in}}%
\pgfpathlineto{\pgfqpoint{4.699290in}{1.399484in}}%
\pgfpathlineto{\pgfqpoint{4.706111in}{1.400658in}}%
\pgfpathlineto{\pgfqpoint{4.717892in}{1.403086in}}%
\pgfpathlineto{\pgfqpoint{4.725333in}{1.402144in}}%
\pgfpathlineto{\pgfqpoint{4.742074in}{1.398779in}}%
\pgfpathlineto{\pgfqpoint{4.750755in}{1.400689in}}%
\pgfpathlineto{\pgfqpoint{4.760056in}{1.402255in}}%
\pgfpathlineto{\pgfqpoint{4.768117in}{1.401010in}}%
\pgfpathlineto{\pgfqpoint{4.782378in}{1.398132in}}%
\pgfpathlineto{\pgfqpoint{4.789819in}{1.399351in}}%
\pgfpathlineto{\pgfqpoint{4.802840in}{1.401882in}}%
\pgfpathlineto{\pgfqpoint{4.810901in}{1.400404in}}%
\pgfpathlineto{\pgfqpoint{4.822682in}{1.398367in}}%
\pgfpathlineto{\pgfqpoint{4.830123in}{1.399391in}}%
\pgfpathlineto{\pgfqpoint{4.845005in}{1.402471in}}%
\pgfpathlineto{\pgfqpoint{4.866087in}{1.399519in}}%
\pgfpathlineto{\pgfqpoint{4.871047in}{1.401117in}}%
\pgfpathlineto{\pgfqpoint{4.875388in}{1.402083in}}%
\pgfpathlineto{\pgfqpoint{4.889029in}{1.397548in}}%
\pgfpathlineto{\pgfqpoint{4.893369in}{1.395999in}}%
\pgfpathlineto{\pgfqpoint{4.900810in}{1.397345in}}%
\pgfpathlineto{\pgfqpoint{4.920032in}{1.399833in}}%
\pgfpathlineto{\pgfqpoint{4.933673in}{1.396990in}}%
\pgfpathlineto{\pgfqpoint{4.942974in}{1.398849in}}%
\pgfpathlineto{\pgfqpoint{4.954755in}{1.400883in}}%
\pgfpathlineto{\pgfqpoint{4.962196in}{1.399769in}}%
\pgfpathlineto{\pgfqpoint{4.975218in}{1.397137in}}%
\pgfpathlineto{\pgfqpoint{4.983898in}{1.398690in}}%
\pgfpathlineto{\pgfqpoint{4.995060in}{1.400482in}}%
\pgfpathlineto{\pgfqpoint{5.003120in}{1.399446in}}%
\pgfpathlineto{\pgfqpoint{5.018622in}{1.396604in}}%
\pgfpathlineto{\pgfqpoint{5.028543in}{1.398821in}}%
\pgfpathlineto{\pgfqpoint{5.037224in}{1.399649in}}%
\pgfpathlineto{\pgfqpoint{5.046525in}{1.398065in}}%
\pgfpathlineto{\pgfqpoint{5.059546in}{1.395635in}}%
\pgfpathlineto{\pgfqpoint{5.066367in}{1.397487in}}%
\pgfpathlineto{\pgfqpoint{5.073807in}{1.398997in}}%
\pgfpathlineto{\pgfqpoint{5.084348in}{1.397793in}}%
\pgfpathlineto{\pgfqpoint{5.097990in}{1.396536in}}%
\pgfpathlineto{\pgfqpoint{5.103570in}{1.399023in}}%
\pgfpathlineto{\pgfqpoint{5.108531in}{1.400476in}}%
\pgfpathlineto{\pgfqpoint{5.113491in}{1.402514in}}%
\pgfpathlineto{\pgfqpoint{5.127753in}{1.400337in}}%
\pgfpathlineto{\pgfqpoint{5.133333in}{1.400198in}}%
\pgfpathlineto{\pgfqpoint{5.150075in}{1.402965in}}%
\pgfpathlineto{\pgfqpoint{5.171157in}{1.399773in}}%
\pgfpathlineto{\pgfqpoint{5.181698in}{1.402132in}}%
\pgfpathlineto{\pgfqpoint{5.189139in}{1.402438in}}%
\pgfpathlineto{\pgfqpoint{5.202160in}{1.399642in}}%
\pgfpathlineto{\pgfqpoint{5.211461in}{1.399022in}}%
\pgfpathlineto{\pgfqpoint{5.222002in}{1.401020in}}%
\pgfpathlineto{\pgfqpoint{5.230683in}{1.401713in}}%
\pgfpathlineto{\pgfqpoint{5.241224in}{1.399462in}}%
\pgfpathlineto{\pgfqpoint{5.250525in}{1.398473in}}%
\pgfpathlineto{\pgfqpoint{5.260446in}{1.399971in}}%
\pgfpathlineto{\pgfqpoint{5.272227in}{1.401720in}}%
\pgfpathlineto{\pgfqpoint{5.280288in}{1.400014in}}%
\pgfpathlineto{\pgfqpoint{5.288969in}{1.398732in}}%
\pgfpathlineto{\pgfqpoint{5.298270in}{1.400221in}}%
\pgfpathlineto{\pgfqpoint{5.314391in}{1.402517in}}%
\pgfpathlineto{\pgfqpoint{5.324932in}{1.399497in}}%
\pgfpathlineto{\pgfqpoint{5.329273in}{1.399088in}}%
\pgfpathlineto{\pgfqpoint{5.333613in}{1.398388in}}%
\pgfpathlineto{\pgfqpoint{5.341054in}{1.400003in}}%
\pgfpathlineto{\pgfqpoint{5.348495in}{1.398653in}}%
\pgfpathlineto{\pgfqpoint{5.354695in}{1.397389in}}%
\pgfpathlineto{\pgfqpoint{5.361516in}{1.396020in}}%
\pgfpathlineto{\pgfqpoint{5.367717in}{1.397759in}}%
\pgfpathlineto{\pgfqpoint{5.377018in}{1.400015in}}%
\pgfpathlineto{\pgfqpoint{5.386319in}{1.399541in}}%
\pgfpathlineto{\pgfqpoint{5.404300in}{1.397395in}}%
\pgfpathlineto{\pgfqpoint{5.423522in}{1.400140in}}%
\pgfpathlineto{\pgfqpoint{5.448325in}{1.397284in}}%
\pgfpathlineto{\pgfqpoint{5.461966in}{1.399584in}}%
\pgfpathlineto{\pgfqpoint{5.470027in}{1.397850in}}%
\pgfpathlineto{\pgfqpoint{5.482428in}{1.395359in}}%
\pgfpathlineto{\pgfqpoint{5.489869in}{1.396191in}}%
\pgfpathlineto{\pgfqpoint{5.502270in}{1.398508in}}%
\pgfpathlineto{\pgfqpoint{5.522112in}{1.395844in}}%
\pgfpathlineto{\pgfqpoint{5.526453in}{1.398822in}}%
\pgfpathlineto{\pgfqpoint{5.530793in}{1.400836in}}%
\pgfpathlineto{\pgfqpoint{5.542574in}{1.402770in}}%
\pgfpathlineto{\pgfqpoint{5.550015in}{1.401028in}}%
\pgfpathlineto{\pgfqpoint{5.560556in}{1.401650in}}%
\pgfpathlineto{\pgfqpoint{5.574817in}{1.403785in}}%
\pgfpathlineto{\pgfqpoint{5.584738in}{1.400855in}}%
\pgfpathlineto{\pgfqpoint{5.592799in}{1.399672in}}%
\pgfpathlineto{\pgfqpoint{5.601480in}{1.400845in}}%
\pgfpathlineto{\pgfqpoint{5.613881in}{1.402673in}}%
\pgfpathlineto{\pgfqpoint{5.621322in}{1.401088in}}%
\pgfpathlineto{\pgfqpoint{5.633103in}{1.398533in}}%
\pgfpathlineto{\pgfqpoint{5.641164in}{1.399593in}}%
\pgfpathlineto{\pgfqpoint{5.655426in}{1.402048in}}%
\pgfpathlineto{\pgfqpoint{5.662866in}{1.400654in}}%
\pgfpathlineto{\pgfqpoint{5.674027in}{1.398433in}}%
\pgfpathlineto{\pgfqpoint{5.680848in}{1.399650in}}%
\pgfpathlineto{\pgfqpoint{5.695110in}{1.402808in}}%
\pgfpathlineto{\pgfqpoint{5.701930in}{1.401806in}}%
\pgfpathlineto{\pgfqpoint{5.713091in}{1.399356in}}%
\pgfpathlineto{\pgfqpoint{5.729213in}{1.400653in}}%
\pgfpathlineto{\pgfqpoint{5.747195in}{1.395906in}}%
\pgfpathlineto{\pgfqpoint{5.753395in}{1.397608in}}%
\pgfpathlineto{\pgfqpoint{5.763316in}{1.400615in}}%
\pgfpathlineto{\pgfqpoint{5.770137in}{1.399898in}}%
\pgfpathlineto{\pgfqpoint{5.786259in}{1.396809in}}%
\pgfpathlineto{\pgfqpoint{5.794319in}{1.398749in}}%
\pgfpathlineto{\pgfqpoint{5.804240in}{1.400957in}}%
\pgfpathlineto{\pgfqpoint{5.811061in}{1.400000in}}%
\pgfpathlineto{\pgfqpoint{5.827183in}{1.396347in}}%
\pgfpathlineto{\pgfqpoint{5.835244in}{1.397972in}}%
\pgfpathlineto{\pgfqpoint{5.845785in}{1.400063in}}%
\pgfpathlineto{\pgfqpoint{5.852605in}{1.398960in}}%
\pgfpathlineto{\pgfqpoint{5.868107in}{1.395226in}}%
\pgfpathlineto{\pgfqpoint{5.876168in}{1.396945in}}%
\pgfpathlineto{\pgfqpoint{5.886709in}{1.398875in}}%
\pgfpathlineto{\pgfqpoint{5.892909in}{1.397646in}}%
\pgfpathlineto{\pgfqpoint{5.902830in}{1.394819in}}%
\pgfpathlineto{\pgfqpoint{5.910271in}{1.396602in}}%
\pgfpathlineto{\pgfqpoint{5.915852in}{1.399427in}}%
\pgfpathlineto{\pgfqpoint{5.920812in}{1.401720in}}%
\pgfpathlineto{\pgfqpoint{5.931973in}{1.401377in}}%
\pgfpathlineto{\pgfqpoint{5.939414in}{1.400346in}}%
\pgfpathlineto{\pgfqpoint{5.948715in}{1.402833in}}%
\pgfpathlineto{\pgfqpoint{5.957396in}{1.403850in}}%
\pgfpathlineto{\pgfqpoint{5.964836in}{1.402510in}}%
\pgfpathlineto{\pgfqpoint{5.977238in}{1.399632in}}%
\pgfpathlineto{\pgfqpoint{5.984678in}{1.400763in}}%
\pgfpathlineto{\pgfqpoint{5.996460in}{1.402772in}}%
\pgfpathlineto{\pgfqpoint{6.004520in}{1.401533in}}%
\pgfpathlineto{\pgfqpoint{6.019402in}{1.398609in}}%
\pgfpathlineto{\pgfqpoint{6.027463in}{1.400152in}}%
\pgfpathlineto{\pgfqpoint{6.038004in}{1.402001in}}%
\pgfpathlineto{\pgfqpoint{6.046065in}{1.400817in}}%
\pgfpathlineto{\pgfqpoint{6.059706in}{1.398629in}}%
\pgfpathlineto{\pgfqpoint{6.067147in}{1.400190in}}%
\pgfpathlineto{\pgfqpoint{6.077688in}{1.402886in}}%
\pgfpathlineto{\pgfqpoint{6.085749in}{1.401486in}}%
\pgfpathlineto{\pgfqpoint{6.101250in}{1.399655in}}%
\pgfpathlineto{\pgfqpoint{6.106211in}{1.400470in}}%
\pgfpathlineto{\pgfqpoint{6.130393in}{1.396266in}}%
\pgfpathlineto{\pgfqpoint{6.153335in}{1.399691in}}%
\pgfpathlineto{\pgfqpoint{6.169457in}{1.396958in}}%
\pgfpathlineto{\pgfqpoint{6.194259in}{1.399152in}}%
\pgfpathlineto{\pgfqpoint{6.211001in}{1.396019in}}%
\pgfpathlineto{\pgfqpoint{6.232703in}{1.397957in}}%
\pgfpathlineto{\pgfqpoint{6.242004in}{1.395195in}}%
\pgfpathlineto{\pgfqpoint{6.248205in}{1.394263in}}%
\pgfpathlineto{\pgfqpoint{6.255026in}{1.396381in}}%
\pgfpathlineto{\pgfqpoint{6.262466in}{1.397542in}}%
\pgfpathlineto{\pgfqpoint{6.271767in}{1.398675in}}%
\pgfpathlineto{\pgfqpoint{6.275488in}{1.398249in}}%
\pgfpathlineto{\pgfqpoint{6.296570in}{1.404893in}}%
\pgfpathlineto{\pgfqpoint{6.312071in}{1.401253in}}%
\pgfpathlineto{\pgfqpoint{6.319512in}{1.400709in}}%
\pgfpathlineto{\pgfqpoint{6.328813in}{1.402968in}}%
\pgfpathlineto{\pgfqpoint{6.335634in}{1.403445in}}%
\pgfpathlineto{\pgfqpoint{6.344934in}{1.401289in}}%
\pgfpathlineto{\pgfqpoint{6.356096in}{1.399266in}}%
\pgfpathlineto{\pgfqpoint{6.364156in}{1.400182in}}%
\pgfpathlineto{\pgfqpoint{6.377178in}{1.402227in}}%
\pgfpathlineto{\pgfqpoint{6.386479in}{1.400108in}}%
\pgfpathlineto{\pgfqpoint{6.395780in}{1.398934in}}%
\pgfpathlineto{\pgfqpoint{6.405081in}{1.400269in}}%
\pgfpathlineto{\pgfqpoint{6.418722in}{1.402890in}}%
\pgfpathlineto{\pgfqpoint{6.434843in}{1.400531in}}%
\pgfpathlineto{\pgfqpoint{6.443524in}{1.400900in}}%
\pgfpathlineto{\pgfqpoint{6.447245in}{1.401810in}}%
\pgfpathlineto{\pgfqpoint{6.468947in}{1.395685in}}%
\pgfpathlineto{\pgfqpoint{6.483828in}{1.400031in}}%
\pgfpathlineto{\pgfqpoint{6.491269in}{1.399641in}}%
\pgfpathlineto{\pgfqpoint{6.508011in}{1.396782in}}%
\pgfpathlineto{\pgfqpoint{6.527853in}{1.400110in}}%
\pgfpathlineto{\pgfqpoint{6.537774in}{1.397365in}}%
\pgfpathlineto{\pgfqpoint{6.546455in}{1.395676in}}%
\pgfpathlineto{\pgfqpoint{6.553275in}{1.396993in}}%
\pgfpathlineto{\pgfqpoint{6.563816in}{1.399294in}}%
\pgfpathlineto{\pgfqpoint{6.571257in}{1.398005in}}%
\pgfpathlineto{\pgfqpoint{6.587999in}{1.394171in}}%
\pgfpathlineto{\pgfqpoint{6.593579in}{1.396073in}}%
\pgfpathlineto{\pgfqpoint{6.600400in}{1.398361in}}%
\pgfpathlineto{\pgfqpoint{6.612181in}{1.397681in}}%
\pgfpathlineto{\pgfqpoint{6.619622in}{1.397072in}}%
\pgfpathlineto{\pgfqpoint{6.633263in}{1.404733in}}%
\pgfpathlineto{\pgfqpoint{6.636984in}{1.405544in}}%
\pgfpathlineto{\pgfqpoint{6.641944in}{1.403834in}}%
\pgfpathlineto{\pgfqpoint{6.651865in}{1.400514in}}%
\pgfpathlineto{\pgfqpoint{6.658686in}{1.400419in}}%
\pgfpathlineto{\pgfqpoint{6.666747in}{1.403184in}}%
\pgfpathlineto{\pgfqpoint{6.672947in}{1.404527in}}%
\pgfpathlineto{\pgfqpoint{6.678528in}{1.403406in}}%
\pgfpathlineto{\pgfqpoint{6.694649in}{1.398642in}}%
\pgfpathlineto{\pgfqpoint{6.701470in}{1.399811in}}%
\pgfpathlineto{\pgfqpoint{6.715112in}{1.403128in}}%
\pgfpathlineto{\pgfqpoint{6.721312in}{1.401439in}}%
\pgfpathlineto{\pgfqpoint{6.732473in}{1.398113in}}%
\pgfpathlineto{\pgfqpoint{6.739294in}{1.399000in}}%
\pgfpathlineto{\pgfqpoint{6.757896in}{1.403351in}}%
\pgfpathlineto{\pgfqpoint{6.770917in}{1.399616in}}%
\pgfpathlineto{\pgfqpoint{6.781458in}{1.401056in}}%
\pgfpathlineto{\pgfqpoint{6.784558in}{1.401476in}}%
\pgfpathlineto{\pgfqpoint{6.795720in}{1.397378in}}%
\pgfpathlineto{\pgfqpoint{6.803780in}{1.394397in}}%
\pgfpathlineto{\pgfqpoint{6.808121in}{1.395672in}}%
\pgfpathlineto{\pgfqpoint{6.821142in}{1.400889in}}%
\pgfpathlineto{\pgfqpoint{6.827343in}{1.400286in}}%
\pgfpathlineto{\pgfqpoint{6.845945in}{1.395895in}}%
\pgfpathlineto{\pgfqpoint{6.864546in}{1.400344in}}%
\pgfpathlineto{\pgfqpoint{6.871987in}{1.397543in}}%
\pgfpathlineto{\pgfqpoint{6.881288in}{1.394011in}}%
\pgfpathlineto{\pgfqpoint{6.886249in}{1.394379in}}%
\pgfpathlineto{\pgfqpoint{6.903610in}{1.398529in}}%
\pgfpathlineto{\pgfqpoint{6.911051in}{1.395615in}}%
\pgfpathlineto{\pgfqpoint{6.918492in}{1.393016in}}%
\pgfpathlineto{\pgfqpoint{6.922212in}{1.394142in}}%
\pgfpathlineto{\pgfqpoint{6.939574in}{1.403364in}}%
\pgfpathlineto{\pgfqpoint{6.955696in}{1.401119in}}%
\pgfpathlineto{\pgfqpoint{6.962516in}{1.403693in}}%
\pgfpathlineto{\pgfqpoint{6.969337in}{1.405400in}}%
\pgfpathlineto{\pgfqpoint{6.975538in}{1.404266in}}%
\pgfpathlineto{\pgfqpoint{6.992899in}{1.399455in}}%
\pgfpathlineto{\pgfqpoint{6.999720in}{1.401107in}}%
\pgfpathlineto{\pgfqpoint{7.009641in}{1.403555in}}%
\pgfpathlineto{\pgfqpoint{7.015842in}{1.402460in}}%
\pgfpathlineto{\pgfqpoint{7.031963in}{1.398318in}}%
\pgfpathlineto{\pgfqpoint{7.038784in}{1.399781in}}%
\pgfpathlineto{\pgfqpoint{7.051185in}{1.403420in}}%
\pgfpathlineto{\pgfqpoint{7.057386in}{1.401915in}}%
\pgfpathlineto{\pgfqpoint{7.066687in}{1.399698in}}%
\pgfpathlineto{\pgfqpoint{7.075368in}{1.400468in}}%
\pgfpathlineto{\pgfqpoint{7.085909in}{1.402555in}}%
\pgfpathlineto{\pgfqpoint{7.102650in}{1.394651in}}%
\pgfpathlineto{\pgfqpoint{7.110091in}{1.396987in}}%
\pgfpathlineto{\pgfqpoint{7.123112in}{1.400734in}}%
\pgfpathlineto{\pgfqpoint{7.128693in}{1.399248in}}%
\pgfpathlineto{\pgfqpoint{7.139234in}{1.395678in}}%
\pgfpathlineto{\pgfqpoint{7.145435in}{1.396732in}}%
\pgfpathlineto{\pgfqpoint{7.160936in}{1.400659in}}%
\pgfpathlineto{\pgfqpoint{7.167137in}{1.399441in}}%
\pgfpathlineto{\pgfqpoint{7.182018in}{1.394768in}}%
\pgfpathlineto{\pgfqpoint{7.188839in}{1.396608in}}%
\pgfpathlineto{\pgfqpoint{7.198140in}{1.398761in}}%
\pgfpathlineto{\pgfqpoint{7.200000in}{1.398785in}}%
\pgfpathlineto{\pgfqpoint{7.200000in}{1.398785in}}%
\pgfusepath{stroke}%
\end{pgfscope}%
\begin{pgfscope}%
\pgfpathrectangle{\pgfqpoint{1.000000in}{0.600000in}}{\pgfqpoint{6.200000in}{4.800000in}} %
\pgfusepath{clip}%
\pgfsetrectcap%
\pgfsetroundjoin%
\pgfsetlinewidth{1.003750pt}%
\definecolor{currentstroke}{rgb}{1.000000,0.000000,0.000000}%
\pgfsetstrokecolor{currentstroke}%
\pgfsetdash{}{0pt}%
\pgfpathmoveto{\pgfqpoint{1.000000in}{1.256000in}}%
\pgfpathlineto{\pgfqpoint{1.010541in}{1.257075in}}%
\pgfpathlineto{\pgfqpoint{1.023562in}{1.261123in}}%
\pgfpathlineto{\pgfqpoint{1.034103in}{1.263441in}}%
\pgfpathlineto{\pgfqpoint{1.057666in}{1.266219in}}%
\pgfpathlineto{\pgfqpoint{1.078748in}{1.272115in}}%
\pgfpathlineto{\pgfqpoint{1.097350in}{1.274325in}}%
\pgfpathlineto{\pgfqpoint{1.109751in}{1.279037in}}%
\pgfpathlineto{\pgfqpoint{1.119052in}{1.281460in}}%
\pgfpathlineto{\pgfqpoint{1.142614in}{1.285251in}}%
\pgfpathlineto{\pgfqpoint{1.164936in}{1.292378in}}%
\pgfpathlineto{\pgfqpoint{1.179198in}{1.294529in}}%
\pgfpathlineto{\pgfqpoint{1.189739in}{1.298419in}}%
\pgfpathlineto{\pgfqpoint{1.202760in}{1.303443in}}%
\pgfpathlineto{\pgfqpoint{1.226943in}{1.308510in}}%
\pgfpathlineto{\pgfqpoint{1.249265in}{1.316621in}}%
\pgfpathlineto{\pgfqpoint{1.260426in}{1.318744in}}%
\pgfpathlineto{\pgfqpoint{1.271587in}{1.323245in}}%
\pgfpathlineto{\pgfqpoint{1.285849in}{1.329603in}}%
\pgfpathlineto{\pgfqpoint{1.304450in}{1.333927in}}%
\pgfpathlineto{\pgfqpoint{1.314371in}{1.338963in}}%
\pgfpathlineto{\pgfqpoint{1.323672in}{1.343618in}}%
\pgfpathlineto{\pgfqpoint{1.348475in}{1.351106in}}%
\pgfpathlineto{\pgfqpoint{1.364596in}{1.359404in}}%
\pgfpathlineto{\pgfqpoint{1.375758in}{1.362874in}}%
\pgfpathlineto{\pgfqpoint{1.385059in}{1.366377in}}%
\pgfpathlineto{\pgfqpoint{1.396220in}{1.373547in}}%
\pgfpathlineto{\pgfqpoint{1.403040in}{1.376708in}}%
\pgfpathlineto{\pgfqpoint{1.408001in}{1.378453in}}%
\pgfpathlineto{\pgfqpoint{1.414821in}{1.379621in}}%
\pgfpathlineto{\pgfqpoint{1.431563in}{1.390824in}}%
\pgfpathlineto{\pgfqpoint{1.435284in}{1.392697in}}%
\pgfpathlineto{\pgfqpoint{1.441484in}{1.394536in}}%
\pgfpathlineto{\pgfqpoint{1.447065in}{1.396809in}}%
\pgfpathlineto{\pgfqpoint{1.452025in}{1.397674in}}%
\pgfpathlineto{\pgfqpoint{1.475588in}{1.411651in}}%
\pgfpathlineto{\pgfqpoint{1.492329in}{1.418609in}}%
\pgfpathlineto{\pgfqpoint{1.499770in}{1.423525in}}%
\pgfpathlineto{\pgfqpoint{1.506591in}{1.427986in}}%
\pgfpathlineto{\pgfqpoint{1.512171in}{1.431034in}}%
\pgfpathlineto{\pgfqpoint{1.525193in}{1.435105in}}%
\pgfpathlineto{\pgfqpoint{1.552475in}{1.449173in}}%
\pgfpathlineto{\pgfqpoint{1.560536in}{1.451792in}}%
\pgfpathlineto{\pgfqpoint{1.577898in}{1.460654in}}%
\pgfpathlineto{\pgfqpoint{1.586579in}{1.465742in}}%
\pgfpathlineto{\pgfqpoint{1.616342in}{1.476586in}}%
\pgfpathlineto{\pgfqpoint{1.629983in}{1.483644in}}%
\pgfpathlineto{\pgfqpoint{1.648585in}{1.488920in}}%
\pgfpathlineto{\pgfqpoint{1.675248in}{1.500569in}}%
\pgfpathlineto{\pgfqpoint{1.687649in}{1.503179in}}%
\pgfpathlineto{\pgfqpoint{1.718032in}{1.515881in}}%
\pgfpathlineto{\pgfqpoint{1.732293in}{1.518779in}}%
\pgfpathlineto{\pgfqpoint{1.751515in}{1.526966in}}%
\pgfpathlineto{\pgfqpoint{1.763916in}{1.530344in}}%
\pgfpathlineto{\pgfqpoint{1.775078in}{1.532283in}}%
\pgfpathlineto{\pgfqpoint{1.799880in}{1.541568in}}%
\pgfpathlineto{\pgfqpoint{1.811041in}{1.543479in}}%
\pgfpathlineto{\pgfqpoint{1.819722in}{1.545253in}}%
\pgfpathlineto{\pgfqpoint{1.840184in}{1.552458in}}%
\pgfpathlineto{\pgfqpoint{1.867467in}{1.557897in}}%
\pgfpathlineto{\pgfqpoint{1.879248in}{1.562113in}}%
\pgfpathlineto{\pgfqpoint{1.915832in}{1.568497in}}%
\pgfpathlineto{\pgfqpoint{1.923892in}{1.570412in}}%
\pgfpathlineto{\pgfqpoint{1.956756in}{1.573609in}}%
\pgfpathlineto{\pgfqpoint{1.968537in}{1.575782in}}%
\pgfpathlineto{\pgfqpoint{1.994579in}{1.575953in}}%
\pgfpathlineto{\pgfqpoint{2.015042in}{1.577220in}}%
\pgfpathlineto{\pgfqpoint{2.029303in}{1.575222in}}%
\pgfpathlineto{\pgfqpoint{2.050385in}{1.575895in}}%
\pgfpathlineto{\pgfqpoint{2.059066in}{1.574610in}}%
\pgfpathlineto{\pgfqpoint{2.072707in}{1.571348in}}%
\pgfpathlineto{\pgfqpoint{2.088829in}{1.570492in}}%
\pgfpathlineto{\pgfqpoint{2.104330in}{1.566571in}}%
\pgfpathlineto{\pgfqpoint{2.116112in}{1.562524in}}%
\pgfpathlineto{\pgfqpoint{2.132853in}{1.559648in}}%
\pgfpathlineto{\pgfqpoint{2.152695in}{1.550938in}}%
\pgfpathlineto{\pgfqpoint{2.161376in}{1.547541in}}%
\pgfpathlineto{\pgfqpoint{2.175018in}{1.543599in}}%
\pgfpathlineto{\pgfqpoint{2.194239in}{1.532123in}}%
\pgfpathlineto{\pgfqpoint{2.207261in}{1.524997in}}%
\pgfpathlineto{\pgfqpoint{2.218422in}{1.519782in}}%
\pgfpathlineto{\pgfqpoint{2.228963in}{1.510980in}}%
\pgfpathlineto{\pgfqpoint{2.249425in}{1.495056in}}%
\pgfpathlineto{\pgfqpoint{2.263686in}{1.484239in}}%
\pgfpathlineto{\pgfqpoint{2.304610in}{1.437761in}}%
\pgfpathlineto{\pgfqpoint{2.315772in}{1.420785in}}%
\pgfpathlineto{\pgfqpoint{2.321352in}{1.411835in}}%
\pgfpathlineto{\pgfqpoint{2.325693in}{1.406157in}}%
\pgfpathlineto{\pgfqpoint{2.330653in}{1.400938in}}%
\pgfpathlineto{\pgfqpoint{2.341194in}{1.385796in}}%
\pgfpathlineto{\pgfqpoint{2.361036in}{1.356212in}}%
\pgfpathlineto{\pgfqpoint{2.368477in}{1.348368in}}%
\pgfpathlineto{\pgfqpoint{2.375918in}{1.340430in}}%
\pgfpathlineto{\pgfqpoint{2.390799in}{1.322402in}}%
\pgfpathlineto{\pgfqpoint{2.405061in}{1.310422in}}%
\pgfpathlineto{\pgfqpoint{2.418702in}{1.300418in}}%
\pgfpathlineto{\pgfqpoint{2.431723in}{1.289389in}}%
\pgfpathlineto{\pgfqpoint{2.444744in}{1.282649in}}%
\pgfpathlineto{\pgfqpoint{2.465827in}{1.272379in}}%
\pgfpathlineto{\pgfqpoint{2.475128in}{1.267704in}}%
\pgfpathlineto{\pgfqpoint{2.485669in}{1.265517in}}%
\pgfpathlineto{\pgfqpoint{2.506751in}{1.261340in}}%
\pgfpathlineto{\pgfqpoint{2.518532in}{1.258470in}}%
\pgfpathlineto{\pgfqpoint{2.527213in}{1.259202in}}%
\pgfpathlineto{\pgfqpoint{2.543954in}{1.260997in}}%
\pgfpathlineto{\pgfqpoint{2.564416in}{1.262456in}}%
\pgfpathlineto{\pgfqpoint{2.573717in}{1.266630in}}%
\pgfpathlineto{\pgfqpoint{2.584878in}{1.270880in}}%
\pgfpathlineto{\pgfqpoint{2.604100in}{1.277396in}}%
\pgfpathlineto{\pgfqpoint{2.612161in}{1.282447in}}%
\pgfpathlineto{\pgfqpoint{2.629523in}{1.294755in}}%
\pgfpathlineto{\pgfqpoint{2.641304in}{1.302343in}}%
\pgfpathlineto{\pgfqpoint{2.653085in}{1.312567in}}%
\pgfpathlineto{\pgfqpoint{2.691529in}{1.356241in}}%
\pgfpathlineto{\pgfqpoint{2.736794in}{1.428160in}}%
\pgfpathlineto{\pgfqpoint{2.742374in}{1.435956in}}%
\pgfpathlineto{\pgfqpoint{2.752295in}{1.448528in}}%
\pgfpathlineto{\pgfqpoint{2.777098in}{1.480731in}}%
\pgfpathlineto{\pgfqpoint{2.785779in}{1.488070in}}%
\pgfpathlineto{\pgfqpoint{2.803140in}{1.503163in}}%
\pgfpathlineto{\pgfqpoint{2.826083in}{1.521703in}}%
\pgfpathlineto{\pgfqpoint{2.839104in}{1.528440in}}%
\pgfpathlineto{\pgfqpoint{2.863286in}{1.543025in}}%
\pgfpathlineto{\pgfqpoint{2.870727in}{1.545131in}}%
\pgfpathlineto{\pgfqpoint{2.881888in}{1.547781in}}%
\pgfpathlineto{\pgfqpoint{2.901730in}{1.555021in}}%
\pgfpathlineto{\pgfqpoint{2.911651in}{1.556990in}}%
\pgfpathlineto{\pgfqpoint{2.931493in}{1.558079in}}%
\pgfpathlineto{\pgfqpoint{2.940174in}{1.559132in}}%
\pgfpathlineto{\pgfqpoint{2.956916in}{1.558057in}}%
\pgfpathlineto{\pgfqpoint{2.973037in}{1.554874in}}%
\pgfpathlineto{\pgfqpoint{2.982958in}{1.554395in}}%
\pgfpathlineto{\pgfqpoint{2.996600in}{1.550103in}}%
\pgfpathlineto{\pgfqpoint{3.007141in}{1.545276in}}%
\pgfpathlineto{\pgfqpoint{3.015822in}{1.541992in}}%
\pgfpathlineto{\pgfqpoint{3.028843in}{1.538211in}}%
\pgfpathlineto{\pgfqpoint{3.045585in}{1.527872in}}%
\pgfpathlineto{\pgfqpoint{3.064806in}{1.516629in}}%
\pgfpathlineto{\pgfqpoint{3.072867in}{1.511862in}}%
\pgfpathlineto{\pgfqpoint{3.084648in}{1.500922in}}%
\pgfpathlineto{\pgfqpoint{3.111311in}{1.478435in}}%
\pgfpathlineto{\pgfqpoint{3.118132in}{1.471142in}}%
\pgfpathlineto{\pgfqpoint{3.146655in}{1.435343in}}%
\pgfpathlineto{\pgfqpoint{3.153475in}{1.427127in}}%
\pgfpathlineto{\pgfqpoint{3.174557in}{1.393791in}}%
\pgfpathlineto{\pgfqpoint{3.178898in}{1.388611in}}%
\pgfpathlineto{\pgfqpoint{3.186959in}{1.377578in}}%
\pgfpathlineto{\pgfqpoint{3.191919in}{1.371112in}}%
\pgfpathlineto{\pgfqpoint{3.204320in}{1.352348in}}%
\pgfpathlineto{\pgfqpoint{3.219202in}{1.335783in}}%
\pgfpathlineto{\pgfqpoint{3.229123in}{1.326461in}}%
\pgfpathlineto{\pgfqpoint{3.244004in}{1.310284in}}%
\pgfpathlineto{\pgfqpoint{3.258266in}{1.299585in}}%
\pgfpathlineto{\pgfqpoint{3.273147in}{1.289909in}}%
\pgfpathlineto{\pgfqpoint{3.285549in}{1.280488in}}%
\pgfpathlineto{\pgfqpoint{3.297950in}{1.274910in}}%
\pgfpathlineto{\pgfqpoint{3.322132in}{1.264738in}}%
\pgfpathlineto{\pgfqpoint{3.330193in}{1.261703in}}%
\pgfpathlineto{\pgfqpoint{3.341354in}{1.260411in}}%
\pgfpathlineto{\pgfqpoint{3.361816in}{1.257895in}}%
\pgfpathlineto{\pgfqpoint{3.372977in}{1.256293in}}%
\pgfpathlineto{\pgfqpoint{3.381658in}{1.257896in}}%
\pgfpathlineto{\pgfqpoint{3.399020in}{1.261492in}}%
\pgfpathlineto{\pgfqpoint{3.418242in}{1.265123in}}%
\pgfpathlineto{\pgfqpoint{3.427543in}{1.270481in}}%
\pgfpathlineto{\pgfqpoint{3.439324in}{1.276707in}}%
\pgfpathlineto{\pgfqpoint{3.456686in}{1.285368in}}%
\pgfpathlineto{\pgfqpoint{3.464746in}{1.291522in}}%
\pgfpathlineto{\pgfqpoint{3.505051in}{1.331191in}}%
\pgfpathlineto{\pgfqpoint{3.531713in}{1.370454in}}%
\pgfpathlineto{\pgfqpoint{3.542874in}{1.389385in}}%
\pgfpathlineto{\pgfqpoint{3.548455in}{1.399366in}}%
\pgfpathlineto{\pgfqpoint{3.552795in}{1.407106in}}%
\pgfpathlineto{\pgfqpoint{3.558996in}{1.417962in}}%
\pgfpathlineto{\pgfqpoint{3.565817in}{1.427266in}}%
\pgfpathlineto{\pgfqpoint{3.576358in}{1.443622in}}%
\pgfpathlineto{\pgfqpoint{3.585659in}{1.459003in}}%
\pgfpathlineto{\pgfqpoint{3.592479in}{1.468056in}}%
\pgfpathlineto{\pgfqpoint{3.618522in}{1.493094in}}%
\pgfpathlineto{\pgfqpoint{3.634643in}{1.507295in}}%
\pgfpathlineto{\pgfqpoint{3.671227in}{1.524123in}}%
\pgfpathlineto{\pgfqpoint{3.678048in}{1.525784in}}%
\pgfpathlineto{\pgfqpoint{3.686729in}{1.525022in}}%
\pgfpathlineto{\pgfqpoint{3.694789in}{1.525322in}}%
\pgfpathlineto{\pgfqpoint{3.705951in}{1.525905in}}%
\pgfpathlineto{\pgfqpoint{3.720832in}{1.523701in}}%
\pgfpathlineto{\pgfqpoint{3.728273in}{1.519735in}}%
\pgfpathlineto{\pgfqpoint{3.737574in}{1.515013in}}%
\pgfpathlineto{\pgfqpoint{3.749355in}{1.510337in}}%
\pgfpathlineto{\pgfqpoint{3.763616in}{1.500150in}}%
\pgfpathlineto{\pgfqpoint{3.771057in}{1.492452in}}%
\pgfpathlineto{\pgfqpoint{3.781598in}{1.481384in}}%
\pgfpathlineto{\pgfqpoint{3.790899in}{1.472718in}}%
\pgfpathlineto{\pgfqpoint{3.802060in}{1.457001in}}%
\pgfpathlineto{\pgfqpoint{3.810121in}{1.443983in}}%
\pgfpathlineto{\pgfqpoint{3.819422in}{1.427963in}}%
\pgfpathlineto{\pgfqpoint{3.825003in}{1.419781in}}%
\pgfpathlineto{\pgfqpoint{3.833683in}{1.404803in}}%
\pgfpathlineto{\pgfqpoint{3.837404in}{1.397631in}}%
\pgfpathlineto{\pgfqpoint{3.843604in}{1.384417in}}%
\pgfpathlineto{\pgfqpoint{3.849805in}{1.373725in}}%
\pgfpathlineto{\pgfqpoint{3.856626in}{1.361356in}}%
\pgfpathlineto{\pgfqpoint{3.870267in}{1.342366in}}%
\pgfpathlineto{\pgfqpoint{3.878948in}{1.330582in}}%
\pgfpathlineto{\pgfqpoint{3.896310in}{1.311015in}}%
\pgfpathlineto{\pgfqpoint{3.910571in}{1.300540in}}%
\pgfpathlineto{\pgfqpoint{3.919872in}{1.293185in}}%
\pgfpathlineto{\pgfqpoint{3.937854in}{1.283832in}}%
\pgfpathlineto{\pgfqpoint{3.970097in}{1.277882in}}%
\pgfpathlineto{\pgfqpoint{3.983738in}{1.279158in}}%
\pgfpathlineto{\pgfqpoint{3.998620in}{1.283295in}}%
\pgfpathlineto{\pgfqpoint{4.006681in}{1.284662in}}%
\pgfpathlineto{\pgfqpoint{4.014741in}{1.289098in}}%
\pgfpathlineto{\pgfqpoint{4.031483in}{1.300495in}}%
\pgfpathlineto{\pgfqpoint{4.040784in}{1.307531in}}%
\pgfpathlineto{\pgfqpoint{4.050085in}{1.313687in}}%
\pgfpathlineto{\pgfqpoint{4.057526in}{1.322368in}}%
\pgfpathlineto{\pgfqpoint{4.090389in}{1.365129in}}%
\pgfpathlineto{\pgfqpoint{4.110231in}{1.400993in}}%
\pgfpathlineto{\pgfqpoint{4.114571in}{1.407341in}}%
\pgfpathlineto{\pgfqpoint{4.121392in}{1.419007in}}%
\pgfpathlineto{\pgfqpoint{4.126973in}{1.426731in}}%
\pgfpathlineto{\pgfqpoint{4.139374in}{1.448319in}}%
\pgfpathlineto{\pgfqpoint{4.154875in}{1.469581in}}%
\pgfpathlineto{\pgfqpoint{4.165417in}{1.480226in}}%
\pgfpathlineto{\pgfqpoint{4.177198in}{1.493702in}}%
\pgfpathlineto{\pgfqpoint{4.194559in}{1.508023in}}%
\pgfpathlineto{\pgfqpoint{4.211921in}{1.517889in}}%
\pgfpathlineto{\pgfqpoint{4.219982in}{1.522848in}}%
\pgfpathlineto{\pgfqpoint{4.234863in}{1.528058in}}%
\pgfpathlineto{\pgfqpoint{4.242304in}{1.528761in}}%
\pgfpathlineto{\pgfqpoint{4.251605in}{1.529628in}}%
\pgfpathlineto{\pgfqpoint{4.264626in}{1.532390in}}%
\pgfpathlineto{\pgfqpoint{4.276408in}{1.530507in}}%
\pgfpathlineto{\pgfqpoint{4.286329in}{1.527624in}}%
\pgfpathlineto{\pgfqpoint{4.297490in}{1.524054in}}%
\pgfpathlineto{\pgfqpoint{4.307411in}{1.521702in}}%
\pgfpathlineto{\pgfqpoint{4.314851in}{1.516915in}}%
\pgfpathlineto{\pgfqpoint{4.352055in}{1.487742in}}%
\pgfpathlineto{\pgfqpoint{4.360116in}{1.476930in}}%
\pgfpathlineto{\pgfqpoint{4.394219in}{1.426022in}}%
\pgfpathlineto{\pgfqpoint{4.401660in}{1.411799in}}%
\pgfpathlineto{\pgfqpoint{4.406621in}{1.401367in}}%
\pgfpathlineto{\pgfqpoint{4.411581in}{1.392107in}}%
\pgfpathlineto{\pgfqpoint{4.417162in}{1.383487in}}%
\pgfpathlineto{\pgfqpoint{4.425223in}{1.369547in}}%
\pgfpathlineto{\pgfqpoint{4.431423in}{1.359071in}}%
\pgfpathlineto{\pgfqpoint{4.441964in}{1.340317in}}%
\pgfpathlineto{\pgfqpoint{4.454985in}{1.324244in}}%
\pgfpathlineto{\pgfqpoint{4.482268in}{1.296724in}}%
\pgfpathlineto{\pgfqpoint{4.490949in}{1.291702in}}%
\pgfpathlineto{\pgfqpoint{4.507691in}{1.284962in}}%
\pgfpathlineto{\pgfqpoint{4.528153in}{1.277965in}}%
\pgfpathlineto{\pgfqpoint{4.536834in}{1.279308in}}%
\pgfpathlineto{\pgfqpoint{4.570317in}{1.287707in}}%
\pgfpathlineto{\pgfqpoint{4.577758in}{1.293635in}}%
\pgfpathlineto{\pgfqpoint{4.613101in}{1.327387in}}%
\pgfpathlineto{\pgfqpoint{4.621162in}{1.341200in}}%
\pgfpathlineto{\pgfqpoint{4.631703in}{1.357976in}}%
\pgfpathlineto{\pgfqpoint{4.646585in}{1.384049in}}%
\pgfpathlineto{\pgfqpoint{4.669527in}{1.430948in}}%
\pgfpathlineto{\pgfqpoint{4.686889in}{1.459983in}}%
\pgfpathlineto{\pgfqpoint{4.701770in}{1.482304in}}%
\pgfpathlineto{\pgfqpoint{4.713551in}{1.492356in}}%
\pgfpathlineto{\pgfqpoint{4.731533in}{1.507008in}}%
\pgfpathlineto{\pgfqpoint{4.742074in}{1.513902in}}%
\pgfpathlineto{\pgfqpoint{4.748275in}{1.515121in}}%
\pgfpathlineto{\pgfqpoint{4.762536in}{1.516741in}}%
\pgfpathlineto{\pgfqpoint{4.771837in}{1.517530in}}%
\pgfpathlineto{\pgfqpoint{4.785479in}{1.516079in}}%
\pgfpathlineto{\pgfqpoint{4.791679in}{1.512808in}}%
\pgfpathlineto{\pgfqpoint{4.805941in}{1.504078in}}%
\pgfpathlineto{\pgfqpoint{4.814001in}{1.499268in}}%
\pgfpathlineto{\pgfqpoint{4.826403in}{1.488052in}}%
\pgfpathlineto{\pgfqpoint{4.833223in}{1.479710in}}%
\pgfpathlineto{\pgfqpoint{4.862366in}{1.434379in}}%
\pgfpathlineto{\pgfqpoint{4.869807in}{1.419950in}}%
\pgfpathlineto{\pgfqpoint{4.876628in}{1.404760in}}%
\pgfpathlineto{\pgfqpoint{4.880968in}{1.397208in}}%
\pgfpathlineto{\pgfqpoint{4.887789in}{1.383283in}}%
\pgfpathlineto{\pgfqpoint{4.892749in}{1.375873in}}%
\pgfpathlineto{\pgfqpoint{4.905771in}{1.350931in}}%
\pgfpathlineto{\pgfqpoint{4.921272in}{1.326691in}}%
\pgfpathlineto{\pgfqpoint{4.934293in}{1.313563in}}%
\pgfpathlineto{\pgfqpoint{4.942974in}{1.304629in}}%
\pgfpathlineto{\pgfqpoint{4.960956in}{1.291341in}}%
\pgfpathlineto{\pgfqpoint{4.967777in}{1.290171in}}%
\pgfpathlineto{\pgfqpoint{4.974597in}{1.288505in}}%
\pgfpathlineto{\pgfqpoint{4.986999in}{1.284097in}}%
\pgfpathlineto{\pgfqpoint{5.002500in}{1.284186in}}%
\pgfpathlineto{\pgfqpoint{5.008081in}{1.286187in}}%
\pgfpathlineto{\pgfqpoint{5.019862in}{1.291470in}}%
\pgfpathlineto{\pgfqpoint{5.029163in}{1.293978in}}%
\pgfpathlineto{\pgfqpoint{5.036604in}{1.299370in}}%
\pgfpathlineto{\pgfqpoint{5.051485in}{1.312504in}}%
\pgfpathlineto{\pgfqpoint{5.063266in}{1.325806in}}%
\pgfpathlineto{\pgfqpoint{5.070707in}{1.333046in}}%
\pgfpathlineto{\pgfqpoint{5.077528in}{1.344123in}}%
\pgfpathlineto{\pgfqpoint{5.093649in}{1.373756in}}%
\pgfpathlineto{\pgfqpoint{5.097990in}{1.380324in}}%
\pgfpathlineto{\pgfqpoint{5.101710in}{1.385464in}}%
\pgfpathlineto{\pgfqpoint{5.111631in}{1.403229in}}%
\pgfpathlineto{\pgfqpoint{5.114731in}{1.410105in}}%
\pgfpathlineto{\pgfqpoint{5.120312in}{1.422923in}}%
\pgfpathlineto{\pgfqpoint{5.128373in}{1.437509in}}%
\pgfpathlineto{\pgfqpoint{5.134573in}{1.449091in}}%
\pgfpathlineto{\pgfqpoint{5.164956in}{1.487810in}}%
\pgfpathlineto{\pgfqpoint{5.173637in}{1.496733in}}%
\pgfpathlineto{\pgfqpoint{5.180458in}{1.500087in}}%
\pgfpathlineto{\pgfqpoint{5.187279in}{1.503916in}}%
\pgfpathlineto{\pgfqpoint{5.199680in}{1.512037in}}%
\pgfpathlineto{\pgfqpoint{5.217042in}{1.516890in}}%
\pgfpathlineto{\pgfqpoint{5.223242in}{1.515790in}}%
\pgfpathlineto{\pgfqpoint{5.231303in}{1.514543in}}%
\pgfpathlineto{\pgfqpoint{5.242464in}{1.514365in}}%
\pgfpathlineto{\pgfqpoint{5.250525in}{1.510503in}}%
\pgfpathlineto{\pgfqpoint{5.263546in}{1.502843in}}%
\pgfpathlineto{\pgfqpoint{5.290829in}{1.476459in}}%
\pgfpathlineto{\pgfqpoint{5.309431in}{1.445966in}}%
\pgfpathlineto{\pgfqpoint{5.316252in}{1.434056in}}%
\pgfpathlineto{\pgfqpoint{5.321832in}{1.425124in}}%
\pgfpathlineto{\pgfqpoint{5.342914in}{1.379546in}}%
\pgfpathlineto{\pgfqpoint{5.347255in}{1.370934in}}%
\pgfpathlineto{\pgfqpoint{5.354075in}{1.357222in}}%
\pgfpathlineto{\pgfqpoint{5.365857in}{1.338614in}}%
\pgfpathlineto{\pgfqpoint{5.373917in}{1.325485in}}%
\pgfpathlineto{\pgfqpoint{5.387559in}{1.310346in}}%
\pgfpathlineto{\pgfqpoint{5.394379in}{1.305106in}}%
\pgfpathlineto{\pgfqpoint{5.407401in}{1.296718in}}%
\pgfpathlineto{\pgfqpoint{5.414841in}{1.291738in}}%
\pgfpathlineto{\pgfqpoint{5.421042in}{1.290387in}}%
\pgfpathlineto{\pgfqpoint{5.435924in}{1.290233in}}%
\pgfpathlineto{\pgfqpoint{5.459486in}{1.296488in}}%
\pgfpathlineto{\pgfqpoint{5.466307in}{1.302205in}}%
\pgfpathlineto{\pgfqpoint{5.484288in}{1.320945in}}%
\pgfpathlineto{\pgfqpoint{5.503510in}{1.347081in}}%
\pgfpathlineto{\pgfqpoint{5.526453in}{1.393463in}}%
\pgfpathlineto{\pgfqpoint{5.530173in}{1.401176in}}%
\pgfpathlineto{\pgfqpoint{5.535754in}{1.412959in}}%
\pgfpathlineto{\pgfqpoint{5.540714in}{1.422712in}}%
\pgfpathlineto{\pgfqpoint{5.550015in}{1.443707in}}%
\pgfpathlineto{\pgfqpoint{5.565517in}{1.469047in}}%
\pgfpathlineto{\pgfqpoint{5.593419in}{1.499874in}}%
\pgfpathlineto{\pgfqpoint{5.603960in}{1.505841in}}%
\pgfpathlineto{\pgfqpoint{5.613261in}{1.508010in}}%
\pgfpathlineto{\pgfqpoint{5.623802in}{1.511505in}}%
\pgfpathlineto{\pgfqpoint{5.630003in}{1.512322in}}%
\pgfpathlineto{\pgfqpoint{5.636824in}{1.510735in}}%
\pgfpathlineto{\pgfqpoint{5.648605in}{1.505518in}}%
\pgfpathlineto{\pgfqpoint{5.669687in}{1.491770in}}%
\pgfpathlineto{\pgfqpoint{5.675888in}{1.485245in}}%
\pgfpathlineto{\pgfqpoint{5.687669in}{1.468104in}}%
\pgfpathlineto{\pgfqpoint{5.700070in}{1.447539in}}%
\pgfpathlineto{\pgfqpoint{5.713711in}{1.421963in}}%
\pgfpathlineto{\pgfqpoint{5.729213in}{1.386583in}}%
\pgfpathlineto{\pgfqpoint{5.762076in}{1.327259in}}%
\pgfpathlineto{\pgfqpoint{5.770137in}{1.317884in}}%
\pgfpathlineto{\pgfqpoint{5.782538in}{1.307355in}}%
\pgfpathlineto{\pgfqpoint{5.804240in}{1.292126in}}%
\pgfpathlineto{\pgfqpoint{5.811061in}{1.290869in}}%
\pgfpathlineto{\pgfqpoint{5.824082in}{1.291804in}}%
\pgfpathlineto{\pgfqpoint{5.842064in}{1.294272in}}%
\pgfpathlineto{\pgfqpoint{5.848885in}{1.297418in}}%
\pgfpathlineto{\pgfqpoint{5.855706in}{1.303136in}}%
\pgfpathlineto{\pgfqpoint{5.884848in}{1.333630in}}%
\pgfpathlineto{\pgfqpoint{5.892289in}{1.345537in}}%
\pgfpathlineto{\pgfqpoint{5.932593in}{1.426111in}}%
\pgfpathlineto{\pgfqpoint{5.940654in}{1.443062in}}%
\pgfpathlineto{\pgfqpoint{5.957396in}{1.468085in}}%
\pgfpathlineto{\pgfqpoint{5.977238in}{1.492569in}}%
\pgfpathlineto{\pgfqpoint{5.984058in}{1.496879in}}%
\pgfpathlineto{\pgfqpoint{6.006381in}{1.506221in}}%
\pgfpathlineto{\pgfqpoint{6.018162in}{1.508528in}}%
\pgfpathlineto{\pgfqpoint{6.023742in}{1.507358in}}%
\pgfpathlineto{\pgfqpoint{6.033663in}{1.502118in}}%
\pgfpathlineto{\pgfqpoint{6.050405in}{1.492474in}}%
\pgfpathlineto{\pgfqpoint{6.061566in}{1.482010in}}%
\pgfpathlineto{\pgfqpoint{6.067767in}{1.473187in}}%
\pgfpathlineto{\pgfqpoint{6.090709in}{1.432530in}}%
\pgfpathlineto{\pgfqpoint{6.114271in}{1.379346in}}%
\pgfpathlineto{\pgfqpoint{6.119852in}{1.368613in}}%
\pgfpathlineto{\pgfqpoint{6.127293in}{1.354845in}}%
\pgfpathlineto{\pgfqpoint{6.136594in}{1.337733in}}%
\pgfpathlineto{\pgfqpoint{6.152095in}{1.315507in}}%
\pgfpathlineto{\pgfqpoint{6.158296in}{1.311077in}}%
\pgfpathlineto{\pgfqpoint{6.168837in}{1.303953in}}%
\pgfpathlineto{\pgfqpoint{6.176898in}{1.299049in}}%
\pgfpathlineto{\pgfqpoint{6.186819in}{1.296425in}}%
\pgfpathlineto{\pgfqpoint{6.193639in}{1.295954in}}%
\pgfpathlineto{\pgfqpoint{6.199220in}{1.297990in}}%
\pgfpathlineto{\pgfqpoint{6.212861in}{1.304381in}}%
\pgfpathlineto{\pgfqpoint{6.219062in}{1.307643in}}%
\pgfpathlineto{\pgfqpoint{6.226503in}{1.314715in}}%
\pgfpathlineto{\pgfqpoint{6.237664in}{1.327918in}}%
\pgfpathlineto{\pgfqpoint{6.257506in}{1.361607in}}%
\pgfpathlineto{\pgfqpoint{6.290989in}{1.434435in}}%
\pgfpathlineto{\pgfqpoint{6.295950in}{1.443931in}}%
\pgfpathlineto{\pgfqpoint{6.305251in}{1.462224in}}%
\pgfpathlineto{\pgfqpoint{6.321992in}{1.486133in}}%
\pgfpathlineto{\pgfqpoint{6.328813in}{1.490576in}}%
\pgfpathlineto{\pgfqpoint{6.336254in}{1.496159in}}%
\pgfpathlineto{\pgfqpoint{6.344934in}{1.502552in}}%
\pgfpathlineto{\pgfqpoint{6.353615in}{1.505188in}}%
\pgfpathlineto{\pgfqpoint{6.362916in}{1.506668in}}%
\pgfpathlineto{\pgfqpoint{6.368497in}{1.504772in}}%
\pgfpathlineto{\pgfqpoint{6.379658in}{1.500087in}}%
\pgfpathlineto{\pgfqpoint{6.387099in}{1.497511in}}%
\pgfpathlineto{\pgfqpoint{6.393299in}{1.492271in}}%
\pgfpathlineto{\pgfqpoint{6.407561in}{1.476497in}}%
\pgfpathlineto{\pgfqpoint{6.428023in}{1.443407in}}%
\pgfpathlineto{\pgfqpoint{6.444144in}{1.405976in}}%
\pgfpathlineto{\pgfqpoint{6.448485in}{1.395428in}}%
\pgfpathlineto{\pgfqpoint{6.453445in}{1.386456in}}%
\pgfpathlineto{\pgfqpoint{6.460266in}{1.370603in}}%
\pgfpathlineto{\pgfqpoint{6.467707in}{1.357921in}}%
\pgfpathlineto{\pgfqpoint{6.477628in}{1.337492in}}%
\pgfpathlineto{\pgfqpoint{6.491889in}{1.317553in}}%
\pgfpathlineto{\pgfqpoint{6.497470in}{1.312957in}}%
\pgfpathlineto{\pgfqpoint{6.508631in}{1.305185in}}%
\pgfpathlineto{\pgfqpoint{6.517312in}{1.298281in}}%
\pgfpathlineto{\pgfqpoint{6.523512in}{1.296605in}}%
\pgfpathlineto{\pgfqpoint{6.535914in}{1.295919in}}%
\pgfpathlineto{\pgfqpoint{6.542114in}{1.298326in}}%
\pgfpathlineto{\pgfqpoint{6.550795in}{1.301608in}}%
\pgfpathlineto{\pgfqpoint{6.559476in}{1.304043in}}%
\pgfpathlineto{\pgfqpoint{6.565057in}{1.308779in}}%
\pgfpathlineto{\pgfqpoint{6.579938in}{1.326224in}}%
\pgfpathlineto{\pgfqpoint{6.601640in}{1.360451in}}%
\pgfpathlineto{\pgfqpoint{6.620862in}{1.403906in}}%
\pgfpathlineto{\pgfqpoint{6.624582in}{1.410946in}}%
\pgfpathlineto{\pgfqpoint{6.631403in}{1.425222in}}%
\pgfpathlineto{\pgfqpoint{6.636984in}{1.434800in}}%
\pgfpathlineto{\pgfqpoint{6.649385in}{1.460359in}}%
\pgfpathlineto{\pgfqpoint{6.661786in}{1.478451in}}%
\pgfpathlineto{\pgfqpoint{6.667987in}{1.483039in}}%
\pgfpathlineto{\pgfqpoint{6.675428in}{1.488809in}}%
\pgfpathlineto{\pgfqpoint{6.686589in}{1.498413in}}%
\pgfpathlineto{\pgfqpoint{6.694649in}{1.501189in}}%
\pgfpathlineto{\pgfqpoint{6.702090in}{1.502010in}}%
\pgfpathlineto{\pgfqpoint{6.707671in}{1.500229in}}%
\pgfpathlineto{\pgfqpoint{6.718832in}{1.495826in}}%
\pgfpathlineto{\pgfqpoint{6.726893in}{1.493403in}}%
\pgfpathlineto{\pgfqpoint{6.732473in}{1.488755in}}%
\pgfpathlineto{\pgfqpoint{6.743014in}{1.476263in}}%
\pgfpathlineto{\pgfqpoint{6.749835in}{1.465089in}}%
\pgfpathlineto{\pgfqpoint{6.760996in}{1.446442in}}%
\pgfpathlineto{\pgfqpoint{6.766577in}{1.436325in}}%
\pgfpathlineto{\pgfqpoint{6.782698in}{1.395795in}}%
\pgfpathlineto{\pgfqpoint{6.787659in}{1.385640in}}%
\pgfpathlineto{\pgfqpoint{6.793239in}{1.372685in}}%
\pgfpathlineto{\pgfqpoint{6.798820in}{1.361979in}}%
\pgfpathlineto{\pgfqpoint{6.806261in}{1.348732in}}%
\pgfpathlineto{\pgfqpoint{6.816802in}{1.328576in}}%
\pgfpathlineto{\pgfqpoint{6.826103in}{1.316914in}}%
\pgfpathlineto{\pgfqpoint{6.832303in}{1.311899in}}%
\pgfpathlineto{\pgfqpoint{6.840984in}{1.308496in}}%
\pgfpathlineto{\pgfqpoint{6.849045in}{1.303811in}}%
\pgfpathlineto{\pgfqpoint{6.855866in}{1.300887in}}%
\pgfpathlineto{\pgfqpoint{6.861446in}{1.301045in}}%
\pgfpathlineto{\pgfqpoint{6.868887in}{1.303614in}}%
\pgfpathlineto{\pgfqpoint{6.875708in}{1.308429in}}%
\pgfpathlineto{\pgfqpoint{6.900510in}{1.331994in}}%
\pgfpathlineto{\pgfqpoint{6.909811in}{1.349128in}}%
\pgfpathlineto{\pgfqpoint{6.941434in}{1.418272in}}%
\pgfpathlineto{\pgfqpoint{6.955076in}{1.449759in}}%
\pgfpathlineto{\pgfqpoint{6.961276in}{1.459606in}}%
\pgfpathlineto{\pgfqpoint{6.989179in}{1.494083in}}%
\pgfpathlineto{\pgfqpoint{6.995380in}{1.498182in}}%
\pgfpathlineto{\pgfqpoint{7.000960in}{1.499174in}}%
\pgfpathlineto{\pgfqpoint{7.015842in}{1.499939in}}%
\pgfpathlineto{\pgfqpoint{7.022042in}{1.499596in}}%
\pgfpathlineto{\pgfqpoint{7.029483in}{1.496614in}}%
\pgfpathlineto{\pgfqpoint{7.036924in}{1.491648in}}%
\pgfpathlineto{\pgfqpoint{7.043124in}{1.484425in}}%
\pgfpathlineto{\pgfqpoint{7.057386in}{1.466224in}}%
\pgfpathlineto{\pgfqpoint{7.062966in}{1.457909in}}%
\pgfpathlineto{\pgfqpoint{7.074747in}{1.433992in}}%
\pgfpathlineto{\pgfqpoint{7.080328in}{1.419197in}}%
\pgfpathlineto{\pgfqpoint{7.085289in}{1.407276in}}%
\pgfpathlineto{\pgfqpoint{7.090249in}{1.396889in}}%
\pgfpathlineto{\pgfqpoint{7.095830in}{1.384116in}}%
\pgfpathlineto{\pgfqpoint{7.101410in}{1.373976in}}%
\pgfpathlineto{\pgfqpoint{7.113191in}{1.347763in}}%
\pgfpathlineto{\pgfqpoint{7.126213in}{1.324785in}}%
\pgfpathlineto{\pgfqpoint{7.131793in}{1.320195in}}%
\pgfpathlineto{\pgfqpoint{7.139854in}{1.313754in}}%
\pgfpathlineto{\pgfqpoint{7.149775in}{1.304863in}}%
\pgfpathlineto{\pgfqpoint{7.158456in}{1.301139in}}%
\pgfpathlineto{\pgfqpoint{7.165897in}{1.299400in}}%
\pgfpathlineto{\pgfqpoint{7.170237in}{1.300578in}}%
\pgfpathlineto{\pgfqpoint{7.186979in}{1.308041in}}%
\pgfpathlineto{\pgfqpoint{7.191939in}{1.310619in}}%
\pgfpathlineto{\pgfqpoint{7.198760in}{1.317422in}}%
\pgfpathlineto{\pgfqpoint{7.200000in}{1.318769in}}%
\pgfpathlineto{\pgfqpoint{7.200000in}{1.318769in}}%
\pgfusepath{stroke}%
\end{pgfscope}%
\begin{pgfscope}%
\pgfpathrectangle{\pgfqpoint{1.000000in}{0.600000in}}{\pgfqpoint{6.200000in}{4.800000in}} %
\pgfusepath{clip}%
\pgfsetrectcap%
\pgfsetroundjoin%
\pgfsetlinewidth{1.003750pt}%
\definecolor{currentstroke}{rgb}{0.000000,0.750000,0.750000}%
\pgfsetstrokecolor{currentstroke}%
\pgfsetdash{}{0pt}%
\pgfpathmoveto{\pgfqpoint{1.000000in}{1.400000in}}%
\pgfpathlineto{\pgfqpoint{1.206481in}{1.400204in}}%
\pgfpathlineto{\pgfqpoint{1.215162in}{1.399688in}}%
\pgfpathlineto{\pgfqpoint{1.224462in}{1.399359in}}%
\pgfpathlineto{\pgfqpoint{1.237484in}{1.400024in}}%
\pgfpathlineto{\pgfqpoint{1.248645in}{1.400265in}}%
\pgfpathlineto{\pgfqpoint{1.256706in}{1.399618in}}%
\pgfpathlineto{\pgfqpoint{1.265387in}{1.399185in}}%
\pgfpathlineto{\pgfqpoint{1.279648in}{1.400045in}}%
\pgfpathlineto{\pgfqpoint{1.308171in}{1.399545in}}%
\pgfpathlineto{\pgfqpoint{1.316852in}{1.400203in}}%
\pgfpathlineto{\pgfqpoint{1.347235in}{1.399696in}}%
\pgfpathlineto{\pgfqpoint{1.354055in}{1.400021in}}%
\pgfpathlineto{\pgfqpoint{1.360876in}{1.400496in}}%
\pgfpathlineto{\pgfqpoint{1.367077in}{1.401058in}}%
\pgfpathlineto{\pgfqpoint{1.380098in}{1.399049in}}%
\pgfpathlineto{\pgfqpoint{1.386299in}{1.399645in}}%
\pgfpathlineto{\pgfqpoint{1.392499in}{1.399535in}}%
\pgfpathlineto{\pgfqpoint{1.400560in}{1.401725in}}%
\pgfpathlineto{\pgfqpoint{1.409861in}{1.399382in}}%
\pgfpathlineto{\pgfqpoint{1.414821in}{1.399438in}}%
\pgfpathlineto{\pgfqpoint{1.421642in}{1.398160in}}%
\pgfpathlineto{\pgfqpoint{1.429703in}{1.400759in}}%
\pgfpathlineto{\pgfqpoint{1.434663in}{1.400543in}}%
\pgfpathlineto{\pgfqpoint{1.439624in}{1.401713in}}%
\pgfpathlineto{\pgfqpoint{1.449545in}{1.399325in}}%
\pgfpathlineto{\pgfqpoint{1.453265in}{1.398969in}}%
\pgfpathlineto{\pgfqpoint{1.457606in}{1.397942in}}%
\pgfpathlineto{\pgfqpoint{1.461946in}{1.400504in}}%
\pgfpathlineto{\pgfqpoint{1.465047in}{1.401086in}}%
\pgfpathlineto{\pgfqpoint{1.471247in}{1.400574in}}%
\pgfpathlineto{\pgfqpoint{1.476828in}{1.401426in}}%
\pgfpathlineto{\pgfqpoint{1.485509in}{1.398785in}}%
\pgfpathlineto{\pgfqpoint{1.490469in}{1.399244in}}%
\pgfpathlineto{\pgfqpoint{1.496670in}{1.398743in}}%
\pgfpathlineto{\pgfqpoint{1.505351in}{1.401430in}}%
\pgfpathlineto{\pgfqpoint{1.514031in}{1.400119in}}%
\pgfpathlineto{\pgfqpoint{1.520232in}{1.400198in}}%
\pgfpathlineto{\pgfqpoint{1.528913in}{1.398185in}}%
\pgfpathlineto{\pgfqpoint{1.540074in}{1.400605in}}%
\pgfpathlineto{\pgfqpoint{1.547515in}{1.399982in}}%
\pgfpathlineto{\pgfqpoint{1.556196in}{1.400600in}}%
\pgfpathlineto{\pgfqpoint{1.566737in}{1.398517in}}%
\pgfpathlineto{\pgfqpoint{1.578518in}{1.400520in}}%
\pgfpathlineto{\pgfqpoint{1.587819in}{1.399849in}}%
\pgfpathlineto{\pgfqpoint{1.597120in}{1.400445in}}%
\pgfpathlineto{\pgfqpoint{1.608901in}{1.398935in}}%
\pgfpathlineto{\pgfqpoint{1.620062in}{1.400660in}}%
\pgfpathlineto{\pgfqpoint{1.631223in}{1.399627in}}%
\pgfpathlineto{\pgfqpoint{1.641144in}{1.400334in}}%
\pgfpathlineto{\pgfqpoint{1.652305in}{1.398990in}}%
\pgfpathlineto{\pgfqpoint{1.663466in}{1.400821in}}%
\pgfpathlineto{\pgfqpoint{1.675868in}{1.399441in}}%
\pgfpathlineto{\pgfqpoint{1.685169in}{1.400586in}}%
\pgfpathlineto{\pgfqpoint{1.697570in}{1.399169in}}%
\pgfpathlineto{\pgfqpoint{1.707491in}{1.400999in}}%
\pgfpathlineto{\pgfqpoint{1.721132in}{1.399422in}}%
\pgfpathlineto{\pgfqpoint{1.729193in}{1.400905in}}%
\pgfpathlineto{\pgfqpoint{1.744074in}{1.399638in}}%
\pgfpathlineto{\pgfqpoint{1.751515in}{1.401157in}}%
\pgfpathlineto{\pgfqpoint{1.758336in}{1.399173in}}%
\pgfpathlineto{\pgfqpoint{1.763296in}{1.398677in}}%
\pgfpathlineto{\pgfqpoint{1.778178in}{1.400018in}}%
\pgfpathlineto{\pgfqpoint{1.784998in}{1.398601in}}%
\pgfpathlineto{\pgfqpoint{1.791199in}{1.400279in}}%
\pgfpathlineto{\pgfqpoint{1.796780in}{1.401090in}}%
\pgfpathlineto{\pgfqpoint{1.803600in}{1.399152in}}%
\pgfpathlineto{\pgfqpoint{1.808561in}{1.398724in}}%
\pgfpathlineto{\pgfqpoint{1.822822in}{1.400267in}}%
\pgfpathlineto{\pgfqpoint{1.830883in}{1.398704in}}%
\pgfpathlineto{\pgfqpoint{1.847005in}{1.399816in}}%
\pgfpathlineto{\pgfqpoint{1.853825in}{1.398880in}}%
\pgfpathlineto{\pgfqpoint{1.868707in}{1.400112in}}%
\pgfpathlineto{\pgfqpoint{1.876148in}{1.398783in}}%
\pgfpathlineto{\pgfqpoint{1.884208in}{1.400593in}}%
\pgfpathlineto{\pgfqpoint{1.889789in}{1.400556in}}%
\pgfpathlineto{\pgfqpoint{1.900950in}{1.399311in}}%
\pgfpathlineto{\pgfqpoint{1.912111in}{1.400602in}}%
\pgfpathlineto{\pgfqpoint{1.924512in}{1.399215in}}%
\pgfpathlineto{\pgfqpoint{1.935674in}{1.400602in}}%
\pgfpathlineto{\pgfqpoint{1.947455in}{1.399480in}}%
\pgfpathlineto{\pgfqpoint{1.957996in}{1.400562in}}%
\pgfpathlineto{\pgfqpoint{1.970397in}{1.399234in}}%
\pgfpathlineto{\pgfqpoint{1.980938in}{1.400857in}}%
\pgfpathlineto{\pgfqpoint{1.994579in}{1.399715in}}%
\pgfpathlineto{\pgfqpoint{2.002640in}{1.400789in}}%
\pgfpathlineto{\pgfqpoint{2.016902in}{1.399579in}}%
\pgfpathlineto{\pgfqpoint{2.024962in}{1.401230in}}%
\pgfpathlineto{\pgfqpoint{2.031783in}{1.399399in}}%
\pgfpathlineto{\pgfqpoint{2.037364in}{1.399054in}}%
\pgfpathlineto{\pgfqpoint{2.049765in}{1.400524in}}%
\pgfpathlineto{\pgfqpoint{2.059066in}{1.398872in}}%
\pgfpathlineto{\pgfqpoint{2.072707in}{1.400699in}}%
\pgfpathlineto{\pgfqpoint{2.081388in}{1.398798in}}%
\pgfpathlineto{\pgfqpoint{2.096270in}{1.400154in}}%
\pgfpathlineto{\pgfqpoint{2.103090in}{1.398815in}}%
\pgfpathlineto{\pgfqpoint{2.109291in}{1.400587in}}%
\pgfpathlineto{\pgfqpoint{2.114871in}{1.401339in}}%
\pgfpathlineto{\pgfqpoint{2.122312in}{1.399083in}}%
\pgfpathlineto{\pgfqpoint{2.127273in}{1.398958in}}%
\pgfpathlineto{\pgfqpoint{2.139674in}{1.400792in}}%
\pgfpathlineto{\pgfqpoint{2.149595in}{1.399089in}}%
\pgfpathlineto{\pgfqpoint{2.162616in}{1.400687in}}%
\pgfpathlineto{\pgfqpoint{2.171917in}{1.399007in}}%
\pgfpathlineto{\pgfqpoint{2.186799in}{1.400402in}}%
\pgfpathlineto{\pgfqpoint{2.194859in}{1.399109in}}%
\pgfpathlineto{\pgfqpoint{2.210361in}{1.400154in}}%
\pgfpathlineto{\pgfqpoint{2.217802in}{1.399331in}}%
\pgfpathlineto{\pgfqpoint{2.232683in}{1.400356in}}%
\pgfpathlineto{\pgfqpoint{2.240744in}{1.399096in}}%
\pgfpathlineto{\pgfqpoint{2.256866in}{1.400207in}}%
\pgfpathlineto{\pgfqpoint{2.263686in}{1.399839in}}%
\pgfpathlineto{\pgfqpoint{2.274847in}{1.400887in}}%
\pgfpathlineto{\pgfqpoint{2.284768in}{1.398915in}}%
\pgfpathlineto{\pgfqpoint{2.297790in}{1.400918in}}%
\pgfpathlineto{\pgfqpoint{2.303370in}{1.400332in}}%
\pgfpathlineto{\pgfqpoint{2.311431in}{1.401080in}}%
\pgfpathlineto{\pgfqpoint{2.318872in}{1.397979in}}%
\pgfpathlineto{\pgfqpoint{2.325693in}{1.400100in}}%
\pgfpathlineto{\pgfqpoint{2.330033in}{1.399268in}}%
\pgfpathlineto{\pgfqpoint{2.336854in}{1.401913in}}%
\pgfpathlineto{\pgfqpoint{2.343674in}{1.399149in}}%
\pgfpathlineto{\pgfqpoint{2.349255in}{1.399521in}}%
\pgfpathlineto{\pgfqpoint{2.357316in}{1.397652in}}%
\pgfpathlineto{\pgfqpoint{2.367857in}{1.400405in}}%
\pgfpathlineto{\pgfqpoint{2.376538in}{1.399218in}}%
\pgfpathlineto{\pgfqpoint{2.385219in}{1.399746in}}%
\pgfpathlineto{\pgfqpoint{2.395760in}{1.398361in}}%
\pgfpathlineto{\pgfqpoint{2.408161in}{1.400079in}}%
\pgfpathlineto{\pgfqpoint{2.418082in}{1.399055in}}%
\pgfpathlineto{\pgfqpoint{2.429243in}{1.399687in}}%
\pgfpathlineto{\pgfqpoint{2.439784in}{1.398855in}}%
\pgfpathlineto{\pgfqpoint{2.452185in}{1.399926in}}%
\pgfpathlineto{\pgfqpoint{2.462726in}{1.399046in}}%
\pgfpathlineto{\pgfqpoint{2.475128in}{1.399616in}}%
\pgfpathlineto{\pgfqpoint{2.485669in}{1.399141in}}%
\pgfpathlineto{\pgfqpoint{2.498070in}{1.399742in}}%
\pgfpathlineto{\pgfqpoint{2.509851in}{1.399224in}}%
\pgfpathlineto{\pgfqpoint{2.522252in}{1.399599in}}%
\pgfpathlineto{\pgfqpoint{2.534033in}{1.399416in}}%
\pgfpathlineto{\pgfqpoint{2.545195in}{1.399711in}}%
\pgfpathlineto{\pgfqpoint{2.556976in}{1.399280in}}%
\pgfpathlineto{\pgfqpoint{2.568757in}{1.399724in}}%
\pgfpathlineto{\pgfqpoint{2.579918in}{1.399402in}}%
\pgfpathlineto{\pgfqpoint{2.590459in}{1.399970in}}%
\pgfpathlineto{\pgfqpoint{2.602240in}{1.399114in}}%
\pgfpathlineto{\pgfqpoint{2.614021in}{1.399852in}}%
\pgfpathlineto{\pgfqpoint{2.623942in}{1.399308in}}%
\pgfpathlineto{\pgfqpoint{2.634483in}{1.400175in}}%
\pgfpathlineto{\pgfqpoint{2.646265in}{1.398776in}}%
\pgfpathlineto{\pgfqpoint{2.657426in}{1.399899in}}%
\pgfpathlineto{\pgfqpoint{2.666107in}{1.399282in}}%
\pgfpathlineto{\pgfqpoint{2.675408in}{1.400318in}}%
\pgfpathlineto{\pgfqpoint{2.687189in}{1.398245in}}%
\pgfpathlineto{\pgfqpoint{2.694629in}{1.399887in}}%
\pgfpathlineto{\pgfqpoint{2.702070in}{1.399809in}}%
\pgfpathlineto{\pgfqpoint{2.707651in}{1.401625in}}%
\pgfpathlineto{\pgfqpoint{2.711371in}{1.399216in}}%
\pgfpathlineto{\pgfqpoint{2.714471in}{1.398151in}}%
\pgfpathlineto{\pgfqpoint{2.721292in}{1.399145in}}%
\pgfpathlineto{\pgfqpoint{2.725013in}{1.397801in}}%
\pgfpathlineto{\pgfqpoint{2.728113in}{1.399796in}}%
\pgfpathlineto{\pgfqpoint{2.731833in}{1.401904in}}%
\pgfpathlineto{\pgfqpoint{2.735554in}{1.400837in}}%
\pgfpathlineto{\pgfqpoint{2.739894in}{1.400354in}}%
\pgfpathlineto{\pgfqpoint{2.747955in}{1.401206in}}%
\pgfpathlineto{\pgfqpoint{2.757256in}{1.398722in}}%
\pgfpathlineto{\pgfqpoint{2.769657in}{1.400843in}}%
\pgfpathlineto{\pgfqpoint{2.777718in}{1.399737in}}%
\pgfpathlineto{\pgfqpoint{2.790119in}{1.400585in}}%
\pgfpathlineto{\pgfqpoint{2.799420in}{1.399020in}}%
\pgfpathlineto{\pgfqpoint{2.813681in}{1.400492in}}%
\pgfpathlineto{\pgfqpoint{2.821742in}{1.399247in}}%
\pgfpathlineto{\pgfqpoint{2.837244in}{1.400068in}}%
\pgfpathlineto{\pgfqpoint{2.844064in}{1.398992in}}%
\pgfpathlineto{\pgfqpoint{2.852125in}{1.400969in}}%
\pgfpathlineto{\pgfqpoint{2.857706in}{1.400825in}}%
\pgfpathlineto{\pgfqpoint{2.868247in}{1.399179in}}%
\pgfpathlineto{\pgfqpoint{2.880648in}{1.400868in}}%
\pgfpathlineto{\pgfqpoint{2.890569in}{1.399067in}}%
\pgfpathlineto{\pgfqpoint{2.904210in}{1.400613in}}%
\pgfpathlineto{\pgfqpoint{2.912271in}{1.398921in}}%
\pgfpathlineto{\pgfqpoint{2.918472in}{1.400737in}}%
\pgfpathlineto{\pgfqpoint{2.924052in}{1.401459in}}%
\pgfpathlineto{\pgfqpoint{2.938934in}{1.399892in}}%
\pgfpathlineto{\pgfqpoint{2.946375in}{1.401414in}}%
\pgfpathlineto{\pgfqpoint{2.953195in}{1.399425in}}%
\pgfpathlineto{\pgfqpoint{2.958156in}{1.399047in}}%
\pgfpathlineto{\pgfqpoint{2.971797in}{1.400845in}}%
\pgfpathlineto{\pgfqpoint{2.980478in}{1.398928in}}%
\pgfpathlineto{\pgfqpoint{2.995980in}{1.400355in}}%
\pgfpathlineto{\pgfqpoint{3.002800in}{1.399038in}}%
\pgfpathlineto{\pgfqpoint{3.009621in}{1.400874in}}%
\pgfpathlineto{\pgfqpoint{3.015202in}{1.401242in}}%
\pgfpathlineto{\pgfqpoint{3.028223in}{1.399542in}}%
\pgfpathlineto{\pgfqpoint{3.037524in}{1.401367in}}%
\pgfpathlineto{\pgfqpoint{3.053645in}{1.400214in}}%
\pgfpathlineto{\pgfqpoint{3.060466in}{1.401131in}}%
\pgfpathlineto{\pgfqpoint{3.075348in}{1.400193in}}%
\pgfpathlineto{\pgfqpoint{3.082788in}{1.401347in}}%
\pgfpathlineto{\pgfqpoint{3.098910in}{1.400106in}}%
\pgfpathlineto{\pgfqpoint{3.105731in}{1.401191in}}%
\pgfpathlineto{\pgfqpoint{3.119992in}{1.400735in}}%
\pgfpathlineto{\pgfqpoint{3.126193in}{1.401385in}}%
\pgfpathlineto{\pgfqpoint{3.139214in}{1.399546in}}%
\pgfpathlineto{\pgfqpoint{3.145415in}{1.401483in}}%
\pgfpathlineto{\pgfqpoint{3.156576in}{1.401597in}}%
\pgfpathlineto{\pgfqpoint{3.159676in}{1.401820in}}%
\pgfpathlineto{\pgfqpoint{3.163396in}{1.399153in}}%
\pgfpathlineto{\pgfqpoint{3.166497in}{1.397948in}}%
\pgfpathlineto{\pgfqpoint{3.174557in}{1.398634in}}%
\pgfpathlineto{\pgfqpoint{3.177658in}{1.398181in}}%
\pgfpathlineto{\pgfqpoint{3.186339in}{1.401049in}}%
\pgfpathlineto{\pgfqpoint{3.191919in}{1.399340in}}%
\pgfpathlineto{\pgfqpoint{3.200600in}{1.399237in}}%
\pgfpathlineto{\pgfqpoint{3.207421in}{1.397617in}}%
\pgfpathlineto{\pgfqpoint{3.213001in}{1.399529in}}%
\pgfpathlineto{\pgfqpoint{3.217962in}{1.400458in}}%
\pgfpathlineto{\pgfqpoint{3.235324in}{1.400123in}}%
\pgfpathlineto{\pgfqpoint{3.241524in}{1.399124in}}%
\pgfpathlineto{\pgfqpoint{3.248345in}{1.398340in}}%
\pgfpathlineto{\pgfqpoint{3.263846in}{1.399609in}}%
\pgfpathlineto{\pgfqpoint{3.271287in}{1.398928in}}%
\pgfpathlineto{\pgfqpoint{3.284308in}{1.399564in}}%
\pgfpathlineto{\pgfqpoint{3.293609in}{1.398816in}}%
\pgfpathlineto{\pgfqpoint{3.306631in}{1.399845in}}%
\pgfpathlineto{\pgfqpoint{3.317172in}{1.399027in}}%
\pgfpathlineto{\pgfqpoint{3.329573in}{1.399609in}}%
\pgfpathlineto{\pgfqpoint{3.340114in}{1.399123in}}%
\pgfpathlineto{\pgfqpoint{3.352515in}{1.399683in}}%
\pgfpathlineto{\pgfqpoint{3.363676in}{1.399136in}}%
\pgfpathlineto{\pgfqpoint{3.376698in}{1.399553in}}%
\pgfpathlineto{\pgfqpoint{3.387859in}{1.399330in}}%
\pgfpathlineto{\pgfqpoint{3.399640in}{1.399655in}}%
\pgfpathlineto{\pgfqpoint{3.410801in}{1.399192in}}%
\pgfpathlineto{\pgfqpoint{3.423202in}{1.399634in}}%
\pgfpathlineto{\pgfqpoint{3.433743in}{1.399345in}}%
\pgfpathlineto{\pgfqpoint{3.444284in}{1.399975in}}%
\pgfpathlineto{\pgfqpoint{3.456686in}{1.399076in}}%
\pgfpathlineto{\pgfqpoint{3.467227in}{1.399836in}}%
\pgfpathlineto{\pgfqpoint{3.477768in}{1.399311in}}%
\pgfpathlineto{\pgfqpoint{3.487689in}{1.400139in}}%
\pgfpathlineto{\pgfqpoint{3.499470in}{1.398472in}}%
\pgfpathlineto{\pgfqpoint{3.509391in}{1.399660in}}%
\pgfpathlineto{\pgfqpoint{3.517452in}{1.399291in}}%
\pgfpathlineto{\pgfqpoint{3.525513in}{1.400554in}}%
\pgfpathlineto{\pgfqpoint{3.534813in}{1.397884in}}%
\pgfpathlineto{\pgfqpoint{3.540394in}{1.399523in}}%
\pgfpathlineto{\pgfqpoint{3.545975in}{1.398926in}}%
\pgfpathlineto{\pgfqpoint{3.552175in}{1.402145in}}%
\pgfpathlineto{\pgfqpoint{3.558376in}{1.400073in}}%
\pgfpathlineto{\pgfqpoint{3.563956in}{1.401048in}}%
\pgfpathlineto{\pgfqpoint{3.572637in}{1.398702in}}%
\pgfpathlineto{\pgfqpoint{3.581938in}{1.401585in}}%
\pgfpathlineto{\pgfqpoint{3.591859in}{1.400232in}}%
\pgfpathlineto{\pgfqpoint{3.601160in}{1.401152in}}%
\pgfpathlineto{\pgfqpoint{3.612321in}{1.399197in}}%
\pgfpathlineto{\pgfqpoint{3.624102in}{1.401112in}}%
\pgfpathlineto{\pgfqpoint{3.634643in}{1.399631in}}%
\pgfpathlineto{\pgfqpoint{3.646425in}{1.400917in}}%
\pgfpathlineto{\pgfqpoint{3.656346in}{1.399209in}}%
\pgfpathlineto{\pgfqpoint{3.670607in}{1.400768in}}%
\pgfpathlineto{\pgfqpoint{3.678668in}{1.399345in}}%
\pgfpathlineto{\pgfqpoint{3.693549in}{1.400589in}}%
\pgfpathlineto{\pgfqpoint{3.700370in}{1.399106in}}%
\pgfpathlineto{\pgfqpoint{3.705951in}{1.400607in}}%
\pgfpathlineto{\pgfqpoint{3.712151in}{1.401835in}}%
\pgfpathlineto{\pgfqpoint{3.718352in}{1.400086in}}%
\pgfpathlineto{\pgfqpoint{3.723932in}{1.399340in}}%
\pgfpathlineto{\pgfqpoint{3.738814in}{1.400598in}}%
\pgfpathlineto{\pgfqpoint{3.745635in}{1.399133in}}%
\pgfpathlineto{\pgfqpoint{3.751215in}{1.400937in}}%
\pgfpathlineto{\pgfqpoint{3.756796in}{1.402180in}}%
\pgfpathlineto{\pgfqpoint{3.762376in}{1.400530in}}%
\pgfpathlineto{\pgfqpoint{3.767957in}{1.399324in}}%
\pgfpathlineto{\pgfqpoint{3.773537in}{1.400866in}}%
\pgfpathlineto{\pgfqpoint{3.779118in}{1.401642in}}%
\pgfpathlineto{\pgfqpoint{3.791519in}{1.400171in}}%
\pgfpathlineto{\pgfqpoint{3.799580in}{1.402783in}}%
\pgfpathlineto{\pgfqpoint{3.804540in}{1.400699in}}%
\pgfpathlineto{\pgfqpoint{3.808881in}{1.399416in}}%
\pgfpathlineto{\pgfqpoint{3.813841in}{1.400797in}}%
\pgfpathlineto{\pgfqpoint{3.817562in}{1.400745in}}%
\pgfpathlineto{\pgfqpoint{3.823762in}{1.398801in}}%
\pgfpathlineto{\pgfqpoint{3.827483in}{1.401611in}}%
\pgfpathlineto{\pgfqpoint{3.829963in}{1.402772in}}%
\pgfpathlineto{\pgfqpoint{3.832443in}{1.401385in}}%
\pgfpathlineto{\pgfqpoint{3.836164in}{1.399333in}}%
\pgfpathlineto{\pgfqpoint{3.842984in}{1.400000in}}%
\pgfpathlineto{\pgfqpoint{3.847945in}{1.396564in}}%
\pgfpathlineto{\pgfqpoint{3.851045in}{1.398020in}}%
\pgfpathlineto{\pgfqpoint{3.855386in}{1.400010in}}%
\pgfpathlineto{\pgfqpoint{3.859106in}{1.399135in}}%
\pgfpathlineto{\pgfqpoint{3.864066in}{1.398155in}}%
\pgfpathlineto{\pgfqpoint{3.875848in}{1.399697in}}%
\pgfpathlineto{\pgfqpoint{3.883288in}{1.397060in}}%
\pgfpathlineto{\pgfqpoint{3.887629in}{1.398503in}}%
\pgfpathlineto{\pgfqpoint{3.893829in}{1.400471in}}%
\pgfpathlineto{\pgfqpoint{3.898790in}{1.399175in}}%
\pgfpathlineto{\pgfqpoint{3.904990in}{1.397828in}}%
\pgfpathlineto{\pgfqpoint{3.910571in}{1.399748in}}%
\pgfpathlineto{\pgfqpoint{3.915532in}{1.400729in}}%
\pgfpathlineto{\pgfqpoint{3.920492in}{1.399076in}}%
\pgfpathlineto{\pgfqpoint{3.926693in}{1.397372in}}%
\pgfpathlineto{\pgfqpoint{3.931653in}{1.398975in}}%
\pgfpathlineto{\pgfqpoint{3.937854in}{1.400824in}}%
\pgfpathlineto{\pgfqpoint{3.942814in}{1.399421in}}%
\pgfpathlineto{\pgfqpoint{3.949015in}{1.397709in}}%
\pgfpathlineto{\pgfqpoint{3.953975in}{1.399182in}}%
\pgfpathlineto{\pgfqpoint{3.960176in}{1.400829in}}%
\pgfpathlineto{\pgfqpoint{3.965137in}{1.399293in}}%
\pgfpathlineto{\pgfqpoint{3.971337in}{1.397517in}}%
\pgfpathlineto{\pgfqpoint{3.976298in}{1.399024in}}%
\pgfpathlineto{\pgfqpoint{3.982498in}{1.400983in}}%
\pgfpathlineto{\pgfqpoint{3.986839in}{1.399936in}}%
\pgfpathlineto{\pgfqpoint{3.994279in}{1.397868in}}%
\pgfpathlineto{\pgfqpoint{3.999240in}{1.399458in}}%
\pgfpathlineto{\pgfqpoint{4.004820in}{1.400885in}}%
\pgfpathlineto{\pgfqpoint{4.009781in}{1.399450in}}%
\pgfpathlineto{\pgfqpoint{4.015982in}{1.397661in}}%
\pgfpathlineto{\pgfqpoint{4.020322in}{1.398845in}}%
\pgfpathlineto{\pgfqpoint{4.027763in}{1.401045in}}%
\pgfpathlineto{\pgfqpoint{4.032723in}{1.399495in}}%
\pgfpathlineto{\pgfqpoint{4.038304in}{1.398084in}}%
\pgfpathlineto{\pgfqpoint{4.043264in}{1.399453in}}%
\pgfpathlineto{\pgfqpoint{4.048845in}{1.400631in}}%
\pgfpathlineto{\pgfqpoint{4.053805in}{1.399033in}}%
\pgfpathlineto{\pgfqpoint{4.059386in}{1.397538in}}%
\pgfpathlineto{\pgfqpoint{4.063726in}{1.398981in}}%
\pgfpathlineto{\pgfqpoint{4.069927in}{1.401069in}}%
\pgfpathlineto{\pgfqpoint{4.074267in}{1.399750in}}%
\pgfpathlineto{\pgfqpoint{4.079848in}{1.398368in}}%
\pgfpathlineto{\pgfqpoint{4.089769in}{1.399145in}}%
\pgfpathlineto{\pgfqpoint{4.095350in}{1.396989in}}%
\pgfpathlineto{\pgfqpoint{4.098450in}{1.399047in}}%
\pgfpathlineto{\pgfqpoint{4.102170in}{1.401252in}}%
\pgfpathlineto{\pgfqpoint{4.105271in}{1.399984in}}%
\pgfpathlineto{\pgfqpoint{4.107751in}{1.399463in}}%
\pgfpathlineto{\pgfqpoint{4.114571in}{1.401693in}}%
\pgfpathlineto{\pgfqpoint{4.120152in}{1.398396in}}%
\pgfpathlineto{\pgfqpoint{4.128833in}{1.399758in}}%
\pgfpathlineto{\pgfqpoint{4.133173in}{1.398976in}}%
\pgfpathlineto{\pgfqpoint{4.137514in}{1.401129in}}%
\pgfpathlineto{\pgfqpoint{4.141854in}{1.402605in}}%
\pgfpathlineto{\pgfqpoint{4.146195in}{1.401129in}}%
\pgfpathlineto{\pgfqpoint{4.151775in}{1.399209in}}%
\pgfpathlineto{\pgfqpoint{4.156736in}{1.400372in}}%
\pgfpathlineto{\pgfqpoint{4.162316in}{1.401335in}}%
\pgfpathlineto{\pgfqpoint{4.175958in}{1.399951in}}%
\pgfpathlineto{\pgfqpoint{4.182778in}{1.401977in}}%
\pgfpathlineto{\pgfqpoint{4.187739in}{1.400583in}}%
\pgfpathlineto{\pgfqpoint{4.193939in}{1.398857in}}%
\pgfpathlineto{\pgfqpoint{4.198900in}{1.400118in}}%
\pgfpathlineto{\pgfqpoint{4.205101in}{1.401584in}}%
\pgfpathlineto{\pgfqpoint{4.210681in}{1.399943in}}%
\pgfpathlineto{\pgfqpoint{4.216262in}{1.398854in}}%
\pgfpathlineto{\pgfqpoint{4.221842in}{1.400622in}}%
\pgfpathlineto{\pgfqpoint{4.227423in}{1.401868in}}%
\pgfpathlineto{\pgfqpoint{4.232383in}{1.400470in}}%
\pgfpathlineto{\pgfqpoint{4.238584in}{1.398926in}}%
\pgfpathlineto{\pgfqpoint{4.243544in}{1.400232in}}%
\pgfpathlineto{\pgfqpoint{4.249745in}{1.401739in}}%
\pgfpathlineto{\pgfqpoint{4.255326in}{1.400220in}}%
\pgfpathlineto{\pgfqpoint{4.261526in}{1.399055in}}%
\pgfpathlineto{\pgfqpoint{4.267727in}{1.401015in}}%
\pgfpathlineto{\pgfqpoint{4.273307in}{1.401827in}}%
\pgfpathlineto{\pgfqpoint{4.280128in}{1.399668in}}%
\pgfpathlineto{\pgfqpoint{4.285089in}{1.399256in}}%
\pgfpathlineto{\pgfqpoint{4.298730in}{1.401002in}}%
\pgfpathlineto{\pgfqpoint{4.306791in}{1.399332in}}%
\pgfpathlineto{\pgfqpoint{4.312991in}{1.401088in}}%
\pgfpathlineto{\pgfqpoint{4.318572in}{1.401858in}}%
\pgfpathlineto{\pgfqpoint{4.325393in}{1.399802in}}%
\pgfpathlineto{\pgfqpoint{4.330353in}{1.399308in}}%
\pgfpathlineto{\pgfqpoint{4.345235in}{1.400763in}}%
\pgfpathlineto{\pgfqpoint{4.352055in}{1.399879in}}%
\pgfpathlineto{\pgfqpoint{4.365697in}{1.401140in}}%
\pgfpathlineto{\pgfqpoint{4.373137in}{1.399073in}}%
\pgfpathlineto{\pgfqpoint{4.378098in}{1.400737in}}%
\pgfpathlineto{\pgfqpoint{4.383058in}{1.401873in}}%
\pgfpathlineto{\pgfqpoint{4.393599in}{1.401581in}}%
\pgfpathlineto{\pgfqpoint{4.397320in}{1.402330in}}%
\pgfpathlineto{\pgfqpoint{4.400420in}{1.400193in}}%
\pgfpathlineto{\pgfqpoint{4.404140in}{1.397852in}}%
\pgfpathlineto{\pgfqpoint{4.407241in}{1.399123in}}%
\pgfpathlineto{\pgfqpoint{4.410341in}{1.399565in}}%
\pgfpathlineto{\pgfqpoint{4.415922in}{1.398123in}}%
\pgfpathlineto{\pgfqpoint{4.423362in}{1.401180in}}%
\pgfpathlineto{\pgfqpoint{4.432663in}{1.398649in}}%
\pgfpathlineto{\pgfqpoint{4.439484in}{1.398857in}}%
\pgfpathlineto{\pgfqpoint{4.448785in}{1.397758in}}%
\pgfpathlineto{\pgfqpoint{4.460566in}{1.399893in}}%
\pgfpathlineto{\pgfqpoint{4.471727in}{1.398591in}}%
\pgfpathlineto{\pgfqpoint{4.482268in}{1.399131in}}%
\pgfpathlineto{\pgfqpoint{4.492809in}{1.398525in}}%
\pgfpathlineto{\pgfqpoint{4.504590in}{1.399622in}}%
\pgfpathlineto{\pgfqpoint{4.516992in}{1.398698in}}%
\pgfpathlineto{\pgfqpoint{4.528773in}{1.399154in}}%
\pgfpathlineto{\pgfqpoint{4.539314in}{1.398856in}}%
\pgfpathlineto{\pgfqpoint{4.551095in}{1.399556in}}%
\pgfpathlineto{\pgfqpoint{4.563496in}{1.398803in}}%
\pgfpathlineto{\pgfqpoint{4.574657in}{1.399308in}}%
\pgfpathlineto{\pgfqpoint{4.584578in}{1.398987in}}%
\pgfpathlineto{\pgfqpoint{4.595120in}{1.399895in}}%
\pgfpathlineto{\pgfqpoint{4.608141in}{1.398511in}}%
\pgfpathlineto{\pgfqpoint{4.616822in}{1.399159in}}%
\pgfpathlineto{\pgfqpoint{4.624882in}{1.398780in}}%
\pgfpathlineto{\pgfqpoint{4.633563in}{1.400468in}}%
\pgfpathlineto{\pgfqpoint{4.642864in}{1.398031in}}%
\pgfpathlineto{\pgfqpoint{4.647825in}{1.399110in}}%
\pgfpathlineto{\pgfqpoint{4.652785in}{1.398503in}}%
\pgfpathlineto{\pgfqpoint{4.659606in}{1.402013in}}%
\pgfpathlineto{\pgfqpoint{4.665187in}{1.400592in}}%
\pgfpathlineto{\pgfqpoint{4.672627in}{1.401121in}}%
\pgfpathlineto{\pgfqpoint{4.680068in}{1.398973in}}%
\pgfpathlineto{\pgfqpoint{4.686269in}{1.401423in}}%
\pgfpathlineto{\pgfqpoint{4.690609in}{1.401889in}}%
\pgfpathlineto{\pgfqpoint{4.703630in}{1.400690in}}%
\pgfpathlineto{\pgfqpoint{4.711071in}{1.401436in}}%
\pgfpathlineto{\pgfqpoint{4.724092in}{1.399835in}}%
\pgfpathlineto{\pgfqpoint{4.733393in}{1.401677in}}%
\pgfpathlineto{\pgfqpoint{4.747655in}{1.400466in}}%
\pgfpathlineto{\pgfqpoint{4.755096in}{1.401740in}}%
\pgfpathlineto{\pgfqpoint{4.762536in}{1.399714in}}%
\pgfpathlineto{\pgfqpoint{4.767497in}{1.399631in}}%
\pgfpathlineto{\pgfqpoint{4.779898in}{1.401688in}}%
\pgfpathlineto{\pgfqpoint{4.789819in}{1.399833in}}%
\pgfpathlineto{\pgfqpoint{4.802840in}{1.401275in}}%
\pgfpathlineto{\pgfqpoint{4.810901in}{1.399430in}}%
\pgfpathlineto{\pgfqpoint{4.816482in}{1.401379in}}%
\pgfpathlineto{\pgfqpoint{4.822062in}{1.402583in}}%
\pgfpathlineto{\pgfqpoint{4.827643in}{1.400822in}}%
\pgfpathlineto{\pgfqpoint{4.833223in}{1.399873in}}%
\pgfpathlineto{\pgfqpoint{4.845625in}{1.400995in}}%
\pgfpathlineto{\pgfqpoint{4.851205in}{1.399332in}}%
\pgfpathlineto{\pgfqpoint{4.854925in}{1.400786in}}%
\pgfpathlineto{\pgfqpoint{4.860506in}{1.403590in}}%
\pgfpathlineto{\pgfqpoint{4.863606in}{1.402326in}}%
\pgfpathlineto{\pgfqpoint{4.868567in}{1.399580in}}%
\pgfpathlineto{\pgfqpoint{4.872287in}{1.400920in}}%
\pgfpathlineto{\pgfqpoint{4.874767in}{1.400701in}}%
\pgfpathlineto{\pgfqpoint{4.880968in}{1.397206in}}%
\pgfpathlineto{\pgfqpoint{4.886549in}{1.400841in}}%
\pgfpathlineto{\pgfqpoint{4.893989in}{1.399261in}}%
\pgfpathlineto{\pgfqpoint{4.898950in}{1.401049in}}%
\pgfpathlineto{\pgfqpoint{4.902670in}{1.398949in}}%
\pgfpathlineto{\pgfqpoint{4.907631in}{1.396242in}}%
\pgfpathlineto{\pgfqpoint{4.911351in}{1.397318in}}%
\pgfpathlineto{\pgfqpoint{4.918172in}{1.400035in}}%
\pgfpathlineto{\pgfqpoint{4.922512in}{1.398843in}}%
\pgfpathlineto{\pgfqpoint{4.928093in}{1.397398in}}%
\pgfpathlineto{\pgfqpoint{4.932433in}{1.398860in}}%
\pgfpathlineto{\pgfqpoint{4.938634in}{1.400745in}}%
\pgfpathlineto{\pgfqpoint{4.942974in}{1.399231in}}%
\pgfpathlineto{\pgfqpoint{4.949175in}{1.396806in}}%
\pgfpathlineto{\pgfqpoint{4.953515in}{1.397950in}}%
\pgfpathlineto{\pgfqpoint{4.960956in}{1.400717in}}%
\pgfpathlineto{\pgfqpoint{4.965297in}{1.399461in}}%
\pgfpathlineto{\pgfqpoint{4.972117in}{1.397302in}}%
\pgfpathlineto{\pgfqpoint{4.976458in}{1.398763in}}%
\pgfpathlineto{\pgfqpoint{4.982658in}{1.400911in}}%
\pgfpathlineto{\pgfqpoint{4.986999in}{1.399619in}}%
\pgfpathlineto{\pgfqpoint{4.993819in}{1.396960in}}%
\pgfpathlineto{\pgfqpoint{4.998160in}{1.398212in}}%
\pgfpathlineto{\pgfqpoint{5.005601in}{1.401036in}}%
\pgfpathlineto{\pgfqpoint{5.009941in}{1.399651in}}%
\pgfpathlineto{\pgfqpoint{5.016142in}{1.397407in}}%
\pgfpathlineto{\pgfqpoint{5.020482in}{1.398689in}}%
\pgfpathlineto{\pgfqpoint{5.027303in}{1.401068in}}%
\pgfpathlineto{\pgfqpoint{5.031643in}{1.399559in}}%
\pgfpathlineto{\pgfqpoint{5.037844in}{1.397014in}}%
\pgfpathlineto{\pgfqpoint{5.041564in}{1.397948in}}%
\pgfpathlineto{\pgfqpoint{5.049625in}{1.401155in}}%
\pgfpathlineto{\pgfqpoint{5.053965in}{1.399594in}}%
\pgfpathlineto{\pgfqpoint{5.059546in}{1.397662in}}%
\pgfpathlineto{\pgfqpoint{5.063886in}{1.399146in}}%
\pgfpathlineto{\pgfqpoint{5.068847in}{1.400839in}}%
\pgfpathlineto{\pgfqpoint{5.072567in}{1.399595in}}%
\pgfpathlineto{\pgfqpoint{5.079388in}{1.396403in}}%
\pgfpathlineto{\pgfqpoint{5.082488in}{1.397870in}}%
\pgfpathlineto{\pgfqpoint{5.088069in}{1.401090in}}%
\pgfpathlineto{\pgfqpoint{5.091169in}{1.399991in}}%
\pgfpathlineto{\pgfqpoint{5.094889in}{1.398426in}}%
\pgfpathlineto{\pgfqpoint{5.097370in}{1.399819in}}%
\pgfpathlineto{\pgfqpoint{5.101090in}{1.401942in}}%
\pgfpathlineto{\pgfqpoint{5.103570in}{1.400049in}}%
\pgfpathlineto{\pgfqpoint{5.107291in}{1.396828in}}%
\pgfpathlineto{\pgfqpoint{5.109771in}{1.398758in}}%
\pgfpathlineto{\pgfqpoint{5.112871in}{1.400925in}}%
\pgfpathlineto{\pgfqpoint{5.115352in}{1.399364in}}%
\pgfpathlineto{\pgfqpoint{5.118452in}{1.398143in}}%
\pgfpathlineto{\pgfqpoint{5.120932in}{1.400003in}}%
\pgfpathlineto{\pgfqpoint{5.125273in}{1.403494in}}%
\pgfpathlineto{\pgfqpoint{5.127753in}{1.402685in}}%
\pgfpathlineto{\pgfqpoint{5.133953in}{1.399248in}}%
\pgfpathlineto{\pgfqpoint{5.137674in}{1.400417in}}%
\pgfpathlineto{\pgfqpoint{5.142634in}{1.402113in}}%
\pgfpathlineto{\pgfqpoint{5.146355in}{1.400757in}}%
\pgfpathlineto{\pgfqpoint{5.151935in}{1.398489in}}%
\pgfpathlineto{\pgfqpoint{5.155656in}{1.399578in}}%
\pgfpathlineto{\pgfqpoint{5.163096in}{1.402767in}}%
\pgfpathlineto{\pgfqpoint{5.166817in}{1.401519in}}%
\pgfpathlineto{\pgfqpoint{5.173637in}{1.398707in}}%
\pgfpathlineto{\pgfqpoint{5.177978in}{1.399963in}}%
\pgfpathlineto{\pgfqpoint{5.184798in}{1.402328in}}%
\pgfpathlineto{\pgfqpoint{5.189139in}{1.400756in}}%
\pgfpathlineto{\pgfqpoint{5.195340in}{1.398513in}}%
\pgfpathlineto{\pgfqpoint{5.199680in}{1.399953in}}%
\pgfpathlineto{\pgfqpoint{5.206501in}{1.402771in}}%
\pgfpathlineto{\pgfqpoint{5.210221in}{1.401777in}}%
\pgfpathlineto{\pgfqpoint{5.218282in}{1.398740in}}%
\pgfpathlineto{\pgfqpoint{5.222622in}{1.400306in}}%
\pgfpathlineto{\pgfqpoint{5.228823in}{1.402639in}}%
\pgfpathlineto{\pgfqpoint{5.233163in}{1.401299in}}%
\pgfpathlineto{\pgfqpoint{5.239984in}{1.398738in}}%
\pgfpathlineto{\pgfqpoint{5.244324in}{1.400183in}}%
\pgfpathlineto{\pgfqpoint{5.251145in}{1.403039in}}%
\pgfpathlineto{\pgfqpoint{5.254865in}{1.402059in}}%
\pgfpathlineto{\pgfqpoint{5.262926in}{1.398922in}}%
\pgfpathlineto{\pgfqpoint{5.267267in}{1.400455in}}%
\pgfpathlineto{\pgfqpoint{5.273467in}{1.402695in}}%
\pgfpathlineto{\pgfqpoint{5.277808in}{1.401267in}}%
\pgfpathlineto{\pgfqpoint{5.284008in}{1.399160in}}%
\pgfpathlineto{\pgfqpoint{5.287729in}{1.400432in}}%
\pgfpathlineto{\pgfqpoint{5.294549in}{1.403574in}}%
\pgfpathlineto{\pgfqpoint{5.298270in}{1.402309in}}%
\pgfpathlineto{\pgfqpoint{5.305091in}{1.399072in}}%
\pgfpathlineto{\pgfqpoint{5.308811in}{1.400479in}}%
\pgfpathlineto{\pgfqpoint{5.313771in}{1.402232in}}%
\pgfpathlineto{\pgfqpoint{5.317492in}{1.400254in}}%
\pgfpathlineto{\pgfqpoint{5.320592in}{1.399226in}}%
\pgfpathlineto{\pgfqpoint{5.323072in}{1.400589in}}%
\pgfpathlineto{\pgfqpoint{5.327413in}{1.403593in}}%
\pgfpathlineto{\pgfqpoint{5.329893in}{1.401307in}}%
\pgfpathlineto{\pgfqpoint{5.332993in}{1.398146in}}%
\pgfpathlineto{\pgfqpoint{5.334853in}{1.398698in}}%
\pgfpathlineto{\pgfqpoint{5.338574in}{1.400656in}}%
\pgfpathlineto{\pgfqpoint{5.341054in}{1.398784in}}%
\pgfpathlineto{\pgfqpoint{5.344774in}{1.395921in}}%
\pgfpathlineto{\pgfqpoint{5.347255in}{1.397085in}}%
\pgfpathlineto{\pgfqpoint{5.352835in}{1.400752in}}%
\pgfpathlineto{\pgfqpoint{5.355936in}{1.399707in}}%
\pgfpathlineto{\pgfqpoint{5.361516in}{1.397265in}}%
\pgfpathlineto{\pgfqpoint{5.365857in}{1.398734in}}%
\pgfpathlineto{\pgfqpoint{5.370817in}{1.399936in}}%
\pgfpathlineto{\pgfqpoint{5.375158in}{1.398328in}}%
\pgfpathlineto{\pgfqpoint{5.380738in}{1.396472in}}%
\pgfpathlineto{\pgfqpoint{5.385079in}{1.397884in}}%
\pgfpathlineto{\pgfqpoint{5.391899in}{1.400470in}}%
\pgfpathlineto{\pgfqpoint{5.396240in}{1.399275in}}%
\pgfpathlineto{\pgfqpoint{5.403060in}{1.397205in}}%
\pgfpathlineto{\pgfqpoint{5.408021in}{1.398745in}}%
\pgfpathlineto{\pgfqpoint{5.413601in}{1.400132in}}%
\pgfpathlineto{\pgfqpoint{5.418562in}{1.398676in}}%
\pgfpathlineto{\pgfqpoint{5.424762in}{1.397033in}}%
\pgfpathlineto{\pgfqpoint{5.429723in}{1.398612in}}%
\pgfpathlineto{\pgfqpoint{5.435924in}{1.400503in}}%
\pgfpathlineto{\pgfqpoint{5.440884in}{1.399151in}}%
\pgfpathlineto{\pgfqpoint{5.447085in}{1.397458in}}%
\pgfpathlineto{\pgfqpoint{5.452045in}{1.398778in}}%
\pgfpathlineto{\pgfqpoint{5.458246in}{1.400125in}}%
\pgfpathlineto{\pgfqpoint{5.463826in}{1.398405in}}%
\pgfpathlineto{\pgfqpoint{5.469407in}{1.397395in}}%
\pgfpathlineto{\pgfqpoint{5.474987in}{1.399456in}}%
\pgfpathlineto{\pgfqpoint{5.479948in}{1.400644in}}%
\pgfpathlineto{\pgfqpoint{5.484908in}{1.399119in}}%
\pgfpathlineto{\pgfqpoint{5.490489in}{1.397652in}}%
\pgfpathlineto{\pgfqpoint{5.496690in}{1.399249in}}%
\pgfpathlineto{\pgfqpoint{5.501030in}{1.399198in}}%
\pgfpathlineto{\pgfqpoint{5.509711in}{1.397218in}}%
\pgfpathlineto{\pgfqpoint{5.518392in}{1.400847in}}%
\pgfpathlineto{\pgfqpoint{5.523352in}{1.399191in}}%
\pgfpathlineto{\pgfqpoint{5.528933in}{1.400682in}}%
\pgfpathlineto{\pgfqpoint{5.533893in}{1.397611in}}%
\pgfpathlineto{\pgfqpoint{5.536994in}{1.399751in}}%
\pgfpathlineto{\pgfqpoint{5.540094in}{1.401095in}}%
\pgfpathlineto{\pgfqpoint{5.548155in}{1.400910in}}%
\pgfpathlineto{\pgfqpoint{5.554355in}{1.403030in}}%
\pgfpathlineto{\pgfqpoint{5.558696in}{1.401316in}}%
\pgfpathlineto{\pgfqpoint{5.564276in}{1.399536in}}%
\pgfpathlineto{\pgfqpoint{5.569237in}{1.400786in}}%
\pgfpathlineto{\pgfqpoint{5.574817in}{1.401587in}}%
\pgfpathlineto{\pgfqpoint{5.587219in}{1.400445in}}%
\pgfpathlineto{\pgfqpoint{5.595280in}{1.402164in}}%
\pgfpathlineto{\pgfqpoint{5.602100in}{1.400206in}}%
\pgfpathlineto{\pgfqpoint{5.607061in}{1.399778in}}%
\pgfpathlineto{\pgfqpoint{5.621322in}{1.400991in}}%
\pgfpathlineto{\pgfqpoint{5.628143in}{1.400062in}}%
\pgfpathlineto{\pgfqpoint{5.644264in}{1.401095in}}%
\pgfpathlineto{\pgfqpoint{5.651085in}{1.400056in}}%
\pgfpathlineto{\pgfqpoint{5.667827in}{1.400847in}}%
\pgfpathlineto{\pgfqpoint{5.674027in}{1.400883in}}%
\pgfpathlineto{\pgfqpoint{5.684568in}{1.402230in}}%
\pgfpathlineto{\pgfqpoint{5.696350in}{1.400349in}}%
\pgfpathlineto{\pgfqpoint{5.705031in}{1.401716in}}%
\pgfpathlineto{\pgfqpoint{5.712471in}{1.401827in}}%
\pgfpathlineto{\pgfqpoint{5.716812in}{1.402215in}}%
\pgfpathlineto{\pgfqpoint{5.721152in}{1.399128in}}%
\pgfpathlineto{\pgfqpoint{5.724252in}{1.398315in}}%
\pgfpathlineto{\pgfqpoint{5.729833in}{1.398422in}}%
\pgfpathlineto{\pgfqpoint{5.734793in}{1.397746in}}%
\pgfpathlineto{\pgfqpoint{5.744094in}{1.399962in}}%
\pgfpathlineto{\pgfqpoint{5.754015in}{1.397942in}}%
\pgfpathlineto{\pgfqpoint{5.765797in}{1.397890in}}%
\pgfpathlineto{\pgfqpoint{5.773857in}{1.398384in}}%
\pgfpathlineto{\pgfqpoint{5.783778in}{1.399306in}}%
\pgfpathlineto{\pgfqpoint{5.800520in}{1.398633in}}%
\pgfpathlineto{\pgfqpoint{5.812921in}{1.398475in}}%
\pgfpathlineto{\pgfqpoint{5.824082in}{1.399310in}}%
\pgfpathlineto{\pgfqpoint{5.832763in}{1.398985in}}%
\pgfpathlineto{\pgfqpoint{5.843304in}{1.398521in}}%
\pgfpathlineto{\pgfqpoint{5.856326in}{1.398841in}}%
\pgfpathlineto{\pgfqpoint{5.865007in}{1.399156in}}%
\pgfpathlineto{\pgfqpoint{5.873687in}{1.399660in}}%
\pgfpathlineto{\pgfqpoint{5.886709in}{1.398135in}}%
\pgfpathlineto{\pgfqpoint{5.894769in}{1.398686in}}%
\pgfpathlineto{\pgfqpoint{5.900970in}{1.398881in}}%
\pgfpathlineto{\pgfqpoint{5.908411in}{1.400920in}}%
\pgfpathlineto{\pgfqpoint{5.915852in}{1.398687in}}%
\pgfpathlineto{\pgfqpoint{5.920192in}{1.399363in}}%
\pgfpathlineto{\pgfqpoint{5.925153in}{1.398304in}}%
\pgfpathlineto{\pgfqpoint{5.933833in}{1.401734in}}%
\pgfpathlineto{\pgfqpoint{5.940034in}{1.401212in}}%
\pgfpathlineto{\pgfqpoint{5.947475in}{1.401649in}}%
\pgfpathlineto{\pgfqpoint{5.958016in}{1.399738in}}%
\pgfpathlineto{\pgfqpoint{5.969797in}{1.401402in}}%
\pgfpathlineto{\pgfqpoint{5.979718in}{1.400507in}}%
\pgfpathlineto{\pgfqpoint{5.990879in}{1.401261in}}%
\pgfpathlineto{\pgfqpoint{6.000800in}{1.399971in}}%
\pgfpathlineto{\pgfqpoint{6.015062in}{1.401310in}}%
\pgfpathlineto{\pgfqpoint{6.023122in}{1.400463in}}%
\pgfpathlineto{\pgfqpoint{6.036764in}{1.401266in}}%
\pgfpathlineto{\pgfqpoint{6.044824in}{1.400105in}}%
\pgfpathlineto{\pgfqpoint{6.059706in}{1.401727in}}%
\pgfpathlineto{\pgfqpoint{6.066527in}{1.400811in}}%
\pgfpathlineto{\pgfqpoint{6.077688in}{1.401624in}}%
\pgfpathlineto{\pgfqpoint{6.084508in}{1.399802in}}%
\pgfpathlineto{\pgfqpoint{6.088849in}{1.401924in}}%
\pgfpathlineto{\pgfqpoint{6.092569in}{1.403325in}}%
\pgfpathlineto{\pgfqpoint{6.095670in}{1.401959in}}%
\pgfpathlineto{\pgfqpoint{6.100010in}{1.400226in}}%
\pgfpathlineto{\pgfqpoint{6.105591in}{1.400446in}}%
\pgfpathlineto{\pgfqpoint{6.110551in}{1.396697in}}%
\pgfpathlineto{\pgfqpoint{6.113651in}{1.398340in}}%
\pgfpathlineto{\pgfqpoint{6.117372in}{1.399411in}}%
\pgfpathlineto{\pgfqpoint{6.126053in}{1.398419in}}%
\pgfpathlineto{\pgfqpoint{6.131633in}{1.399897in}}%
\pgfpathlineto{\pgfqpoint{6.135974in}{1.398266in}}%
\pgfpathlineto{\pgfqpoint{6.141554in}{1.396258in}}%
\pgfpathlineto{\pgfqpoint{6.145895in}{1.397406in}}%
\pgfpathlineto{\pgfqpoint{6.152715in}{1.399469in}}%
\pgfpathlineto{\pgfqpoint{6.158296in}{1.398018in}}%
\pgfpathlineto{\pgfqpoint{6.163876in}{1.397294in}}%
\pgfpathlineto{\pgfqpoint{6.176898in}{1.399133in}}%
\pgfpathlineto{\pgfqpoint{6.184958in}{1.396724in}}%
\pgfpathlineto{\pgfqpoint{6.189919in}{1.398352in}}%
\pgfpathlineto{\pgfqpoint{6.196120in}{1.400020in}}%
\pgfpathlineto{\pgfqpoint{6.201080in}{1.398583in}}%
\pgfpathlineto{\pgfqpoint{6.206661in}{1.397355in}}%
\pgfpathlineto{\pgfqpoint{6.211621in}{1.398989in}}%
\pgfpathlineto{\pgfqpoint{6.217202in}{1.400270in}}%
\pgfpathlineto{\pgfqpoint{6.222162in}{1.398421in}}%
\pgfpathlineto{\pgfqpoint{6.227743in}{1.396720in}}%
\pgfpathlineto{\pgfqpoint{6.232083in}{1.398158in}}%
\pgfpathlineto{\pgfqpoint{6.238284in}{1.400134in}}%
\pgfpathlineto{\pgfqpoint{6.242624in}{1.398701in}}%
\pgfpathlineto{\pgfqpoint{6.247585in}{1.397626in}}%
\pgfpathlineto{\pgfqpoint{6.252545in}{1.399960in}}%
\pgfpathlineto{\pgfqpoint{6.255646in}{1.400389in}}%
\pgfpathlineto{\pgfqpoint{6.258746in}{1.398515in}}%
\pgfpathlineto{\pgfqpoint{6.263086in}{1.396096in}}%
\pgfpathlineto{\pgfqpoint{6.265567in}{1.397785in}}%
\pgfpathlineto{\pgfqpoint{6.269287in}{1.400598in}}%
\pgfpathlineto{\pgfqpoint{6.271767in}{1.399409in}}%
\pgfpathlineto{\pgfqpoint{6.274247in}{1.398371in}}%
\pgfpathlineto{\pgfqpoint{6.276728in}{1.400592in}}%
\pgfpathlineto{\pgfqpoint{6.279828in}{1.403394in}}%
\pgfpathlineto{\pgfqpoint{6.281688in}{1.402567in}}%
\pgfpathlineto{\pgfqpoint{6.286649in}{1.399337in}}%
\pgfpathlineto{\pgfqpoint{6.290369in}{1.401154in}}%
\pgfpathlineto{\pgfqpoint{6.293469in}{1.402047in}}%
\pgfpathlineto{\pgfqpoint{6.297190in}{1.400369in}}%
\pgfpathlineto{\pgfqpoint{6.301530in}{1.399119in}}%
\pgfpathlineto{\pgfqpoint{6.305251in}{1.400785in}}%
\pgfpathlineto{\pgfqpoint{6.310831in}{1.403358in}}%
\pgfpathlineto{\pgfqpoint{6.314551in}{1.402270in}}%
\pgfpathlineto{\pgfqpoint{6.321372in}{1.399391in}}%
\pgfpathlineto{\pgfqpoint{6.325713in}{1.400594in}}%
\pgfpathlineto{\pgfqpoint{6.331913in}{1.402635in}}%
\pgfpathlineto{\pgfqpoint{6.336254in}{1.401167in}}%
\pgfpathlineto{\pgfqpoint{6.342454in}{1.399071in}}%
\pgfpathlineto{\pgfqpoint{6.346795in}{1.400585in}}%
\pgfpathlineto{\pgfqpoint{6.353615in}{1.403262in}}%
\pgfpathlineto{\pgfqpoint{6.357956in}{1.401855in}}%
\pgfpathlineto{\pgfqpoint{6.364776in}{1.399404in}}%
\pgfpathlineto{\pgfqpoint{6.369117in}{1.400823in}}%
\pgfpathlineto{\pgfqpoint{6.375318in}{1.403041in}}%
\pgfpathlineto{\pgfqpoint{6.379658in}{1.401657in}}%
\pgfpathlineto{\pgfqpoint{6.385859in}{1.399338in}}%
\pgfpathlineto{\pgfqpoint{6.389579in}{1.400397in}}%
\pgfpathlineto{\pgfqpoint{6.397640in}{1.403738in}}%
\pgfpathlineto{\pgfqpoint{6.401980in}{1.402008in}}%
\pgfpathlineto{\pgfqpoint{6.407561in}{1.399663in}}%
\pgfpathlineto{\pgfqpoint{6.411281in}{1.400591in}}%
\pgfpathlineto{\pgfqpoint{6.418102in}{1.402921in}}%
\pgfpathlineto{\pgfqpoint{6.422442in}{1.400824in}}%
\pgfpathlineto{\pgfqpoint{6.426163in}{1.399650in}}%
\pgfpathlineto{\pgfqpoint{6.429263in}{1.401100in}}%
\pgfpathlineto{\pgfqpoint{6.434843in}{1.404384in}}%
\pgfpathlineto{\pgfqpoint{6.437324in}{1.402269in}}%
\pgfpathlineto{\pgfqpoint{6.441664in}{1.398500in}}%
\pgfpathlineto{\pgfqpoint{6.444144in}{1.400003in}}%
\pgfpathlineto{\pgfqpoint{6.446625in}{1.400986in}}%
\pgfpathlineto{\pgfqpoint{6.449105in}{1.398516in}}%
\pgfpathlineto{\pgfqpoint{6.452205in}{1.395897in}}%
\pgfpathlineto{\pgfqpoint{6.454065in}{1.397130in}}%
\pgfpathlineto{\pgfqpoint{6.458406in}{1.401043in}}%
\pgfpathlineto{\pgfqpoint{6.460886in}{1.400284in}}%
\pgfpathlineto{\pgfqpoint{6.466467in}{1.397391in}}%
\pgfpathlineto{\pgfqpoint{6.471427in}{1.399530in}}%
\pgfpathlineto{\pgfqpoint{6.475148in}{1.399554in}}%
\pgfpathlineto{\pgfqpoint{6.485689in}{1.396112in}}%
\pgfpathlineto{\pgfqpoint{6.496230in}{1.400061in}}%
\pgfpathlineto{\pgfqpoint{6.507391in}{1.397365in}}%
\pgfpathlineto{\pgfqpoint{6.516692in}{1.399986in}}%
\pgfpathlineto{\pgfqpoint{6.522892in}{1.397125in}}%
\pgfpathlineto{\pgfqpoint{6.527233in}{1.396482in}}%
\pgfpathlineto{\pgfqpoint{6.532193in}{1.398714in}}%
\pgfpathlineto{\pgfqpoint{6.537154in}{1.400559in}}%
\pgfpathlineto{\pgfqpoint{6.541494in}{1.399503in}}%
\pgfpathlineto{\pgfqpoint{6.548315in}{1.397148in}}%
\pgfpathlineto{\pgfqpoint{6.552655in}{1.398443in}}%
\pgfpathlineto{\pgfqpoint{6.558856in}{1.400260in}}%
\pgfpathlineto{\pgfqpoint{6.563196in}{1.398823in}}%
\pgfpathlineto{\pgfqpoint{6.569397in}{1.396657in}}%
\pgfpathlineto{\pgfqpoint{6.573117in}{1.397797in}}%
\pgfpathlineto{\pgfqpoint{6.580558in}{1.400650in}}%
\pgfpathlineto{\pgfqpoint{6.584898in}{1.399121in}}%
\pgfpathlineto{\pgfqpoint{6.589859in}{1.397677in}}%
\pgfpathlineto{\pgfqpoint{6.594819in}{1.399261in}}%
\pgfpathlineto{\pgfqpoint{6.598540in}{1.399479in}}%
\pgfpathlineto{\pgfqpoint{6.602880in}{1.396889in}}%
\pgfpathlineto{\pgfqpoint{6.605981in}{1.395961in}}%
\pgfpathlineto{\pgfqpoint{6.608461in}{1.397673in}}%
\pgfpathlineto{\pgfqpoint{6.612801in}{1.401257in}}%
\pgfpathlineto{\pgfqpoint{6.615282in}{1.400337in}}%
\pgfpathlineto{\pgfqpoint{6.617762in}{1.399564in}}%
\pgfpathlineto{\pgfqpoint{6.620242in}{1.401382in}}%
\pgfpathlineto{\pgfqpoint{6.623342in}{1.403061in}}%
\pgfpathlineto{\pgfqpoint{6.625823in}{1.401079in}}%
\pgfpathlineto{\pgfqpoint{6.629543in}{1.398489in}}%
\pgfpathlineto{\pgfqpoint{6.632643in}{1.399740in}}%
\pgfpathlineto{\pgfqpoint{6.636364in}{1.401073in}}%
\pgfpathlineto{\pgfqpoint{6.645665in}{1.400609in}}%
\pgfpathlineto{\pgfqpoint{6.652485in}{1.403454in}}%
\pgfpathlineto{\pgfqpoint{6.656206in}{1.402207in}}%
\pgfpathlineto{\pgfqpoint{6.663026in}{1.399608in}}%
\pgfpathlineto{\pgfqpoint{6.667987in}{1.401023in}}%
\pgfpathlineto{\pgfqpoint{6.672947in}{1.402115in}}%
\pgfpathlineto{\pgfqpoint{6.677908in}{1.400552in}}%
\pgfpathlineto{\pgfqpoint{6.682868in}{1.399469in}}%
\pgfpathlineto{\pgfqpoint{6.687209in}{1.400887in}}%
\pgfpathlineto{\pgfqpoint{6.694029in}{1.403138in}}%
\pgfpathlineto{\pgfqpoint{6.698370in}{1.401829in}}%
\pgfpathlineto{\pgfqpoint{6.705191in}{1.399793in}}%
\pgfpathlineto{\pgfqpoint{6.710151in}{1.401303in}}%
\pgfpathlineto{\pgfqpoint{6.715732in}{1.402570in}}%
\pgfpathlineto{\pgfqpoint{6.721312in}{1.400774in}}%
\pgfpathlineto{\pgfqpoint{6.726273in}{1.400011in}}%
\pgfpathlineto{\pgfqpoint{6.731233in}{1.401970in}}%
\pgfpathlineto{\pgfqpoint{6.736814in}{1.403721in}}%
\pgfpathlineto{\pgfqpoint{6.741154in}{1.402414in}}%
\pgfpathlineto{\pgfqpoint{6.747355in}{1.400260in}}%
\pgfpathlineto{\pgfqpoint{6.752315in}{1.401459in}}%
\pgfpathlineto{\pgfqpoint{6.757276in}{1.402097in}}%
\pgfpathlineto{\pgfqpoint{6.765957in}{1.401123in}}%
\pgfpathlineto{\pgfqpoint{6.772777in}{1.404156in}}%
\pgfpathlineto{\pgfqpoint{6.775878in}{1.401066in}}%
\pgfpathlineto{\pgfqpoint{6.778978in}{1.398888in}}%
\pgfpathlineto{\pgfqpoint{6.785179in}{1.398369in}}%
\pgfpathlineto{\pgfqpoint{6.788899in}{1.395699in}}%
\pgfpathlineto{\pgfqpoint{6.791379in}{1.396896in}}%
\pgfpathlineto{\pgfqpoint{6.796340in}{1.400256in}}%
\pgfpathlineto{\pgfqpoint{6.799440in}{1.399564in}}%
\pgfpathlineto{\pgfqpoint{6.805021in}{1.397874in}}%
\pgfpathlineto{\pgfqpoint{6.815562in}{1.397645in}}%
\pgfpathlineto{\pgfqpoint{6.822382in}{1.396039in}}%
\pgfpathlineto{\pgfqpoint{6.827343in}{1.397774in}}%
\pgfpathlineto{\pgfqpoint{6.833543in}{1.399429in}}%
\pgfpathlineto{\pgfqpoint{6.839744in}{1.397934in}}%
\pgfpathlineto{\pgfqpoint{6.844704in}{1.397626in}}%
\pgfpathlineto{\pgfqpoint{6.855866in}{1.398634in}}%
\pgfpathlineto{\pgfqpoint{6.865167in}{1.396974in}}%
\pgfpathlineto{\pgfqpoint{6.879428in}{1.398968in}}%
\pgfpathlineto{\pgfqpoint{6.886249in}{1.398242in}}%
\pgfpathlineto{\pgfqpoint{6.896170in}{1.398662in}}%
\pgfpathlineto{\pgfqpoint{6.905471in}{1.396723in}}%
\pgfpathlineto{\pgfqpoint{6.925933in}{1.400534in}}%
\pgfpathlineto{\pgfqpoint{6.933373in}{1.398036in}}%
\pgfpathlineto{\pgfqpoint{6.937714in}{1.399107in}}%
\pgfpathlineto{\pgfqpoint{6.942054in}{1.399422in}}%
\pgfpathlineto{\pgfqpoint{6.950735in}{1.403193in}}%
\pgfpathlineto{\pgfqpoint{6.960656in}{1.401151in}}%
\pgfpathlineto{\pgfqpoint{6.968097in}{1.401303in}}%
\pgfpathlineto{\pgfqpoint{6.977398in}{1.400155in}}%
\pgfpathlineto{\pgfqpoint{6.989179in}{1.402331in}}%
\pgfpathlineto{\pgfqpoint{6.999100in}{1.400544in}}%
\pgfpathlineto{\pgfqpoint{7.010881in}{1.401736in}}%
\pgfpathlineto{\pgfqpoint{7.018942in}{1.400081in}}%
\pgfpathlineto{\pgfqpoint{7.024522in}{1.402031in}}%
\pgfpathlineto{\pgfqpoint{7.030103in}{1.403231in}}%
\pgfpathlineto{\pgfqpoint{7.035684in}{1.401554in}}%
\pgfpathlineto{\pgfqpoint{7.041264in}{1.400640in}}%
\pgfpathlineto{\pgfqpoint{7.053665in}{1.401657in}}%
\pgfpathlineto{\pgfqpoint{7.059866in}{1.400169in}}%
\pgfpathlineto{\pgfqpoint{7.064206in}{1.402206in}}%
\pgfpathlineto{\pgfqpoint{7.069167in}{1.404544in}}%
\pgfpathlineto{\pgfqpoint{7.072267in}{1.403496in}}%
\pgfpathlineto{\pgfqpoint{7.078468in}{1.400340in}}%
\pgfpathlineto{\pgfqpoint{7.084048in}{1.399959in}}%
\pgfpathlineto{\pgfqpoint{7.089009in}{1.396694in}}%
\pgfpathlineto{\pgfqpoint{7.092109in}{1.399521in}}%
\pgfpathlineto{\pgfqpoint{7.094589in}{1.400821in}}%
\pgfpathlineto{\pgfqpoint{7.097690in}{1.399503in}}%
\pgfpathlineto{\pgfqpoint{7.100790in}{1.398825in}}%
\pgfpathlineto{\pgfqpoint{7.107611in}{1.400049in}}%
\pgfpathlineto{\pgfqpoint{7.118152in}{1.395611in}}%
\pgfpathlineto{\pgfqpoint{7.128073in}{1.399239in}}%
\pgfpathlineto{\pgfqpoint{7.138614in}{1.397329in}}%
\pgfpathlineto{\pgfqpoint{7.147295in}{1.400118in}}%
\pgfpathlineto{\pgfqpoint{7.151635in}{1.398085in}}%
\pgfpathlineto{\pgfqpoint{7.157216in}{1.395791in}}%
\pgfpathlineto{\pgfqpoint{7.160936in}{1.396872in}}%
\pgfpathlineto{\pgfqpoint{7.168997in}{1.400320in}}%
\pgfpathlineto{\pgfqpoint{7.173337in}{1.398741in}}%
\pgfpathlineto{\pgfqpoint{7.178918in}{1.396754in}}%
\pgfpathlineto{\pgfqpoint{7.182638in}{1.398031in}}%
\pgfpathlineto{\pgfqpoint{7.188839in}{1.400744in}}%
\pgfpathlineto{\pgfqpoint{7.192559in}{1.399532in}}%
\pgfpathlineto{\pgfqpoint{7.200000in}{1.395729in}}%
\pgfpathlineto{\pgfqpoint{7.200000in}{1.395729in}}%
\pgfusepath{stroke}%
\end{pgfscope}%
\begin{pgfscope}%
\pgfsetbuttcap%
\pgfsetroundjoin%
\definecolor{currentfill}{rgb}{0.000000,0.000000,0.000000}%
\pgfsetfillcolor{currentfill}%
\pgfsetlinewidth{0.501875pt}%
\definecolor{currentstroke}{rgb}{0.000000,0.000000,0.000000}%
\pgfsetstrokecolor{currentstroke}%
\pgfsetdash{}{0pt}%
\pgfsys@defobject{currentmarker}{\pgfqpoint{0.000000in}{0.000000in}}{\pgfqpoint{0.000000in}{0.055556in}}{%
\pgfpathmoveto{\pgfqpoint{0.000000in}{0.000000in}}%
\pgfpathlineto{\pgfqpoint{0.000000in}{0.055556in}}%
\pgfusepath{stroke,fill}%
}%
\begin{pgfscope}%
\pgfsys@transformshift{1.000000in}{0.600000in}%
\pgfsys@useobject{currentmarker}{}%
\end{pgfscope}%
\end{pgfscope}%
\begin{pgfscope}%
\pgfsetbuttcap%
\pgfsetroundjoin%
\definecolor{currentfill}{rgb}{0.000000,0.000000,0.000000}%
\pgfsetfillcolor{currentfill}%
\pgfsetlinewidth{0.501875pt}%
\definecolor{currentstroke}{rgb}{0.000000,0.000000,0.000000}%
\pgfsetstrokecolor{currentstroke}%
\pgfsetdash{}{0pt}%
\pgfsys@defobject{currentmarker}{\pgfqpoint{0.000000in}{-0.055556in}}{\pgfqpoint{0.000000in}{0.000000in}}{%
\pgfpathmoveto{\pgfqpoint{0.000000in}{0.000000in}}%
\pgfpathlineto{\pgfqpoint{0.000000in}{-0.055556in}}%
\pgfusepath{stroke,fill}%
}%
\begin{pgfscope}%
\pgfsys@transformshift{1.000000in}{5.400000in}%
\pgfsys@useobject{currentmarker}{}%
\end{pgfscope}%
\end{pgfscope}%
\begin{pgfscope}%
\pgftext[x=1.000000in,y=0.544444in,,top]{{\rmfamily\fontsize{12.000000}{14.400000}\selectfont 0}}%
\end{pgfscope}%
\begin{pgfscope}%
\pgfsetbuttcap%
\pgfsetroundjoin%
\definecolor{currentfill}{rgb}{0.000000,0.000000,0.000000}%
\pgfsetfillcolor{currentfill}%
\pgfsetlinewidth{0.501875pt}%
\definecolor{currentstroke}{rgb}{0.000000,0.000000,0.000000}%
\pgfsetstrokecolor{currentstroke}%
\pgfsetdash{}{0pt}%
\pgfsys@defobject{currentmarker}{\pgfqpoint{0.000000in}{0.000000in}}{\pgfqpoint{0.000000in}{0.055556in}}{%
\pgfpathmoveto{\pgfqpoint{0.000000in}{0.000000in}}%
\pgfpathlineto{\pgfqpoint{0.000000in}{0.055556in}}%
\pgfusepath{stroke,fill}%
}%
\begin{pgfscope}%
\pgfsys@transformshift{2.240000in}{0.600000in}%
\pgfsys@useobject{currentmarker}{}%
\end{pgfscope}%
\end{pgfscope}%
\begin{pgfscope}%
\pgfsetbuttcap%
\pgfsetroundjoin%
\definecolor{currentfill}{rgb}{0.000000,0.000000,0.000000}%
\pgfsetfillcolor{currentfill}%
\pgfsetlinewidth{0.501875pt}%
\definecolor{currentstroke}{rgb}{0.000000,0.000000,0.000000}%
\pgfsetstrokecolor{currentstroke}%
\pgfsetdash{}{0pt}%
\pgfsys@defobject{currentmarker}{\pgfqpoint{0.000000in}{-0.055556in}}{\pgfqpoint{0.000000in}{0.000000in}}{%
\pgfpathmoveto{\pgfqpoint{0.000000in}{0.000000in}}%
\pgfpathlineto{\pgfqpoint{0.000000in}{-0.055556in}}%
\pgfusepath{stroke,fill}%
}%
\begin{pgfscope}%
\pgfsys@transformshift{2.240000in}{5.400000in}%
\pgfsys@useobject{currentmarker}{}%
\end{pgfscope}%
\end{pgfscope}%
\begin{pgfscope}%
\pgftext[x=2.240000in,y=0.544444in,,top]{{\rmfamily\fontsize{12.000000}{14.400000}\selectfont 2}}%
\end{pgfscope}%
\begin{pgfscope}%
\pgfsetbuttcap%
\pgfsetroundjoin%
\definecolor{currentfill}{rgb}{0.000000,0.000000,0.000000}%
\pgfsetfillcolor{currentfill}%
\pgfsetlinewidth{0.501875pt}%
\definecolor{currentstroke}{rgb}{0.000000,0.000000,0.000000}%
\pgfsetstrokecolor{currentstroke}%
\pgfsetdash{}{0pt}%
\pgfsys@defobject{currentmarker}{\pgfqpoint{0.000000in}{0.000000in}}{\pgfqpoint{0.000000in}{0.055556in}}{%
\pgfpathmoveto{\pgfqpoint{0.000000in}{0.000000in}}%
\pgfpathlineto{\pgfqpoint{0.000000in}{0.055556in}}%
\pgfusepath{stroke,fill}%
}%
\begin{pgfscope}%
\pgfsys@transformshift{3.480000in}{0.600000in}%
\pgfsys@useobject{currentmarker}{}%
\end{pgfscope}%
\end{pgfscope}%
\begin{pgfscope}%
\pgfsetbuttcap%
\pgfsetroundjoin%
\definecolor{currentfill}{rgb}{0.000000,0.000000,0.000000}%
\pgfsetfillcolor{currentfill}%
\pgfsetlinewidth{0.501875pt}%
\definecolor{currentstroke}{rgb}{0.000000,0.000000,0.000000}%
\pgfsetstrokecolor{currentstroke}%
\pgfsetdash{}{0pt}%
\pgfsys@defobject{currentmarker}{\pgfqpoint{0.000000in}{-0.055556in}}{\pgfqpoint{0.000000in}{0.000000in}}{%
\pgfpathmoveto{\pgfqpoint{0.000000in}{0.000000in}}%
\pgfpathlineto{\pgfqpoint{0.000000in}{-0.055556in}}%
\pgfusepath{stroke,fill}%
}%
\begin{pgfscope}%
\pgfsys@transformshift{3.480000in}{5.400000in}%
\pgfsys@useobject{currentmarker}{}%
\end{pgfscope}%
\end{pgfscope}%
\begin{pgfscope}%
\pgftext[x=3.480000in,y=0.544444in,,top]{{\rmfamily\fontsize{12.000000}{14.400000}\selectfont 4}}%
\end{pgfscope}%
\begin{pgfscope}%
\pgfsetbuttcap%
\pgfsetroundjoin%
\definecolor{currentfill}{rgb}{0.000000,0.000000,0.000000}%
\pgfsetfillcolor{currentfill}%
\pgfsetlinewidth{0.501875pt}%
\definecolor{currentstroke}{rgb}{0.000000,0.000000,0.000000}%
\pgfsetstrokecolor{currentstroke}%
\pgfsetdash{}{0pt}%
\pgfsys@defobject{currentmarker}{\pgfqpoint{0.000000in}{0.000000in}}{\pgfqpoint{0.000000in}{0.055556in}}{%
\pgfpathmoveto{\pgfqpoint{0.000000in}{0.000000in}}%
\pgfpathlineto{\pgfqpoint{0.000000in}{0.055556in}}%
\pgfusepath{stroke,fill}%
}%
\begin{pgfscope}%
\pgfsys@transformshift{4.720000in}{0.600000in}%
\pgfsys@useobject{currentmarker}{}%
\end{pgfscope}%
\end{pgfscope}%
\begin{pgfscope}%
\pgfsetbuttcap%
\pgfsetroundjoin%
\definecolor{currentfill}{rgb}{0.000000,0.000000,0.000000}%
\pgfsetfillcolor{currentfill}%
\pgfsetlinewidth{0.501875pt}%
\definecolor{currentstroke}{rgb}{0.000000,0.000000,0.000000}%
\pgfsetstrokecolor{currentstroke}%
\pgfsetdash{}{0pt}%
\pgfsys@defobject{currentmarker}{\pgfqpoint{0.000000in}{-0.055556in}}{\pgfqpoint{0.000000in}{0.000000in}}{%
\pgfpathmoveto{\pgfqpoint{0.000000in}{0.000000in}}%
\pgfpathlineto{\pgfqpoint{0.000000in}{-0.055556in}}%
\pgfusepath{stroke,fill}%
}%
\begin{pgfscope}%
\pgfsys@transformshift{4.720000in}{5.400000in}%
\pgfsys@useobject{currentmarker}{}%
\end{pgfscope}%
\end{pgfscope}%
\begin{pgfscope}%
\pgftext[x=4.720000in,y=0.544444in,,top]{{\rmfamily\fontsize{12.000000}{14.400000}\selectfont 6}}%
\end{pgfscope}%
\begin{pgfscope}%
\pgfsetbuttcap%
\pgfsetroundjoin%
\definecolor{currentfill}{rgb}{0.000000,0.000000,0.000000}%
\pgfsetfillcolor{currentfill}%
\pgfsetlinewidth{0.501875pt}%
\definecolor{currentstroke}{rgb}{0.000000,0.000000,0.000000}%
\pgfsetstrokecolor{currentstroke}%
\pgfsetdash{}{0pt}%
\pgfsys@defobject{currentmarker}{\pgfqpoint{0.000000in}{0.000000in}}{\pgfqpoint{0.000000in}{0.055556in}}{%
\pgfpathmoveto{\pgfqpoint{0.000000in}{0.000000in}}%
\pgfpathlineto{\pgfqpoint{0.000000in}{0.055556in}}%
\pgfusepath{stroke,fill}%
}%
\begin{pgfscope}%
\pgfsys@transformshift{5.960000in}{0.600000in}%
\pgfsys@useobject{currentmarker}{}%
\end{pgfscope}%
\end{pgfscope}%
\begin{pgfscope}%
\pgfsetbuttcap%
\pgfsetroundjoin%
\definecolor{currentfill}{rgb}{0.000000,0.000000,0.000000}%
\pgfsetfillcolor{currentfill}%
\pgfsetlinewidth{0.501875pt}%
\definecolor{currentstroke}{rgb}{0.000000,0.000000,0.000000}%
\pgfsetstrokecolor{currentstroke}%
\pgfsetdash{}{0pt}%
\pgfsys@defobject{currentmarker}{\pgfqpoint{0.000000in}{-0.055556in}}{\pgfqpoint{0.000000in}{0.000000in}}{%
\pgfpathmoveto{\pgfqpoint{0.000000in}{0.000000in}}%
\pgfpathlineto{\pgfqpoint{0.000000in}{-0.055556in}}%
\pgfusepath{stroke,fill}%
}%
\begin{pgfscope}%
\pgfsys@transformshift{5.960000in}{5.400000in}%
\pgfsys@useobject{currentmarker}{}%
\end{pgfscope}%
\end{pgfscope}%
\begin{pgfscope}%
\pgftext[x=5.960000in,y=0.544444in,,top]{{\rmfamily\fontsize{12.000000}{14.400000}\selectfont 8}}%
\end{pgfscope}%
\begin{pgfscope}%
\pgfsetbuttcap%
\pgfsetroundjoin%
\definecolor{currentfill}{rgb}{0.000000,0.000000,0.000000}%
\pgfsetfillcolor{currentfill}%
\pgfsetlinewidth{0.501875pt}%
\definecolor{currentstroke}{rgb}{0.000000,0.000000,0.000000}%
\pgfsetstrokecolor{currentstroke}%
\pgfsetdash{}{0pt}%
\pgfsys@defobject{currentmarker}{\pgfqpoint{0.000000in}{0.000000in}}{\pgfqpoint{0.000000in}{0.055556in}}{%
\pgfpathmoveto{\pgfqpoint{0.000000in}{0.000000in}}%
\pgfpathlineto{\pgfqpoint{0.000000in}{0.055556in}}%
\pgfusepath{stroke,fill}%
}%
\begin{pgfscope}%
\pgfsys@transformshift{7.200000in}{0.600000in}%
\pgfsys@useobject{currentmarker}{}%
\end{pgfscope}%
\end{pgfscope}%
\begin{pgfscope}%
\pgfsetbuttcap%
\pgfsetroundjoin%
\definecolor{currentfill}{rgb}{0.000000,0.000000,0.000000}%
\pgfsetfillcolor{currentfill}%
\pgfsetlinewidth{0.501875pt}%
\definecolor{currentstroke}{rgb}{0.000000,0.000000,0.000000}%
\pgfsetstrokecolor{currentstroke}%
\pgfsetdash{}{0pt}%
\pgfsys@defobject{currentmarker}{\pgfqpoint{0.000000in}{-0.055556in}}{\pgfqpoint{0.000000in}{0.000000in}}{%
\pgfpathmoveto{\pgfqpoint{0.000000in}{0.000000in}}%
\pgfpathlineto{\pgfqpoint{0.000000in}{-0.055556in}}%
\pgfusepath{stroke,fill}%
}%
\begin{pgfscope}%
\pgfsys@transformshift{7.200000in}{5.400000in}%
\pgfsys@useobject{currentmarker}{}%
\end{pgfscope}%
\end{pgfscope}%
\begin{pgfscope}%
\pgftext[x=7.200000in,y=0.544444in,,top]{{\rmfamily\fontsize{12.000000}{14.400000}\selectfont 10}}%
\end{pgfscope}%
\begin{pgfscope}%
\pgftext[x=4.100000in,y=0.327000in,,top]{{\rmfamily\fontsize{12.000000}{14.400000}\selectfont \(\displaystyle t\) in s}}%
\end{pgfscope}%
\begin{pgfscope}%
\pgfsetbuttcap%
\pgfsetroundjoin%
\definecolor{currentfill}{rgb}{0.000000,0.000000,0.000000}%
\pgfsetfillcolor{currentfill}%
\pgfsetlinewidth{0.501875pt}%
\definecolor{currentstroke}{rgb}{0.000000,0.000000,0.000000}%
\pgfsetstrokecolor{currentstroke}%
\pgfsetdash{}{0pt}%
\pgfsys@defobject{currentmarker}{\pgfqpoint{0.000000in}{0.000000in}}{\pgfqpoint{0.055556in}{0.000000in}}{%
\pgfpathmoveto{\pgfqpoint{0.000000in}{0.000000in}}%
\pgfpathlineto{\pgfqpoint{0.055556in}{0.000000in}}%
\pgfusepath{stroke,fill}%
}%
\begin{pgfscope}%
\pgfsys@transformshift{1.000000in}{0.600000in}%
\pgfsys@useobject{currentmarker}{}%
\end{pgfscope}%
\end{pgfscope}%
\begin{pgfscope}%
\pgfsetbuttcap%
\pgfsetroundjoin%
\definecolor{currentfill}{rgb}{0.000000,0.000000,0.000000}%
\pgfsetfillcolor{currentfill}%
\pgfsetlinewidth{0.501875pt}%
\definecolor{currentstroke}{rgb}{0.000000,0.000000,0.000000}%
\pgfsetstrokecolor{currentstroke}%
\pgfsetdash{}{0pt}%
\pgfsys@defobject{currentmarker}{\pgfqpoint{-0.055556in}{0.000000in}}{\pgfqpoint{0.000000in}{0.000000in}}{%
\pgfpathmoveto{\pgfqpoint{0.000000in}{0.000000in}}%
\pgfpathlineto{\pgfqpoint{-0.055556in}{0.000000in}}%
\pgfusepath{stroke,fill}%
}%
\begin{pgfscope}%
\pgfsys@transformshift{7.200000in}{0.600000in}%
\pgfsys@useobject{currentmarker}{}%
\end{pgfscope}%
\end{pgfscope}%
\begin{pgfscope}%
\pgftext[x=0.944444in,y=0.600000in,right,]{{\rmfamily\fontsize{12.000000}{14.400000}\selectfont −500}}%
\end{pgfscope}%
\begin{pgfscope}%
\pgfsetbuttcap%
\pgfsetroundjoin%
\definecolor{currentfill}{rgb}{0.000000,0.000000,0.000000}%
\pgfsetfillcolor{currentfill}%
\pgfsetlinewidth{0.501875pt}%
\definecolor{currentstroke}{rgb}{0.000000,0.000000,0.000000}%
\pgfsetstrokecolor{currentstroke}%
\pgfsetdash{}{0pt}%
\pgfsys@defobject{currentmarker}{\pgfqpoint{0.000000in}{0.000000in}}{\pgfqpoint{0.055556in}{0.000000in}}{%
\pgfpathmoveto{\pgfqpoint{0.000000in}{0.000000in}}%
\pgfpathlineto{\pgfqpoint{0.055556in}{0.000000in}}%
\pgfusepath{stroke,fill}%
}%
\begin{pgfscope}%
\pgfsys@transformshift{1.000000in}{1.400000in}%
\pgfsys@useobject{currentmarker}{}%
\end{pgfscope}%
\end{pgfscope}%
\begin{pgfscope}%
\pgfsetbuttcap%
\pgfsetroundjoin%
\definecolor{currentfill}{rgb}{0.000000,0.000000,0.000000}%
\pgfsetfillcolor{currentfill}%
\pgfsetlinewidth{0.501875pt}%
\definecolor{currentstroke}{rgb}{0.000000,0.000000,0.000000}%
\pgfsetstrokecolor{currentstroke}%
\pgfsetdash{}{0pt}%
\pgfsys@defobject{currentmarker}{\pgfqpoint{-0.055556in}{0.000000in}}{\pgfqpoint{0.000000in}{0.000000in}}{%
\pgfpathmoveto{\pgfqpoint{0.000000in}{0.000000in}}%
\pgfpathlineto{\pgfqpoint{-0.055556in}{0.000000in}}%
\pgfusepath{stroke,fill}%
}%
\begin{pgfscope}%
\pgfsys@transformshift{7.200000in}{1.400000in}%
\pgfsys@useobject{currentmarker}{}%
\end{pgfscope}%
\end{pgfscope}%
\begin{pgfscope}%
\pgftext[x=0.944444in,y=1.400000in,right,]{{\rmfamily\fontsize{12.000000}{14.400000}\selectfont 0}}%
\end{pgfscope}%
\begin{pgfscope}%
\pgfsetbuttcap%
\pgfsetroundjoin%
\definecolor{currentfill}{rgb}{0.000000,0.000000,0.000000}%
\pgfsetfillcolor{currentfill}%
\pgfsetlinewidth{0.501875pt}%
\definecolor{currentstroke}{rgb}{0.000000,0.000000,0.000000}%
\pgfsetstrokecolor{currentstroke}%
\pgfsetdash{}{0pt}%
\pgfsys@defobject{currentmarker}{\pgfqpoint{0.000000in}{0.000000in}}{\pgfqpoint{0.055556in}{0.000000in}}{%
\pgfpathmoveto{\pgfqpoint{0.000000in}{0.000000in}}%
\pgfpathlineto{\pgfqpoint{0.055556in}{0.000000in}}%
\pgfusepath{stroke,fill}%
}%
\begin{pgfscope}%
\pgfsys@transformshift{1.000000in}{2.200000in}%
\pgfsys@useobject{currentmarker}{}%
\end{pgfscope}%
\end{pgfscope}%
\begin{pgfscope}%
\pgfsetbuttcap%
\pgfsetroundjoin%
\definecolor{currentfill}{rgb}{0.000000,0.000000,0.000000}%
\pgfsetfillcolor{currentfill}%
\pgfsetlinewidth{0.501875pt}%
\definecolor{currentstroke}{rgb}{0.000000,0.000000,0.000000}%
\pgfsetstrokecolor{currentstroke}%
\pgfsetdash{}{0pt}%
\pgfsys@defobject{currentmarker}{\pgfqpoint{-0.055556in}{0.000000in}}{\pgfqpoint{0.000000in}{0.000000in}}{%
\pgfpathmoveto{\pgfqpoint{0.000000in}{0.000000in}}%
\pgfpathlineto{\pgfqpoint{-0.055556in}{0.000000in}}%
\pgfusepath{stroke,fill}%
}%
\begin{pgfscope}%
\pgfsys@transformshift{7.200000in}{2.200000in}%
\pgfsys@useobject{currentmarker}{}%
\end{pgfscope}%
\end{pgfscope}%
\begin{pgfscope}%
\pgftext[x=0.944444in,y=2.200000in,right,]{{\rmfamily\fontsize{12.000000}{14.400000}\selectfont 500}}%
\end{pgfscope}%
\begin{pgfscope}%
\pgfsetbuttcap%
\pgfsetroundjoin%
\definecolor{currentfill}{rgb}{0.000000,0.000000,0.000000}%
\pgfsetfillcolor{currentfill}%
\pgfsetlinewidth{0.501875pt}%
\definecolor{currentstroke}{rgb}{0.000000,0.000000,0.000000}%
\pgfsetstrokecolor{currentstroke}%
\pgfsetdash{}{0pt}%
\pgfsys@defobject{currentmarker}{\pgfqpoint{0.000000in}{0.000000in}}{\pgfqpoint{0.055556in}{0.000000in}}{%
\pgfpathmoveto{\pgfqpoint{0.000000in}{0.000000in}}%
\pgfpathlineto{\pgfqpoint{0.055556in}{0.000000in}}%
\pgfusepath{stroke,fill}%
}%
\begin{pgfscope}%
\pgfsys@transformshift{1.000000in}{3.000000in}%
\pgfsys@useobject{currentmarker}{}%
\end{pgfscope}%
\end{pgfscope}%
\begin{pgfscope}%
\pgfsetbuttcap%
\pgfsetroundjoin%
\definecolor{currentfill}{rgb}{0.000000,0.000000,0.000000}%
\pgfsetfillcolor{currentfill}%
\pgfsetlinewidth{0.501875pt}%
\definecolor{currentstroke}{rgb}{0.000000,0.000000,0.000000}%
\pgfsetstrokecolor{currentstroke}%
\pgfsetdash{}{0pt}%
\pgfsys@defobject{currentmarker}{\pgfqpoint{-0.055556in}{0.000000in}}{\pgfqpoint{0.000000in}{0.000000in}}{%
\pgfpathmoveto{\pgfqpoint{0.000000in}{0.000000in}}%
\pgfpathlineto{\pgfqpoint{-0.055556in}{0.000000in}}%
\pgfusepath{stroke,fill}%
}%
\begin{pgfscope}%
\pgfsys@transformshift{7.200000in}{3.000000in}%
\pgfsys@useobject{currentmarker}{}%
\end{pgfscope}%
\end{pgfscope}%
\begin{pgfscope}%
\pgftext[x=0.944444in,y=3.000000in,right,]{{\rmfamily\fontsize{12.000000}{14.400000}\selectfont 1000}}%
\end{pgfscope}%
\begin{pgfscope}%
\pgfsetbuttcap%
\pgfsetroundjoin%
\definecolor{currentfill}{rgb}{0.000000,0.000000,0.000000}%
\pgfsetfillcolor{currentfill}%
\pgfsetlinewidth{0.501875pt}%
\definecolor{currentstroke}{rgb}{0.000000,0.000000,0.000000}%
\pgfsetstrokecolor{currentstroke}%
\pgfsetdash{}{0pt}%
\pgfsys@defobject{currentmarker}{\pgfqpoint{0.000000in}{0.000000in}}{\pgfqpoint{0.055556in}{0.000000in}}{%
\pgfpathmoveto{\pgfqpoint{0.000000in}{0.000000in}}%
\pgfpathlineto{\pgfqpoint{0.055556in}{0.000000in}}%
\pgfusepath{stroke,fill}%
}%
\begin{pgfscope}%
\pgfsys@transformshift{1.000000in}{3.800000in}%
\pgfsys@useobject{currentmarker}{}%
\end{pgfscope}%
\end{pgfscope}%
\begin{pgfscope}%
\pgfsetbuttcap%
\pgfsetroundjoin%
\definecolor{currentfill}{rgb}{0.000000,0.000000,0.000000}%
\pgfsetfillcolor{currentfill}%
\pgfsetlinewidth{0.501875pt}%
\definecolor{currentstroke}{rgb}{0.000000,0.000000,0.000000}%
\pgfsetstrokecolor{currentstroke}%
\pgfsetdash{}{0pt}%
\pgfsys@defobject{currentmarker}{\pgfqpoint{-0.055556in}{0.000000in}}{\pgfqpoint{0.000000in}{0.000000in}}{%
\pgfpathmoveto{\pgfqpoint{0.000000in}{0.000000in}}%
\pgfpathlineto{\pgfqpoint{-0.055556in}{0.000000in}}%
\pgfusepath{stroke,fill}%
}%
\begin{pgfscope}%
\pgfsys@transformshift{7.200000in}{3.800000in}%
\pgfsys@useobject{currentmarker}{}%
\end{pgfscope}%
\end{pgfscope}%
\begin{pgfscope}%
\pgftext[x=0.944444in,y=3.800000in,right,]{{\rmfamily\fontsize{12.000000}{14.400000}\selectfont 1500}}%
\end{pgfscope}%
\begin{pgfscope}%
\pgfsetbuttcap%
\pgfsetroundjoin%
\definecolor{currentfill}{rgb}{0.000000,0.000000,0.000000}%
\pgfsetfillcolor{currentfill}%
\pgfsetlinewidth{0.501875pt}%
\definecolor{currentstroke}{rgb}{0.000000,0.000000,0.000000}%
\pgfsetstrokecolor{currentstroke}%
\pgfsetdash{}{0pt}%
\pgfsys@defobject{currentmarker}{\pgfqpoint{0.000000in}{0.000000in}}{\pgfqpoint{0.055556in}{0.000000in}}{%
\pgfpathmoveto{\pgfqpoint{0.000000in}{0.000000in}}%
\pgfpathlineto{\pgfqpoint{0.055556in}{0.000000in}}%
\pgfusepath{stroke,fill}%
}%
\begin{pgfscope}%
\pgfsys@transformshift{1.000000in}{4.600000in}%
\pgfsys@useobject{currentmarker}{}%
\end{pgfscope}%
\end{pgfscope}%
\begin{pgfscope}%
\pgfsetbuttcap%
\pgfsetroundjoin%
\definecolor{currentfill}{rgb}{0.000000,0.000000,0.000000}%
\pgfsetfillcolor{currentfill}%
\pgfsetlinewidth{0.501875pt}%
\definecolor{currentstroke}{rgb}{0.000000,0.000000,0.000000}%
\pgfsetstrokecolor{currentstroke}%
\pgfsetdash{}{0pt}%
\pgfsys@defobject{currentmarker}{\pgfqpoint{-0.055556in}{0.000000in}}{\pgfqpoint{0.000000in}{0.000000in}}{%
\pgfpathmoveto{\pgfqpoint{0.000000in}{0.000000in}}%
\pgfpathlineto{\pgfqpoint{-0.055556in}{0.000000in}}%
\pgfusepath{stroke,fill}%
}%
\begin{pgfscope}%
\pgfsys@transformshift{7.200000in}{4.600000in}%
\pgfsys@useobject{currentmarker}{}%
\end{pgfscope}%
\end{pgfscope}%
\begin{pgfscope}%
\pgftext[x=0.944444in,y=4.600000in,right,]{{\rmfamily\fontsize{12.000000}{14.400000}\selectfont 2000}}%
\end{pgfscope}%
\begin{pgfscope}%
\pgfsetbuttcap%
\pgfsetroundjoin%
\definecolor{currentfill}{rgb}{0.000000,0.000000,0.000000}%
\pgfsetfillcolor{currentfill}%
\pgfsetlinewidth{0.501875pt}%
\definecolor{currentstroke}{rgb}{0.000000,0.000000,0.000000}%
\pgfsetstrokecolor{currentstroke}%
\pgfsetdash{}{0pt}%
\pgfsys@defobject{currentmarker}{\pgfqpoint{0.000000in}{0.000000in}}{\pgfqpoint{0.055556in}{0.000000in}}{%
\pgfpathmoveto{\pgfqpoint{0.000000in}{0.000000in}}%
\pgfpathlineto{\pgfqpoint{0.055556in}{0.000000in}}%
\pgfusepath{stroke,fill}%
}%
\begin{pgfscope}%
\pgfsys@transformshift{1.000000in}{5.400000in}%
\pgfsys@useobject{currentmarker}{}%
\end{pgfscope}%
\end{pgfscope}%
\begin{pgfscope}%
\pgfsetbuttcap%
\pgfsetroundjoin%
\definecolor{currentfill}{rgb}{0.000000,0.000000,0.000000}%
\pgfsetfillcolor{currentfill}%
\pgfsetlinewidth{0.501875pt}%
\definecolor{currentstroke}{rgb}{0.000000,0.000000,0.000000}%
\pgfsetstrokecolor{currentstroke}%
\pgfsetdash{}{0pt}%
\pgfsys@defobject{currentmarker}{\pgfqpoint{-0.055556in}{0.000000in}}{\pgfqpoint{0.000000in}{0.000000in}}{%
\pgfpathmoveto{\pgfqpoint{0.000000in}{0.000000in}}%
\pgfpathlineto{\pgfqpoint{-0.055556in}{0.000000in}}%
\pgfusepath{stroke,fill}%
}%
\begin{pgfscope}%
\pgfsys@transformshift{7.200000in}{5.400000in}%
\pgfsys@useobject{currentmarker}{}%
\end{pgfscope}%
\end{pgfscope}%
\begin{pgfscope}%
\pgftext[x=0.944444in,y=5.400000in,right,]{{\rmfamily\fontsize{12.000000}{14.400000}\selectfont 2500}}%
\end{pgfscope}%
\begin{pgfscope}%
\pgftext[x=0.503000in,y=3.000000in,,bottom,rotate=90.000000]{{\rmfamily\fontsize{12.000000}{14.400000}\selectfont \(\displaystyle q\) in \(\displaystyle ^{\circ}\)}}%
\end{pgfscope}%
\begin{pgfscope}%
\pgfsetbuttcap%
\pgfsetroundjoin%
\pgfsetlinewidth{1.003750pt}%
\definecolor{currentstroke}{rgb}{0.000000,0.000000,0.000000}%
\pgfsetstrokecolor{currentstroke}%
\pgfsetdash{}{0pt}%
\pgfpathmoveto{\pgfqpoint{1.000000in}{5.400000in}}%
\pgfpathlineto{\pgfqpoint{7.200000in}{5.400000in}}%
\pgfusepath{stroke}%
\end{pgfscope}%
\begin{pgfscope}%
\pgfsetbuttcap%
\pgfsetroundjoin%
\pgfsetlinewidth{1.003750pt}%
\definecolor{currentstroke}{rgb}{0.000000,0.000000,0.000000}%
\pgfsetstrokecolor{currentstroke}%
\pgfsetdash{}{0pt}%
\pgfpathmoveto{\pgfqpoint{7.200000in}{0.600000in}}%
\pgfpathlineto{\pgfqpoint{7.200000in}{5.400000in}}%
\pgfusepath{stroke}%
\end{pgfscope}%
\begin{pgfscope}%
\pgfsetbuttcap%
\pgfsetroundjoin%
\pgfsetlinewidth{1.003750pt}%
\definecolor{currentstroke}{rgb}{0.000000,0.000000,0.000000}%
\pgfsetstrokecolor{currentstroke}%
\pgfsetdash{}{0pt}%
\pgfpathmoveto{\pgfqpoint{1.000000in}{0.600000in}}%
\pgfpathlineto{\pgfqpoint{7.200000in}{0.600000in}}%
\pgfusepath{stroke}%
\end{pgfscope}%
\begin{pgfscope}%
\pgfsetbuttcap%
\pgfsetroundjoin%
\pgfsetlinewidth{1.003750pt}%
\definecolor{currentstroke}{rgb}{0.000000,0.000000,0.000000}%
\pgfsetstrokecolor{currentstroke}%
\pgfsetdash{}{0pt}%
\pgfpathmoveto{\pgfqpoint{1.000000in}{0.600000in}}%
\pgfpathlineto{\pgfqpoint{1.000000in}{5.400000in}}%
\pgfusepath{stroke}%
\end{pgfscope}%
\begin{pgfscope}%
\pgfsetbuttcap%
\pgfsetroundjoin%
\definecolor{currentfill}{rgb}{1.000000,1.000000,1.000000}%
\pgfsetfillcolor{currentfill}%
\pgfsetlinewidth{1.003750pt}%
\definecolor{currentstroke}{rgb}{0.000000,0.000000,0.000000}%
\pgfsetstrokecolor{currentstroke}%
\pgfsetdash{}{0pt}%
\pgfpathmoveto{\pgfqpoint{6.147338in}{4.124445in}}%
\pgfpathlineto{\pgfqpoint{7.100000in}{4.124445in}}%
\pgfpathlineto{\pgfqpoint{7.100000in}{5.300000in}}%
\pgfpathlineto{\pgfqpoint{6.147338in}{5.300000in}}%
\pgfpathlineto{\pgfqpoint{6.147338in}{4.124445in}}%
\pgfpathclose%
\pgfusepath{stroke,fill}%
\end{pgfscope}%
\begin{pgfscope}%
\pgfsetrectcap%
\pgfsetroundjoin%
\pgfsetlinewidth{1.003750pt}%
\definecolor{currentstroke}{rgb}{0.000000,0.000000,1.000000}%
\pgfsetstrokecolor{currentstroke}%
\pgfsetdash{}{0pt}%
\pgfpathmoveto{\pgfqpoint{6.287338in}{5.150000in}}%
\pgfpathlineto{\pgfqpoint{6.567338in}{5.150000in}}%
\pgfusepath{stroke}%
\end{pgfscope}%
\begin{pgfscope}%
\pgftext[x=6.787338in,y=5.080000in,left,base]{{\rmfamily\fontsize{14.400000}{17.280000}\selectfont \(\displaystyle q_{11}\)}}%
\end{pgfscope}%
\begin{pgfscope}%
\pgfsetrectcap%
\pgfsetroundjoin%
\pgfsetlinewidth{1.003750pt}%
\definecolor{currentstroke}{rgb}{0.000000,0.500000,0.000000}%
\pgfsetstrokecolor{currentstroke}%
\pgfsetdash{}{0pt}%
\pgfpathmoveto{\pgfqpoint{6.287338in}{4.871111in}}%
\pgfpathlineto{\pgfqpoint{6.567338in}{4.871111in}}%
\pgfusepath{stroke}%
\end{pgfscope}%
\begin{pgfscope}%
\pgftext[x=6.787338in,y=4.801111in,left,base]{{\rmfamily\fontsize{14.400000}{17.280000}\selectfont \(\displaystyle q_{12}\)}}%
\end{pgfscope}%
\begin{pgfscope}%
\pgfsetrectcap%
\pgfsetroundjoin%
\pgfsetlinewidth{1.003750pt}%
\definecolor{currentstroke}{rgb}{1.000000,0.000000,0.000000}%
\pgfsetstrokecolor{currentstroke}%
\pgfsetdash{}{0pt}%
\pgfpathmoveto{\pgfqpoint{6.287338in}{4.592223in}}%
\pgfpathlineto{\pgfqpoint{6.567338in}{4.592223in}}%
\pgfusepath{stroke}%
\end{pgfscope}%
\begin{pgfscope}%
\pgftext[x=6.787338in,y=4.522223in,left,base]{{\rmfamily\fontsize{14.400000}{17.280000}\selectfont \(\displaystyle q_{21}\)}}%
\end{pgfscope}%
\begin{pgfscope}%
\pgfsetrectcap%
\pgfsetroundjoin%
\pgfsetlinewidth{1.003750pt}%
\definecolor{currentstroke}{rgb}{0.000000,0.750000,0.750000}%
\pgfsetstrokecolor{currentstroke}%
\pgfsetdash{}{0pt}%
\pgfpathmoveto{\pgfqpoint{6.287338in}{4.313334in}}%
\pgfpathlineto{\pgfqpoint{6.567338in}{4.313334in}}%
\pgfusepath{stroke}%
\end{pgfscope}%
\begin{pgfscope}%
\pgftext[x=6.787338in,y=4.243334in,left,base]{{\rmfamily\fontsize{14.400000}{17.280000}\selectfont \(\displaystyle q_{22}\)}}%
\end{pgfscope}%
\end{pgfpicture}%
\makeatother%
\endgroup%
}
\caption{Verlauf der Gelenkwinkel bei Verwendung einer Vorsteuerung, kein Regler vorhanden}
\end{figure}

Es wurde kein Regler verwendet

%\begin{figure}[htb!]
%%% Creator: Matplotlib, PGF backend
%%
%% To include the figure in your LaTeX document, write
%%   \input{<filename>.pgf}
%%
%% Make sure the required packages are loaded in your preamble
%%   \usepackage{pgf}
%%
%% Figures using additional raster images can only be included by \input if
%% they are in the same directory as the main LaTeX file. For loading figures
%% from other directories you can use the `import` package
%%   \usepackage{import}
%% and then include the figures with
%%   \import{<path to file>}{<filename>.pgf}
%%
%% Matplotlib used the following preamble
%%   \usepackage{fontspec}
%%   \setsansfont{DejaVu Sans}
%%   \setmonofont{DejaVu Sans Mono}
%%
\begingroup%
\makeatletter%
\begin{pgfpicture}%
\pgfpathrectangle{\pgfpointorigin}{\pgfqpoint{8.000000in}{6.000000in}}%
\pgfusepath{use as bounding box}%
\begin{pgfscope}%
\pgfsetbuttcap%
\pgfsetroundjoin%
\definecolor{currentfill}{rgb}{1.000000,1.000000,1.000000}%
\pgfsetfillcolor{currentfill}%
\pgfsetlinewidth{0.000000pt}%
\definecolor{currentstroke}{rgb}{1.000000,1.000000,1.000000}%
\pgfsetstrokecolor{currentstroke}%
\pgfsetdash{}{0pt}%
\pgfpathmoveto{\pgfqpoint{0.000000in}{0.000000in}}%
\pgfpathlineto{\pgfqpoint{8.000000in}{0.000000in}}%
\pgfpathlineto{\pgfqpoint{8.000000in}{6.000000in}}%
\pgfpathlineto{\pgfqpoint{0.000000in}{6.000000in}}%
\pgfpathclose%
\pgfusepath{fill}%
\end{pgfscope}%
\begin{pgfscope}%
\pgfsetbuttcap%
\pgfsetroundjoin%
\definecolor{currentfill}{rgb}{1.000000,1.000000,1.000000}%
\pgfsetfillcolor{currentfill}%
\pgfsetlinewidth{0.000000pt}%
\definecolor{currentstroke}{rgb}{0.000000,0.000000,0.000000}%
\pgfsetstrokecolor{currentstroke}%
\pgfsetstrokeopacity{0.000000}%
\pgfsetdash{}{0pt}%
\pgfpathmoveto{\pgfqpoint{1.000000in}{3.218182in}}%
\pgfpathlineto{\pgfqpoint{3.818182in}{3.218182in}}%
\pgfpathlineto{\pgfqpoint{3.818182in}{5.400000in}}%
\pgfpathlineto{\pgfqpoint{1.000000in}{5.400000in}}%
\pgfpathclose%
\pgfusepath{fill}%
\end{pgfscope}%
\begin{pgfscope}%
\pgfpathrectangle{\pgfqpoint{1.000000in}{3.218182in}}{\pgfqpoint{2.818182in}{2.181818in}} %
\pgfusepath{clip}%
\pgfsetrectcap%
\pgfsetroundjoin%
\pgfsetlinewidth{1.003750pt}%
\definecolor{currentstroke}{rgb}{0.000000,0.000000,1.000000}%
\pgfsetstrokecolor{currentstroke}%
\pgfsetdash{}{0pt}%
\pgfpathmoveto{\pgfqpoint{1.000000in}{5.036364in}}%
\pgfpathlineto{\pgfqpoint{1.008455in}{5.035238in}}%
\pgfpathlineto{\pgfqpoint{1.017756in}{5.033946in}}%
\pgfpathlineto{\pgfqpoint{1.027621in}{5.033030in}}%
\pgfpathlineto{\pgfqpoint{1.037204in}{5.031790in}}%
\pgfpathlineto{\pgfqpoint{1.047068in}{5.031001in}}%
\pgfpathlineto{\pgfqpoint{1.056651in}{5.029906in}}%
\pgfpathlineto{\pgfqpoint{1.066516in}{5.029226in}}%
\pgfpathlineto{\pgfqpoint{1.075817in}{5.028293in}}%
\pgfpathlineto{\pgfqpoint{1.085681in}{5.027760in}}%
\pgfpathlineto{\pgfqpoint{1.094982in}{5.026971in}}%
\pgfpathlineto{\pgfqpoint{1.104565in}{5.026590in}}%
\pgfpathlineto{\pgfqpoint{1.113866in}{5.025936in}}%
\pgfpathlineto{\pgfqpoint{1.123167in}{5.025700in}}%
\pgfpathlineto{\pgfqpoint{1.132468in}{5.025200in}}%
\pgfpathlineto{\pgfqpoint{1.141487in}{5.025077in}}%
\pgfpathlineto{\pgfqpoint{1.150506in}{5.024745in}}%
\pgfpathlineto{\pgfqpoint{1.159525in}{5.024705in}}%
\pgfpathlineto{\pgfqpoint{1.167699in}{5.024526in}}%
\pgfpathlineto{\pgfqpoint{1.177281in}{5.024533in}}%
\pgfpathlineto{\pgfqpoint{1.184891in}{5.024561in}}%
\pgfpathlineto{\pgfqpoint{1.193628in}{5.024799in}}%
\pgfpathlineto{\pgfqpoint{1.202366in}{5.024924in}}%
\pgfpathlineto{\pgfqpoint{1.210257in}{5.025056in}}%
\pgfpathlineto{\pgfqpoint{1.218995in}{5.025234in}}%
\pgfpathlineto{\pgfqpoint{1.226886in}{5.025258in}}%
\pgfpathlineto{\pgfqpoint{1.236187in}{5.025525in}}%
\pgfpathlineto{\pgfqpoint{1.244079in}{5.025335in}}%
\pgfpathlineto{\pgfqpoint{1.252252in}{5.025270in}}%
\pgfpathlineto{\pgfqpoint{1.260990in}{5.025195in}}%
\pgfpathlineto{\pgfqpoint{1.270291in}{5.024816in}}%
\pgfpathlineto{\pgfqpoint{1.279028in}{5.024573in}}%
\pgfpathlineto{\pgfqpoint{1.289174in}{5.023921in}}%
\pgfpathlineto{\pgfqpoint{1.297630in}{5.023446in}}%
\pgfpathlineto{\pgfqpoint{1.308340in}{5.022471in}}%
\pgfpathlineto{\pgfqpoint{1.316795in}{5.021716in}}%
\pgfpathlineto{\pgfqpoint{1.327505in}{5.020419in}}%
\pgfpathlineto{\pgfqpoint{1.336243in}{5.019339in}}%
\pgfpathlineto{\pgfqpoint{1.346953in}{5.017720in}}%
\pgfpathlineto{\pgfqpoint{1.355690in}{5.016302in}}%
\pgfpathlineto{\pgfqpoint{1.366400in}{5.014312in}}%
\pgfpathlineto{\pgfqpoint{1.375138in}{5.012542in}}%
\pgfpathlineto{\pgfqpoint{1.385848in}{5.010149in}}%
\pgfpathlineto{\pgfqpoint{1.394303in}{5.008120in}}%
\pgfpathlineto{\pgfqpoint{1.405295in}{5.005203in}}%
\pgfpathlineto{\pgfqpoint{1.413750in}{5.002825in}}%
\pgfpathlineto{\pgfqpoint{1.424461in}{4.999544in}}%
\pgfpathlineto{\pgfqpoint{1.432916in}{4.996901in}}%
\pgfpathlineto{\pgfqpoint{1.443626in}{4.993201in}}%
\pgfpathlineto{\pgfqpoint{1.452363in}{4.990174in}}%
\pgfpathlineto{\pgfqpoint{1.462792in}{4.986273in}}%
\pgfpathlineto{\pgfqpoint{1.471811in}{4.982872in}}%
\pgfpathlineto{\pgfqpoint{1.481957in}{4.978875in}}%
\pgfpathlineto{\pgfqpoint{1.491258in}{4.975132in}}%
\pgfpathlineto{\pgfqpoint{1.501405in}{4.971030in}}%
\pgfpathlineto{\pgfqpoint{1.510424in}{4.967223in}}%
\pgfpathlineto{\pgfqpoint{1.520570in}{4.962968in}}%
\pgfpathlineto{\pgfqpoint{1.529589in}{4.959056in}}%
\pgfpathlineto{\pgfqpoint{1.540018in}{4.954508in}}%
\pgfpathlineto{\pgfqpoint{1.548755in}{4.950557in}}%
\pgfpathlineto{\pgfqpoint{1.559465in}{4.945576in}}%
\pgfpathlineto{\pgfqpoint{1.567920in}{4.941419in}}%
\pgfpathlineto{\pgfqpoint{1.578912in}{4.935760in}}%
\pgfpathlineto{\pgfqpoint{1.586804in}{4.931114in}}%
\pgfpathlineto{\pgfqpoint{1.613016in}{4.913645in}}%
\pgfpathlineto{\pgfqpoint{1.668540in}{4.882116in}}%
\pgfpathlineto{\pgfqpoint{1.676995in}{4.877236in}}%
\pgfpathlineto{\pgfqpoint{1.685732in}{4.872919in}}%
\pgfpathlineto{\pgfqpoint{1.694751in}{4.868320in}}%
\pgfpathlineto{\pgfqpoint{1.704334in}{4.863502in}}%
\pgfpathlineto{\pgfqpoint{1.714762in}{4.858676in}}%
\pgfpathlineto{\pgfqpoint{1.723500in}{4.854936in}}%
\pgfpathlineto{\pgfqpoint{1.734773in}{4.850724in}}%
\pgfpathlineto{\pgfqpoint{1.742383in}{4.848449in}}%
\pgfpathlineto{\pgfqpoint{1.755348in}{4.845420in}}%
\pgfpathlineto{\pgfqpoint{1.761831in}{4.845009in}}%
\pgfpathlineto{\pgfqpoint{1.807772in}{4.845609in}}%
\pgfpathlineto{\pgfqpoint{1.823273in}{4.842579in}}%
\pgfpathlineto{\pgfqpoint{1.833702in}{4.839709in}}%
\pgfpathlineto{\pgfqpoint{1.841311in}{4.836539in}}%
\pgfpathlineto{\pgfqpoint{1.852585in}{4.831883in}}%
\pgfpathlineto{\pgfqpoint{1.860759in}{4.827723in}}%
\pgfpathlineto{\pgfqpoint{1.871469in}{4.822448in}}%
\pgfpathlineto{\pgfqpoint{1.880206in}{4.817564in}}%
\pgfpathlineto{\pgfqpoint{1.890353in}{4.812214in}}%
\pgfpathlineto{\pgfqpoint{1.899372in}{4.807055in}}%
\pgfpathlineto{\pgfqpoint{1.909236in}{4.801759in}}%
\pgfpathlineto{\pgfqpoint{1.918537in}{4.796499in}}%
\pgfpathlineto{\pgfqpoint{1.928402in}{4.791196in}}%
\pgfpathlineto{\pgfqpoint{1.937703in}{4.785952in}}%
\pgfpathlineto{\pgfqpoint{1.947849in}{4.780384in}}%
\pgfpathlineto{\pgfqpoint{1.956587in}{4.775208in}}%
\pgfpathlineto{\pgfqpoint{1.967297in}{4.768843in}}%
\pgfpathlineto{\pgfqpoint{1.975470in}{4.763073in}}%
\pgfpathlineto{\pgfqpoint{1.985617in}{4.756072in}}%
\pgfpathlineto{\pgfqpoint{1.993790in}{4.749380in}}%
\pgfpathlineto{\pgfqpoint{2.001682in}{4.744464in}}%
\pgfpathlineto{\pgfqpoint{2.039731in}{4.721601in}}%
\pgfpathlineto{\pgfqpoint{2.110475in}{4.682297in}}%
\pgfpathlineto{\pgfqpoint{2.142887in}{4.672755in}}%
\pgfpathlineto{\pgfqpoint{2.149088in}{4.672757in}}%
\pgfpathlineto{\pgfqpoint{2.171917in}{4.673657in}}%
\pgfpathlineto{\pgfqpoint{2.192210in}{4.668417in}}%
\pgfpathlineto{\pgfqpoint{2.227159in}{4.649735in}}%
\pgfpathlineto{\pgfqpoint{2.259290in}{4.626638in}}%
\pgfpathlineto{\pgfqpoint{2.267463in}{4.619898in}}%
\pgfpathlineto{\pgfqpoint{2.319887in}{4.572705in}}%
\pgfpathlineto{\pgfqpoint{2.330315in}{4.565537in}}%
\pgfpathlineto{\pgfqpoint{2.366391in}{4.541710in}}%
\pgfpathlineto{\pgfqpoint{2.392885in}{4.530355in}}%
\pgfpathlineto{\pgfqpoint{2.400495in}{4.528815in}}%
\pgfpathlineto{\pgfqpoint{2.427834in}{4.528181in}}%
\pgfpathlineto{\pgfqpoint{2.444744in}{4.523817in}}%
\pgfpathlineto{\pgfqpoint{2.470674in}{4.510735in}}%
\pgfpathlineto{\pgfqpoint{2.480257in}{4.504661in}}%
\pgfpathlineto{\pgfqpoint{2.493222in}{4.495867in}}%
\pgfpathlineto{\pgfqpoint{2.502523in}{4.488544in}}%
\pgfpathlineto{\pgfqpoint{2.549309in}{4.449088in}}%
\pgfpathlineto{\pgfqpoint{2.567348in}{4.431341in}}%
\pgfpathlineto{\pgfqpoint{2.580594in}{4.420072in}}%
\pgfpathlineto{\pgfqpoint{2.589614in}{4.412414in}}%
\pgfpathlineto{\pgfqpoint{2.598914in}{4.405374in}}%
\pgfpathlineto{\pgfqpoint{2.608215in}{4.398078in}}%
\pgfpathlineto{\pgfqpoint{2.616953in}{4.391945in}}%
\pgfpathlineto{\pgfqpoint{2.629918in}{4.383609in}}%
\pgfpathlineto{\pgfqpoint{2.637246in}{4.380620in}}%
\pgfpathlineto{\pgfqpoint{2.654720in}{4.376356in}}%
\pgfpathlineto{\pgfqpoint{2.669376in}{4.376580in}}%
\pgfpathlineto{\pgfqpoint{2.675577in}{4.375422in}}%
\pgfpathlineto{\pgfqpoint{2.693333in}{4.367554in}}%
\pgfpathlineto{\pgfqpoint{2.718135in}{4.348706in}}%
\pgfpathlineto{\pgfqpoint{2.751111in}{4.318912in}}%
\pgfpathlineto{\pgfqpoint{2.783524in}{4.283429in}}%
\pgfpathlineto{\pgfqpoint{2.792543in}{4.275237in}}%
\pgfpathlineto{\pgfqpoint{2.805226in}{4.264043in}}%
\pgfpathlineto{\pgfqpoint{2.815654in}{4.256329in}}%
\pgfpathlineto{\pgfqpoint{2.835102in}{4.242933in}}%
\pgfpathlineto{\pgfqpoint{2.848348in}{4.236280in}}%
\pgfpathlineto{\pgfqpoint{2.866387in}{4.233407in}}%
\pgfpathlineto{\pgfqpoint{2.874842in}{4.233081in}}%
\pgfpathlineto{\pgfqpoint{2.888371in}{4.229933in}}%
\pgfpathlineto{\pgfqpoint{2.906973in}{4.219606in}}%
\pgfpathlineto{\pgfqpoint{2.942485in}{4.190187in}}%
\pgfpathlineto{\pgfqpoint{2.964469in}{4.166652in}}%
\pgfpathlineto{\pgfqpoint{2.986453in}{4.140641in}}%
\pgfpathlineto{\pgfqpoint{3.030703in}{4.101112in}}%
\pgfpathlineto{\pgfqpoint{3.040004in}{4.095240in}}%
\pgfpathlineto{\pgfqpoint{3.053814in}{4.091153in}}%
\pgfpathlineto{\pgfqpoint{3.058888in}{4.091578in}}%
\pgfpathlineto{\pgfqpoint{3.066216in}{4.091045in}}%
\pgfpathlineto{\pgfqpoint{3.080872in}{4.085838in}}%
\pgfpathlineto{\pgfqpoint{3.100319in}{4.071277in}}%
\pgfpathlineto{\pgfqpoint{3.140341in}{4.029756in}}%
\pgfpathlineto{\pgfqpoint{3.162043in}{4.002823in}}%
\pgfpathlineto{\pgfqpoint{3.173317in}{3.990587in}}%
\pgfpathlineto{\pgfqpoint{3.182900in}{3.982030in}}%
\pgfpathlineto{\pgfqpoint{3.210803in}{3.958455in}}%
\pgfpathlineto{\pgfqpoint{3.217567in}{3.955175in}}%
\pgfpathlineto{\pgfqpoint{3.233069in}{3.950745in}}%
\pgfpathlineto{\pgfqpoint{3.241806in}{3.950443in}}%
\pgfpathlineto{\pgfqpoint{3.256744in}{3.945541in}}%
\pgfpathlineto{\pgfqpoint{3.275628in}{3.931682in}}%
\pgfpathlineto{\pgfqpoint{3.310858in}{3.894904in}}%
\pgfpathlineto{\pgfqpoint{3.343553in}{3.852529in}}%
\pgfpathlineto{\pgfqpoint{3.353981in}{3.841402in}}%
\pgfpathlineto{\pgfqpoint{3.377938in}{3.820776in}}%
\pgfpathlineto{\pgfqpoint{3.387239in}{3.815661in}}%
\pgfpathlineto{\pgfqpoint{3.400204in}{3.814076in}}%
\pgfpathlineto{\pgfqpoint{3.409786in}{3.812308in}}%
\pgfpathlineto{\pgfqpoint{3.423315in}{3.804214in}}%
\pgfpathlineto{\pgfqpoint{3.430643in}{3.797579in}}%
\pgfpathlineto{\pgfqpoint{3.462492in}{3.763948in}}%
\pgfpathlineto{\pgfqpoint{3.478275in}{3.741955in}}%
\pgfpathlineto{\pgfqpoint{3.489267in}{3.727252in}}%
\pgfpathlineto{\pgfqpoint{3.524216in}{3.690481in}}%
\pgfpathlineto{\pgfqpoint{3.543100in}{3.678443in}}%
\pgfpathlineto{\pgfqpoint{3.559729in}{3.676158in}}%
\pgfpathlineto{\pgfqpoint{3.563956in}{3.675111in}}%
\pgfpathlineto{\pgfqpoint{3.580022in}{3.665718in}}%
\pgfpathlineto{\pgfqpoint{3.598905in}{3.647265in}}%
\pgfpathlineto{\pgfqpoint{3.622862in}{3.616650in}}%
\pgfpathlineto{\pgfqpoint{3.649074in}{3.580276in}}%
\pgfpathlineto{\pgfqpoint{3.659784in}{3.567597in}}%
\pgfpathlineto{\pgfqpoint{3.670212in}{3.556795in}}%
\pgfpathlineto{\pgfqpoint{3.676695in}{3.551170in}}%
\pgfpathlineto{\pgfqpoint{3.692478in}{3.543251in}}%
\pgfpathlineto{\pgfqpoint{3.702907in}{3.542406in}}%
\pgfpathlineto{\pgfqpoint{3.715590in}{3.536706in}}%
\pgfpathlineto{\pgfqpoint{3.732219in}{3.521566in}}%
\pgfpathlineto{\pgfqpoint{3.751948in}{3.498364in}}%
\pgfpathlineto{\pgfqpoint{3.771113in}{3.469742in}}%
\pgfpathlineto{\pgfqpoint{3.777314in}{3.460750in}}%
\pgfpathlineto{\pgfqpoint{3.810008in}{3.422435in}}%
\pgfpathlineto{\pgfqpoint{3.818182in}{3.416328in}}%
\pgfpathlineto{\pgfqpoint{3.818182in}{3.416328in}}%
\pgfusepath{stroke}%
\end{pgfscope}%
\begin{pgfscope}%
\pgfsetbuttcap%
\pgfsetroundjoin%
\definecolor{currentfill}{rgb}{0.000000,0.000000,0.000000}%
\pgfsetfillcolor{currentfill}%
\pgfsetlinewidth{0.501875pt}%
\definecolor{currentstroke}{rgb}{0.000000,0.000000,0.000000}%
\pgfsetstrokecolor{currentstroke}%
\pgfsetdash{}{0pt}%
\pgfsys@defobject{currentmarker}{\pgfqpoint{0.000000in}{0.000000in}}{\pgfqpoint{0.000000in}{0.055556in}}{%
\pgfpathmoveto{\pgfqpoint{0.000000in}{0.000000in}}%
\pgfpathlineto{\pgfqpoint{0.000000in}{0.055556in}}%
\pgfusepath{stroke,fill}%
}%
\begin{pgfscope}%
\pgfsys@transformshift{1.000000in}{3.218182in}%
\pgfsys@useobject{currentmarker}{}%
\end{pgfscope}%
\end{pgfscope}%
\begin{pgfscope}%
\pgfsetbuttcap%
\pgfsetroundjoin%
\definecolor{currentfill}{rgb}{0.000000,0.000000,0.000000}%
\pgfsetfillcolor{currentfill}%
\pgfsetlinewidth{0.501875pt}%
\definecolor{currentstroke}{rgb}{0.000000,0.000000,0.000000}%
\pgfsetstrokecolor{currentstroke}%
\pgfsetdash{}{0pt}%
\pgfsys@defobject{currentmarker}{\pgfqpoint{0.000000in}{-0.055556in}}{\pgfqpoint{0.000000in}{0.000000in}}{%
\pgfpathmoveto{\pgfqpoint{0.000000in}{0.000000in}}%
\pgfpathlineto{\pgfqpoint{0.000000in}{-0.055556in}}%
\pgfusepath{stroke,fill}%
}%
\begin{pgfscope}%
\pgfsys@transformshift{1.000000in}{5.400000in}%
\pgfsys@useobject{currentmarker}{}%
\end{pgfscope}%
\end{pgfscope}%
\begin{pgfscope}%
\pgftext[x=1.000000in,y=3.162626in,,top]{{\rmfamily\fontsize{12.000000}{14.400000}\selectfont 0}}%
\end{pgfscope}%
\begin{pgfscope}%
\pgfsetbuttcap%
\pgfsetroundjoin%
\definecolor{currentfill}{rgb}{0.000000,0.000000,0.000000}%
\pgfsetfillcolor{currentfill}%
\pgfsetlinewidth{0.501875pt}%
\definecolor{currentstroke}{rgb}{0.000000,0.000000,0.000000}%
\pgfsetstrokecolor{currentstroke}%
\pgfsetdash{}{0pt}%
\pgfsys@defobject{currentmarker}{\pgfqpoint{0.000000in}{0.000000in}}{\pgfqpoint{0.000000in}{0.055556in}}{%
\pgfpathmoveto{\pgfqpoint{0.000000in}{0.000000in}}%
\pgfpathlineto{\pgfqpoint{0.000000in}{0.055556in}}%
\pgfusepath{stroke,fill}%
}%
\begin{pgfscope}%
\pgfsys@transformshift{1.563636in}{3.218182in}%
\pgfsys@useobject{currentmarker}{}%
\end{pgfscope}%
\end{pgfscope}%
\begin{pgfscope}%
\pgfsetbuttcap%
\pgfsetroundjoin%
\definecolor{currentfill}{rgb}{0.000000,0.000000,0.000000}%
\pgfsetfillcolor{currentfill}%
\pgfsetlinewidth{0.501875pt}%
\definecolor{currentstroke}{rgb}{0.000000,0.000000,0.000000}%
\pgfsetstrokecolor{currentstroke}%
\pgfsetdash{}{0pt}%
\pgfsys@defobject{currentmarker}{\pgfqpoint{0.000000in}{-0.055556in}}{\pgfqpoint{0.000000in}{0.000000in}}{%
\pgfpathmoveto{\pgfqpoint{0.000000in}{0.000000in}}%
\pgfpathlineto{\pgfqpoint{0.000000in}{-0.055556in}}%
\pgfusepath{stroke,fill}%
}%
\begin{pgfscope}%
\pgfsys@transformshift{1.563636in}{5.400000in}%
\pgfsys@useobject{currentmarker}{}%
\end{pgfscope}%
\end{pgfscope}%
\begin{pgfscope}%
\pgftext[x=1.563636in,y=3.162626in,,top]{{\rmfamily\fontsize{12.000000}{14.400000}\selectfont 2}}%
\end{pgfscope}%
\begin{pgfscope}%
\pgfsetbuttcap%
\pgfsetroundjoin%
\definecolor{currentfill}{rgb}{0.000000,0.000000,0.000000}%
\pgfsetfillcolor{currentfill}%
\pgfsetlinewidth{0.501875pt}%
\definecolor{currentstroke}{rgb}{0.000000,0.000000,0.000000}%
\pgfsetstrokecolor{currentstroke}%
\pgfsetdash{}{0pt}%
\pgfsys@defobject{currentmarker}{\pgfqpoint{0.000000in}{0.000000in}}{\pgfqpoint{0.000000in}{0.055556in}}{%
\pgfpathmoveto{\pgfqpoint{0.000000in}{0.000000in}}%
\pgfpathlineto{\pgfqpoint{0.000000in}{0.055556in}}%
\pgfusepath{stroke,fill}%
}%
\begin{pgfscope}%
\pgfsys@transformshift{2.127273in}{3.218182in}%
\pgfsys@useobject{currentmarker}{}%
\end{pgfscope}%
\end{pgfscope}%
\begin{pgfscope}%
\pgfsetbuttcap%
\pgfsetroundjoin%
\definecolor{currentfill}{rgb}{0.000000,0.000000,0.000000}%
\pgfsetfillcolor{currentfill}%
\pgfsetlinewidth{0.501875pt}%
\definecolor{currentstroke}{rgb}{0.000000,0.000000,0.000000}%
\pgfsetstrokecolor{currentstroke}%
\pgfsetdash{}{0pt}%
\pgfsys@defobject{currentmarker}{\pgfqpoint{0.000000in}{-0.055556in}}{\pgfqpoint{0.000000in}{0.000000in}}{%
\pgfpathmoveto{\pgfqpoint{0.000000in}{0.000000in}}%
\pgfpathlineto{\pgfqpoint{0.000000in}{-0.055556in}}%
\pgfusepath{stroke,fill}%
}%
\begin{pgfscope}%
\pgfsys@transformshift{2.127273in}{5.400000in}%
\pgfsys@useobject{currentmarker}{}%
\end{pgfscope}%
\end{pgfscope}%
\begin{pgfscope}%
\pgftext[x=2.127273in,y=3.162626in,,top]{{\rmfamily\fontsize{12.000000}{14.400000}\selectfont 4}}%
\end{pgfscope}%
\begin{pgfscope}%
\pgfsetbuttcap%
\pgfsetroundjoin%
\definecolor{currentfill}{rgb}{0.000000,0.000000,0.000000}%
\pgfsetfillcolor{currentfill}%
\pgfsetlinewidth{0.501875pt}%
\definecolor{currentstroke}{rgb}{0.000000,0.000000,0.000000}%
\pgfsetstrokecolor{currentstroke}%
\pgfsetdash{}{0pt}%
\pgfsys@defobject{currentmarker}{\pgfqpoint{0.000000in}{0.000000in}}{\pgfqpoint{0.000000in}{0.055556in}}{%
\pgfpathmoveto{\pgfqpoint{0.000000in}{0.000000in}}%
\pgfpathlineto{\pgfqpoint{0.000000in}{0.055556in}}%
\pgfusepath{stroke,fill}%
}%
\begin{pgfscope}%
\pgfsys@transformshift{2.690909in}{3.218182in}%
\pgfsys@useobject{currentmarker}{}%
\end{pgfscope}%
\end{pgfscope}%
\begin{pgfscope}%
\pgfsetbuttcap%
\pgfsetroundjoin%
\definecolor{currentfill}{rgb}{0.000000,0.000000,0.000000}%
\pgfsetfillcolor{currentfill}%
\pgfsetlinewidth{0.501875pt}%
\definecolor{currentstroke}{rgb}{0.000000,0.000000,0.000000}%
\pgfsetstrokecolor{currentstroke}%
\pgfsetdash{}{0pt}%
\pgfsys@defobject{currentmarker}{\pgfqpoint{0.000000in}{-0.055556in}}{\pgfqpoint{0.000000in}{0.000000in}}{%
\pgfpathmoveto{\pgfqpoint{0.000000in}{0.000000in}}%
\pgfpathlineto{\pgfqpoint{0.000000in}{-0.055556in}}%
\pgfusepath{stroke,fill}%
}%
\begin{pgfscope}%
\pgfsys@transformshift{2.690909in}{5.400000in}%
\pgfsys@useobject{currentmarker}{}%
\end{pgfscope}%
\end{pgfscope}%
\begin{pgfscope}%
\pgftext[x=2.690909in,y=3.162626in,,top]{{\rmfamily\fontsize{12.000000}{14.400000}\selectfont 6}}%
\end{pgfscope}%
\begin{pgfscope}%
\pgfsetbuttcap%
\pgfsetroundjoin%
\definecolor{currentfill}{rgb}{0.000000,0.000000,0.000000}%
\pgfsetfillcolor{currentfill}%
\pgfsetlinewidth{0.501875pt}%
\definecolor{currentstroke}{rgb}{0.000000,0.000000,0.000000}%
\pgfsetstrokecolor{currentstroke}%
\pgfsetdash{}{0pt}%
\pgfsys@defobject{currentmarker}{\pgfqpoint{0.000000in}{0.000000in}}{\pgfqpoint{0.000000in}{0.055556in}}{%
\pgfpathmoveto{\pgfqpoint{0.000000in}{0.000000in}}%
\pgfpathlineto{\pgfqpoint{0.000000in}{0.055556in}}%
\pgfusepath{stroke,fill}%
}%
\begin{pgfscope}%
\pgfsys@transformshift{3.254545in}{3.218182in}%
\pgfsys@useobject{currentmarker}{}%
\end{pgfscope}%
\end{pgfscope}%
\begin{pgfscope}%
\pgfsetbuttcap%
\pgfsetroundjoin%
\definecolor{currentfill}{rgb}{0.000000,0.000000,0.000000}%
\pgfsetfillcolor{currentfill}%
\pgfsetlinewidth{0.501875pt}%
\definecolor{currentstroke}{rgb}{0.000000,0.000000,0.000000}%
\pgfsetstrokecolor{currentstroke}%
\pgfsetdash{}{0pt}%
\pgfsys@defobject{currentmarker}{\pgfqpoint{0.000000in}{-0.055556in}}{\pgfqpoint{0.000000in}{0.000000in}}{%
\pgfpathmoveto{\pgfqpoint{0.000000in}{0.000000in}}%
\pgfpathlineto{\pgfqpoint{0.000000in}{-0.055556in}}%
\pgfusepath{stroke,fill}%
}%
\begin{pgfscope}%
\pgfsys@transformshift{3.254545in}{5.400000in}%
\pgfsys@useobject{currentmarker}{}%
\end{pgfscope}%
\end{pgfscope}%
\begin{pgfscope}%
\pgftext[x=3.254545in,y=3.162626in,,top]{{\rmfamily\fontsize{12.000000}{14.400000}\selectfont 8}}%
\end{pgfscope}%
\begin{pgfscope}%
\pgfsetbuttcap%
\pgfsetroundjoin%
\definecolor{currentfill}{rgb}{0.000000,0.000000,0.000000}%
\pgfsetfillcolor{currentfill}%
\pgfsetlinewidth{0.501875pt}%
\definecolor{currentstroke}{rgb}{0.000000,0.000000,0.000000}%
\pgfsetstrokecolor{currentstroke}%
\pgfsetdash{}{0pt}%
\pgfsys@defobject{currentmarker}{\pgfqpoint{0.000000in}{0.000000in}}{\pgfqpoint{0.000000in}{0.055556in}}{%
\pgfpathmoveto{\pgfqpoint{0.000000in}{0.000000in}}%
\pgfpathlineto{\pgfqpoint{0.000000in}{0.055556in}}%
\pgfusepath{stroke,fill}%
}%
\begin{pgfscope}%
\pgfsys@transformshift{3.818182in}{3.218182in}%
\pgfsys@useobject{currentmarker}{}%
\end{pgfscope}%
\end{pgfscope}%
\begin{pgfscope}%
\pgfsetbuttcap%
\pgfsetroundjoin%
\definecolor{currentfill}{rgb}{0.000000,0.000000,0.000000}%
\pgfsetfillcolor{currentfill}%
\pgfsetlinewidth{0.501875pt}%
\definecolor{currentstroke}{rgb}{0.000000,0.000000,0.000000}%
\pgfsetstrokecolor{currentstroke}%
\pgfsetdash{}{0pt}%
\pgfsys@defobject{currentmarker}{\pgfqpoint{0.000000in}{-0.055556in}}{\pgfqpoint{0.000000in}{0.000000in}}{%
\pgfpathmoveto{\pgfqpoint{0.000000in}{0.000000in}}%
\pgfpathlineto{\pgfqpoint{0.000000in}{-0.055556in}}%
\pgfusepath{stroke,fill}%
}%
\begin{pgfscope}%
\pgfsys@transformshift{3.818182in}{5.400000in}%
\pgfsys@useobject{currentmarker}{}%
\end{pgfscope}%
\end{pgfscope}%
\begin{pgfscope}%
\pgftext[x=3.818182in,y=3.162626in,,top]{{\rmfamily\fontsize{12.000000}{14.400000}\selectfont 10}}%
\end{pgfscope}%
\begin{pgfscope}%
\pgfsetbuttcap%
\pgfsetroundjoin%
\definecolor{currentfill}{rgb}{0.000000,0.000000,0.000000}%
\pgfsetfillcolor{currentfill}%
\pgfsetlinewidth{0.501875pt}%
\definecolor{currentstroke}{rgb}{0.000000,0.000000,0.000000}%
\pgfsetstrokecolor{currentstroke}%
\pgfsetdash{}{0pt}%
\pgfsys@defobject{currentmarker}{\pgfqpoint{0.000000in}{0.000000in}}{\pgfqpoint{0.055556in}{0.000000in}}{%
\pgfpathmoveto{\pgfqpoint{0.000000in}{0.000000in}}%
\pgfpathlineto{\pgfqpoint{0.055556in}{0.000000in}}%
\pgfusepath{stroke,fill}%
}%
\begin{pgfscope}%
\pgfsys@transformshift{1.000000in}{3.218182in}%
\pgfsys@useobject{currentmarker}{}%
\end{pgfscope}%
\end{pgfscope}%
\begin{pgfscope}%
\pgfsetbuttcap%
\pgfsetroundjoin%
\definecolor{currentfill}{rgb}{0.000000,0.000000,0.000000}%
\pgfsetfillcolor{currentfill}%
\pgfsetlinewidth{0.501875pt}%
\definecolor{currentstroke}{rgb}{0.000000,0.000000,0.000000}%
\pgfsetstrokecolor{currentstroke}%
\pgfsetdash{}{0pt}%
\pgfsys@defobject{currentmarker}{\pgfqpoint{-0.055556in}{0.000000in}}{\pgfqpoint{0.000000in}{0.000000in}}{%
\pgfpathmoveto{\pgfqpoint{0.000000in}{0.000000in}}%
\pgfpathlineto{\pgfqpoint{-0.055556in}{0.000000in}}%
\pgfusepath{stroke,fill}%
}%
\begin{pgfscope}%
\pgfsys@transformshift{3.818182in}{3.218182in}%
\pgfsys@useobject{currentmarker}{}%
\end{pgfscope}%
\end{pgfscope}%
\begin{pgfscope}%
\pgftext[x=0.944444in,y=3.218182in,right,]{{\rmfamily\fontsize{12.000000}{14.400000}\selectfont -2500}}%
\end{pgfscope}%
\begin{pgfscope}%
\pgfsetbuttcap%
\pgfsetroundjoin%
\definecolor{currentfill}{rgb}{0.000000,0.000000,0.000000}%
\pgfsetfillcolor{currentfill}%
\pgfsetlinewidth{0.501875pt}%
\definecolor{currentstroke}{rgb}{0.000000,0.000000,0.000000}%
\pgfsetstrokecolor{currentstroke}%
\pgfsetdash{}{0pt}%
\pgfsys@defobject{currentmarker}{\pgfqpoint{0.000000in}{0.000000in}}{\pgfqpoint{0.055556in}{0.000000in}}{%
\pgfpathmoveto{\pgfqpoint{0.000000in}{0.000000in}}%
\pgfpathlineto{\pgfqpoint{0.055556in}{0.000000in}}%
\pgfusepath{stroke,fill}%
}%
\begin{pgfscope}%
\pgfsys@transformshift{1.000000in}{3.581818in}%
\pgfsys@useobject{currentmarker}{}%
\end{pgfscope}%
\end{pgfscope}%
\begin{pgfscope}%
\pgfsetbuttcap%
\pgfsetroundjoin%
\definecolor{currentfill}{rgb}{0.000000,0.000000,0.000000}%
\pgfsetfillcolor{currentfill}%
\pgfsetlinewidth{0.501875pt}%
\definecolor{currentstroke}{rgb}{0.000000,0.000000,0.000000}%
\pgfsetstrokecolor{currentstroke}%
\pgfsetdash{}{0pt}%
\pgfsys@defobject{currentmarker}{\pgfqpoint{-0.055556in}{0.000000in}}{\pgfqpoint{0.000000in}{0.000000in}}{%
\pgfpathmoveto{\pgfqpoint{0.000000in}{0.000000in}}%
\pgfpathlineto{\pgfqpoint{-0.055556in}{0.000000in}}%
\pgfusepath{stroke,fill}%
}%
\begin{pgfscope}%
\pgfsys@transformshift{3.818182in}{3.581818in}%
\pgfsys@useobject{currentmarker}{}%
\end{pgfscope}%
\end{pgfscope}%
\begin{pgfscope}%
\pgftext[x=0.944444in,y=3.581818in,right,]{{\rmfamily\fontsize{12.000000}{14.400000}\selectfont -2000}}%
\end{pgfscope}%
\begin{pgfscope}%
\pgfsetbuttcap%
\pgfsetroundjoin%
\definecolor{currentfill}{rgb}{0.000000,0.000000,0.000000}%
\pgfsetfillcolor{currentfill}%
\pgfsetlinewidth{0.501875pt}%
\definecolor{currentstroke}{rgb}{0.000000,0.000000,0.000000}%
\pgfsetstrokecolor{currentstroke}%
\pgfsetdash{}{0pt}%
\pgfsys@defobject{currentmarker}{\pgfqpoint{0.000000in}{0.000000in}}{\pgfqpoint{0.055556in}{0.000000in}}{%
\pgfpathmoveto{\pgfqpoint{0.000000in}{0.000000in}}%
\pgfpathlineto{\pgfqpoint{0.055556in}{0.000000in}}%
\pgfusepath{stroke,fill}%
}%
\begin{pgfscope}%
\pgfsys@transformshift{1.000000in}{3.945455in}%
\pgfsys@useobject{currentmarker}{}%
\end{pgfscope}%
\end{pgfscope}%
\begin{pgfscope}%
\pgfsetbuttcap%
\pgfsetroundjoin%
\definecolor{currentfill}{rgb}{0.000000,0.000000,0.000000}%
\pgfsetfillcolor{currentfill}%
\pgfsetlinewidth{0.501875pt}%
\definecolor{currentstroke}{rgb}{0.000000,0.000000,0.000000}%
\pgfsetstrokecolor{currentstroke}%
\pgfsetdash{}{0pt}%
\pgfsys@defobject{currentmarker}{\pgfqpoint{-0.055556in}{0.000000in}}{\pgfqpoint{0.000000in}{0.000000in}}{%
\pgfpathmoveto{\pgfqpoint{0.000000in}{0.000000in}}%
\pgfpathlineto{\pgfqpoint{-0.055556in}{0.000000in}}%
\pgfusepath{stroke,fill}%
}%
\begin{pgfscope}%
\pgfsys@transformshift{3.818182in}{3.945455in}%
\pgfsys@useobject{currentmarker}{}%
\end{pgfscope}%
\end{pgfscope}%
\begin{pgfscope}%
\pgftext[x=0.944444in,y=3.945455in,right,]{{\rmfamily\fontsize{12.000000}{14.400000}\selectfont -1500}}%
\end{pgfscope}%
\begin{pgfscope}%
\pgfsetbuttcap%
\pgfsetroundjoin%
\definecolor{currentfill}{rgb}{0.000000,0.000000,0.000000}%
\pgfsetfillcolor{currentfill}%
\pgfsetlinewidth{0.501875pt}%
\definecolor{currentstroke}{rgb}{0.000000,0.000000,0.000000}%
\pgfsetstrokecolor{currentstroke}%
\pgfsetdash{}{0pt}%
\pgfsys@defobject{currentmarker}{\pgfqpoint{0.000000in}{0.000000in}}{\pgfqpoint{0.055556in}{0.000000in}}{%
\pgfpathmoveto{\pgfqpoint{0.000000in}{0.000000in}}%
\pgfpathlineto{\pgfqpoint{0.055556in}{0.000000in}}%
\pgfusepath{stroke,fill}%
}%
\begin{pgfscope}%
\pgfsys@transformshift{1.000000in}{4.309091in}%
\pgfsys@useobject{currentmarker}{}%
\end{pgfscope}%
\end{pgfscope}%
\begin{pgfscope}%
\pgfsetbuttcap%
\pgfsetroundjoin%
\definecolor{currentfill}{rgb}{0.000000,0.000000,0.000000}%
\pgfsetfillcolor{currentfill}%
\pgfsetlinewidth{0.501875pt}%
\definecolor{currentstroke}{rgb}{0.000000,0.000000,0.000000}%
\pgfsetstrokecolor{currentstroke}%
\pgfsetdash{}{0pt}%
\pgfsys@defobject{currentmarker}{\pgfqpoint{-0.055556in}{0.000000in}}{\pgfqpoint{0.000000in}{0.000000in}}{%
\pgfpathmoveto{\pgfqpoint{0.000000in}{0.000000in}}%
\pgfpathlineto{\pgfqpoint{-0.055556in}{0.000000in}}%
\pgfusepath{stroke,fill}%
}%
\begin{pgfscope}%
\pgfsys@transformshift{3.818182in}{4.309091in}%
\pgfsys@useobject{currentmarker}{}%
\end{pgfscope}%
\end{pgfscope}%
\begin{pgfscope}%
\pgftext[x=0.944444in,y=4.309091in,right,]{{\rmfamily\fontsize{12.000000}{14.400000}\selectfont -1000}}%
\end{pgfscope}%
\begin{pgfscope}%
\pgfsetbuttcap%
\pgfsetroundjoin%
\definecolor{currentfill}{rgb}{0.000000,0.000000,0.000000}%
\pgfsetfillcolor{currentfill}%
\pgfsetlinewidth{0.501875pt}%
\definecolor{currentstroke}{rgb}{0.000000,0.000000,0.000000}%
\pgfsetstrokecolor{currentstroke}%
\pgfsetdash{}{0pt}%
\pgfsys@defobject{currentmarker}{\pgfqpoint{0.000000in}{0.000000in}}{\pgfqpoint{0.055556in}{0.000000in}}{%
\pgfpathmoveto{\pgfqpoint{0.000000in}{0.000000in}}%
\pgfpathlineto{\pgfqpoint{0.055556in}{0.000000in}}%
\pgfusepath{stroke,fill}%
}%
\begin{pgfscope}%
\pgfsys@transformshift{1.000000in}{4.672727in}%
\pgfsys@useobject{currentmarker}{}%
\end{pgfscope}%
\end{pgfscope}%
\begin{pgfscope}%
\pgfsetbuttcap%
\pgfsetroundjoin%
\definecolor{currentfill}{rgb}{0.000000,0.000000,0.000000}%
\pgfsetfillcolor{currentfill}%
\pgfsetlinewidth{0.501875pt}%
\definecolor{currentstroke}{rgb}{0.000000,0.000000,0.000000}%
\pgfsetstrokecolor{currentstroke}%
\pgfsetdash{}{0pt}%
\pgfsys@defobject{currentmarker}{\pgfqpoint{-0.055556in}{0.000000in}}{\pgfqpoint{0.000000in}{0.000000in}}{%
\pgfpathmoveto{\pgfqpoint{0.000000in}{0.000000in}}%
\pgfpathlineto{\pgfqpoint{-0.055556in}{0.000000in}}%
\pgfusepath{stroke,fill}%
}%
\begin{pgfscope}%
\pgfsys@transformshift{3.818182in}{4.672727in}%
\pgfsys@useobject{currentmarker}{}%
\end{pgfscope}%
\end{pgfscope}%
\begin{pgfscope}%
\pgftext[x=0.944444in,y=4.672727in,right,]{{\rmfamily\fontsize{12.000000}{14.400000}\selectfont -500}}%
\end{pgfscope}%
\begin{pgfscope}%
\pgfsetbuttcap%
\pgfsetroundjoin%
\definecolor{currentfill}{rgb}{0.000000,0.000000,0.000000}%
\pgfsetfillcolor{currentfill}%
\pgfsetlinewidth{0.501875pt}%
\definecolor{currentstroke}{rgb}{0.000000,0.000000,0.000000}%
\pgfsetstrokecolor{currentstroke}%
\pgfsetdash{}{0pt}%
\pgfsys@defobject{currentmarker}{\pgfqpoint{0.000000in}{0.000000in}}{\pgfqpoint{0.055556in}{0.000000in}}{%
\pgfpathmoveto{\pgfqpoint{0.000000in}{0.000000in}}%
\pgfpathlineto{\pgfqpoint{0.055556in}{0.000000in}}%
\pgfusepath{stroke,fill}%
}%
\begin{pgfscope}%
\pgfsys@transformshift{1.000000in}{5.036364in}%
\pgfsys@useobject{currentmarker}{}%
\end{pgfscope}%
\end{pgfscope}%
\begin{pgfscope}%
\pgfsetbuttcap%
\pgfsetroundjoin%
\definecolor{currentfill}{rgb}{0.000000,0.000000,0.000000}%
\pgfsetfillcolor{currentfill}%
\pgfsetlinewidth{0.501875pt}%
\definecolor{currentstroke}{rgb}{0.000000,0.000000,0.000000}%
\pgfsetstrokecolor{currentstroke}%
\pgfsetdash{}{0pt}%
\pgfsys@defobject{currentmarker}{\pgfqpoint{-0.055556in}{0.000000in}}{\pgfqpoint{0.000000in}{0.000000in}}{%
\pgfpathmoveto{\pgfqpoint{0.000000in}{0.000000in}}%
\pgfpathlineto{\pgfqpoint{-0.055556in}{0.000000in}}%
\pgfusepath{stroke,fill}%
}%
\begin{pgfscope}%
\pgfsys@transformshift{3.818182in}{5.036364in}%
\pgfsys@useobject{currentmarker}{}%
\end{pgfscope}%
\end{pgfscope}%
\begin{pgfscope}%
\pgftext[x=0.944444in,y=5.036364in,right,]{{\rmfamily\fontsize{12.000000}{14.400000}\selectfont 0}}%
\end{pgfscope}%
\begin{pgfscope}%
\pgfsetbuttcap%
\pgfsetroundjoin%
\definecolor{currentfill}{rgb}{0.000000,0.000000,0.000000}%
\pgfsetfillcolor{currentfill}%
\pgfsetlinewidth{0.501875pt}%
\definecolor{currentstroke}{rgb}{0.000000,0.000000,0.000000}%
\pgfsetstrokecolor{currentstroke}%
\pgfsetdash{}{0pt}%
\pgfsys@defobject{currentmarker}{\pgfqpoint{0.000000in}{0.000000in}}{\pgfqpoint{0.055556in}{0.000000in}}{%
\pgfpathmoveto{\pgfqpoint{0.000000in}{0.000000in}}%
\pgfpathlineto{\pgfqpoint{0.055556in}{0.000000in}}%
\pgfusepath{stroke,fill}%
}%
\begin{pgfscope}%
\pgfsys@transformshift{1.000000in}{5.400000in}%
\pgfsys@useobject{currentmarker}{}%
\end{pgfscope}%
\end{pgfscope}%
\begin{pgfscope}%
\pgfsetbuttcap%
\pgfsetroundjoin%
\definecolor{currentfill}{rgb}{0.000000,0.000000,0.000000}%
\pgfsetfillcolor{currentfill}%
\pgfsetlinewidth{0.501875pt}%
\definecolor{currentstroke}{rgb}{0.000000,0.000000,0.000000}%
\pgfsetstrokecolor{currentstroke}%
\pgfsetdash{}{0pt}%
\pgfsys@defobject{currentmarker}{\pgfqpoint{-0.055556in}{0.000000in}}{\pgfqpoint{0.000000in}{0.000000in}}{%
\pgfpathmoveto{\pgfqpoint{0.000000in}{0.000000in}}%
\pgfpathlineto{\pgfqpoint{-0.055556in}{0.000000in}}%
\pgfusepath{stroke,fill}%
}%
\begin{pgfscope}%
\pgfsys@transformshift{3.818182in}{5.400000in}%
\pgfsys@useobject{currentmarker}{}%
\end{pgfscope}%
\end{pgfscope}%
\begin{pgfscope}%
\pgftext[x=0.944444in,y=5.400000in,right,]{{\rmfamily\fontsize{12.000000}{14.400000}\selectfont 500}}%
\end{pgfscope}%
\begin{pgfscope}%
\pgfsetbuttcap%
\pgfsetroundjoin%
\pgfsetlinewidth{1.003750pt}%
\definecolor{currentstroke}{rgb}{0.000000,0.000000,0.000000}%
\pgfsetstrokecolor{currentstroke}%
\pgfsetdash{}{0pt}%
\pgfpathmoveto{\pgfqpoint{1.000000in}{5.400000in}}%
\pgfpathlineto{\pgfqpoint{3.818182in}{5.400000in}}%
\pgfusepath{stroke}%
\end{pgfscope}%
\begin{pgfscope}%
\pgfsetbuttcap%
\pgfsetroundjoin%
\pgfsetlinewidth{1.003750pt}%
\definecolor{currentstroke}{rgb}{0.000000,0.000000,0.000000}%
\pgfsetstrokecolor{currentstroke}%
\pgfsetdash{}{0pt}%
\pgfpathmoveto{\pgfqpoint{3.818182in}{3.218182in}}%
\pgfpathlineto{\pgfqpoint{3.818182in}{5.400000in}}%
\pgfusepath{stroke}%
\end{pgfscope}%
\begin{pgfscope}%
\pgfsetbuttcap%
\pgfsetroundjoin%
\pgfsetlinewidth{1.003750pt}%
\definecolor{currentstroke}{rgb}{0.000000,0.000000,0.000000}%
\pgfsetstrokecolor{currentstroke}%
\pgfsetdash{}{0pt}%
\pgfpathmoveto{\pgfqpoint{1.000000in}{3.218182in}}%
\pgfpathlineto{\pgfqpoint{3.818182in}{3.218182in}}%
\pgfusepath{stroke}%
\end{pgfscope}%
\begin{pgfscope}%
\pgfsetbuttcap%
\pgfsetroundjoin%
\pgfsetlinewidth{1.003750pt}%
\definecolor{currentstroke}{rgb}{0.000000,0.000000,0.000000}%
\pgfsetstrokecolor{currentstroke}%
\pgfsetdash{}{0pt}%
\pgfpathmoveto{\pgfqpoint{1.000000in}{3.218182in}}%
\pgfpathlineto{\pgfqpoint{1.000000in}{5.400000in}}%
\pgfusepath{stroke}%
\end{pgfscope}%
\begin{pgfscope}%
\pgftext[x=2.409091in,y=5.469444in,,base]{{\rmfamily\fontsize{14.400000}{17.280000}\selectfont e1}}%
\end{pgfscope}%
\begin{pgfscope}%
\pgfsetbuttcap%
\pgfsetroundjoin%
\definecolor{currentfill}{rgb}{1.000000,1.000000,1.000000}%
\pgfsetfillcolor{currentfill}%
\pgfsetlinewidth{0.000000pt}%
\definecolor{currentstroke}{rgb}{0.000000,0.000000,0.000000}%
\pgfsetstrokecolor{currentstroke}%
\pgfsetstrokeopacity{0.000000}%
\pgfsetdash{}{0pt}%
\pgfpathmoveto{\pgfqpoint{4.381818in}{3.218182in}}%
\pgfpathlineto{\pgfqpoint{7.200000in}{3.218182in}}%
\pgfpathlineto{\pgfqpoint{7.200000in}{5.400000in}}%
\pgfpathlineto{\pgfqpoint{4.381818in}{5.400000in}}%
\pgfpathclose%
\pgfusepath{fill}%
\end{pgfscope}%
\begin{pgfscope}%
\pgfpathrectangle{\pgfqpoint{4.381818in}{3.218182in}}{\pgfqpoint{2.818182in}{2.181818in}} %
\pgfusepath{clip}%
\pgfsetrectcap%
\pgfsetroundjoin%
\pgfsetlinewidth{1.003750pt}%
\definecolor{currentstroke}{rgb}{0.000000,0.000000,1.000000}%
\pgfsetstrokecolor{currentstroke}%
\pgfsetdash{}{0pt}%
\pgfpathmoveto{\pgfqpoint{4.381818in}{4.915152in}}%
\pgfpathlineto{\pgfqpoint{4.383227in}{4.888552in}}%
\pgfpathlineto{\pgfqpoint{4.386610in}{4.765386in}}%
\pgfpathlineto{\pgfqpoint{4.389992in}{4.676050in}}%
\pgfpathlineto{\pgfqpoint{4.390555in}{4.673245in}}%
\pgfpathlineto{\pgfqpoint{4.391119in}{4.674583in}}%
\pgfpathlineto{\pgfqpoint{4.392246in}{4.689433in}}%
\pgfpathlineto{\pgfqpoint{4.394501in}{4.758960in}}%
\pgfpathlineto{\pgfqpoint{4.399856in}{4.929053in}}%
\pgfpathlineto{\pgfqpoint{4.400420in}{4.931525in}}%
\pgfpathlineto{\pgfqpoint{4.400984in}{4.929827in}}%
\pgfpathlineto{\pgfqpoint{4.402111in}{4.914353in}}%
\pgfpathlineto{\pgfqpoint{4.404648in}{4.834225in}}%
\pgfpathlineto{\pgfqpoint{4.409439in}{4.688285in}}%
\pgfpathlineto{\pgfqpoint{4.410003in}{4.685276in}}%
\pgfpathlineto{\pgfqpoint{4.410567in}{4.686433in}}%
\pgfpathlineto{\pgfqpoint{4.411694in}{4.701064in}}%
\pgfpathlineto{\pgfqpoint{4.413949in}{4.770913in}}%
\pgfpathlineto{\pgfqpoint{4.419304in}{4.941495in}}%
\pgfpathlineto{\pgfqpoint{4.419867in}{4.943839in}}%
\pgfpathlineto{\pgfqpoint{4.420431in}{4.942013in}}%
\pgfpathlineto{\pgfqpoint{4.421840in}{4.920087in}}%
\pgfpathlineto{\pgfqpoint{4.424659in}{4.823840in}}%
\pgfpathlineto{\pgfqpoint{4.428605in}{4.703626in}}%
\pgfpathlineto{\pgfqpoint{4.429450in}{4.697693in}}%
\pgfpathlineto{\pgfqpoint{4.430014in}{4.698976in}}%
\pgfpathlineto{\pgfqpoint{4.431141in}{4.714054in}}%
\pgfpathlineto{\pgfqpoint{4.433396in}{4.785536in}}%
\pgfpathlineto{\pgfqpoint{4.438469in}{4.952407in}}%
\pgfpathlineto{\pgfqpoint{4.439315in}{4.956347in}}%
\pgfpathlineto{\pgfqpoint{4.439597in}{4.955567in}}%
\pgfpathlineto{\pgfqpoint{4.440724in}{4.942353in}}%
\pgfpathlineto{\pgfqpoint{4.442979in}{4.876597in}}%
\pgfpathlineto{\pgfqpoint{4.448334in}{4.712601in}}%
\pgfpathlineto{\pgfqpoint{4.448898in}{4.710363in}}%
\pgfpathlineto{\pgfqpoint{4.449461in}{4.712409in}}%
\pgfpathlineto{\pgfqpoint{4.450871in}{4.735953in}}%
\pgfpathlineto{\pgfqpoint{4.453689in}{4.840695in}}%
\pgfpathlineto{\pgfqpoint{4.457635in}{4.964976in}}%
\pgfpathlineto{\pgfqpoint{4.458480in}{4.969428in}}%
\pgfpathlineto{\pgfqpoint{4.458762in}{4.968786in}}%
\pgfpathlineto{\pgfqpoint{4.459890in}{4.956012in}}%
\pgfpathlineto{\pgfqpoint{4.462144in}{4.891013in}}%
\pgfpathlineto{\pgfqpoint{4.467499in}{4.725953in}}%
\pgfpathlineto{\pgfqpoint{4.468063in}{4.723347in}}%
\pgfpathlineto{\pgfqpoint{4.468627in}{4.725095in}}%
\pgfpathlineto{\pgfqpoint{4.469754in}{4.741707in}}%
\pgfpathlineto{\pgfqpoint{4.472009in}{4.818761in}}%
\pgfpathlineto{\pgfqpoint{4.476800in}{4.980285in}}%
\pgfpathlineto{\pgfqpoint{4.477364in}{4.983257in}}%
\pgfpathlineto{\pgfqpoint{4.477928in}{4.981825in}}%
\pgfpathlineto{\pgfqpoint{4.479055in}{4.966526in}}%
\pgfpathlineto{\pgfqpoint{4.481592in}{4.887284in}}%
\pgfpathlineto{\pgfqpoint{4.486383in}{4.740054in}}%
\pgfpathlineto{\pgfqpoint{4.487229in}{4.736452in}}%
\pgfpathlineto{\pgfqpoint{4.487511in}{4.737477in}}%
\pgfpathlineto{\pgfqpoint{4.488638in}{4.752942in}}%
\pgfpathlineto{\pgfqpoint{4.490893in}{4.831048in}}%
\pgfpathlineto{\pgfqpoint{4.495402in}{4.993211in}}%
\pgfpathlineto{\pgfqpoint{4.496248in}{4.997953in}}%
\pgfpathlineto{\pgfqpoint{4.496530in}{4.997206in}}%
\pgfpathlineto{\pgfqpoint{4.497657in}{4.983360in}}%
\pgfpathlineto{\pgfqpoint{4.499912in}{4.915886in}}%
\pgfpathlineto{\pgfqpoint{4.505267in}{4.751249in}}%
\pgfpathlineto{\pgfqpoint{4.505831in}{4.749158in}}%
\pgfpathlineto{\pgfqpoint{4.506112in}{4.749860in}}%
\pgfpathlineto{\pgfqpoint{4.507240in}{4.764734in}}%
\pgfpathlineto{\pgfqpoint{4.509213in}{4.833088in}}%
\pgfpathlineto{\pgfqpoint{4.514004in}{5.009971in}}%
\pgfpathlineto{\pgfqpoint{4.514568in}{5.012491in}}%
\pgfpathlineto{\pgfqpoint{4.515132in}{5.010137in}}%
\pgfpathlineto{\pgfqpoint{4.516541in}{4.985482in}}%
\pgfpathlineto{\pgfqpoint{4.519923in}{4.864867in}}%
\pgfpathlineto{\pgfqpoint{4.523587in}{4.762939in}}%
\pgfpathlineto{\pgfqpoint{4.524151in}{4.760794in}}%
\pgfpathlineto{\pgfqpoint{4.524432in}{4.761666in}}%
\pgfpathlineto{\pgfqpoint{4.525560in}{4.778890in}}%
\pgfpathlineto{\pgfqpoint{4.527815in}{4.871309in}}%
\pgfpathlineto{\pgfqpoint{4.531760in}{5.020689in}}%
\pgfpathlineto{\pgfqpoint{4.532324in}{5.023856in}}%
\pgfpathlineto{\pgfqpoint{4.532888in}{5.022125in}}%
\pgfpathlineto{\pgfqpoint{4.534297in}{4.999978in}}%
\pgfpathlineto{\pgfqpoint{4.537397in}{4.898329in}}%
\pgfpathlineto{\pgfqpoint{4.541625in}{4.768311in}}%
\pgfpathlineto{\pgfqpoint{4.541907in}{4.767784in}}%
\pgfpathlineto{\pgfqpoint{4.541907in}{4.767784in}}%
\pgfpathlineto{\pgfqpoint{4.541907in}{4.767784in}}%
\pgfpathlineto{\pgfqpoint{4.542471in}{4.772113in}}%
\pgfpathlineto{\pgfqpoint{4.543880in}{4.815058in}}%
\pgfpathlineto{\pgfqpoint{4.549235in}{5.020341in}}%
\pgfpathlineto{\pgfqpoint{4.549799in}{5.021694in}}%
\pgfpathlineto{\pgfqpoint{4.550362in}{5.020698in}}%
\pgfpathlineto{\pgfqpoint{4.551772in}{5.010740in}}%
\pgfpathlineto{\pgfqpoint{4.553181in}{4.984229in}}%
\pgfpathlineto{\pgfqpoint{4.555436in}{4.883809in}}%
\pgfpathlineto{\pgfqpoint{4.558254in}{4.771694in}}%
\pgfpathlineto{\pgfqpoint{4.558536in}{4.771893in}}%
\pgfpathlineto{\pgfqpoint{4.559381in}{4.789379in}}%
\pgfpathlineto{\pgfqpoint{4.567837in}{5.075280in}}%
\pgfpathlineto{\pgfqpoint{4.568400in}{5.066235in}}%
\pgfpathlineto{\pgfqpoint{4.570092in}{4.971954in}}%
\pgfpathlineto{\pgfqpoint{4.572628in}{4.859413in}}%
\pgfpathlineto{\pgfqpoint{4.574319in}{4.851282in}}%
\pgfpathlineto{\pgfqpoint{4.576856in}{4.788672in}}%
\pgfpathlineto{\pgfqpoint{4.577420in}{4.799721in}}%
\pgfpathlineto{\pgfqpoint{4.579111in}{4.925855in}}%
\pgfpathlineto{\pgfqpoint{4.581084in}{5.014028in}}%
\pgfpathlineto{\pgfqpoint{4.582493in}{5.003266in}}%
\pgfpathlineto{\pgfqpoint{4.582775in}{5.005219in}}%
\pgfpathlineto{\pgfqpoint{4.583902in}{5.039295in}}%
\pgfpathlineto{\pgfqpoint{4.585029in}{5.073010in}}%
\pgfpathlineto{\pgfqpoint{4.585593in}{5.065992in}}%
\pgfpathlineto{\pgfqpoint{4.587002in}{4.967346in}}%
\pgfpathlineto{\pgfqpoint{4.589257in}{4.847794in}}%
\pgfpathlineto{\pgfqpoint{4.590666in}{4.860661in}}%
\pgfpathlineto{\pgfqpoint{4.591230in}{4.853909in}}%
\pgfpathlineto{\pgfqpoint{4.593203in}{4.790805in}}%
\pgfpathlineto{\pgfqpoint{4.593767in}{4.797458in}}%
\pgfpathlineto{\pgfqpoint{4.595176in}{4.888073in}}%
\pgfpathlineto{\pgfqpoint{4.597712in}{5.018914in}}%
\pgfpathlineto{\pgfqpoint{4.598276in}{5.016972in}}%
\pgfpathlineto{\pgfqpoint{4.599122in}{5.012949in}}%
\pgfpathlineto{\pgfqpoint{4.599404in}{5.014207in}}%
\pgfpathlineto{\pgfqpoint{4.600813in}{5.041229in}}%
\pgfpathlineto{\pgfqpoint{4.601658in}{5.053962in}}%
\pgfpathlineto{\pgfqpoint{4.602222in}{5.049575in}}%
\pgfpathlineto{\pgfqpoint{4.603349in}{4.997204in}}%
\pgfpathlineto{\pgfqpoint{4.606732in}{4.816096in}}%
\pgfpathlineto{\pgfqpoint{4.607013in}{4.816268in}}%
\pgfpathlineto{\pgfqpoint{4.611241in}{4.842609in}}%
\pgfpathlineto{\pgfqpoint{4.612650in}{4.897614in}}%
\pgfpathlineto{\pgfqpoint{4.616314in}{5.062937in}}%
\pgfpathlineto{\pgfqpoint{4.616596in}{5.062448in}}%
\pgfpathlineto{\pgfqpoint{4.617442in}{5.046423in}}%
\pgfpathlineto{\pgfqpoint{4.626179in}{4.780287in}}%
\pgfpathlineto{\pgfqpoint{4.626461in}{4.780404in}}%
\pgfpathlineto{\pgfqpoint{4.627306in}{4.792541in}}%
\pgfpathlineto{\pgfqpoint{4.629279in}{4.874847in}}%
\pgfpathlineto{\pgfqpoint{4.632943in}{5.022305in}}%
\pgfpathlineto{\pgfqpoint{4.633507in}{5.026291in}}%
\pgfpathlineto{\pgfqpoint{4.634071in}{5.024634in}}%
\pgfpathlineto{\pgfqpoint{4.635480in}{5.001858in}}%
\pgfpathlineto{\pgfqpoint{4.639144in}{4.887916in}}%
\pgfpathlineto{\pgfqpoint{4.643372in}{4.764829in}}%
\pgfpathlineto{\pgfqpoint{4.643653in}{4.763620in}}%
\pgfpathlineto{\pgfqpoint{4.643935in}{4.763868in}}%
\pgfpathlineto{\pgfqpoint{4.643935in}{4.763868in}}%
\pgfpathlineto{\pgfqpoint{4.644781in}{4.773454in}}%
\pgfpathlineto{\pgfqpoint{4.646754in}{4.838879in}}%
\pgfpathlineto{\pgfqpoint{4.651545in}{5.004959in}}%
\pgfpathlineto{\pgfqpoint{4.651827in}{5.006085in}}%
\pgfpathlineto{\pgfqpoint{4.652109in}{5.005939in}}%
\pgfpathlineto{\pgfqpoint{4.652109in}{5.005939in}}%
\pgfpathlineto{\pgfqpoint{4.652954in}{4.998325in}}%
\pgfpathlineto{\pgfqpoint{4.654927in}{4.948374in}}%
\pgfpathlineto{\pgfqpoint{4.661692in}{4.748854in}}%
\pgfpathlineto{\pgfqpoint{4.661973in}{4.748205in}}%
\pgfpathlineto{\pgfqpoint{4.662255in}{4.748656in}}%
\pgfpathlineto{\pgfqpoint{4.663101in}{4.756768in}}%
\pgfpathlineto{\pgfqpoint{4.664792in}{4.801115in}}%
\pgfpathlineto{\pgfqpoint{4.670711in}{4.982869in}}%
\pgfpathlineto{\pgfqpoint{4.671274in}{4.980412in}}%
\pgfpathlineto{\pgfqpoint{4.672684in}{4.954122in}}%
\pgfpathlineto{\pgfqpoint{4.676629in}{4.808500in}}%
\pgfpathlineto{\pgfqpoint{4.679730in}{4.733184in}}%
\pgfpathlineto{\pgfqpoint{4.680575in}{4.728788in}}%
\pgfpathlineto{\pgfqpoint{4.681139in}{4.730292in}}%
\pgfpathlineto{\pgfqpoint{4.682266in}{4.744051in}}%
\pgfpathlineto{\pgfqpoint{4.684521in}{4.808864in}}%
\pgfpathlineto{\pgfqpoint{4.689313in}{4.950595in}}%
\pgfpathlineto{\pgfqpoint{4.689876in}{4.952628in}}%
\pgfpathlineto{\pgfqpoint{4.690158in}{4.951912in}}%
\pgfpathlineto{\pgfqpoint{4.691285in}{4.937500in}}%
\pgfpathlineto{\pgfqpoint{4.693540in}{4.864751in}}%
\pgfpathlineto{\pgfqpoint{4.698332in}{4.711079in}}%
\pgfpathlineto{\pgfqpoint{4.699459in}{4.703075in}}%
\pgfpathlineto{\pgfqpoint{4.699741in}{4.703263in}}%
\pgfpathlineto{\pgfqpoint{4.700586in}{4.708939in}}%
\pgfpathlineto{\pgfqpoint{4.702278in}{4.741063in}}%
\pgfpathlineto{\pgfqpoint{4.709042in}{4.917384in}}%
\pgfpathlineto{\pgfqpoint{4.709606in}{4.915746in}}%
\pgfpathlineto{\pgfqpoint{4.710733in}{4.900156in}}%
\pgfpathlineto{\pgfqpoint{4.712988in}{4.827982in}}%
\pgfpathlineto{\pgfqpoint{4.717779in}{4.678020in}}%
\pgfpathlineto{\pgfqpoint{4.718906in}{4.672207in}}%
\pgfpathlineto{\pgfqpoint{4.719188in}{4.673021in}}%
\pgfpathlineto{\pgfqpoint{4.720316in}{4.684962in}}%
\pgfpathlineto{\pgfqpoint{4.722570in}{4.741970in}}%
\pgfpathlineto{\pgfqpoint{4.727926in}{4.880195in}}%
\pgfpathlineto{\pgfqpoint{4.728489in}{4.881029in}}%
\pgfpathlineto{\pgfqpoint{4.728771in}{4.880041in}}%
\pgfpathlineto{\pgfqpoint{4.729898in}{4.866868in}}%
\pgfpathlineto{\pgfqpoint{4.732153in}{4.802083in}}%
\pgfpathlineto{\pgfqpoint{4.737508in}{4.640864in}}%
\pgfpathlineto{\pgfqpoint{4.738354in}{4.638043in}}%
\pgfpathlineto{\pgfqpoint{4.738636in}{4.638910in}}%
\pgfpathlineto{\pgfqpoint{4.739763in}{4.651210in}}%
\pgfpathlineto{\pgfqpoint{4.742018in}{4.710462in}}%
\pgfpathlineto{\pgfqpoint{4.747091in}{4.844948in}}%
\pgfpathlineto{\pgfqpoint{4.747655in}{4.846608in}}%
\pgfpathlineto{\pgfqpoint{4.747937in}{4.845999in}}%
\pgfpathlineto{\pgfqpoint{4.749064in}{4.834205in}}%
\pgfpathlineto{\pgfqpoint{4.751319in}{4.773155in}}%
\pgfpathlineto{\pgfqpoint{4.756956in}{4.603792in}}%
\pgfpathlineto{\pgfqpoint{4.757801in}{4.601205in}}%
\pgfpathlineto{\pgfqpoint{4.758083in}{4.602121in}}%
\pgfpathlineto{\pgfqpoint{4.759210in}{4.614376in}}%
\pgfpathlineto{\pgfqpoint{4.761465in}{4.673790in}}%
\pgfpathlineto{\pgfqpoint{4.766538in}{4.811409in}}%
\pgfpathlineto{\pgfqpoint{4.767102in}{4.812799in}}%
\pgfpathlineto{\pgfqpoint{4.767384in}{4.811988in}}%
\pgfpathlineto{\pgfqpoint{4.768511in}{4.798875in}}%
\pgfpathlineto{\pgfqpoint{4.770766in}{4.733924in}}%
\pgfpathlineto{\pgfqpoint{4.776403in}{4.563939in}}%
\pgfpathlineto{\pgfqpoint{4.776967in}{4.561806in}}%
\pgfpathlineto{\pgfqpoint{4.777530in}{4.563361in}}%
\pgfpathlineto{\pgfqpoint{4.778940in}{4.582170in}}%
\pgfpathlineto{\pgfqpoint{4.781758in}{4.666373in}}%
\pgfpathlineto{\pgfqpoint{4.785704in}{4.774249in}}%
\pgfpathlineto{\pgfqpoint{4.786550in}{4.777992in}}%
\pgfpathlineto{\pgfqpoint{4.786831in}{4.777192in}}%
\pgfpathlineto{\pgfqpoint{4.787959in}{4.763813in}}%
\pgfpathlineto{\pgfqpoint{4.790214in}{4.696203in}}%
\pgfpathlineto{\pgfqpoint{4.795569in}{4.525416in}}%
\pgfpathlineto{\pgfqpoint{4.796414in}{4.521849in}}%
\pgfpathlineto{\pgfqpoint{4.796696in}{4.522640in}}%
\pgfpathlineto{\pgfqpoint{4.797823in}{4.535135in}}%
\pgfpathlineto{\pgfqpoint{4.800078in}{4.594640in}}%
\pgfpathlineto{\pgfqpoint{4.805433in}{4.740451in}}%
\pgfpathlineto{\pgfqpoint{4.805997in}{4.741751in}}%
\pgfpathlineto{\pgfqpoint{4.806279in}{4.740875in}}%
\pgfpathlineto{\pgfqpoint{4.807406in}{4.727294in}}%
\pgfpathlineto{\pgfqpoint{4.809661in}{4.659303in}}%
\pgfpathlineto{\pgfqpoint{4.815016in}{4.487105in}}%
\pgfpathlineto{\pgfqpoint{4.815580in}{4.484591in}}%
\pgfpathlineto{\pgfqpoint{4.816143in}{4.486299in}}%
\pgfpathlineto{\pgfqpoint{4.817553in}{4.507852in}}%
\pgfpathlineto{\pgfqpoint{4.820653in}{4.608114in}}%
\pgfpathlineto{\pgfqpoint{4.824317in}{4.701672in}}%
\pgfpathlineto{\pgfqpoint{4.825163in}{4.706177in}}%
\pgfpathlineto{\pgfqpoint{4.825726in}{4.704470in}}%
\pgfpathlineto{\pgfqpoint{4.826854in}{4.689669in}}%
\pgfpathlineto{\pgfqpoint{4.829108in}{4.622175in}}%
\pgfpathlineto{\pgfqpoint{4.834463in}{4.455614in}}%
\pgfpathlineto{\pgfqpoint{4.835027in}{4.454152in}}%
\pgfpathlineto{\pgfqpoint{4.835309in}{4.455085in}}%
\pgfpathlineto{\pgfqpoint{4.836436in}{4.469492in}}%
\pgfpathlineto{\pgfqpoint{4.838691in}{4.537802in}}%
\pgfpathlineto{\pgfqpoint{4.843201in}{4.669267in}}%
\pgfpathlineto{\pgfqpoint{4.844328in}{4.675529in}}%
\pgfpathlineto{\pgfqpoint{4.844610in}{4.674963in}}%
\pgfpathlineto{\pgfqpoint{4.845737in}{4.664061in}}%
\pgfpathlineto{\pgfqpoint{4.847710in}{4.614960in}}%
\pgfpathlineto{\pgfqpoint{4.854193in}{4.432965in}}%
\pgfpathlineto{\pgfqpoint{4.854756in}{4.434996in}}%
\pgfpathlineto{\pgfqpoint{4.856166in}{4.458679in}}%
\pgfpathlineto{\pgfqpoint{4.859548in}{4.579213in}}%
\pgfpathlineto{\pgfqpoint{4.862648in}{4.652257in}}%
\pgfpathlineto{\pgfqpoint{4.863212in}{4.654425in}}%
\pgfpathlineto{\pgfqpoint{4.863775in}{4.653072in}}%
\pgfpathlineto{\pgfqpoint{4.864903in}{4.640417in}}%
\pgfpathlineto{\pgfqpoint{4.867158in}{4.581798in}}%
\pgfpathlineto{\pgfqpoint{4.872795in}{4.422744in}}%
\pgfpathlineto{\pgfqpoint{4.873358in}{4.420199in}}%
\pgfpathlineto{\pgfqpoint{4.873922in}{4.421660in}}%
\pgfpathlineto{\pgfqpoint{4.875049in}{4.436809in}}%
\pgfpathlineto{\pgfqpoint{4.877304in}{4.506455in}}%
\pgfpathlineto{\pgfqpoint{4.881814in}{4.638957in}}%
\pgfpathlineto{\pgfqpoint{4.882377in}{4.641563in}}%
\pgfpathlineto{\pgfqpoint{4.882941in}{4.640225in}}%
\pgfpathlineto{\pgfqpoint{4.884068in}{4.626652in}}%
\pgfpathlineto{\pgfqpoint{4.886323in}{4.565930in}}%
\pgfpathlineto{\pgfqpoint{4.891678in}{4.413364in}}%
\pgfpathlineto{\pgfqpoint{4.892524in}{4.409396in}}%
\pgfpathlineto{\pgfqpoint{4.892806in}{4.409827in}}%
\pgfpathlineto{\pgfqpoint{4.893651in}{4.416516in}}%
\pgfpathlineto{\pgfqpoint{4.895342in}{4.453071in}}%
\pgfpathlineto{\pgfqpoint{4.901825in}{4.630123in}}%
\pgfpathlineto{\pgfqpoint{4.902107in}{4.629759in}}%
\pgfpathlineto{\pgfqpoint{4.902952in}{4.622748in}}%
\pgfpathlineto{\pgfqpoint{4.904925in}{4.576859in}}%
\pgfpathlineto{\pgfqpoint{4.911689in}{4.392411in}}%
\pgfpathlineto{\pgfqpoint{4.912253in}{4.394294in}}%
\pgfpathlineto{\pgfqpoint{4.913662in}{4.414243in}}%
\pgfpathlineto{\pgfqpoint{4.916481in}{4.501832in}}%
\pgfpathlineto{\pgfqpoint{4.920427in}{4.609084in}}%
\pgfpathlineto{\pgfqpoint{4.921272in}{4.613110in}}%
\pgfpathlineto{\pgfqpoint{4.921554in}{4.612549in}}%
\pgfpathlineto{\pgfqpoint{4.922681in}{4.600896in}}%
\pgfpathlineto{\pgfqpoint{4.924936in}{4.539357in}}%
\pgfpathlineto{\pgfqpoint{4.930573in}{4.367246in}}%
\pgfpathlineto{\pgfqpoint{4.931137in}{4.365107in}}%
\pgfpathlineto{\pgfqpoint{4.931700in}{4.367129in}}%
\pgfpathlineto{\pgfqpoint{4.933110in}{4.389128in}}%
\pgfpathlineto{\pgfqpoint{4.936210in}{4.491470in}}%
\pgfpathlineto{\pgfqpoint{4.939592in}{4.580615in}}%
\pgfpathlineto{\pgfqpoint{4.940438in}{4.584323in}}%
\pgfpathlineto{\pgfqpoint{4.940720in}{4.583567in}}%
\pgfpathlineto{\pgfqpoint{4.941847in}{4.570980in}}%
\pgfpathlineto{\pgfqpoint{4.944102in}{4.508127in}}%
\pgfpathlineto{\pgfqpoint{4.950302in}{4.320671in}}%
\pgfpathlineto{\pgfqpoint{4.950584in}{4.319888in}}%
\pgfpathlineto{\pgfqpoint{4.950866in}{4.320193in}}%
\pgfpathlineto{\pgfqpoint{4.950866in}{4.320193in}}%
\pgfpathlineto{\pgfqpoint{4.951712in}{4.327658in}}%
\pgfpathlineto{\pgfqpoint{4.953403in}{4.369189in}}%
\pgfpathlineto{\pgfqpoint{4.959040in}{4.533168in}}%
\pgfpathlineto{\pgfqpoint{4.959603in}{4.530005in}}%
\pgfpathlineto{\pgfqpoint{4.961012in}{4.499791in}}%
\pgfpathlineto{\pgfqpoint{4.969750in}{4.234220in}}%
\pgfpathlineto{\pgfqpoint{4.970032in}{4.234751in}}%
\pgfpathlineto{\pgfqpoint{4.970877in}{4.244676in}}%
\pgfpathlineto{\pgfqpoint{4.972568in}{4.303059in}}%
\pgfpathlineto{\pgfqpoint{4.975950in}{4.442003in}}%
\pgfpathlineto{\pgfqpoint{4.976232in}{4.441703in}}%
\pgfpathlineto{\pgfqpoint{4.977078in}{4.421370in}}%
\pgfpathlineto{\pgfqpoint{4.981024in}{4.293971in}}%
\pgfpathlineto{\pgfqpoint{4.981869in}{4.278346in}}%
\pgfpathlineto{\pgfqpoint{4.983278in}{4.176483in}}%
\pgfpathlineto{\pgfqpoint{4.984969in}{4.072950in}}%
\pgfpathlineto{\pgfqpoint{4.985251in}{4.078392in}}%
\pgfpathlineto{\pgfqpoint{4.986942in}{4.199820in}}%
\pgfpathlineto{\pgfqpoint{4.988070in}{4.238581in}}%
\pgfpathlineto{\pgfqpoint{4.988352in}{4.236471in}}%
\pgfpathlineto{\pgfqpoint{4.989479in}{4.223213in}}%
\pgfpathlineto{\pgfqpoint{4.989761in}{4.226575in}}%
\pgfpathlineto{\pgfqpoint{4.990888in}{4.288208in}}%
\pgfpathlineto{\pgfqpoint{4.993143in}{4.426833in}}%
\pgfpathlineto{\pgfqpoint{4.993425in}{4.424238in}}%
\pgfpathlineto{\pgfqpoint{4.994834in}{4.367605in}}%
\pgfpathlineto{\pgfqpoint{4.997934in}{4.261062in}}%
\pgfpathlineto{\pgfqpoint{5.002162in}{4.200232in}}%
\pgfpathlineto{\pgfqpoint{5.002726in}{4.203691in}}%
\pgfpathlineto{\pgfqpoint{5.003853in}{4.230825in}}%
\pgfpathlineto{\pgfqpoint{5.006390in}{4.370001in}}%
\pgfpathlineto{\pgfqpoint{5.009208in}{4.483959in}}%
\pgfpathlineto{\pgfqpoint{5.010054in}{4.489117in}}%
\pgfpathlineto{\pgfqpoint{5.010336in}{4.488111in}}%
\pgfpathlineto{\pgfqpoint{5.011463in}{4.472864in}}%
\pgfpathlineto{\pgfqpoint{5.014281in}{4.389544in}}%
\pgfpathlineto{\pgfqpoint{5.018791in}{4.274020in}}%
\pgfpathlineto{\pgfqpoint{5.019355in}{4.271287in}}%
\pgfpathlineto{\pgfqpoint{5.019918in}{4.272599in}}%
\pgfpathlineto{\pgfqpoint{5.021046in}{4.287944in}}%
\pgfpathlineto{\pgfqpoint{5.023301in}{4.364227in}}%
\pgfpathlineto{\pgfqpoint{5.028092in}{4.530403in}}%
\pgfpathlineto{\pgfqpoint{5.028656in}{4.532751in}}%
\pgfpathlineto{\pgfqpoint{5.029219in}{4.530553in}}%
\pgfpathlineto{\pgfqpoint{5.030629in}{4.507147in}}%
\pgfpathlineto{\pgfqpoint{5.033729in}{4.400776in}}%
\pgfpathlineto{\pgfqpoint{5.037393in}{4.301167in}}%
\pgfpathlineto{\pgfqpoint{5.038238in}{4.296641in}}%
\pgfpathlineto{\pgfqpoint{5.038802in}{4.298491in}}%
\pgfpathlineto{\pgfqpoint{5.040211in}{4.319849in}}%
\pgfpathlineto{\pgfqpoint{5.043030in}{4.416276in}}%
\pgfpathlineto{\pgfqpoint{5.046976in}{4.539923in}}%
\pgfpathlineto{\pgfqpoint{5.047821in}{4.545413in}}%
\pgfpathlineto{\pgfqpoint{5.048385in}{4.543713in}}%
\pgfpathlineto{\pgfqpoint{5.049512in}{4.528060in}}%
\pgfpathlineto{\pgfqpoint{5.052049in}{4.447859in}}%
\pgfpathlineto{\pgfqpoint{5.056558in}{4.307762in}}%
\pgfpathlineto{\pgfqpoint{5.057404in}{4.302487in}}%
\pgfpathlineto{\pgfqpoint{5.057968in}{4.304046in}}%
\pgfpathlineto{\pgfqpoint{5.059095in}{4.319028in}}%
\pgfpathlineto{\pgfqpoint{5.061350in}{4.387290in}}%
\pgfpathlineto{\pgfqpoint{5.066705in}{4.552885in}}%
\pgfpathlineto{\pgfqpoint{5.067269in}{4.555090in}}%
\pgfpathlineto{\pgfqpoint{5.067832in}{4.553130in}}%
\pgfpathlineto{\pgfqpoint{5.069241in}{4.530955in}}%
\pgfpathlineto{\pgfqpoint{5.072060in}{4.435780in}}%
\pgfpathlineto{\pgfqpoint{5.076006in}{4.322982in}}%
\pgfpathlineto{\pgfqpoint{5.076851in}{4.319515in}}%
\pgfpathlineto{\pgfqpoint{5.077133in}{4.320460in}}%
\pgfpathlineto{\pgfqpoint{5.078261in}{4.334527in}}%
\pgfpathlineto{\pgfqpoint{5.080515in}{4.403375in}}%
\pgfpathlineto{\pgfqpoint{5.086152in}{4.585143in}}%
\pgfpathlineto{\pgfqpoint{5.086716in}{4.587759in}}%
\pgfpathlineto{\pgfqpoint{5.087280in}{4.586320in}}%
\pgfpathlineto{\pgfqpoint{5.088407in}{4.571758in}}%
\pgfpathlineto{\pgfqpoint{5.090944in}{4.496361in}}%
\pgfpathlineto{\pgfqpoint{5.095171in}{4.376010in}}%
\pgfpathlineto{\pgfqpoint{5.096017in}{4.371910in}}%
\pgfpathlineto{\pgfqpoint{5.096299in}{4.372650in}}%
\pgfpathlineto{\pgfqpoint{5.097426in}{4.386263in}}%
\pgfpathlineto{\pgfqpoint{5.099681in}{4.457342in}}%
\pgfpathlineto{\pgfqpoint{5.105600in}{4.659344in}}%
\pgfpathlineto{\pgfqpoint{5.106445in}{4.662381in}}%
\pgfpathlineto{\pgfqpoint{5.106727in}{4.661453in}}%
\pgfpathlineto{\pgfqpoint{5.107854in}{4.648625in}}%
\pgfpathlineto{\pgfqpoint{5.110109in}{4.589140in}}%
\pgfpathlineto{\pgfqpoint{5.114619in}{4.472922in}}%
\pgfpathlineto{\pgfqpoint{5.115182in}{4.471139in}}%
\pgfpathlineto{\pgfqpoint{5.115464in}{4.471854in}}%
\pgfpathlineto{\pgfqpoint{5.116592in}{4.485865in}}%
\pgfpathlineto{\pgfqpoint{5.118565in}{4.550356in}}%
\pgfpathlineto{\pgfqpoint{5.124765in}{4.781248in}}%
\pgfpathlineto{\pgfqpoint{5.125611in}{4.784510in}}%
\pgfpathlineto{\pgfqpoint{5.125893in}{4.783720in}}%
\pgfpathlineto{\pgfqpoint{5.127020in}{4.772159in}}%
\pgfpathlineto{\pgfqpoint{5.129557in}{4.711323in}}%
\pgfpathlineto{\pgfqpoint{5.133784in}{4.616437in}}%
\pgfpathlineto{\pgfqpoint{5.134066in}{4.615789in}}%
\pgfpathlineto{\pgfqpoint{5.134348in}{4.616297in}}%
\pgfpathlineto{\pgfqpoint{5.135194in}{4.625312in}}%
\pgfpathlineto{\pgfqpoint{5.136885in}{4.679288in}}%
\pgfpathlineto{\pgfqpoint{5.143649in}{4.950436in}}%
\pgfpathlineto{\pgfqpoint{5.144494in}{4.945392in}}%
\pgfpathlineto{\pgfqpoint{5.147031in}{4.905381in}}%
\pgfpathlineto{\pgfqpoint{5.150695in}{4.814815in}}%
\pgfpathlineto{\pgfqpoint{5.152104in}{4.792201in}}%
\pgfpathlineto{\pgfqpoint{5.152386in}{4.792905in}}%
\pgfpathlineto{\pgfqpoint{5.153232in}{4.811841in}}%
\pgfpathlineto{\pgfqpoint{5.154923in}{4.931882in}}%
\pgfpathlineto{\pgfqpoint{5.157459in}{5.075651in}}%
\pgfpathlineto{\pgfqpoint{5.157741in}{5.076629in}}%
\pgfpathlineto{\pgfqpoint{5.158023in}{5.075934in}}%
\pgfpathlineto{\pgfqpoint{5.158587in}{5.073298in}}%
\pgfpathlineto{\pgfqpoint{5.158869in}{5.073488in}}%
\pgfpathlineto{\pgfqpoint{5.159714in}{5.091621in}}%
\pgfpathlineto{\pgfqpoint{5.161687in}{5.182089in}}%
\pgfpathlineto{\pgfqpoint{5.162251in}{5.167213in}}%
\pgfpathlineto{\pgfqpoint{5.165069in}{4.940538in}}%
\pgfpathlineto{\pgfqpoint{5.166197in}{4.960634in}}%
\pgfpathlineto{\pgfqpoint{5.167042in}{4.970532in}}%
\pgfpathlineto{\pgfqpoint{5.167324in}{4.966527in}}%
\pgfpathlineto{\pgfqpoint{5.169861in}{4.876372in}}%
\pgfpathlineto{\pgfqpoint{5.170706in}{4.900870in}}%
\pgfpathlineto{\pgfqpoint{5.174652in}{5.069814in}}%
\pgfpathlineto{\pgfqpoint{5.174934in}{5.068233in}}%
\pgfpathlineto{\pgfqpoint{5.176343in}{5.037812in}}%
\pgfpathlineto{\pgfqpoint{5.181416in}{4.851105in}}%
\pgfpathlineto{\pgfqpoint{5.185362in}{4.703438in}}%
\pgfpathlineto{\pgfqpoint{5.185926in}{4.701161in}}%
\pgfpathlineto{\pgfqpoint{5.186208in}{4.702310in}}%
\pgfpathlineto{\pgfqpoint{5.187335in}{4.720191in}}%
\pgfpathlineto{\pgfqpoint{5.193254in}{4.868825in}}%
\pgfpathlineto{\pgfqpoint{5.193818in}{4.866071in}}%
\pgfpathlineto{\pgfqpoint{5.195227in}{4.841550in}}%
\pgfpathlineto{\pgfqpoint{5.198045in}{4.740067in}}%
\pgfpathlineto{\pgfqpoint{5.203400in}{4.553651in}}%
\pgfpathlineto{\pgfqpoint{5.204528in}{4.547376in}}%
\pgfpathlineto{\pgfqpoint{5.204810in}{4.548383in}}%
\pgfpathlineto{\pgfqpoint{5.205937in}{4.562145in}}%
\pgfpathlineto{\pgfqpoint{5.208474in}{4.631842in}}%
\pgfpathlineto{\pgfqpoint{5.212138in}{4.715187in}}%
\pgfpathlineto{\pgfqpoint{5.212701in}{4.716673in}}%
\pgfpathlineto{\pgfqpoint{5.212983in}{4.715894in}}%
\pgfpathlineto{\pgfqpoint{5.214111in}{4.702587in}}%
\pgfpathlineto{\pgfqpoint{5.216365in}{4.634915in}}%
\pgfpathlineto{\pgfqpoint{5.222566in}{4.427550in}}%
\pgfpathlineto{\pgfqpoint{5.223411in}{4.423438in}}%
\pgfpathlineto{\pgfqpoint{5.223693in}{4.423943in}}%
\pgfpathlineto{\pgfqpoint{5.224821in}{4.435008in}}%
\pgfpathlineto{\pgfqpoint{5.227075in}{4.491563in}}%
\pgfpathlineto{\pgfqpoint{5.231585in}{4.600826in}}%
\pgfpathlineto{\pgfqpoint{5.232149in}{4.602875in}}%
\pgfpathlineto{\pgfqpoint{5.232712in}{4.601266in}}%
\pgfpathlineto{\pgfqpoint{5.233840in}{4.586871in}}%
\pgfpathlineto{\pgfqpoint{5.236095in}{4.520081in}}%
\pgfpathlineto{\pgfqpoint{5.241731in}{4.342536in}}%
\pgfpathlineto{\pgfqpoint{5.242577in}{4.338474in}}%
\pgfpathlineto{\pgfqpoint{5.242859in}{4.339067in}}%
\pgfpathlineto{\pgfqpoint{5.243986in}{4.350850in}}%
\pgfpathlineto{\pgfqpoint{5.246241in}{4.410339in}}%
\pgfpathlineto{\pgfqpoint{5.250751in}{4.530985in}}%
\pgfpathlineto{\pgfqpoint{5.251596in}{4.534662in}}%
\pgfpathlineto{\pgfqpoint{5.251878in}{4.534144in}}%
\pgfpathlineto{\pgfqpoint{5.253005in}{4.523506in}}%
\pgfpathlineto{\pgfqpoint{5.254978in}{4.475274in}}%
\pgfpathlineto{\pgfqpoint{5.261461in}{4.297776in}}%
\pgfpathlineto{\pgfqpoint{5.261743in}{4.297387in}}%
\pgfpathlineto{\pgfqpoint{5.262024in}{4.297955in}}%
\pgfpathlineto{\pgfqpoint{5.263152in}{4.309773in}}%
\pgfpathlineto{\pgfqpoint{5.265407in}{4.371393in}}%
\pgfpathlineto{\pgfqpoint{5.270198in}{4.506875in}}%
\pgfpathlineto{\pgfqpoint{5.271043in}{4.509599in}}%
\pgfpathlineto{\pgfqpoint{5.271325in}{4.508637in}}%
\pgfpathlineto{\pgfqpoint{5.272453in}{4.496066in}}%
\pgfpathlineto{\pgfqpoint{5.274707in}{4.437259in}}%
\pgfpathlineto{\pgfqpoint{5.280063in}{4.290643in}}%
\pgfpathlineto{\pgfqpoint{5.280908in}{4.288068in}}%
\pgfpathlineto{\pgfqpoint{5.281190in}{4.288942in}}%
\pgfpathlineto{\pgfqpoint{5.282317in}{4.301176in}}%
\pgfpathlineto{\pgfqpoint{5.284572in}{4.362229in}}%
\pgfpathlineto{\pgfqpoint{5.289645in}{4.506576in}}%
\pgfpathlineto{\pgfqpoint{5.290209in}{4.509121in}}%
\pgfpathlineto{\pgfqpoint{5.290773in}{4.507805in}}%
\pgfpathlineto{\pgfqpoint{5.291900in}{4.494103in}}%
\pgfpathlineto{\pgfqpoint{5.294437in}{4.422626in}}%
\pgfpathlineto{\pgfqpoint{5.299228in}{4.289315in}}%
\pgfpathlineto{\pgfqpoint{5.299792in}{4.286751in}}%
\pgfpathlineto{\pgfqpoint{5.300355in}{4.287904in}}%
\pgfpathlineto{\pgfqpoint{5.301483in}{4.300854in}}%
\pgfpathlineto{\pgfqpoint{5.303738in}{4.361535in}}%
\pgfpathlineto{\pgfqpoint{5.308811in}{4.506298in}}%
\pgfpathlineto{\pgfqpoint{5.309656in}{4.509976in}}%
\pgfpathlineto{\pgfqpoint{5.309938in}{4.509383in}}%
\pgfpathlineto{\pgfqpoint{5.311066in}{4.498051in}}%
\pgfpathlineto{\pgfqpoint{5.313320in}{4.439062in}}%
\pgfpathlineto{\pgfqpoint{5.318957in}{4.276943in}}%
\pgfpathlineto{\pgfqpoint{5.319521in}{4.276013in}}%
\pgfpathlineto{\pgfqpoint{5.319803in}{4.277129in}}%
\pgfpathlineto{\pgfqpoint{5.320930in}{4.291634in}}%
\pgfpathlineto{\pgfqpoint{5.323467in}{4.368078in}}%
\pgfpathlineto{\pgfqpoint{5.327695in}{4.491107in}}%
\pgfpathlineto{\pgfqpoint{5.328540in}{4.495363in}}%
\pgfpathlineto{\pgfqpoint{5.328822in}{4.494809in}}%
\pgfpathlineto{\pgfqpoint{5.329949in}{4.483262in}}%
\pgfpathlineto{\pgfqpoint{5.332204in}{4.424028in}}%
\pgfpathlineto{\pgfqpoint{5.338405in}{4.239143in}}%
\pgfpathlineto{\pgfqpoint{5.338968in}{4.238076in}}%
\pgfpathlineto{\pgfqpoint{5.339250in}{4.239265in}}%
\pgfpathlineto{\pgfqpoint{5.340378in}{4.255521in}}%
\pgfpathlineto{\pgfqpoint{5.342914in}{4.343240in}}%
\pgfpathlineto{\pgfqpoint{5.346296in}{4.447773in}}%
\pgfpathlineto{\pgfqpoint{5.346860in}{4.449439in}}%
\pgfpathlineto{\pgfqpoint{5.347142in}{4.447932in}}%
\pgfpathlineto{\pgfqpoint{5.348269in}{4.427197in}}%
\pgfpathlineto{\pgfqpoint{5.352497in}{4.269407in}}%
\pgfpathlineto{\pgfqpoint{5.357570in}{4.134499in}}%
\pgfpathlineto{\pgfqpoint{5.358416in}{4.148595in}}%
\pgfpathlineto{\pgfqpoint{5.360107in}{4.236090in}}%
\pgfpathlineto{\pgfqpoint{5.362362in}{4.345248in}}%
\pgfpathlineto{\pgfqpoint{5.362644in}{4.344648in}}%
\pgfpathlineto{\pgfqpoint{5.363771in}{4.309837in}}%
\pgfpathlineto{\pgfqpoint{5.364898in}{4.279287in}}%
\pgfpathlineto{\pgfqpoint{5.365462in}{4.283294in}}%
\pgfpathlineto{\pgfqpoint{5.366871in}{4.314719in}}%
\pgfpathlineto{\pgfqpoint{5.367435in}{4.304899in}}%
\pgfpathlineto{\pgfqpoint{5.368844in}{4.181980in}}%
\pgfpathlineto{\pgfqpoint{5.370535in}{4.072237in}}%
\pgfpathlineto{\pgfqpoint{5.370817in}{4.075029in}}%
\pgfpathlineto{\pgfqpoint{5.373072in}{4.149904in}}%
\pgfpathlineto{\pgfqpoint{5.374199in}{4.140564in}}%
\pgfpathlineto{\pgfqpoint{5.374763in}{4.138011in}}%
\pgfpathlineto{\pgfqpoint{5.375045in}{4.140272in}}%
\pgfpathlineto{\pgfqpoint{5.375890in}{4.167134in}}%
\pgfpathlineto{\pgfqpoint{5.380118in}{4.400760in}}%
\pgfpathlineto{\pgfqpoint{5.380682in}{4.395099in}}%
\pgfpathlineto{\pgfqpoint{5.383218in}{4.329531in}}%
\pgfpathlineto{\pgfqpoint{5.389137in}{4.193818in}}%
\pgfpathlineto{\pgfqpoint{5.389419in}{4.192947in}}%
\pgfpathlineto{\pgfqpoint{5.389701in}{4.193365in}}%
\pgfpathlineto{\pgfqpoint{5.390546in}{4.202949in}}%
\pgfpathlineto{\pgfqpoint{5.392237in}{4.259698in}}%
\pgfpathlineto{\pgfqpoint{5.397592in}{4.469965in}}%
\pgfpathlineto{\pgfqpoint{5.397874in}{4.470831in}}%
\pgfpathlineto{\pgfqpoint{5.398156in}{4.470514in}}%
\pgfpathlineto{\pgfqpoint{5.398156in}{4.470514in}}%
\pgfpathlineto{\pgfqpoint{5.399002in}{4.463020in}}%
\pgfpathlineto{\pgfqpoint{5.400975in}{4.416256in}}%
\pgfpathlineto{\pgfqpoint{5.407175in}{4.249378in}}%
\pgfpathlineto{\pgfqpoint{5.407457in}{4.248511in}}%
\pgfpathlineto{\pgfqpoint{5.407739in}{4.248634in}}%
\pgfpathlineto{\pgfqpoint{5.407739in}{4.248634in}}%
\pgfpathlineto{\pgfqpoint{5.408584in}{4.255069in}}%
\pgfpathlineto{\pgfqpoint{5.410276in}{4.294216in}}%
\pgfpathlineto{\pgfqpoint{5.416758in}{4.497433in}}%
\pgfpathlineto{\pgfqpoint{5.417322in}{4.496067in}}%
\pgfpathlineto{\pgfqpoint{5.418449in}{4.481115in}}%
\pgfpathlineto{\pgfqpoint{5.420986in}{4.403276in}}%
\pgfpathlineto{\pgfqpoint{5.425777in}{4.262048in}}%
\pgfpathlineto{\pgfqpoint{5.426623in}{4.258238in}}%
\pgfpathlineto{\pgfqpoint{5.426905in}{4.258905in}}%
\pgfpathlineto{\pgfqpoint{5.428032in}{4.271136in}}%
\pgfpathlineto{\pgfqpoint{5.430287in}{4.334319in}}%
\pgfpathlineto{\pgfqpoint{5.435642in}{4.496307in}}%
\pgfpathlineto{\pgfqpoint{5.436205in}{4.498066in}}%
\pgfpathlineto{\pgfqpoint{5.436487in}{4.497379in}}%
\pgfpathlineto{\pgfqpoint{5.437615in}{4.484552in}}%
\pgfpathlineto{\pgfqpoint{5.439869in}{4.419442in}}%
\pgfpathlineto{\pgfqpoint{5.445225in}{4.256708in}}%
\pgfpathlineto{\pgfqpoint{5.445788in}{4.254477in}}%
\pgfpathlineto{\pgfqpoint{5.446352in}{4.256290in}}%
\pgfpathlineto{\pgfqpoint{5.447761in}{4.277738in}}%
\pgfpathlineto{\pgfqpoint{5.450580in}{4.372502in}}%
\pgfpathlineto{\pgfqpoint{5.454807in}{4.498175in}}%
\pgfpathlineto{\pgfqpoint{5.455653in}{4.502305in}}%
\pgfpathlineto{\pgfqpoint{5.455935in}{4.501635in}}%
\pgfpathlineto{\pgfqpoint{5.457062in}{4.488993in}}%
\pgfpathlineto{\pgfqpoint{5.459317in}{4.424988in}}%
\pgfpathlineto{\pgfqpoint{5.464390in}{4.273462in}}%
\pgfpathlineto{\pgfqpoint{5.464954in}{4.270743in}}%
\pgfpathlineto{\pgfqpoint{5.465517in}{4.272195in}}%
\pgfpathlineto{\pgfqpoint{5.466645in}{4.287353in}}%
\pgfpathlineto{\pgfqpoint{5.468900in}{4.357474in}}%
\pgfpathlineto{\pgfqpoint{5.474537in}{4.538272in}}%
\pgfpathlineto{\pgfqpoint{5.475100in}{4.540835in}}%
\pgfpathlineto{\pgfqpoint{5.475664in}{4.539409in}}%
\pgfpathlineto{\pgfqpoint{5.476791in}{4.525108in}}%
\pgfpathlineto{\pgfqpoint{5.479328in}{4.451602in}}%
\pgfpathlineto{\pgfqpoint{5.483556in}{4.338038in}}%
\pgfpathlineto{\pgfqpoint{5.484119in}{4.335455in}}%
\pgfpathlineto{\pgfqpoint{5.484683in}{4.337109in}}%
\pgfpathlineto{\pgfqpoint{5.485810in}{4.353261in}}%
\pgfpathlineto{\pgfqpoint{5.488065in}{4.428824in}}%
\pgfpathlineto{\pgfqpoint{5.493702in}{4.630649in}}%
\pgfpathlineto{\pgfqpoint{5.494829in}{4.636865in}}%
\pgfpathlineto{\pgfqpoint{5.495111in}{4.636054in}}%
\pgfpathlineto{\pgfqpoint{5.496239in}{4.623999in}}%
\pgfpathlineto{\pgfqpoint{5.498493in}{4.567454in}}%
\pgfpathlineto{\pgfqpoint{5.502721in}{4.463759in}}%
\pgfpathlineto{\pgfqpoint{5.503285in}{4.462024in}}%
\pgfpathlineto{\pgfqpoint{5.503567in}{4.462818in}}%
\pgfpathlineto{\pgfqpoint{5.504694in}{4.477755in}}%
\pgfpathlineto{\pgfqpoint{5.506667in}{4.546991in}}%
\pgfpathlineto{\pgfqpoint{5.512868in}{4.792046in}}%
\pgfpathlineto{\pgfqpoint{5.513713in}{4.796057in}}%
\pgfpathlineto{\pgfqpoint{5.513995in}{4.795619in}}%
\pgfpathlineto{\pgfqpoint{5.515122in}{4.786207in}}%
\pgfpathlineto{\pgfqpoint{5.517377in}{4.740440in}}%
\pgfpathlineto{\pgfqpoint{5.521887in}{4.648937in}}%
\pgfpathlineto{\pgfqpoint{5.522732in}{4.657372in}}%
\pgfpathlineto{\pgfqpoint{5.524142in}{4.707406in}}%
\pgfpathlineto{\pgfqpoint{5.530342in}{4.995648in}}%
\pgfpathlineto{\pgfqpoint{5.531470in}{4.998379in}}%
\pgfpathlineto{\pgfqpoint{5.533442in}{5.018679in}}%
\pgfpathlineto{\pgfqpoint{5.534006in}{5.015303in}}%
\pgfpathlineto{\pgfqpoint{5.535134in}{4.971598in}}%
\pgfpathlineto{\pgfqpoint{5.537670in}{4.831771in}}%
\pgfpathlineto{\pgfqpoint{5.537952in}{4.838750in}}%
\pgfpathlineto{\pgfqpoint{5.540207in}{4.968284in}}%
\pgfpathlineto{\pgfqpoint{5.541052in}{4.953775in}}%
\pgfpathlineto{\pgfqpoint{5.541898in}{4.932826in}}%
\pgfpathlineto{\pgfqpoint{5.542462in}{4.941645in}}%
\pgfpathlineto{\pgfqpoint{5.545562in}{5.121633in}}%
\pgfpathlineto{\pgfqpoint{5.546407in}{5.094656in}}%
\pgfpathlineto{\pgfqpoint{5.553172in}{4.758547in}}%
\pgfpathlineto{\pgfqpoint{5.555708in}{4.670893in}}%
\pgfpathlineto{\pgfqpoint{5.555990in}{4.670761in}}%
\pgfpathlineto{\pgfqpoint{5.556836in}{4.681328in}}%
\pgfpathlineto{\pgfqpoint{5.559372in}{4.764150in}}%
\pgfpathlineto{\pgfqpoint{5.561909in}{4.819269in}}%
\pgfpathlineto{\pgfqpoint{5.562191in}{4.819768in}}%
\pgfpathlineto{\pgfqpoint{5.562473in}{4.818992in}}%
\pgfpathlineto{\pgfqpoint{5.563600in}{4.803811in}}%
\pgfpathlineto{\pgfqpoint{5.565855in}{4.729879in}}%
\pgfpathlineto{\pgfqpoint{5.573465in}{4.446736in}}%
\pgfpathlineto{\pgfqpoint{5.574028in}{4.445709in}}%
\pgfpathlineto{\pgfqpoint{5.574310in}{4.446869in}}%
\pgfpathlineto{\pgfqpoint{5.575438in}{4.461717in}}%
\pgfpathlineto{\pgfqpoint{5.578820in}{4.555933in}}%
\pgfpathlineto{\pgfqpoint{5.581356in}{4.595476in}}%
\pgfpathlineto{\pgfqpoint{5.581920in}{4.593801in}}%
\pgfpathlineto{\pgfqpoint{5.583047in}{4.577412in}}%
\pgfpathlineto{\pgfqpoint{5.585302in}{4.502143in}}%
\pgfpathlineto{\pgfqpoint{5.591503in}{4.280587in}}%
\pgfpathlineto{\pgfqpoint{5.592630in}{4.274336in}}%
\pgfpathlineto{\pgfqpoint{5.592912in}{4.275095in}}%
\pgfpathlineto{\pgfqpoint{5.594039in}{4.287079in}}%
\pgfpathlineto{\pgfqpoint{5.596294in}{4.344564in}}%
\pgfpathlineto{\pgfqpoint{5.600522in}{4.444059in}}%
\pgfpathlineto{\pgfqpoint{5.601086in}{4.445406in}}%
\pgfpathlineto{\pgfqpoint{5.601367in}{4.444579in}}%
\pgfpathlineto{\pgfqpoint{5.602495in}{4.430985in}}%
\pgfpathlineto{\pgfqpoint{5.604750in}{4.361192in}}%
\pgfpathlineto{\pgfqpoint{5.610386in}{4.177749in}}%
\pgfpathlineto{\pgfqpoint{5.611232in}{4.173077in}}%
\pgfpathlineto{\pgfqpoint{5.611796in}{4.174492in}}%
\pgfpathlineto{\pgfqpoint{5.612923in}{4.187497in}}%
\pgfpathlineto{\pgfqpoint{5.615460in}{4.254206in}}%
\pgfpathlineto{\pgfqpoint{5.619687in}{4.363315in}}%
\pgfpathlineto{\pgfqpoint{5.620533in}{4.366746in}}%
\pgfpathlineto{\pgfqpoint{5.620815in}{4.365956in}}%
\pgfpathlineto{\pgfqpoint{5.621942in}{4.352798in}}%
\pgfpathlineto{\pgfqpoint{5.623915in}{4.294315in}}%
\pgfpathlineto{\pgfqpoint{5.629270in}{4.122209in}}%
\pgfpathlineto{\pgfqpoint{5.630116in}{4.118538in}}%
\pgfpathlineto{\pgfqpoint{5.630398in}{4.119151in}}%
\pgfpathlineto{\pgfqpoint{5.631525in}{4.130173in}}%
\pgfpathlineto{\pgfqpoint{5.634062in}{4.190835in}}%
\pgfpathlineto{\pgfqpoint{5.639135in}{4.309094in}}%
\pgfpathlineto{\pgfqpoint{5.639699in}{4.310192in}}%
\pgfpathlineto{\pgfqpoint{5.639980in}{4.309266in}}%
\pgfpathlineto{\pgfqpoint{5.641108in}{4.295085in}}%
\pgfpathlineto{\pgfqpoint{5.643081in}{4.229349in}}%
\pgfpathlineto{\pgfqpoint{5.648154in}{4.037356in}}%
\pgfpathlineto{\pgfqpoint{5.648436in}{4.036457in}}%
\pgfpathlineto{\pgfqpoint{5.648718in}{4.036815in}}%
\pgfpathlineto{\pgfqpoint{5.648718in}{4.036815in}}%
\pgfpathlineto{\pgfqpoint{5.649563in}{4.044768in}}%
\pgfpathlineto{\pgfqpoint{5.651818in}{4.098350in}}%
\pgfpathlineto{\pgfqpoint{5.655200in}{4.164326in}}%
\pgfpathlineto{\pgfqpoint{5.658019in}{4.188766in}}%
\pgfpathlineto{\pgfqpoint{5.658300in}{4.188336in}}%
\pgfpathlineto{\pgfqpoint{5.659146in}{4.178929in}}%
\pgfpathlineto{\pgfqpoint{5.660555in}{4.120164in}}%
\pgfpathlineto{\pgfqpoint{5.664219in}{3.858427in}}%
\pgfpathlineto{\pgfqpoint{5.665065in}{3.871815in}}%
\pgfpathlineto{\pgfqpoint{5.666756in}{3.929119in}}%
\pgfpathlineto{\pgfqpoint{5.667319in}{3.923734in}}%
\pgfpathlineto{\pgfqpoint{5.669011in}{3.860776in}}%
\pgfpathlineto{\pgfqpoint{5.669574in}{3.878700in}}%
\pgfpathlineto{\pgfqpoint{5.672111in}{4.118647in}}%
\pgfpathlineto{\pgfqpoint{5.672956in}{4.088621in}}%
\pgfpathlineto{\pgfqpoint{5.674929in}{3.997863in}}%
\pgfpathlineto{\pgfqpoint{5.675211in}{4.001605in}}%
\pgfpathlineto{\pgfqpoint{5.676902in}{4.050407in}}%
\pgfpathlineto{\pgfqpoint{5.677466in}{4.046229in}}%
\pgfpathlineto{\pgfqpoint{5.679157in}{3.970389in}}%
\pgfpathlineto{\pgfqpoint{5.681130in}{3.909877in}}%
\pgfpathlineto{\pgfqpoint{5.681412in}{3.911727in}}%
\pgfpathlineto{\pgfqpoint{5.682539in}{3.948732in}}%
\pgfpathlineto{\pgfqpoint{5.688176in}{4.206914in}}%
\pgfpathlineto{\pgfqpoint{5.690713in}{4.230484in}}%
\pgfpathlineto{\pgfqpoint{5.691276in}{4.231117in}}%
\pgfpathlineto{\pgfqpoint{5.691558in}{4.230568in}}%
\pgfpathlineto{\pgfqpoint{5.692686in}{4.221025in}}%
\pgfpathlineto{\pgfqpoint{5.694659in}{4.173941in}}%
\pgfpathlineto{\pgfqpoint{5.698604in}{4.077742in}}%
\pgfpathlineto{\pgfqpoint{5.698886in}{4.077004in}}%
\pgfpathlineto{\pgfqpoint{5.699168in}{4.077503in}}%
\pgfpathlineto{\pgfqpoint{5.700014in}{4.086653in}}%
\pgfpathlineto{\pgfqpoint{5.701987in}{4.147501in}}%
\pgfpathlineto{\pgfqpoint{5.706778in}{4.307548in}}%
\pgfpathlineto{\pgfqpoint{5.708187in}{4.318087in}}%
\pgfpathlineto{\pgfqpoint{5.708469in}{4.317910in}}%
\pgfpathlineto{\pgfqpoint{5.709315in}{4.313236in}}%
\pgfpathlineto{\pgfqpoint{5.711287in}{4.282769in}}%
\pgfpathlineto{\pgfqpoint{5.717770in}{4.148412in}}%
\pgfpathlineto{\pgfqpoint{5.718334in}{4.149718in}}%
\pgfpathlineto{\pgfqpoint{5.719461in}{4.163279in}}%
\pgfpathlineto{\pgfqpoint{5.721716in}{4.230627in}}%
\pgfpathlineto{\pgfqpoint{5.726225in}{4.369963in}}%
\pgfpathlineto{\pgfqpoint{5.727353in}{4.377038in}}%
\pgfpathlineto{\pgfqpoint{5.727635in}{4.376721in}}%
\pgfpathlineto{\pgfqpoint{5.728762in}{4.367640in}}%
\pgfpathlineto{\pgfqpoint{5.731017in}{4.322017in}}%
\pgfpathlineto{\pgfqpoint{5.735808in}{4.225750in}}%
\pgfpathlineto{\pgfqpoint{5.736654in}{4.223107in}}%
\pgfpathlineto{\pgfqpoint{5.736936in}{4.223766in}}%
\pgfpathlineto{\pgfqpoint{5.738063in}{4.234387in}}%
\pgfpathlineto{\pgfqpoint{5.740036in}{4.282685in}}%
\pgfpathlineto{\pgfqpoint{5.746800in}{4.483003in}}%
\pgfpathlineto{\pgfqpoint{5.747364in}{4.481004in}}%
\pgfpathlineto{\pgfqpoint{5.748773in}{4.462635in}}%
\pgfpathlineto{\pgfqpoint{5.754974in}{4.360049in}}%
\pgfpathlineto{\pgfqpoint{5.755537in}{4.361491in}}%
\pgfpathlineto{\pgfqpoint{5.756665in}{4.372464in}}%
\pgfpathlineto{\pgfqpoint{5.758638in}{4.418931in}}%
\pgfpathlineto{\pgfqpoint{5.762020in}{4.564724in}}%
\pgfpathlineto{\pgfqpoint{5.765120in}{4.669771in}}%
\pgfpathlineto{\pgfqpoint{5.765966in}{4.674235in}}%
\pgfpathlineto{\pgfqpoint{5.766248in}{4.673289in}}%
\pgfpathlineto{\pgfqpoint{5.767375in}{4.659164in}}%
\pgfpathlineto{\pgfqpoint{5.772448in}{4.563849in}}%
\pgfpathlineto{\pgfqpoint{5.773012in}{4.566354in}}%
\pgfpathlineto{\pgfqpoint{5.774703in}{4.589007in}}%
\pgfpathlineto{\pgfqpoint{5.777521in}{4.661915in}}%
\pgfpathlineto{\pgfqpoint{5.780340in}{4.807201in}}%
\pgfpathlineto{\pgfqpoint{5.783440in}{4.952005in}}%
\pgfpathlineto{\pgfqpoint{5.783722in}{4.951651in}}%
\pgfpathlineto{\pgfqpoint{5.784568in}{4.935479in}}%
\pgfpathlineto{\pgfqpoint{5.787668in}{4.824008in}}%
\pgfpathlineto{\pgfqpoint{5.788513in}{4.835771in}}%
\pgfpathlineto{\pgfqpoint{5.790204in}{4.876200in}}%
\pgfpathlineto{\pgfqpoint{5.790486in}{4.873682in}}%
\pgfpathlineto{\pgfqpoint{5.791896in}{4.820425in}}%
\pgfpathlineto{\pgfqpoint{5.792741in}{4.801910in}}%
\pgfpathlineto{\pgfqpoint{5.793023in}{4.809024in}}%
\pgfpathlineto{\pgfqpoint{5.794150in}{4.918383in}}%
\pgfpathlineto{\pgfqpoint{5.795841in}{5.073957in}}%
\pgfpathlineto{\pgfqpoint{5.796123in}{5.073168in}}%
\pgfpathlineto{\pgfqpoint{5.798096in}{5.001743in}}%
\pgfpathlineto{\pgfqpoint{5.798660in}{5.015906in}}%
\pgfpathlineto{\pgfqpoint{5.800351in}{5.077820in}}%
\pgfpathlineto{\pgfqpoint{5.800915in}{5.068074in}}%
\pgfpathlineto{\pgfqpoint{5.802324in}{4.950937in}}%
\pgfpathlineto{\pgfqpoint{5.805142in}{4.752532in}}%
\pgfpathlineto{\pgfqpoint{5.806833in}{4.743066in}}%
\pgfpathlineto{\pgfqpoint{5.807961in}{4.740641in}}%
\pgfpathlineto{\pgfqpoint{5.809652in}{4.733866in}}%
\pgfpathlineto{\pgfqpoint{5.810216in}{4.735439in}}%
\pgfpathlineto{\pgfqpoint{5.811343in}{4.750548in}}%
\pgfpathlineto{\pgfqpoint{5.815289in}{4.829234in}}%
\pgfpathlineto{\pgfqpoint{5.815852in}{4.825282in}}%
\pgfpathlineto{\pgfqpoint{5.817262in}{4.786754in}}%
\pgfpathlineto{\pgfqpoint{5.825153in}{4.479240in}}%
\pgfpathlineto{\pgfqpoint{5.825435in}{4.478974in}}%
\pgfpathlineto{\pgfqpoint{5.825717in}{4.479371in}}%
\pgfpathlineto{\pgfqpoint{5.826845in}{4.486792in}}%
\pgfpathlineto{\pgfqpoint{5.829099in}{4.522976in}}%
\pgfpathlineto{\pgfqpoint{5.833327in}{4.593298in}}%
\pgfpathlineto{\pgfqpoint{5.833891in}{4.592299in}}%
\pgfpathlineto{\pgfqpoint{5.835018in}{4.578478in}}%
\pgfpathlineto{\pgfqpoint{5.836991in}{4.516696in}}%
\pgfpathlineto{\pgfqpoint{5.843192in}{4.293164in}}%
\pgfpathlineto{\pgfqpoint{5.844037in}{4.289603in}}%
\pgfpathlineto{\pgfqpoint{5.844319in}{4.290132in}}%
\pgfpathlineto{\pgfqpoint{5.845446in}{4.300114in}}%
\pgfpathlineto{\pgfqpoint{5.847983in}{4.354558in}}%
\pgfpathlineto{\pgfqpoint{5.852211in}{4.436783in}}%
\pgfpathlineto{\pgfqpoint{5.852774in}{4.437643in}}%
\pgfpathlineto{\pgfqpoint{5.853056in}{4.436758in}}%
\pgfpathlineto{\pgfqpoint{5.854184in}{4.424321in}}%
\pgfpathlineto{\pgfqpoint{5.856157in}{4.371406in}}%
\pgfpathlineto{\pgfqpoint{5.862639in}{4.174273in}}%
\pgfpathlineto{\pgfqpoint{5.862921in}{4.174020in}}%
\pgfpathlineto{\pgfqpoint{5.863203in}{4.174660in}}%
\pgfpathlineto{\pgfqpoint{5.864330in}{4.185688in}}%
\pgfpathlineto{\pgfqpoint{5.866585in}{4.240105in}}%
\pgfpathlineto{\pgfqpoint{5.871376in}{4.354866in}}%
\pgfpathlineto{\pgfqpoint{5.871940in}{4.357147in}}%
\pgfpathlineto{\pgfqpoint{5.872504in}{4.355974in}}%
\pgfpathlineto{\pgfqpoint{5.873631in}{4.343130in}}%
\pgfpathlineto{\pgfqpoint{5.875886in}{4.281538in}}%
\pgfpathlineto{\pgfqpoint{5.881523in}{4.114607in}}%
\pgfpathlineto{\pgfqpoint{5.881805in}{4.113727in}}%
\pgfpathlineto{\pgfqpoint{5.882086in}{4.113873in}}%
\pgfpathlineto{\pgfqpoint{5.882086in}{4.113873in}}%
\pgfpathlineto{\pgfqpoint{5.882932in}{4.120217in}}%
\pgfpathlineto{\pgfqpoint{5.884623in}{4.155533in}}%
\pgfpathlineto{\pgfqpoint{5.891105in}{4.314041in}}%
\pgfpathlineto{\pgfqpoint{5.891951in}{4.309732in}}%
\pgfpathlineto{\pgfqpoint{5.893642in}{4.279478in}}%
\pgfpathlineto{\pgfqpoint{5.897024in}{4.162119in}}%
\pgfpathlineto{\pgfqpoint{5.900688in}{4.059700in}}%
\pgfpathlineto{\pgfqpoint{5.901252in}{4.056821in}}%
\pgfpathlineto{\pgfqpoint{5.901816in}{4.058588in}}%
\pgfpathlineto{\pgfqpoint{5.902943in}{4.075512in}}%
\pgfpathlineto{\pgfqpoint{5.905761in}{4.168751in}}%
\pgfpathlineto{\pgfqpoint{5.908862in}{4.247712in}}%
\pgfpathlineto{\pgfqpoint{5.909144in}{4.248620in}}%
\pgfpathlineto{\pgfqpoint{5.909425in}{4.248213in}}%
\pgfpathlineto{\pgfqpoint{5.910271in}{4.239352in}}%
\pgfpathlineto{\pgfqpoint{5.912244in}{4.184383in}}%
\pgfpathlineto{\pgfqpoint{5.920136in}{3.904823in}}%
\pgfpathlineto{\pgfqpoint{5.920699in}{3.907212in}}%
\pgfpathlineto{\pgfqpoint{5.921827in}{3.938861in}}%
\pgfpathlineto{\pgfqpoint{5.924927in}{4.082011in}}%
\pgfpathlineto{\pgfqpoint{5.925491in}{4.076974in}}%
\pgfpathlineto{\pgfqpoint{5.928027in}{3.985326in}}%
\pgfpathlineto{\pgfqpoint{5.928873in}{4.004001in}}%
\pgfpathlineto{\pgfqpoint{5.929718in}{4.019282in}}%
\pgfpathlineto{\pgfqpoint{5.930000in}{4.014369in}}%
\pgfpathlineto{\pgfqpoint{5.931128in}{3.923705in}}%
\pgfpathlineto{\pgfqpoint{5.933101in}{3.767872in}}%
\pgfpathlineto{\pgfqpoint{5.933382in}{3.771152in}}%
\pgfpathlineto{\pgfqpoint{5.939865in}{4.019566in}}%
\pgfpathlineto{\pgfqpoint{5.942683in}{4.177805in}}%
\pgfpathlineto{\pgfqpoint{5.942965in}{4.179504in}}%
\pgfpathlineto{\pgfqpoint{5.943247in}{4.178930in}}%
\pgfpathlineto{\pgfqpoint{5.943247in}{4.178930in}}%
\pgfpathlineto{\pgfqpoint{5.944374in}{4.159130in}}%
\pgfpathlineto{\pgfqpoint{5.951421in}{3.983779in}}%
\pgfpathlineto{\pgfqpoint{5.951984in}{3.985363in}}%
\pgfpathlineto{\pgfqpoint{5.953112in}{4.000284in}}%
\pgfpathlineto{\pgfqpoint{5.955085in}{4.064103in}}%
\pgfpathlineto{\pgfqpoint{5.960722in}{4.272583in}}%
\pgfpathlineto{\pgfqpoint{5.961003in}{4.273000in}}%
\pgfpathlineto{\pgfqpoint{5.961003in}{4.273000in}}%
\pgfpathlineto{\pgfqpoint{5.961003in}{4.273000in}}%
\pgfpathlineto{\pgfqpoint{5.961567in}{4.269983in}}%
\pgfpathlineto{\pgfqpoint{5.962976in}{4.242591in}}%
\pgfpathlineto{\pgfqpoint{5.970022in}{4.038539in}}%
\pgfpathlineto{\pgfqpoint{5.970868in}{4.042865in}}%
\pgfpathlineto{\pgfqpoint{5.972277in}{4.069671in}}%
\pgfpathlineto{\pgfqpoint{5.975378in}{4.187470in}}%
\pgfpathlineto{\pgfqpoint{5.979042in}{4.305813in}}%
\pgfpathlineto{\pgfqpoint{5.979887in}{4.311996in}}%
\pgfpathlineto{\pgfqpoint{5.980451in}{4.310476in}}%
\pgfpathlineto{\pgfqpoint{5.981578in}{4.294525in}}%
\pgfpathlineto{\pgfqpoint{5.984115in}{4.213105in}}%
\pgfpathlineto{\pgfqpoint{5.988342in}{4.090962in}}%
\pgfpathlineto{\pgfqpoint{5.988906in}{4.088784in}}%
\pgfpathlineto{\pgfqpoint{5.989188in}{4.089383in}}%
\pgfpathlineto{\pgfqpoint{5.990315in}{4.102943in}}%
\pgfpathlineto{\pgfqpoint{5.992288in}{4.164040in}}%
\pgfpathlineto{\pgfqpoint{5.999053in}{4.403757in}}%
\pgfpathlineto{\pgfqpoint{5.999616in}{4.404972in}}%
\pgfpathlineto{\pgfqpoint{5.999898in}{4.404028in}}%
\pgfpathlineto{\pgfqpoint{6.001026in}{4.390523in}}%
\pgfpathlineto{\pgfqpoint{6.003562in}{4.319753in}}%
\pgfpathlineto{\pgfqpoint{6.007226in}{4.233717in}}%
\pgfpathlineto{\pgfqpoint{6.007508in}{4.233006in}}%
\pgfpathlineto{\pgfqpoint{6.007790in}{4.233493in}}%
\pgfpathlineto{\pgfqpoint{6.008635in}{4.242447in}}%
\pgfpathlineto{\pgfqpoint{6.010326in}{4.293309in}}%
\pgfpathlineto{\pgfqpoint{6.019064in}{4.626804in}}%
\pgfpathlineto{\pgfqpoint{6.019909in}{4.621901in}}%
\pgfpathlineto{\pgfqpoint{6.021600in}{4.591819in}}%
\pgfpathlineto{\pgfqpoint{6.025828in}{4.504793in}}%
\pgfpathlineto{\pgfqpoint{6.026110in}{4.505942in}}%
\pgfpathlineto{\pgfqpoint{6.026955in}{4.520131in}}%
\pgfpathlineto{\pgfqpoint{6.028647in}{4.601986in}}%
\pgfpathlineto{\pgfqpoint{6.034283in}{4.924036in}}%
\pgfpathlineto{\pgfqpoint{6.037384in}{5.010668in}}%
\pgfpathlineto{\pgfqpoint{6.037947in}{5.002724in}}%
\pgfpathlineto{\pgfqpoint{6.039357in}{4.915450in}}%
\pgfpathlineto{\pgfqpoint{6.041048in}{4.827116in}}%
\pgfpathlineto{\pgfqpoint{6.041330in}{4.833330in}}%
\pgfpathlineto{\pgfqpoint{6.043303in}{4.914755in}}%
\pgfpathlineto{\pgfqpoint{6.043866in}{4.902202in}}%
\pgfpathlineto{\pgfqpoint{6.045275in}{4.860192in}}%
\pgfpathlineto{\pgfqpoint{6.045557in}{4.864265in}}%
\pgfpathlineto{\pgfqpoint{6.046967in}{4.943474in}}%
\pgfpathlineto{\pgfqpoint{6.048939in}{5.017776in}}%
\pgfpathlineto{\pgfqpoint{6.049785in}{4.995618in}}%
\pgfpathlineto{\pgfqpoint{6.055422in}{4.678487in}}%
\pgfpathlineto{\pgfqpoint{6.059650in}{4.490404in}}%
\pgfpathlineto{\pgfqpoint{6.060213in}{4.488206in}}%
\pgfpathlineto{\pgfqpoint{6.060495in}{4.489491in}}%
\pgfpathlineto{\pgfqpoint{6.061623in}{4.507510in}}%
\pgfpathlineto{\pgfqpoint{6.066132in}{4.607594in}}%
\pgfpathlineto{\pgfqpoint{6.066696in}{4.605869in}}%
\pgfpathlineto{\pgfqpoint{6.067823in}{4.587517in}}%
\pgfpathlineto{\pgfqpoint{6.070078in}{4.502571in}}%
\pgfpathlineto{\pgfqpoint{6.077124in}{4.206961in}}%
\pgfpathlineto{\pgfqpoint{6.078251in}{4.200114in}}%
\pgfpathlineto{\pgfqpoint{6.078533in}{4.201066in}}%
\pgfpathlineto{\pgfqpoint{6.079661in}{4.214809in}}%
\pgfpathlineto{\pgfqpoint{6.082479in}{4.291810in}}%
\pgfpathlineto{\pgfqpoint{6.085298in}{4.348706in}}%
\pgfpathlineto{\pgfqpoint{6.085861in}{4.349899in}}%
\pgfpathlineto{\pgfqpoint{6.086143in}{4.348851in}}%
\pgfpathlineto{\pgfqpoint{6.087271in}{4.333395in}}%
\pgfpathlineto{\pgfqpoint{6.089525in}{4.256540in}}%
\pgfpathlineto{\pgfqpoint{6.095444in}{4.044064in}}%
\pgfpathlineto{\pgfqpoint{6.096571in}{4.038366in}}%
\pgfpathlineto{\pgfqpoint{6.096853in}{4.039265in}}%
\pgfpathlineto{\pgfqpoint{6.097981in}{4.051677in}}%
\pgfpathlineto{\pgfqpoint{6.100235in}{4.110941in}}%
\pgfpathlineto{\pgfqpoint{6.104745in}{4.229425in}}%
\pgfpathlineto{\pgfqpoint{6.105309in}{4.231076in}}%
\pgfpathlineto{\pgfqpoint{6.105591in}{4.230254in}}%
\pgfpathlineto{\pgfqpoint{6.106718in}{4.215523in}}%
\pgfpathlineto{\pgfqpoint{6.108973in}{4.137677in}}%
\pgfpathlineto{\pgfqpoint{6.113764in}{3.969821in}}%
\pgfpathlineto{\pgfqpoint{6.114891in}{3.962803in}}%
\pgfpathlineto{\pgfqpoint{6.115173in}{3.963350in}}%
\pgfpathlineto{\pgfqpoint{6.116301in}{3.973720in}}%
\pgfpathlineto{\pgfqpoint{6.118837in}{4.030378in}}%
\pgfpathlineto{\pgfqpoint{6.123911in}{4.148815in}}%
\pgfpathlineto{\pgfqpoint{6.124192in}{4.149082in}}%
\pgfpathlineto{\pgfqpoint{6.124192in}{4.149082in}}%
\pgfpathlineto{\pgfqpoint{6.124192in}{4.149082in}}%
\pgfpathlineto{\pgfqpoint{6.124756in}{4.145766in}}%
\pgfpathlineto{\pgfqpoint{6.125883in}{4.121672in}}%
\pgfpathlineto{\pgfqpoint{6.128138in}{4.003228in}}%
\pgfpathlineto{\pgfqpoint{6.131520in}{3.838237in}}%
\pgfpathlineto{\pgfqpoint{6.132366in}{3.832758in}}%
\pgfpathlineto{\pgfqpoint{6.132648in}{3.833993in}}%
\pgfpathlineto{\pgfqpoint{6.134057in}{3.854790in}}%
\pgfpathlineto{\pgfqpoint{6.135748in}{3.876007in}}%
\pgfpathlineto{\pgfqpoint{6.136030in}{3.875829in}}%
\pgfpathlineto{\pgfqpoint{6.137439in}{3.867450in}}%
\pgfpathlineto{\pgfqpoint{6.138003in}{3.870158in}}%
\pgfpathlineto{\pgfqpoint{6.139130in}{3.902750in}}%
\pgfpathlineto{\pgfqpoint{6.141103in}{3.988822in}}%
\pgfpathlineto{\pgfqpoint{6.141667in}{3.977421in}}%
\pgfpathlineto{\pgfqpoint{6.143076in}{3.817015in}}%
\pgfpathlineto{\pgfqpoint{6.144485in}{3.710350in}}%
\pgfpathlineto{\pgfqpoint{6.144767in}{3.715565in}}%
\pgfpathlineto{\pgfqpoint{6.146176in}{3.779994in}}%
\pgfpathlineto{\pgfqpoint{6.146740in}{3.770023in}}%
\pgfpathlineto{\pgfqpoint{6.148995in}{3.623051in}}%
\pgfpathlineto{\pgfqpoint{6.149559in}{3.637166in}}%
\pgfpathlineto{\pgfqpoint{6.151250in}{3.818288in}}%
\pgfpathlineto{\pgfqpoint{6.153504in}{3.966171in}}%
\pgfpathlineto{\pgfqpoint{6.158578in}{4.065891in}}%
\pgfpathlineto{\pgfqpoint{6.159141in}{4.062984in}}%
\pgfpathlineto{\pgfqpoint{6.160551in}{4.025282in}}%
\pgfpathlineto{\pgfqpoint{6.164778in}{3.890833in}}%
\pgfpathlineto{\pgfqpoint{6.165342in}{3.893699in}}%
\pgfpathlineto{\pgfqpoint{6.166469in}{3.920281in}}%
\pgfpathlineto{\pgfqpoint{6.174643in}{4.188882in}}%
\pgfpathlineto{\pgfqpoint{6.175207in}{4.188019in}}%
\pgfpathlineto{\pgfqpoint{6.176334in}{4.178489in}}%
\pgfpathlineto{\pgfqpoint{6.178307in}{4.136772in}}%
\pgfpathlineto{\pgfqpoint{6.183662in}{4.003969in}}%
\pgfpathlineto{\pgfqpoint{6.184226in}{4.005409in}}%
\pgfpathlineto{\pgfqpoint{6.185353in}{4.022751in}}%
\pgfpathlineto{\pgfqpoint{6.187608in}{4.106973in}}%
\pgfpathlineto{\pgfqpoint{6.191836in}{4.259081in}}%
\pgfpathlineto{\pgfqpoint{6.193245in}{4.271593in}}%
\pgfpathlineto{\pgfqpoint{6.193527in}{4.271435in}}%
\pgfpathlineto{\pgfqpoint{6.194372in}{4.266127in}}%
\pgfpathlineto{\pgfqpoint{6.196345in}{4.232559in}}%
\pgfpathlineto{\pgfqpoint{6.202546in}{4.113861in}}%
\pgfpathlineto{\pgfqpoint{6.203391in}{4.119575in}}%
\pgfpathlineto{\pgfqpoint{6.204800in}{4.149380in}}%
\pgfpathlineto{\pgfqpoint{6.207337in}{4.261828in}}%
\pgfpathlineto{\pgfqpoint{6.211283in}{4.416247in}}%
\pgfpathlineto{\pgfqpoint{6.212410in}{4.425460in}}%
\pgfpathlineto{\pgfqpoint{6.212974in}{4.424206in}}%
\pgfpathlineto{\pgfqpoint{6.214383in}{4.407588in}}%
\pgfpathlineto{\pgfqpoint{6.220020in}{4.325527in}}%
\pgfpathlineto{\pgfqpoint{6.220584in}{4.324320in}}%
\pgfpathlineto{\pgfqpoint{6.221148in}{4.325574in}}%
\pgfpathlineto{\pgfqpoint{6.222275in}{4.337278in}}%
\pgfpathlineto{\pgfqpoint{6.223966in}{4.383729in}}%
\pgfpathlineto{\pgfqpoint{6.226503in}{4.528739in}}%
\pgfpathlineto{\pgfqpoint{6.229885in}{4.704182in}}%
\pgfpathlineto{\pgfqpoint{6.230730in}{4.711044in}}%
\pgfpathlineto{\pgfqpoint{6.231012in}{4.710023in}}%
\pgfpathlineto{\pgfqpoint{6.232140in}{4.693099in}}%
\pgfpathlineto{\pgfqpoint{6.235240in}{4.634749in}}%
\pgfpathlineto{\pgfqpoint{6.235522in}{4.635431in}}%
\pgfpathlineto{\pgfqpoint{6.236931in}{4.652113in}}%
\pgfpathlineto{\pgfqpoint{6.238340in}{4.664181in}}%
\pgfpathlineto{\pgfqpoint{6.238622in}{4.664106in}}%
\pgfpathlineto{\pgfqpoint{6.240031in}{4.658918in}}%
\pgfpathlineto{\pgfqpoint{6.240595in}{4.661804in}}%
\pgfpathlineto{\pgfqpoint{6.241441in}{4.687025in}}%
\pgfpathlineto{\pgfqpoint{6.242850in}{4.829180in}}%
\pgfpathlineto{\pgfqpoint{6.245105in}{5.051161in}}%
\pgfpathlineto{\pgfqpoint{6.245386in}{5.046609in}}%
\pgfpathlineto{\pgfqpoint{6.247359in}{4.943219in}}%
\pgfpathlineto{\pgfqpoint{6.247923in}{4.961268in}}%
\pgfpathlineto{\pgfqpoint{6.249614in}{5.082250in}}%
\pgfpathlineto{\pgfqpoint{6.250178in}{5.054717in}}%
\pgfpathlineto{\pgfqpoint{6.252714in}{4.744457in}}%
\pgfpathlineto{\pgfqpoint{6.253560in}{4.770319in}}%
\pgfpathlineto{\pgfqpoint{6.255251in}{4.826516in}}%
\pgfpathlineto{\pgfqpoint{6.255533in}{4.823064in}}%
\pgfpathlineto{\pgfqpoint{6.258069in}{4.732979in}}%
\pgfpathlineto{\pgfqpoint{6.258915in}{4.746926in}}%
\pgfpathlineto{\pgfqpoint{6.262015in}{4.842012in}}%
\pgfpathlineto{\pgfqpoint{6.262579in}{4.837452in}}%
\pgfpathlineto{\pgfqpoint{6.263706in}{4.795951in}}%
\pgfpathlineto{\pgfqpoint{6.270753in}{4.444029in}}%
\pgfpathlineto{\pgfqpoint{6.273571in}{4.411941in}}%
\pgfpathlineto{\pgfqpoint{6.273853in}{4.411435in}}%
\pgfpathlineto{\pgfqpoint{6.274135in}{4.411628in}}%
\pgfpathlineto{\pgfqpoint{6.274135in}{4.411628in}}%
\pgfpathlineto{\pgfqpoint{6.274980in}{4.416735in}}%
\pgfpathlineto{\pgfqpoint{6.276953in}{4.450385in}}%
\pgfpathlineto{\pgfqpoint{6.279772in}{4.487863in}}%
\pgfpathlineto{\pgfqpoint{6.280335in}{4.485668in}}%
\pgfpathlineto{\pgfqpoint{6.281463in}{4.467441in}}%
\pgfpathlineto{\pgfqpoint{6.283717in}{4.377661in}}%
\pgfpathlineto{\pgfqpoint{6.288509in}{4.181594in}}%
\pgfpathlineto{\pgfqpoint{6.290764in}{4.159233in}}%
\pgfpathlineto{\pgfqpoint{6.291609in}{4.162293in}}%
\pgfpathlineto{\pgfqpoint{6.293300in}{4.181501in}}%
\pgfpathlineto{\pgfqpoint{6.298937in}{4.270042in}}%
\pgfpathlineto{\pgfqpoint{6.299501in}{4.268287in}}%
\pgfpathlineto{\pgfqpoint{6.300628in}{4.254312in}}%
\pgfpathlineto{\pgfqpoint{6.302601in}{4.196481in}}%
\pgfpathlineto{\pgfqpoint{6.308238in}{4.014196in}}%
\pgfpathlineto{\pgfqpoint{6.308802in}{4.011965in}}%
\pgfpathlineto{\pgfqpoint{6.309365in}{4.013055in}}%
\pgfpathlineto{\pgfqpoint{6.310493in}{4.024397in}}%
\pgfpathlineto{\pgfqpoint{6.313593in}{4.091457in}}%
\pgfpathlineto{\pgfqpoint{6.317539in}{4.157268in}}%
\pgfpathlineto{\pgfqpoint{6.318103in}{4.159025in}}%
\pgfpathlineto{\pgfqpoint{6.318666in}{4.157763in}}%
\pgfpathlineto{\pgfqpoint{6.319794in}{4.145498in}}%
\pgfpathlineto{\pgfqpoint{6.321767in}{4.092367in}}%
\pgfpathlineto{\pgfqpoint{6.327404in}{3.901629in}}%
\pgfpathlineto{\pgfqpoint{6.327685in}{3.901968in}}%
\pgfpathlineto{\pgfqpoint{6.328531in}{3.909846in}}%
\pgfpathlineto{\pgfqpoint{6.330786in}{3.965424in}}%
\pgfpathlineto{\pgfqpoint{6.333886in}{4.026218in}}%
\pgfpathlineto{\pgfqpoint{6.334450in}{4.026714in}}%
\pgfpathlineto{\pgfqpoint{6.334732in}{4.026032in}}%
\pgfpathlineto{\pgfqpoint{6.336705in}{4.013697in}}%
\pgfpathlineto{\pgfqpoint{6.338678in}{3.989867in}}%
\pgfpathlineto{\pgfqpoint{6.340369in}{3.931960in}}%
\pgfpathlineto{\pgfqpoint{6.342623in}{3.744778in}}%
\pgfpathlineto{\pgfqpoint{6.344596in}{3.642931in}}%
\pgfpathlineto{\pgfqpoint{6.345442in}{3.667211in}}%
\pgfpathlineto{\pgfqpoint{6.347697in}{3.782783in}}%
\pgfpathlineto{\pgfqpoint{6.348260in}{3.764299in}}%
\pgfpathlineto{\pgfqpoint{6.349951in}{3.665004in}}%
\pgfpathlineto{\pgfqpoint{6.350233in}{3.673671in}}%
\pgfpathlineto{\pgfqpoint{6.351642in}{3.834313in}}%
\pgfpathlineto{\pgfqpoint{6.352488in}{3.901286in}}%
\pgfpathlineto{\pgfqpoint{6.353052in}{3.880519in}}%
\pgfpathlineto{\pgfqpoint{6.355588in}{3.673324in}}%
\pgfpathlineto{\pgfqpoint{6.356152in}{3.688183in}}%
\pgfpathlineto{\pgfqpoint{6.358125in}{3.768199in}}%
\pgfpathlineto{\pgfqpoint{6.358689in}{3.762464in}}%
\pgfpathlineto{\pgfqpoint{6.360943in}{3.719012in}}%
\pgfpathlineto{\pgfqpoint{6.361225in}{3.721040in}}%
\pgfpathlineto{\pgfqpoint{6.362353in}{3.758107in}}%
\pgfpathlineto{\pgfqpoint{6.367708in}{4.047547in}}%
\pgfpathlineto{\pgfqpoint{6.368271in}{4.045400in}}%
\pgfpathlineto{\pgfqpoint{6.369681in}{4.023593in}}%
\pgfpathlineto{\pgfqpoint{6.376727in}{3.894162in}}%
\pgfpathlineto{\pgfqpoint{6.377290in}{3.892658in}}%
\pgfpathlineto{\pgfqpoint{6.377572in}{3.893130in}}%
\pgfpathlineto{\pgfqpoint{6.378418in}{3.899925in}}%
\pgfpathlineto{\pgfqpoint{6.380109in}{3.939033in}}%
\pgfpathlineto{\pgfqpoint{6.386591in}{4.149170in}}%
\pgfpathlineto{\pgfqpoint{6.387155in}{4.146027in}}%
\pgfpathlineto{\pgfqpoint{6.388564in}{4.122238in}}%
\pgfpathlineto{\pgfqpoint{6.395047in}{3.983936in}}%
\pgfpathlineto{\pgfqpoint{6.395610in}{3.985389in}}%
\pgfpathlineto{\pgfqpoint{6.396738in}{3.997603in}}%
\pgfpathlineto{\pgfqpoint{6.398711in}{4.048423in}}%
\pgfpathlineto{\pgfqpoint{6.406039in}{4.299143in}}%
\pgfpathlineto{\pgfqpoint{6.406602in}{4.296639in}}%
\pgfpathlineto{\pgfqpoint{6.408012in}{4.274546in}}%
\pgfpathlineto{\pgfqpoint{6.412521in}{4.185790in}}%
\pgfpathlineto{\pgfqpoint{6.412803in}{4.186049in}}%
\pgfpathlineto{\pgfqpoint{6.413649in}{4.192988in}}%
\pgfpathlineto{\pgfqpoint{6.415340in}{4.231185in}}%
\pgfpathlineto{\pgfqpoint{6.418722in}{4.373978in}}%
\pgfpathlineto{\pgfqpoint{6.424922in}{4.652877in}}%
\pgfpathlineto{\pgfqpoint{6.425204in}{4.652673in}}%
\pgfpathlineto{\pgfqpoint{6.426050in}{4.642461in}}%
\pgfpathlineto{\pgfqpoint{6.429714in}{4.555663in}}%
\pgfpathlineto{\pgfqpoint{6.430278in}{4.563024in}}%
\pgfpathlineto{\pgfqpoint{6.431687in}{4.626039in}}%
\pgfpathlineto{\pgfqpoint{6.435069in}{4.767162in}}%
\pgfpathlineto{\pgfqpoint{6.436196in}{4.824395in}}%
\pgfpathlineto{\pgfqpoint{6.439015in}{5.111367in}}%
\pgfpathlineto{\pgfqpoint{6.439579in}{5.088625in}}%
\pgfpathlineto{\pgfqpoint{6.441551in}{4.966225in}}%
\pgfpathlineto{\pgfqpoint{6.442115in}{4.977884in}}%
\pgfpathlineto{\pgfqpoint{6.442961in}{4.994424in}}%
\pgfpathlineto{\pgfqpoint{6.443243in}{4.990493in}}%
\pgfpathlineto{\pgfqpoint{6.444088in}{4.934800in}}%
\pgfpathlineto{\pgfqpoint{6.447470in}{4.648181in}}%
\pgfpathlineto{\pgfqpoint{6.448034in}{4.651421in}}%
\pgfpathlineto{\pgfqpoint{6.450289in}{4.671886in}}%
\pgfpathlineto{\pgfqpoint{6.450571in}{4.671424in}}%
\pgfpathlineto{\pgfqpoint{6.453389in}{4.659399in}}%
\pgfpathlineto{\pgfqpoint{6.454516in}{4.660275in}}%
\pgfpathlineto{\pgfqpoint{6.455080in}{4.658806in}}%
\pgfpathlineto{\pgfqpoint{6.456207in}{4.645306in}}%
\pgfpathlineto{\pgfqpoint{6.457899in}{4.586106in}}%
\pgfpathlineto{\pgfqpoint{6.465790in}{4.217016in}}%
\pgfpathlineto{\pgfqpoint{6.466636in}{4.221845in}}%
\pgfpathlineto{\pgfqpoint{6.468891in}{4.255779in}}%
\pgfpathlineto{\pgfqpoint{6.472555in}{4.301194in}}%
\pgfpathlineto{\pgfqpoint{6.472836in}{4.301386in}}%
\pgfpathlineto{\pgfqpoint{6.473118in}{4.300843in}}%
\pgfpathlineto{\pgfqpoint{6.474246in}{4.290469in}}%
\pgfpathlineto{\pgfqpoint{6.476219in}{4.237966in}}%
\pgfpathlineto{\pgfqpoint{6.483828in}{3.961052in}}%
\pgfpathlineto{\pgfqpoint{6.484392in}{3.964139in}}%
\pgfpathlineto{\pgfqpoint{6.485801in}{3.987591in}}%
\pgfpathlineto{\pgfqpoint{6.492002in}{4.126780in}}%
\pgfpathlineto{\pgfqpoint{6.492566in}{4.125826in}}%
\pgfpathlineto{\pgfqpoint{6.493693in}{4.113416in}}%
\pgfpathlineto{\pgfqpoint{6.495666in}{4.060091in}}%
\pgfpathlineto{\pgfqpoint{6.502430in}{3.842013in}}%
\pgfpathlineto{\pgfqpoint{6.503276in}{3.848544in}}%
\pgfpathlineto{\pgfqpoint{6.504967in}{3.888905in}}%
\pgfpathlineto{\pgfqpoint{6.510040in}{4.024944in}}%
\pgfpathlineto{\pgfqpoint{6.510604in}{4.023073in}}%
\pgfpathlineto{\pgfqpoint{6.511731in}{4.005650in}}%
\pgfpathlineto{\pgfqpoint{6.514268in}{3.916970in}}%
\pgfpathlineto{\pgfqpoint{6.520750in}{3.640530in}}%
\pgfpathlineto{\pgfqpoint{6.521314in}{3.644941in}}%
\pgfpathlineto{\pgfqpoint{6.522723in}{3.692474in}}%
\pgfpathlineto{\pgfqpoint{6.524978in}{3.770217in}}%
\pgfpathlineto{\pgfqpoint{6.525260in}{3.767933in}}%
\pgfpathlineto{\pgfqpoint{6.527233in}{3.722319in}}%
\pgfpathlineto{\pgfqpoint{6.527796in}{3.729230in}}%
\pgfpathlineto{\pgfqpoint{6.528924in}{3.755949in}}%
\pgfpathlineto{\pgfqpoint{6.529487in}{3.750293in}}%
\pgfpathlineto{\pgfqpoint{6.530897in}{3.654678in}}%
\pgfpathlineto{\pgfqpoint{6.532870in}{3.568651in}}%
\pgfpathlineto{\pgfqpoint{6.533997in}{3.595565in}}%
\pgfpathlineto{\pgfqpoint{6.538788in}{3.788603in}}%
\pgfpathlineto{\pgfqpoint{6.543016in}{4.020280in}}%
\pgfpathlineto{\pgfqpoint{6.543580in}{4.023591in}}%
\pgfpathlineto{\pgfqpoint{6.543862in}{4.022552in}}%
\pgfpathlineto{\pgfqpoint{6.544989in}{4.003485in}}%
\pgfpathlineto{\pgfqpoint{6.548935in}{3.864410in}}%
\pgfpathlineto{\pgfqpoint{6.551190in}{3.825005in}}%
\pgfpathlineto{\pgfqpoint{6.551472in}{3.824673in}}%
\pgfpathlineto{\pgfqpoint{6.551753in}{3.825451in}}%
\pgfpathlineto{\pgfqpoint{6.552881in}{3.839634in}}%
\pgfpathlineto{\pgfqpoint{6.554854in}{3.902857in}}%
\pgfpathlineto{\pgfqpoint{6.561336in}{4.142957in}}%
\pgfpathlineto{\pgfqpoint{6.561618in}{4.143376in}}%
\pgfpathlineto{\pgfqpoint{6.561618in}{4.143376in}}%
\pgfpathlineto{\pgfqpoint{6.561618in}{4.143376in}}%
\pgfpathlineto{\pgfqpoint{6.562182in}{4.140586in}}%
\pgfpathlineto{\pgfqpoint{6.563591in}{4.114382in}}%
\pgfpathlineto{\pgfqpoint{6.570073in}{3.929410in}}%
\pgfpathlineto{\pgfqpoint{6.570919in}{3.934832in}}%
\pgfpathlineto{\pgfqpoint{6.572328in}{3.967372in}}%
\pgfpathlineto{\pgfqpoint{6.575428in}{4.109075in}}%
\pgfpathlineto{\pgfqpoint{6.579656in}{4.271646in}}%
\pgfpathlineto{\pgfqpoint{6.580784in}{4.280829in}}%
\pgfpathlineto{\pgfqpoint{6.581065in}{4.280450in}}%
\pgfpathlineto{\pgfqpoint{6.581911in}{4.273271in}}%
\pgfpathlineto{\pgfqpoint{6.583884in}{4.228479in}}%
\pgfpathlineto{\pgfqpoint{6.588112in}{4.127858in}}%
\pgfpathlineto{\pgfqpoint{6.588393in}{4.127258in}}%
\pgfpathlineto{\pgfqpoint{6.588675in}{4.127910in}}%
\pgfpathlineto{\pgfqpoint{6.589521in}{4.137846in}}%
\pgfpathlineto{\pgfqpoint{6.591212in}{4.194322in}}%
\pgfpathlineto{\pgfqpoint{6.600231in}{4.580825in}}%
\pgfpathlineto{\pgfqpoint{6.601076in}{4.576004in}}%
\pgfpathlineto{\pgfqpoint{6.602768in}{4.547761in}}%
\pgfpathlineto{\pgfqpoint{6.606150in}{4.485297in}}%
\pgfpathlineto{\pgfqpoint{6.606432in}{4.486448in}}%
\pgfpathlineto{\pgfqpoint{6.607277in}{4.503099in}}%
\pgfpathlineto{\pgfqpoint{6.608968in}{4.608001in}}%
\pgfpathlineto{\pgfqpoint{6.613196in}{4.897494in}}%
\pgfpathlineto{\pgfqpoint{6.614887in}{4.985952in}}%
\pgfpathlineto{\pgfqpoint{6.616296in}{5.054417in}}%
\pgfpathlineto{\pgfqpoint{6.616578in}{5.050977in}}%
\pgfpathlineto{\pgfqpoint{6.617705in}{4.969900in}}%
\pgfpathlineto{\pgfqpoint{6.619960in}{4.826152in}}%
\pgfpathlineto{\pgfqpoint{6.620524in}{4.826006in}}%
\pgfpathlineto{\pgfqpoint{6.621088in}{4.819914in}}%
\pgfpathlineto{\pgfqpoint{6.622215in}{4.767770in}}%
\pgfpathlineto{\pgfqpoint{6.624470in}{4.668174in}}%
\pgfpathlineto{\pgfqpoint{6.624752in}{4.670551in}}%
\pgfpathlineto{\pgfqpoint{6.626443in}{4.721112in}}%
\pgfpathlineto{\pgfqpoint{6.628416in}{4.760188in}}%
\pgfpathlineto{\pgfqpoint{6.628697in}{4.759343in}}%
\pgfpathlineto{\pgfqpoint{6.629825in}{4.739387in}}%
\pgfpathlineto{\pgfqpoint{6.632361in}{4.633176in}}%
\pgfpathlineto{\pgfqpoint{6.640817in}{4.247097in}}%
\pgfpathlineto{\pgfqpoint{6.641099in}{4.247035in}}%
\pgfpathlineto{\pgfqpoint{6.641944in}{4.253417in}}%
\pgfpathlineto{\pgfqpoint{6.644199in}{4.299163in}}%
\pgfpathlineto{\pgfqpoint{6.647017in}{4.340503in}}%
\pgfpathlineto{\pgfqpoint{6.647581in}{4.338883in}}%
\pgfpathlineto{\pgfqpoint{6.648709in}{4.322886in}}%
\pgfpathlineto{\pgfqpoint{6.650963in}{4.246239in}}%
\pgfpathlineto{\pgfqpoint{6.658291in}{3.967286in}}%
\pgfpathlineto{\pgfqpoint{6.658855in}{3.964658in}}%
\pgfpathlineto{\pgfqpoint{6.659419in}{3.965991in}}%
\pgfpathlineto{\pgfqpoint{6.660546in}{3.979791in}}%
\pgfpathlineto{\pgfqpoint{6.663365in}{4.054794in}}%
\pgfpathlineto{\pgfqpoint{6.666465in}{4.113380in}}%
\pgfpathlineto{\pgfqpoint{6.666747in}{4.113632in}}%
\pgfpathlineto{\pgfqpoint{6.667029in}{4.112790in}}%
\pgfpathlineto{\pgfqpoint{6.668156in}{4.098185in}}%
\pgfpathlineto{\pgfqpoint{6.670411in}{4.022648in}}%
\pgfpathlineto{\pgfqpoint{6.675766in}{3.831193in}}%
\pgfpathlineto{\pgfqpoint{6.677175in}{3.820387in}}%
\pgfpathlineto{\pgfqpoint{6.677457in}{3.820641in}}%
\pgfpathlineto{\pgfqpoint{6.678302in}{3.825906in}}%
\pgfpathlineto{\pgfqpoint{6.679993in}{3.854516in}}%
\pgfpathlineto{\pgfqpoint{6.685349in}{3.974748in}}%
\pgfpathlineto{\pgfqpoint{6.685912in}{3.971231in}}%
\pgfpathlineto{\pgfqpoint{6.687040in}{3.944087in}}%
\pgfpathlineto{\pgfqpoint{6.689294in}{3.815001in}}%
\pgfpathlineto{\pgfqpoint{6.692677in}{3.647148in}}%
\pgfpathlineto{\pgfqpoint{6.697750in}{3.562981in}}%
\pgfpathlineto{\pgfqpoint{6.698032in}{3.566340in}}%
\pgfpathlineto{\pgfqpoint{6.699159in}{3.620337in}}%
\pgfpathlineto{\pgfqpoint{6.700850in}{3.715959in}}%
\pgfpathlineto{\pgfqpoint{6.701132in}{3.710500in}}%
\pgfpathlineto{\pgfqpoint{6.703387in}{3.544231in}}%
\pgfpathlineto{\pgfqpoint{6.704514in}{3.577693in}}%
\pgfpathlineto{\pgfqpoint{6.705360in}{3.607284in}}%
\pgfpathlineto{\pgfqpoint{6.705923in}{3.600113in}}%
\pgfpathlineto{\pgfqpoint{6.708460in}{3.487370in}}%
\pgfpathlineto{\pgfqpoint{6.709305in}{3.503911in}}%
\pgfpathlineto{\pgfqpoint{6.711278in}{3.651750in}}%
\pgfpathlineto{\pgfqpoint{6.714379in}{3.829855in}}%
\pgfpathlineto{\pgfqpoint{6.717761in}{3.907090in}}%
\pgfpathlineto{\pgfqpoint{6.718888in}{3.913312in}}%
\pgfpathlineto{\pgfqpoint{6.719170in}{3.912293in}}%
\pgfpathlineto{\pgfqpoint{6.720297in}{3.896746in}}%
\pgfpathlineto{\pgfqpoint{6.723398in}{3.798380in}}%
\pgfpathlineto{\pgfqpoint{6.725653in}{3.758289in}}%
\pgfpathlineto{\pgfqpoint{6.726216in}{3.760589in}}%
\pgfpathlineto{\pgfqpoint{6.727344in}{3.782371in}}%
\pgfpathlineto{\pgfqpoint{6.730162in}{3.904325in}}%
\pgfpathlineto{\pgfqpoint{6.733826in}{4.033251in}}%
\pgfpathlineto{\pgfqpoint{6.735799in}{4.049299in}}%
\pgfpathlineto{\pgfqpoint{6.736645in}{4.045669in}}%
\pgfpathlineto{\pgfqpoint{6.738336in}{4.024062in}}%
\pgfpathlineto{\pgfqpoint{6.743973in}{3.921351in}}%
\pgfpathlineto{\pgfqpoint{6.744536in}{3.923692in}}%
\pgfpathlineto{\pgfqpoint{6.745664in}{3.942609in}}%
\pgfpathlineto{\pgfqpoint{6.747637in}{4.023324in}}%
\pgfpathlineto{\pgfqpoint{6.753274in}{4.279032in}}%
\pgfpathlineto{\pgfqpoint{6.754965in}{4.290351in}}%
\pgfpathlineto{\pgfqpoint{6.755810in}{4.288383in}}%
\pgfpathlineto{\pgfqpoint{6.761165in}{4.268992in}}%
\pgfpathlineto{\pgfqpoint{6.761447in}{4.269152in}}%
\pgfpathlineto{\pgfqpoint{6.762293in}{4.273298in}}%
\pgfpathlineto{\pgfqpoint{6.763420in}{4.292353in}}%
\pgfpathlineto{\pgfqpoint{6.765111in}{4.369836in}}%
\pgfpathlineto{\pgfqpoint{6.770748in}{4.749938in}}%
\pgfpathlineto{\pgfqpoint{6.771312in}{4.746462in}}%
\pgfpathlineto{\pgfqpoint{6.772157in}{4.740587in}}%
\pgfpathlineto{\pgfqpoint{6.772439in}{4.741630in}}%
\pgfpathlineto{\pgfqpoint{6.773285in}{4.764500in}}%
\pgfpathlineto{\pgfqpoint{6.775258in}{4.870209in}}%
\pgfpathlineto{\pgfqpoint{6.775821in}{4.860072in}}%
\pgfpathlineto{\pgfqpoint{6.778076in}{4.699014in}}%
\pgfpathlineto{\pgfqpoint{6.778640in}{4.733290in}}%
\pgfpathlineto{\pgfqpoint{6.780613in}{4.944767in}}%
\pgfpathlineto{\pgfqpoint{6.781176in}{4.927258in}}%
\pgfpathlineto{\pgfqpoint{6.782867in}{4.824652in}}%
\pgfpathlineto{\pgfqpoint{6.783431in}{4.839399in}}%
\pgfpathlineto{\pgfqpoint{6.785404in}{4.921991in}}%
\pgfpathlineto{\pgfqpoint{6.785686in}{4.919471in}}%
\pgfpathlineto{\pgfqpoint{6.786531in}{4.877449in}}%
\pgfpathlineto{\pgfqpoint{6.793014in}{4.398172in}}%
\pgfpathlineto{\pgfqpoint{6.796678in}{4.323096in}}%
\pgfpathlineto{\pgfqpoint{6.797242in}{4.324454in}}%
\pgfpathlineto{\pgfqpoint{6.798651in}{4.343526in}}%
\pgfpathlineto{\pgfqpoint{6.801187in}{4.377157in}}%
\pgfpathlineto{\pgfqpoint{6.801469in}{4.376819in}}%
\pgfpathlineto{\pgfqpoint{6.802315in}{4.367986in}}%
\pgfpathlineto{\pgfqpoint{6.804006in}{4.311025in}}%
\pgfpathlineto{\pgfqpoint{6.811616in}{3.972109in}}%
\pgfpathlineto{\pgfqpoint{6.812743in}{3.967516in}}%
\pgfpathlineto{\pgfqpoint{6.813025in}{3.968024in}}%
\pgfpathlineto{\pgfqpoint{6.814152in}{3.975503in}}%
\pgfpathlineto{\pgfqpoint{6.816407in}{4.011831in}}%
\pgfpathlineto{\pgfqpoint{6.820353in}{4.076152in}}%
\pgfpathlineto{\pgfqpoint{6.820917in}{4.074745in}}%
\pgfpathlineto{\pgfqpoint{6.822044in}{4.059394in}}%
\pgfpathlineto{\pgfqpoint{6.824017in}{3.990928in}}%
\pgfpathlineto{\pgfqpoint{6.829090in}{3.794148in}}%
\pgfpathlineto{\pgfqpoint{6.829936in}{3.789780in}}%
\pgfpathlineto{\pgfqpoint{6.830218in}{3.790304in}}%
\pgfpathlineto{\pgfqpoint{6.831345in}{3.801029in}}%
\pgfpathlineto{\pgfqpoint{6.835009in}{3.876875in}}%
\pgfpathlineto{\pgfqpoint{6.838955in}{3.931181in}}%
\pgfpathlineto{\pgfqpoint{6.839237in}{3.931676in}}%
\pgfpathlineto{\pgfqpoint{6.839518in}{3.931313in}}%
\pgfpathlineto{\pgfqpoint{6.840364in}{3.924208in}}%
\pgfpathlineto{\pgfqpoint{6.841773in}{3.886826in}}%
\pgfpathlineto{\pgfqpoint{6.844028in}{3.746168in}}%
\pgfpathlineto{\pgfqpoint{6.847128in}{3.584692in}}%
\pgfpathlineto{\pgfqpoint{6.847410in}{3.584479in}}%
\pgfpathlineto{\pgfqpoint{6.848256in}{3.597716in}}%
\pgfpathlineto{\pgfqpoint{6.850792in}{3.671767in}}%
\pgfpathlineto{\pgfqpoint{6.851356in}{3.663478in}}%
\pgfpathlineto{\pgfqpoint{6.853329in}{3.574427in}}%
\pgfpathlineto{\pgfqpoint{6.854175in}{3.593994in}}%
\pgfpathlineto{\pgfqpoint{6.856147in}{3.730288in}}%
\pgfpathlineto{\pgfqpoint{6.856711in}{3.699167in}}%
\pgfpathlineto{\pgfqpoint{6.858966in}{3.406222in}}%
\pgfpathlineto{\pgfqpoint{6.859530in}{3.426115in}}%
\pgfpathlineto{\pgfqpoint{6.861221in}{3.545874in}}%
\pgfpathlineto{\pgfqpoint{6.861784in}{3.533716in}}%
\pgfpathlineto{\pgfqpoint{6.863757in}{3.457184in}}%
\pgfpathlineto{\pgfqpoint{6.864039in}{3.460839in}}%
\pgfpathlineto{\pgfqpoint{6.865167in}{3.533697in}}%
\pgfpathlineto{\pgfqpoint{6.869394in}{3.888878in}}%
\pgfpathlineto{\pgfqpoint{6.869958in}{3.892848in}}%
\pgfpathlineto{\pgfqpoint{6.870522in}{3.891491in}}%
\pgfpathlineto{\pgfqpoint{6.872213in}{3.871660in}}%
\pgfpathlineto{\pgfqpoint{6.875877in}{3.803805in}}%
\pgfpathlineto{\pgfqpoint{6.879541in}{3.730877in}}%
\pgfpathlineto{\pgfqpoint{6.880104in}{3.733635in}}%
\pgfpathlineto{\pgfqpoint{6.881232in}{3.756591in}}%
\pgfpathlineto{\pgfqpoint{6.883487in}{3.867705in}}%
\pgfpathlineto{\pgfqpoint{6.887432in}{4.050020in}}%
\pgfpathlineto{\pgfqpoint{6.888560in}{4.057735in}}%
\pgfpathlineto{\pgfqpoint{6.888842in}{4.056579in}}%
\pgfpathlineto{\pgfqpoint{6.889969in}{4.041445in}}%
\pgfpathlineto{\pgfqpoint{6.893351in}{3.945960in}}%
\pgfpathlineto{\pgfqpoint{6.896733in}{3.881392in}}%
\pgfpathlineto{\pgfqpoint{6.897297in}{3.879857in}}%
\pgfpathlineto{\pgfqpoint{6.897579in}{3.880492in}}%
\pgfpathlineto{\pgfqpoint{6.898706in}{3.893280in}}%
\pgfpathlineto{\pgfqpoint{6.900679in}{3.957222in}}%
\pgfpathlineto{\pgfqpoint{6.907725in}{4.271633in}}%
\pgfpathlineto{\pgfqpoint{6.908289in}{4.267997in}}%
\pgfpathlineto{\pgfqpoint{6.909980in}{4.234546in}}%
\pgfpathlineto{\pgfqpoint{6.913644in}{4.164911in}}%
\pgfpathlineto{\pgfqpoint{6.914208in}{4.165577in}}%
\pgfpathlineto{\pgfqpoint{6.915335in}{4.176542in}}%
\pgfpathlineto{\pgfqpoint{6.917308in}{4.224435in}}%
\pgfpathlineto{\pgfqpoint{6.919845in}{4.353789in}}%
\pgfpathlineto{\pgfqpoint{6.925200in}{4.726229in}}%
\pgfpathlineto{\pgfqpoint{6.925763in}{4.722462in}}%
\pgfpathlineto{\pgfqpoint{6.927173in}{4.678068in}}%
\pgfpathlineto{\pgfqpoint{6.928864in}{4.631929in}}%
\pgfpathlineto{\pgfqpoint{6.929146in}{4.634062in}}%
\pgfpathlineto{\pgfqpoint{6.930273in}{4.676168in}}%
\pgfpathlineto{\pgfqpoint{6.931682in}{4.720900in}}%
\pgfpathlineto{\pgfqpoint{6.931964in}{4.717748in}}%
\pgfpathlineto{\pgfqpoint{6.933655in}{4.659914in}}%
\pgfpathlineto{\pgfqpoint{6.934219in}{4.678480in}}%
\pgfpathlineto{\pgfqpoint{6.937037in}{5.006578in}}%
\pgfpathlineto{\pgfqpoint{6.937883in}{4.963973in}}%
\pgfpathlineto{\pgfqpoint{6.939010in}{4.915522in}}%
\pgfpathlineto{\pgfqpoint{6.939574in}{4.924495in}}%
\pgfpathlineto{\pgfqpoint{6.940701in}{4.954089in}}%
\pgfpathlineto{\pgfqpoint{6.941265in}{4.947891in}}%
\pgfpathlineto{\pgfqpoint{6.942392in}{4.860786in}}%
\pgfpathlineto{\pgfqpoint{6.946902in}{4.395838in}}%
\pgfpathlineto{\pgfqpoint{6.948875in}{4.369557in}}%
\pgfpathlineto{\pgfqpoint{6.949720in}{4.367826in}}%
\pgfpathlineto{\pgfqpoint{6.950002in}{4.368185in}}%
\pgfpathlineto{\pgfqpoint{6.950848in}{4.372748in}}%
\pgfpathlineto{\pgfqpoint{6.952539in}{4.400762in}}%
\pgfpathlineto{\pgfqpoint{6.955639in}{4.454118in}}%
\pgfpathlineto{\pgfqpoint{6.956485in}{4.444799in}}%
\pgfpathlineto{\pgfqpoint{6.957894in}{4.392063in}}%
\pgfpathlineto{\pgfqpoint{6.960712in}{4.173039in}}%
\pgfpathlineto{\pgfqpoint{6.964376in}{3.948525in}}%
\pgfpathlineto{\pgfqpoint{6.966068in}{3.926898in}}%
\pgfpathlineto{\pgfqpoint{6.966913in}{3.933067in}}%
\pgfpathlineto{\pgfqpoint{6.968886in}{3.977164in}}%
\pgfpathlineto{\pgfqpoint{6.974241in}{4.105966in}}%
\pgfpathlineto{\pgfqpoint{6.975087in}{4.098348in}}%
\pgfpathlineto{\pgfqpoint{6.976496in}{4.056984in}}%
\pgfpathlineto{\pgfqpoint{6.979314in}{3.884026in}}%
\pgfpathlineto{\pgfqpoint{6.982978in}{3.699928in}}%
\pgfpathlineto{\pgfqpoint{6.983824in}{3.693398in}}%
\pgfpathlineto{\pgfqpoint{6.984106in}{3.694341in}}%
\pgfpathlineto{\pgfqpoint{6.985233in}{3.711932in}}%
\pgfpathlineto{\pgfqpoint{6.988052in}{3.814057in}}%
\pgfpathlineto{\pgfqpoint{6.991716in}{3.915261in}}%
\pgfpathlineto{\pgfqpoint{6.992561in}{3.919058in}}%
\pgfpathlineto{\pgfqpoint{6.992843in}{3.918014in}}%
\pgfpathlineto{\pgfqpoint{6.993970in}{3.900292in}}%
\pgfpathlineto{\pgfqpoint{6.995661in}{3.825484in}}%
\pgfpathlineto{\pgfqpoint{6.999044in}{3.526905in}}%
\pgfpathlineto{\pgfqpoint{7.001016in}{3.436211in}}%
\pgfpathlineto{\pgfqpoint{7.001298in}{3.436435in}}%
\pgfpathlineto{\pgfqpoint{7.002144in}{3.454223in}}%
\pgfpathlineto{\pgfqpoint{7.004680in}{3.542222in}}%
\pgfpathlineto{\pgfqpoint{7.005244in}{3.534964in}}%
\pgfpathlineto{\pgfqpoint{7.006653in}{3.503539in}}%
\pgfpathlineto{\pgfqpoint{7.006935in}{3.507733in}}%
\pgfpathlineto{\pgfqpoint{7.008063in}{3.576724in}}%
\pgfpathlineto{\pgfqpoint{7.009190in}{3.644744in}}%
\pgfpathlineto{\pgfqpoint{7.009472in}{3.639583in}}%
\pgfpathlineto{\pgfqpoint{7.010599in}{3.515542in}}%
\pgfpathlineto{\pgfqpoint{7.012290in}{3.358872in}}%
\pgfpathlineto{\pgfqpoint{7.012572in}{3.358983in}}%
\pgfpathlineto{\pgfqpoint{7.014263in}{3.403486in}}%
\pgfpathlineto{\pgfqpoint{7.015109in}{3.394376in}}%
\pgfpathlineto{\pgfqpoint{7.016518in}{3.374439in}}%
\pgfpathlineto{\pgfqpoint{7.016800in}{3.377603in}}%
\pgfpathlineto{\pgfqpoint{7.017927in}{3.428914in}}%
\pgfpathlineto{\pgfqpoint{7.021028in}{3.767326in}}%
\pgfpathlineto{\pgfqpoint{7.023282in}{3.889859in}}%
\pgfpathlineto{\pgfqpoint{7.023846in}{3.894973in}}%
\pgfpathlineto{\pgfqpoint{7.024410in}{3.892429in}}%
\pgfpathlineto{\pgfqpoint{7.025819in}{3.861017in}}%
\pgfpathlineto{\pgfqpoint{7.030047in}{3.682661in}}%
\pgfpathlineto{\pgfqpoint{7.032583in}{3.622728in}}%
\pgfpathlineto{\pgfqpoint{7.033147in}{3.625213in}}%
\pgfpathlineto{\pgfqpoint{7.034274in}{3.651487in}}%
\pgfpathlineto{\pgfqpoint{7.036529in}{3.782955in}}%
\pgfpathlineto{\pgfqpoint{7.041320in}{4.068406in}}%
\pgfpathlineto{\pgfqpoint{7.042166in}{4.074515in}}%
\pgfpathlineto{\pgfqpoint{7.042448in}{4.073202in}}%
\pgfpathlineto{\pgfqpoint{7.043575in}{4.053285in}}%
\pgfpathlineto{\pgfqpoint{7.046394in}{3.942576in}}%
\pgfpathlineto{\pgfqpoint{7.049776in}{3.840740in}}%
\pgfpathlineto{\pgfqpoint{7.050340in}{3.838762in}}%
\pgfpathlineto{\pgfqpoint{7.050621in}{3.839981in}}%
\pgfpathlineto{\pgfqpoint{7.051749in}{3.860784in}}%
\pgfpathlineto{\pgfqpoint{7.053722in}{3.961138in}}%
\pgfpathlineto{\pgfqpoint{7.060768in}{4.429784in}}%
\pgfpathlineto{\pgfqpoint{7.061050in}{4.429055in}}%
\pgfpathlineto{\pgfqpoint{7.062177in}{4.411021in}}%
\pgfpathlineto{\pgfqpoint{7.066687in}{4.303298in}}%
\pgfpathlineto{\pgfqpoint{7.067250in}{4.304729in}}%
\pgfpathlineto{\pgfqpoint{7.068378in}{4.323906in}}%
\pgfpathlineto{\pgfqpoint{7.070069in}{4.408457in}}%
\pgfpathlineto{\pgfqpoint{7.072324in}{4.676190in}}%
\pgfpathlineto{\pgfqpoint{7.075142in}{4.964408in}}%
\pgfpathlineto{\pgfqpoint{7.078242in}{5.018633in}}%
\pgfpathlineto{\pgfqpoint{7.079088in}{4.978330in}}%
\pgfpathlineto{\pgfqpoint{7.083879in}{4.607951in}}%
\pgfpathlineto{\pgfqpoint{7.086416in}{4.543138in}}%
\pgfpathlineto{\pgfqpoint{7.086698in}{4.544531in}}%
\pgfpathlineto{\pgfqpoint{7.087825in}{4.568065in}}%
\pgfpathlineto{\pgfqpoint{7.090362in}{4.623588in}}%
\pgfpathlineto{\pgfqpoint{7.090644in}{4.622317in}}%
\pgfpathlineto{\pgfqpoint{7.091489in}{4.604531in}}%
\pgfpathlineto{\pgfqpoint{7.093180in}{4.506665in}}%
\pgfpathlineto{\pgfqpoint{7.100226in}{4.018750in}}%
\pgfpathlineto{\pgfqpoint{7.101636in}{4.000970in}}%
\pgfpathlineto{\pgfqpoint{7.101917in}{4.001019in}}%
\pgfpathlineto{\pgfqpoint{7.102763in}{4.007911in}}%
\pgfpathlineto{\pgfqpoint{7.104736in}{4.055543in}}%
\pgfpathlineto{\pgfqpoint{7.108400in}{4.143491in}}%
\pgfpathlineto{\pgfqpoint{7.108682in}{4.143721in}}%
\pgfpathlineto{\pgfqpoint{7.109527in}{4.135544in}}%
\pgfpathlineto{\pgfqpoint{7.110937in}{4.090740in}}%
\pgfpathlineto{\pgfqpoint{7.114319in}{3.878317in}}%
\pgfpathlineto{\pgfqpoint{7.117701in}{3.727364in}}%
\pgfpathlineto{\pgfqpoint{7.118828in}{3.716468in}}%
\pgfpathlineto{\pgfqpoint{7.119110in}{3.716888in}}%
\pgfpathlineto{\pgfqpoint{7.119956in}{3.724989in}}%
\pgfpathlineto{\pgfqpoint{7.121929in}{3.775291in}}%
\pgfpathlineto{\pgfqpoint{7.127565in}{3.939897in}}%
\pgfpathlineto{\pgfqpoint{7.128129in}{3.936573in}}%
\pgfpathlineto{\pgfqpoint{7.129257in}{3.909738in}}%
\pgfpathlineto{\pgfqpoint{7.131229in}{3.797716in}}%
\pgfpathlineto{\pgfqpoint{7.135739in}{3.541140in}}%
\pgfpathlineto{\pgfqpoint{7.136303in}{3.537773in}}%
\pgfpathlineto{\pgfqpoint{7.136866in}{3.540875in}}%
\pgfpathlineto{\pgfqpoint{7.138557in}{3.574600in}}%
\pgfpathlineto{\pgfqpoint{7.141376in}{3.615698in}}%
\pgfpathlineto{\pgfqpoint{7.142785in}{3.638046in}}%
\pgfpathlineto{\pgfqpoint{7.144758in}{3.686802in}}%
\pgfpathlineto{\pgfqpoint{7.145322in}{3.676877in}}%
\pgfpathlineto{\pgfqpoint{7.146449in}{3.565049in}}%
\pgfpathlineto{\pgfqpoint{7.148704in}{3.312845in}}%
\pgfpathlineto{\pgfqpoint{7.148986in}{3.316302in}}%
\pgfpathlineto{\pgfqpoint{7.150113in}{3.348744in}}%
\pgfpathlineto{\pgfqpoint{7.150677in}{3.335702in}}%
\pgfpathlineto{\pgfqpoint{7.152650in}{3.220194in}}%
\pgfpathlineto{\pgfqpoint{7.153214in}{3.239298in}}%
\pgfpathlineto{\pgfqpoint{7.154905in}{3.463618in}}%
\pgfpathlineto{\pgfqpoint{7.157441in}{3.704767in}}%
\pgfpathlineto{\pgfqpoint{7.160823in}{3.800364in}}%
\pgfpathlineto{\pgfqpoint{7.161669in}{3.806277in}}%
\pgfpathlineto{\pgfqpoint{7.161951in}{3.804774in}}%
\pgfpathlineto{\pgfqpoint{7.163078in}{3.779124in}}%
\pgfpathlineto{\pgfqpoint{7.168151in}{3.572391in}}%
\pgfpathlineto{\pgfqpoint{7.168997in}{3.581104in}}%
\pgfpathlineto{\pgfqpoint{7.170406in}{3.640513in}}%
\pgfpathlineto{\pgfqpoint{7.177170in}{3.992091in}}%
\pgfpathlineto{\pgfqpoint{7.177734in}{3.993851in}}%
\pgfpathlineto{\pgfqpoint{7.178016in}{3.993083in}}%
\pgfpathlineto{\pgfqpoint{7.179143in}{3.980429in}}%
\pgfpathlineto{\pgfqpoint{7.181398in}{3.918849in}}%
\pgfpathlineto{\pgfqpoint{7.186471in}{3.771025in}}%
\pgfpathlineto{\pgfqpoint{7.187317in}{3.780518in}}%
\pgfpathlineto{\pgfqpoint{7.188726in}{3.833937in}}%
\pgfpathlineto{\pgfqpoint{7.192108in}{4.095284in}}%
\pgfpathlineto{\pgfqpoint{7.195209in}{4.244731in}}%
\pgfpathlineto{\pgfqpoint{7.196336in}{4.253943in}}%
\pgfpathlineto{\pgfqpoint{7.196618in}{4.253232in}}%
\pgfpathlineto{\pgfqpoint{7.198027in}{4.238356in}}%
\pgfpathlineto{\pgfqpoint{7.200000in}{4.209328in}}%
\pgfpathlineto{\pgfqpoint{7.200000in}{4.209328in}}%
\pgfusepath{stroke}%
\end{pgfscope}%
\begin{pgfscope}%
\pgfsetbuttcap%
\pgfsetroundjoin%
\definecolor{currentfill}{rgb}{0.000000,0.000000,0.000000}%
\pgfsetfillcolor{currentfill}%
\pgfsetlinewidth{0.501875pt}%
\definecolor{currentstroke}{rgb}{0.000000,0.000000,0.000000}%
\pgfsetstrokecolor{currentstroke}%
\pgfsetdash{}{0pt}%
\pgfsys@defobject{currentmarker}{\pgfqpoint{0.000000in}{0.000000in}}{\pgfqpoint{0.000000in}{0.055556in}}{%
\pgfpathmoveto{\pgfqpoint{0.000000in}{0.000000in}}%
\pgfpathlineto{\pgfqpoint{0.000000in}{0.055556in}}%
\pgfusepath{stroke,fill}%
}%
\begin{pgfscope}%
\pgfsys@transformshift{4.381818in}{3.218182in}%
\pgfsys@useobject{currentmarker}{}%
\end{pgfscope}%
\end{pgfscope}%
\begin{pgfscope}%
\pgfsetbuttcap%
\pgfsetroundjoin%
\definecolor{currentfill}{rgb}{0.000000,0.000000,0.000000}%
\pgfsetfillcolor{currentfill}%
\pgfsetlinewidth{0.501875pt}%
\definecolor{currentstroke}{rgb}{0.000000,0.000000,0.000000}%
\pgfsetstrokecolor{currentstroke}%
\pgfsetdash{}{0pt}%
\pgfsys@defobject{currentmarker}{\pgfqpoint{0.000000in}{-0.055556in}}{\pgfqpoint{0.000000in}{0.000000in}}{%
\pgfpathmoveto{\pgfqpoint{0.000000in}{0.000000in}}%
\pgfpathlineto{\pgfqpoint{0.000000in}{-0.055556in}}%
\pgfusepath{stroke,fill}%
}%
\begin{pgfscope}%
\pgfsys@transformshift{4.381818in}{5.400000in}%
\pgfsys@useobject{currentmarker}{}%
\end{pgfscope}%
\end{pgfscope}%
\begin{pgfscope}%
\pgftext[x=4.381818in,y=3.162626in,,top]{{\rmfamily\fontsize{12.000000}{14.400000}\selectfont 0}}%
\end{pgfscope}%
\begin{pgfscope}%
\pgfsetbuttcap%
\pgfsetroundjoin%
\definecolor{currentfill}{rgb}{0.000000,0.000000,0.000000}%
\pgfsetfillcolor{currentfill}%
\pgfsetlinewidth{0.501875pt}%
\definecolor{currentstroke}{rgb}{0.000000,0.000000,0.000000}%
\pgfsetstrokecolor{currentstroke}%
\pgfsetdash{}{0pt}%
\pgfsys@defobject{currentmarker}{\pgfqpoint{0.000000in}{0.000000in}}{\pgfqpoint{0.000000in}{0.055556in}}{%
\pgfpathmoveto{\pgfqpoint{0.000000in}{0.000000in}}%
\pgfpathlineto{\pgfqpoint{0.000000in}{0.055556in}}%
\pgfusepath{stroke,fill}%
}%
\begin{pgfscope}%
\pgfsys@transformshift{4.945455in}{3.218182in}%
\pgfsys@useobject{currentmarker}{}%
\end{pgfscope}%
\end{pgfscope}%
\begin{pgfscope}%
\pgfsetbuttcap%
\pgfsetroundjoin%
\definecolor{currentfill}{rgb}{0.000000,0.000000,0.000000}%
\pgfsetfillcolor{currentfill}%
\pgfsetlinewidth{0.501875pt}%
\definecolor{currentstroke}{rgb}{0.000000,0.000000,0.000000}%
\pgfsetstrokecolor{currentstroke}%
\pgfsetdash{}{0pt}%
\pgfsys@defobject{currentmarker}{\pgfqpoint{0.000000in}{-0.055556in}}{\pgfqpoint{0.000000in}{0.000000in}}{%
\pgfpathmoveto{\pgfqpoint{0.000000in}{0.000000in}}%
\pgfpathlineto{\pgfqpoint{0.000000in}{-0.055556in}}%
\pgfusepath{stroke,fill}%
}%
\begin{pgfscope}%
\pgfsys@transformshift{4.945455in}{5.400000in}%
\pgfsys@useobject{currentmarker}{}%
\end{pgfscope}%
\end{pgfscope}%
\begin{pgfscope}%
\pgftext[x=4.945455in,y=3.162626in,,top]{{\rmfamily\fontsize{12.000000}{14.400000}\selectfont 2}}%
\end{pgfscope}%
\begin{pgfscope}%
\pgfsetbuttcap%
\pgfsetroundjoin%
\definecolor{currentfill}{rgb}{0.000000,0.000000,0.000000}%
\pgfsetfillcolor{currentfill}%
\pgfsetlinewidth{0.501875pt}%
\definecolor{currentstroke}{rgb}{0.000000,0.000000,0.000000}%
\pgfsetstrokecolor{currentstroke}%
\pgfsetdash{}{0pt}%
\pgfsys@defobject{currentmarker}{\pgfqpoint{0.000000in}{0.000000in}}{\pgfqpoint{0.000000in}{0.055556in}}{%
\pgfpathmoveto{\pgfqpoint{0.000000in}{0.000000in}}%
\pgfpathlineto{\pgfqpoint{0.000000in}{0.055556in}}%
\pgfusepath{stroke,fill}%
}%
\begin{pgfscope}%
\pgfsys@transformshift{5.509091in}{3.218182in}%
\pgfsys@useobject{currentmarker}{}%
\end{pgfscope}%
\end{pgfscope}%
\begin{pgfscope}%
\pgfsetbuttcap%
\pgfsetroundjoin%
\definecolor{currentfill}{rgb}{0.000000,0.000000,0.000000}%
\pgfsetfillcolor{currentfill}%
\pgfsetlinewidth{0.501875pt}%
\definecolor{currentstroke}{rgb}{0.000000,0.000000,0.000000}%
\pgfsetstrokecolor{currentstroke}%
\pgfsetdash{}{0pt}%
\pgfsys@defobject{currentmarker}{\pgfqpoint{0.000000in}{-0.055556in}}{\pgfqpoint{0.000000in}{0.000000in}}{%
\pgfpathmoveto{\pgfqpoint{0.000000in}{0.000000in}}%
\pgfpathlineto{\pgfqpoint{0.000000in}{-0.055556in}}%
\pgfusepath{stroke,fill}%
}%
\begin{pgfscope}%
\pgfsys@transformshift{5.509091in}{5.400000in}%
\pgfsys@useobject{currentmarker}{}%
\end{pgfscope}%
\end{pgfscope}%
\begin{pgfscope}%
\pgftext[x=5.509091in,y=3.162626in,,top]{{\rmfamily\fontsize{12.000000}{14.400000}\selectfont 4}}%
\end{pgfscope}%
\begin{pgfscope}%
\pgfsetbuttcap%
\pgfsetroundjoin%
\definecolor{currentfill}{rgb}{0.000000,0.000000,0.000000}%
\pgfsetfillcolor{currentfill}%
\pgfsetlinewidth{0.501875pt}%
\definecolor{currentstroke}{rgb}{0.000000,0.000000,0.000000}%
\pgfsetstrokecolor{currentstroke}%
\pgfsetdash{}{0pt}%
\pgfsys@defobject{currentmarker}{\pgfqpoint{0.000000in}{0.000000in}}{\pgfqpoint{0.000000in}{0.055556in}}{%
\pgfpathmoveto{\pgfqpoint{0.000000in}{0.000000in}}%
\pgfpathlineto{\pgfqpoint{0.000000in}{0.055556in}}%
\pgfusepath{stroke,fill}%
}%
\begin{pgfscope}%
\pgfsys@transformshift{6.072727in}{3.218182in}%
\pgfsys@useobject{currentmarker}{}%
\end{pgfscope}%
\end{pgfscope}%
\begin{pgfscope}%
\pgfsetbuttcap%
\pgfsetroundjoin%
\definecolor{currentfill}{rgb}{0.000000,0.000000,0.000000}%
\pgfsetfillcolor{currentfill}%
\pgfsetlinewidth{0.501875pt}%
\definecolor{currentstroke}{rgb}{0.000000,0.000000,0.000000}%
\pgfsetstrokecolor{currentstroke}%
\pgfsetdash{}{0pt}%
\pgfsys@defobject{currentmarker}{\pgfqpoint{0.000000in}{-0.055556in}}{\pgfqpoint{0.000000in}{0.000000in}}{%
\pgfpathmoveto{\pgfqpoint{0.000000in}{0.000000in}}%
\pgfpathlineto{\pgfqpoint{0.000000in}{-0.055556in}}%
\pgfusepath{stroke,fill}%
}%
\begin{pgfscope}%
\pgfsys@transformshift{6.072727in}{5.400000in}%
\pgfsys@useobject{currentmarker}{}%
\end{pgfscope}%
\end{pgfscope}%
\begin{pgfscope}%
\pgftext[x=6.072727in,y=3.162626in,,top]{{\rmfamily\fontsize{12.000000}{14.400000}\selectfont 6}}%
\end{pgfscope}%
\begin{pgfscope}%
\pgfsetbuttcap%
\pgfsetroundjoin%
\definecolor{currentfill}{rgb}{0.000000,0.000000,0.000000}%
\pgfsetfillcolor{currentfill}%
\pgfsetlinewidth{0.501875pt}%
\definecolor{currentstroke}{rgb}{0.000000,0.000000,0.000000}%
\pgfsetstrokecolor{currentstroke}%
\pgfsetdash{}{0pt}%
\pgfsys@defobject{currentmarker}{\pgfqpoint{0.000000in}{0.000000in}}{\pgfqpoint{0.000000in}{0.055556in}}{%
\pgfpathmoveto{\pgfqpoint{0.000000in}{0.000000in}}%
\pgfpathlineto{\pgfqpoint{0.000000in}{0.055556in}}%
\pgfusepath{stroke,fill}%
}%
\begin{pgfscope}%
\pgfsys@transformshift{6.636364in}{3.218182in}%
\pgfsys@useobject{currentmarker}{}%
\end{pgfscope}%
\end{pgfscope}%
\begin{pgfscope}%
\pgfsetbuttcap%
\pgfsetroundjoin%
\definecolor{currentfill}{rgb}{0.000000,0.000000,0.000000}%
\pgfsetfillcolor{currentfill}%
\pgfsetlinewidth{0.501875pt}%
\definecolor{currentstroke}{rgb}{0.000000,0.000000,0.000000}%
\pgfsetstrokecolor{currentstroke}%
\pgfsetdash{}{0pt}%
\pgfsys@defobject{currentmarker}{\pgfqpoint{0.000000in}{-0.055556in}}{\pgfqpoint{0.000000in}{0.000000in}}{%
\pgfpathmoveto{\pgfqpoint{0.000000in}{0.000000in}}%
\pgfpathlineto{\pgfqpoint{0.000000in}{-0.055556in}}%
\pgfusepath{stroke,fill}%
}%
\begin{pgfscope}%
\pgfsys@transformshift{6.636364in}{5.400000in}%
\pgfsys@useobject{currentmarker}{}%
\end{pgfscope}%
\end{pgfscope}%
\begin{pgfscope}%
\pgftext[x=6.636364in,y=3.162626in,,top]{{\rmfamily\fontsize{12.000000}{14.400000}\selectfont 8}}%
\end{pgfscope}%
\begin{pgfscope}%
\pgfsetbuttcap%
\pgfsetroundjoin%
\definecolor{currentfill}{rgb}{0.000000,0.000000,0.000000}%
\pgfsetfillcolor{currentfill}%
\pgfsetlinewidth{0.501875pt}%
\definecolor{currentstroke}{rgb}{0.000000,0.000000,0.000000}%
\pgfsetstrokecolor{currentstroke}%
\pgfsetdash{}{0pt}%
\pgfsys@defobject{currentmarker}{\pgfqpoint{0.000000in}{0.000000in}}{\pgfqpoint{0.000000in}{0.055556in}}{%
\pgfpathmoveto{\pgfqpoint{0.000000in}{0.000000in}}%
\pgfpathlineto{\pgfqpoint{0.000000in}{0.055556in}}%
\pgfusepath{stroke,fill}%
}%
\begin{pgfscope}%
\pgfsys@transformshift{7.200000in}{3.218182in}%
\pgfsys@useobject{currentmarker}{}%
\end{pgfscope}%
\end{pgfscope}%
\begin{pgfscope}%
\pgfsetbuttcap%
\pgfsetroundjoin%
\definecolor{currentfill}{rgb}{0.000000,0.000000,0.000000}%
\pgfsetfillcolor{currentfill}%
\pgfsetlinewidth{0.501875pt}%
\definecolor{currentstroke}{rgb}{0.000000,0.000000,0.000000}%
\pgfsetstrokecolor{currentstroke}%
\pgfsetdash{}{0pt}%
\pgfsys@defobject{currentmarker}{\pgfqpoint{0.000000in}{-0.055556in}}{\pgfqpoint{0.000000in}{0.000000in}}{%
\pgfpathmoveto{\pgfqpoint{0.000000in}{0.000000in}}%
\pgfpathlineto{\pgfqpoint{0.000000in}{-0.055556in}}%
\pgfusepath{stroke,fill}%
}%
\begin{pgfscope}%
\pgfsys@transformshift{7.200000in}{5.400000in}%
\pgfsys@useobject{currentmarker}{}%
\end{pgfscope}%
\end{pgfscope}%
\begin{pgfscope}%
\pgftext[x=7.200000in,y=3.162626in,,top]{{\rmfamily\fontsize{12.000000}{14.400000}\selectfont 10}}%
\end{pgfscope}%
\begin{pgfscope}%
\pgfsetbuttcap%
\pgfsetroundjoin%
\definecolor{currentfill}{rgb}{0.000000,0.000000,0.000000}%
\pgfsetfillcolor{currentfill}%
\pgfsetlinewidth{0.501875pt}%
\definecolor{currentstroke}{rgb}{0.000000,0.000000,0.000000}%
\pgfsetstrokecolor{currentstroke}%
\pgfsetdash{}{0pt}%
\pgfsys@defobject{currentmarker}{\pgfqpoint{0.000000in}{0.000000in}}{\pgfqpoint{0.055556in}{0.000000in}}{%
\pgfpathmoveto{\pgfqpoint{0.000000in}{0.000000in}}%
\pgfpathlineto{\pgfqpoint{0.055556in}{0.000000in}}%
\pgfusepath{stroke,fill}%
}%
\begin{pgfscope}%
\pgfsys@transformshift{4.381818in}{3.218182in}%
\pgfsys@useobject{currentmarker}{}%
\end{pgfscope}%
\end{pgfscope}%
\begin{pgfscope}%
\pgfsetbuttcap%
\pgfsetroundjoin%
\definecolor{currentfill}{rgb}{0.000000,0.000000,0.000000}%
\pgfsetfillcolor{currentfill}%
\pgfsetlinewidth{0.501875pt}%
\definecolor{currentstroke}{rgb}{0.000000,0.000000,0.000000}%
\pgfsetstrokecolor{currentstroke}%
\pgfsetdash{}{0pt}%
\pgfsys@defobject{currentmarker}{\pgfqpoint{-0.055556in}{0.000000in}}{\pgfqpoint{0.000000in}{0.000000in}}{%
\pgfpathmoveto{\pgfqpoint{0.000000in}{0.000000in}}%
\pgfpathlineto{\pgfqpoint{-0.055556in}{0.000000in}}%
\pgfusepath{stroke,fill}%
}%
\begin{pgfscope}%
\pgfsys@transformshift{7.200000in}{3.218182in}%
\pgfsys@useobject{currentmarker}{}%
\end{pgfscope}%
\end{pgfscope}%
\begin{pgfscope}%
\pgftext[x=4.326263in,y=3.218182in,right,]{{\rmfamily\fontsize{12.000000}{14.400000}\selectfont -700}}%
\end{pgfscope}%
\begin{pgfscope}%
\pgfsetbuttcap%
\pgfsetroundjoin%
\definecolor{currentfill}{rgb}{0.000000,0.000000,0.000000}%
\pgfsetfillcolor{currentfill}%
\pgfsetlinewidth{0.501875pt}%
\definecolor{currentstroke}{rgb}{0.000000,0.000000,0.000000}%
\pgfsetstrokecolor{currentstroke}%
\pgfsetdash{}{0pt}%
\pgfsys@defobject{currentmarker}{\pgfqpoint{0.000000in}{0.000000in}}{\pgfqpoint{0.055556in}{0.000000in}}{%
\pgfpathmoveto{\pgfqpoint{0.000000in}{0.000000in}}%
\pgfpathlineto{\pgfqpoint{0.055556in}{0.000000in}}%
\pgfusepath{stroke,fill}%
}%
\begin{pgfscope}%
\pgfsys@transformshift{4.381818in}{3.460606in}%
\pgfsys@useobject{currentmarker}{}%
\end{pgfscope}%
\end{pgfscope}%
\begin{pgfscope}%
\pgfsetbuttcap%
\pgfsetroundjoin%
\definecolor{currentfill}{rgb}{0.000000,0.000000,0.000000}%
\pgfsetfillcolor{currentfill}%
\pgfsetlinewidth{0.501875pt}%
\definecolor{currentstroke}{rgb}{0.000000,0.000000,0.000000}%
\pgfsetstrokecolor{currentstroke}%
\pgfsetdash{}{0pt}%
\pgfsys@defobject{currentmarker}{\pgfqpoint{-0.055556in}{0.000000in}}{\pgfqpoint{0.000000in}{0.000000in}}{%
\pgfpathmoveto{\pgfqpoint{0.000000in}{0.000000in}}%
\pgfpathlineto{\pgfqpoint{-0.055556in}{0.000000in}}%
\pgfusepath{stroke,fill}%
}%
\begin{pgfscope}%
\pgfsys@transformshift{7.200000in}{3.460606in}%
\pgfsys@useobject{currentmarker}{}%
\end{pgfscope}%
\end{pgfscope}%
\begin{pgfscope}%
\pgftext[x=4.326263in,y=3.460606in,right,]{{\rmfamily\fontsize{12.000000}{14.400000}\selectfont -600}}%
\end{pgfscope}%
\begin{pgfscope}%
\pgfsetbuttcap%
\pgfsetroundjoin%
\definecolor{currentfill}{rgb}{0.000000,0.000000,0.000000}%
\pgfsetfillcolor{currentfill}%
\pgfsetlinewidth{0.501875pt}%
\definecolor{currentstroke}{rgb}{0.000000,0.000000,0.000000}%
\pgfsetstrokecolor{currentstroke}%
\pgfsetdash{}{0pt}%
\pgfsys@defobject{currentmarker}{\pgfqpoint{0.000000in}{0.000000in}}{\pgfqpoint{0.055556in}{0.000000in}}{%
\pgfpathmoveto{\pgfqpoint{0.000000in}{0.000000in}}%
\pgfpathlineto{\pgfqpoint{0.055556in}{0.000000in}}%
\pgfusepath{stroke,fill}%
}%
\begin{pgfscope}%
\pgfsys@transformshift{4.381818in}{3.703030in}%
\pgfsys@useobject{currentmarker}{}%
\end{pgfscope}%
\end{pgfscope}%
\begin{pgfscope}%
\pgfsetbuttcap%
\pgfsetroundjoin%
\definecolor{currentfill}{rgb}{0.000000,0.000000,0.000000}%
\pgfsetfillcolor{currentfill}%
\pgfsetlinewidth{0.501875pt}%
\definecolor{currentstroke}{rgb}{0.000000,0.000000,0.000000}%
\pgfsetstrokecolor{currentstroke}%
\pgfsetdash{}{0pt}%
\pgfsys@defobject{currentmarker}{\pgfqpoint{-0.055556in}{0.000000in}}{\pgfqpoint{0.000000in}{0.000000in}}{%
\pgfpathmoveto{\pgfqpoint{0.000000in}{0.000000in}}%
\pgfpathlineto{\pgfqpoint{-0.055556in}{0.000000in}}%
\pgfusepath{stroke,fill}%
}%
\begin{pgfscope}%
\pgfsys@transformshift{7.200000in}{3.703030in}%
\pgfsys@useobject{currentmarker}{}%
\end{pgfscope}%
\end{pgfscope}%
\begin{pgfscope}%
\pgftext[x=4.326263in,y=3.703030in,right,]{{\rmfamily\fontsize{12.000000}{14.400000}\selectfont -500}}%
\end{pgfscope}%
\begin{pgfscope}%
\pgfsetbuttcap%
\pgfsetroundjoin%
\definecolor{currentfill}{rgb}{0.000000,0.000000,0.000000}%
\pgfsetfillcolor{currentfill}%
\pgfsetlinewidth{0.501875pt}%
\definecolor{currentstroke}{rgb}{0.000000,0.000000,0.000000}%
\pgfsetstrokecolor{currentstroke}%
\pgfsetdash{}{0pt}%
\pgfsys@defobject{currentmarker}{\pgfqpoint{0.000000in}{0.000000in}}{\pgfqpoint{0.055556in}{0.000000in}}{%
\pgfpathmoveto{\pgfqpoint{0.000000in}{0.000000in}}%
\pgfpathlineto{\pgfqpoint{0.055556in}{0.000000in}}%
\pgfusepath{stroke,fill}%
}%
\begin{pgfscope}%
\pgfsys@transformshift{4.381818in}{3.945455in}%
\pgfsys@useobject{currentmarker}{}%
\end{pgfscope}%
\end{pgfscope}%
\begin{pgfscope}%
\pgfsetbuttcap%
\pgfsetroundjoin%
\definecolor{currentfill}{rgb}{0.000000,0.000000,0.000000}%
\pgfsetfillcolor{currentfill}%
\pgfsetlinewidth{0.501875pt}%
\definecolor{currentstroke}{rgb}{0.000000,0.000000,0.000000}%
\pgfsetstrokecolor{currentstroke}%
\pgfsetdash{}{0pt}%
\pgfsys@defobject{currentmarker}{\pgfqpoint{-0.055556in}{0.000000in}}{\pgfqpoint{0.000000in}{0.000000in}}{%
\pgfpathmoveto{\pgfqpoint{0.000000in}{0.000000in}}%
\pgfpathlineto{\pgfqpoint{-0.055556in}{0.000000in}}%
\pgfusepath{stroke,fill}%
}%
\begin{pgfscope}%
\pgfsys@transformshift{7.200000in}{3.945455in}%
\pgfsys@useobject{currentmarker}{}%
\end{pgfscope}%
\end{pgfscope}%
\begin{pgfscope}%
\pgftext[x=4.326263in,y=3.945455in,right,]{{\rmfamily\fontsize{12.000000}{14.400000}\selectfont -400}}%
\end{pgfscope}%
\begin{pgfscope}%
\pgfsetbuttcap%
\pgfsetroundjoin%
\definecolor{currentfill}{rgb}{0.000000,0.000000,0.000000}%
\pgfsetfillcolor{currentfill}%
\pgfsetlinewidth{0.501875pt}%
\definecolor{currentstroke}{rgb}{0.000000,0.000000,0.000000}%
\pgfsetstrokecolor{currentstroke}%
\pgfsetdash{}{0pt}%
\pgfsys@defobject{currentmarker}{\pgfqpoint{0.000000in}{0.000000in}}{\pgfqpoint{0.055556in}{0.000000in}}{%
\pgfpathmoveto{\pgfqpoint{0.000000in}{0.000000in}}%
\pgfpathlineto{\pgfqpoint{0.055556in}{0.000000in}}%
\pgfusepath{stroke,fill}%
}%
\begin{pgfscope}%
\pgfsys@transformshift{4.381818in}{4.187879in}%
\pgfsys@useobject{currentmarker}{}%
\end{pgfscope}%
\end{pgfscope}%
\begin{pgfscope}%
\pgfsetbuttcap%
\pgfsetroundjoin%
\definecolor{currentfill}{rgb}{0.000000,0.000000,0.000000}%
\pgfsetfillcolor{currentfill}%
\pgfsetlinewidth{0.501875pt}%
\definecolor{currentstroke}{rgb}{0.000000,0.000000,0.000000}%
\pgfsetstrokecolor{currentstroke}%
\pgfsetdash{}{0pt}%
\pgfsys@defobject{currentmarker}{\pgfqpoint{-0.055556in}{0.000000in}}{\pgfqpoint{0.000000in}{0.000000in}}{%
\pgfpathmoveto{\pgfqpoint{0.000000in}{0.000000in}}%
\pgfpathlineto{\pgfqpoint{-0.055556in}{0.000000in}}%
\pgfusepath{stroke,fill}%
}%
\begin{pgfscope}%
\pgfsys@transformshift{7.200000in}{4.187879in}%
\pgfsys@useobject{currentmarker}{}%
\end{pgfscope}%
\end{pgfscope}%
\begin{pgfscope}%
\pgftext[x=4.326263in,y=4.187879in,right,]{{\rmfamily\fontsize{12.000000}{14.400000}\selectfont -300}}%
\end{pgfscope}%
\begin{pgfscope}%
\pgfsetbuttcap%
\pgfsetroundjoin%
\definecolor{currentfill}{rgb}{0.000000,0.000000,0.000000}%
\pgfsetfillcolor{currentfill}%
\pgfsetlinewidth{0.501875pt}%
\definecolor{currentstroke}{rgb}{0.000000,0.000000,0.000000}%
\pgfsetstrokecolor{currentstroke}%
\pgfsetdash{}{0pt}%
\pgfsys@defobject{currentmarker}{\pgfqpoint{0.000000in}{0.000000in}}{\pgfqpoint{0.055556in}{0.000000in}}{%
\pgfpathmoveto{\pgfqpoint{0.000000in}{0.000000in}}%
\pgfpathlineto{\pgfqpoint{0.055556in}{0.000000in}}%
\pgfusepath{stroke,fill}%
}%
\begin{pgfscope}%
\pgfsys@transformshift{4.381818in}{4.430303in}%
\pgfsys@useobject{currentmarker}{}%
\end{pgfscope}%
\end{pgfscope}%
\begin{pgfscope}%
\pgfsetbuttcap%
\pgfsetroundjoin%
\definecolor{currentfill}{rgb}{0.000000,0.000000,0.000000}%
\pgfsetfillcolor{currentfill}%
\pgfsetlinewidth{0.501875pt}%
\definecolor{currentstroke}{rgb}{0.000000,0.000000,0.000000}%
\pgfsetstrokecolor{currentstroke}%
\pgfsetdash{}{0pt}%
\pgfsys@defobject{currentmarker}{\pgfqpoint{-0.055556in}{0.000000in}}{\pgfqpoint{0.000000in}{0.000000in}}{%
\pgfpathmoveto{\pgfqpoint{0.000000in}{0.000000in}}%
\pgfpathlineto{\pgfqpoint{-0.055556in}{0.000000in}}%
\pgfusepath{stroke,fill}%
}%
\begin{pgfscope}%
\pgfsys@transformshift{7.200000in}{4.430303in}%
\pgfsys@useobject{currentmarker}{}%
\end{pgfscope}%
\end{pgfscope}%
\begin{pgfscope}%
\pgftext[x=4.326263in,y=4.430303in,right,]{{\rmfamily\fontsize{12.000000}{14.400000}\selectfont -200}}%
\end{pgfscope}%
\begin{pgfscope}%
\pgfsetbuttcap%
\pgfsetroundjoin%
\definecolor{currentfill}{rgb}{0.000000,0.000000,0.000000}%
\pgfsetfillcolor{currentfill}%
\pgfsetlinewidth{0.501875pt}%
\definecolor{currentstroke}{rgb}{0.000000,0.000000,0.000000}%
\pgfsetstrokecolor{currentstroke}%
\pgfsetdash{}{0pt}%
\pgfsys@defobject{currentmarker}{\pgfqpoint{0.000000in}{0.000000in}}{\pgfqpoint{0.055556in}{0.000000in}}{%
\pgfpathmoveto{\pgfqpoint{0.000000in}{0.000000in}}%
\pgfpathlineto{\pgfqpoint{0.055556in}{0.000000in}}%
\pgfusepath{stroke,fill}%
}%
\begin{pgfscope}%
\pgfsys@transformshift{4.381818in}{4.672727in}%
\pgfsys@useobject{currentmarker}{}%
\end{pgfscope}%
\end{pgfscope}%
\begin{pgfscope}%
\pgfsetbuttcap%
\pgfsetroundjoin%
\definecolor{currentfill}{rgb}{0.000000,0.000000,0.000000}%
\pgfsetfillcolor{currentfill}%
\pgfsetlinewidth{0.501875pt}%
\definecolor{currentstroke}{rgb}{0.000000,0.000000,0.000000}%
\pgfsetstrokecolor{currentstroke}%
\pgfsetdash{}{0pt}%
\pgfsys@defobject{currentmarker}{\pgfqpoint{-0.055556in}{0.000000in}}{\pgfqpoint{0.000000in}{0.000000in}}{%
\pgfpathmoveto{\pgfqpoint{0.000000in}{0.000000in}}%
\pgfpathlineto{\pgfqpoint{-0.055556in}{0.000000in}}%
\pgfusepath{stroke,fill}%
}%
\begin{pgfscope}%
\pgfsys@transformshift{7.200000in}{4.672727in}%
\pgfsys@useobject{currentmarker}{}%
\end{pgfscope}%
\end{pgfscope}%
\begin{pgfscope}%
\pgftext[x=4.326263in,y=4.672727in,right,]{{\rmfamily\fontsize{12.000000}{14.400000}\selectfont -100}}%
\end{pgfscope}%
\begin{pgfscope}%
\pgfsetbuttcap%
\pgfsetroundjoin%
\definecolor{currentfill}{rgb}{0.000000,0.000000,0.000000}%
\pgfsetfillcolor{currentfill}%
\pgfsetlinewidth{0.501875pt}%
\definecolor{currentstroke}{rgb}{0.000000,0.000000,0.000000}%
\pgfsetstrokecolor{currentstroke}%
\pgfsetdash{}{0pt}%
\pgfsys@defobject{currentmarker}{\pgfqpoint{0.000000in}{0.000000in}}{\pgfqpoint{0.055556in}{0.000000in}}{%
\pgfpathmoveto{\pgfqpoint{0.000000in}{0.000000in}}%
\pgfpathlineto{\pgfqpoint{0.055556in}{0.000000in}}%
\pgfusepath{stroke,fill}%
}%
\begin{pgfscope}%
\pgfsys@transformshift{4.381818in}{4.915152in}%
\pgfsys@useobject{currentmarker}{}%
\end{pgfscope}%
\end{pgfscope}%
\begin{pgfscope}%
\pgfsetbuttcap%
\pgfsetroundjoin%
\definecolor{currentfill}{rgb}{0.000000,0.000000,0.000000}%
\pgfsetfillcolor{currentfill}%
\pgfsetlinewidth{0.501875pt}%
\definecolor{currentstroke}{rgb}{0.000000,0.000000,0.000000}%
\pgfsetstrokecolor{currentstroke}%
\pgfsetdash{}{0pt}%
\pgfsys@defobject{currentmarker}{\pgfqpoint{-0.055556in}{0.000000in}}{\pgfqpoint{0.000000in}{0.000000in}}{%
\pgfpathmoveto{\pgfqpoint{0.000000in}{0.000000in}}%
\pgfpathlineto{\pgfqpoint{-0.055556in}{0.000000in}}%
\pgfusepath{stroke,fill}%
}%
\begin{pgfscope}%
\pgfsys@transformshift{7.200000in}{4.915152in}%
\pgfsys@useobject{currentmarker}{}%
\end{pgfscope}%
\end{pgfscope}%
\begin{pgfscope}%
\pgftext[x=4.326263in,y=4.915152in,right,]{{\rmfamily\fontsize{12.000000}{14.400000}\selectfont 0}}%
\end{pgfscope}%
\begin{pgfscope}%
\pgfsetbuttcap%
\pgfsetroundjoin%
\definecolor{currentfill}{rgb}{0.000000,0.000000,0.000000}%
\pgfsetfillcolor{currentfill}%
\pgfsetlinewidth{0.501875pt}%
\definecolor{currentstroke}{rgb}{0.000000,0.000000,0.000000}%
\pgfsetstrokecolor{currentstroke}%
\pgfsetdash{}{0pt}%
\pgfsys@defobject{currentmarker}{\pgfqpoint{0.000000in}{0.000000in}}{\pgfqpoint{0.055556in}{0.000000in}}{%
\pgfpathmoveto{\pgfqpoint{0.000000in}{0.000000in}}%
\pgfpathlineto{\pgfqpoint{0.055556in}{0.000000in}}%
\pgfusepath{stroke,fill}%
}%
\begin{pgfscope}%
\pgfsys@transformshift{4.381818in}{5.157576in}%
\pgfsys@useobject{currentmarker}{}%
\end{pgfscope}%
\end{pgfscope}%
\begin{pgfscope}%
\pgfsetbuttcap%
\pgfsetroundjoin%
\definecolor{currentfill}{rgb}{0.000000,0.000000,0.000000}%
\pgfsetfillcolor{currentfill}%
\pgfsetlinewidth{0.501875pt}%
\definecolor{currentstroke}{rgb}{0.000000,0.000000,0.000000}%
\pgfsetstrokecolor{currentstroke}%
\pgfsetdash{}{0pt}%
\pgfsys@defobject{currentmarker}{\pgfqpoint{-0.055556in}{0.000000in}}{\pgfqpoint{0.000000in}{0.000000in}}{%
\pgfpathmoveto{\pgfqpoint{0.000000in}{0.000000in}}%
\pgfpathlineto{\pgfqpoint{-0.055556in}{0.000000in}}%
\pgfusepath{stroke,fill}%
}%
\begin{pgfscope}%
\pgfsys@transformshift{7.200000in}{5.157576in}%
\pgfsys@useobject{currentmarker}{}%
\end{pgfscope}%
\end{pgfscope}%
\begin{pgfscope}%
\pgftext[x=4.326263in,y=5.157576in,right,]{{\rmfamily\fontsize{12.000000}{14.400000}\selectfont 100}}%
\end{pgfscope}%
\begin{pgfscope}%
\pgfsetbuttcap%
\pgfsetroundjoin%
\definecolor{currentfill}{rgb}{0.000000,0.000000,0.000000}%
\pgfsetfillcolor{currentfill}%
\pgfsetlinewidth{0.501875pt}%
\definecolor{currentstroke}{rgb}{0.000000,0.000000,0.000000}%
\pgfsetstrokecolor{currentstroke}%
\pgfsetdash{}{0pt}%
\pgfsys@defobject{currentmarker}{\pgfqpoint{0.000000in}{0.000000in}}{\pgfqpoint{0.055556in}{0.000000in}}{%
\pgfpathmoveto{\pgfqpoint{0.000000in}{0.000000in}}%
\pgfpathlineto{\pgfqpoint{0.055556in}{0.000000in}}%
\pgfusepath{stroke,fill}%
}%
\begin{pgfscope}%
\pgfsys@transformshift{4.381818in}{5.400000in}%
\pgfsys@useobject{currentmarker}{}%
\end{pgfscope}%
\end{pgfscope}%
\begin{pgfscope}%
\pgfsetbuttcap%
\pgfsetroundjoin%
\definecolor{currentfill}{rgb}{0.000000,0.000000,0.000000}%
\pgfsetfillcolor{currentfill}%
\pgfsetlinewidth{0.501875pt}%
\definecolor{currentstroke}{rgb}{0.000000,0.000000,0.000000}%
\pgfsetstrokecolor{currentstroke}%
\pgfsetdash{}{0pt}%
\pgfsys@defobject{currentmarker}{\pgfqpoint{-0.055556in}{0.000000in}}{\pgfqpoint{0.000000in}{0.000000in}}{%
\pgfpathmoveto{\pgfqpoint{0.000000in}{0.000000in}}%
\pgfpathlineto{\pgfqpoint{-0.055556in}{0.000000in}}%
\pgfusepath{stroke,fill}%
}%
\begin{pgfscope}%
\pgfsys@transformshift{7.200000in}{5.400000in}%
\pgfsys@useobject{currentmarker}{}%
\end{pgfscope}%
\end{pgfscope}%
\begin{pgfscope}%
\pgftext[x=4.326263in,y=5.400000in,right,]{{\rmfamily\fontsize{12.000000}{14.400000}\selectfont 200}}%
\end{pgfscope}%
\begin{pgfscope}%
\pgfsetbuttcap%
\pgfsetroundjoin%
\pgfsetlinewidth{1.003750pt}%
\definecolor{currentstroke}{rgb}{0.000000,0.000000,0.000000}%
\pgfsetstrokecolor{currentstroke}%
\pgfsetdash{}{0pt}%
\pgfpathmoveto{\pgfqpoint{4.381818in}{5.400000in}}%
\pgfpathlineto{\pgfqpoint{7.200000in}{5.400000in}}%
\pgfusepath{stroke}%
\end{pgfscope}%
\begin{pgfscope}%
\pgfsetbuttcap%
\pgfsetroundjoin%
\pgfsetlinewidth{1.003750pt}%
\definecolor{currentstroke}{rgb}{0.000000,0.000000,0.000000}%
\pgfsetstrokecolor{currentstroke}%
\pgfsetdash{}{0pt}%
\pgfpathmoveto{\pgfqpoint{7.200000in}{3.218182in}}%
\pgfpathlineto{\pgfqpoint{7.200000in}{5.400000in}}%
\pgfusepath{stroke}%
\end{pgfscope}%
\begin{pgfscope}%
\pgfsetbuttcap%
\pgfsetroundjoin%
\pgfsetlinewidth{1.003750pt}%
\definecolor{currentstroke}{rgb}{0.000000,0.000000,0.000000}%
\pgfsetstrokecolor{currentstroke}%
\pgfsetdash{}{0pt}%
\pgfpathmoveto{\pgfqpoint{4.381818in}{3.218182in}}%
\pgfpathlineto{\pgfqpoint{7.200000in}{3.218182in}}%
\pgfusepath{stroke}%
\end{pgfscope}%
\begin{pgfscope}%
\pgfsetbuttcap%
\pgfsetroundjoin%
\pgfsetlinewidth{1.003750pt}%
\definecolor{currentstroke}{rgb}{0.000000,0.000000,0.000000}%
\pgfsetstrokecolor{currentstroke}%
\pgfsetdash{}{0pt}%
\pgfpathmoveto{\pgfqpoint{4.381818in}{3.218182in}}%
\pgfpathlineto{\pgfqpoint{4.381818in}{5.400000in}}%
\pgfusepath{stroke}%
\end{pgfscope}%
\begin{pgfscope}%
\pgftext[x=5.790909in,y=5.469444in,,base]{{\rmfamily\fontsize{14.400000}{17.280000}\selectfont e1\_d}}%
\end{pgfscope}%
\begin{pgfscope}%
\pgfsetbuttcap%
\pgfsetroundjoin%
\definecolor{currentfill}{rgb}{1.000000,1.000000,1.000000}%
\pgfsetfillcolor{currentfill}%
\pgfsetlinewidth{0.000000pt}%
\definecolor{currentstroke}{rgb}{0.000000,0.000000,0.000000}%
\pgfsetstrokecolor{currentstroke}%
\pgfsetstrokeopacity{0.000000}%
\pgfsetdash{}{0pt}%
\pgfpathmoveto{\pgfqpoint{1.000000in}{0.600000in}}%
\pgfpathlineto{\pgfqpoint{3.818182in}{0.600000in}}%
\pgfpathlineto{\pgfqpoint{3.818182in}{2.781818in}}%
\pgfpathlineto{\pgfqpoint{1.000000in}{2.781818in}}%
\pgfpathclose%
\pgfusepath{fill}%
\end{pgfscope}%
\begin{pgfscope}%
\pgfpathrectangle{\pgfqpoint{1.000000in}{0.600000in}}{\pgfqpoint{2.818182in}{2.181818in}} %
\pgfusepath{clip}%
\pgfsetrectcap%
\pgfsetroundjoin%
\pgfsetlinewidth{1.003750pt}%
\definecolor{currentstroke}{rgb}{0.000000,0.000000,1.000000}%
\pgfsetstrokecolor{currentstroke}%
\pgfsetdash{}{0pt}%
\pgfpathmoveto{\pgfqpoint{1.000000in}{2.781818in}}%
\pgfpathlineto{\pgfqpoint{1.002255in}{2.780686in}}%
\pgfpathlineto{\pgfqpoint{1.004791in}{2.775956in}}%
\pgfpathlineto{\pgfqpoint{1.008455in}{2.763228in}}%
\pgfpathlineto{\pgfqpoint{1.014656in}{2.742310in}}%
\pgfpathlineto{\pgfqpoint{1.017193in}{2.740243in}}%
\pgfpathlineto{\pgfqpoint{1.021420in}{2.738362in}}%
\pgfpathlineto{\pgfqpoint{1.023957in}{2.733555in}}%
\pgfpathlineto{\pgfqpoint{1.027339in}{2.721374in}}%
\pgfpathlineto{\pgfqpoint{1.034103in}{2.696008in}}%
\pgfpathlineto{\pgfqpoint{1.036922in}{2.693180in}}%
\pgfpathlineto{\pgfqpoint{1.040868in}{2.689931in}}%
\pgfpathlineto{\pgfqpoint{1.043686in}{2.683665in}}%
\pgfpathlineto{\pgfqpoint{1.047350in}{2.668731in}}%
\pgfpathlineto{\pgfqpoint{1.053269in}{2.644447in}}%
\pgfpathlineto{\pgfqpoint{1.056369in}{2.640653in}}%
\pgfpathlineto{\pgfqpoint{1.060033in}{2.636458in}}%
\pgfpathlineto{\pgfqpoint{1.063134in}{2.628692in}}%
\pgfpathlineto{\pgfqpoint{1.066798in}{2.612718in}}%
\pgfpathlineto{\pgfqpoint{1.072716in}{2.586624in}}%
\pgfpathlineto{\pgfqpoint{1.075817in}{2.582602in}}%
\pgfpathlineto{\pgfqpoint{1.078917in}{2.578263in}}%
\pgfpathlineto{\pgfqpoint{1.082299in}{2.568816in}}%
\pgfpathlineto{\pgfqpoint{1.086245in}{2.550440in}}%
\pgfpathlineto{\pgfqpoint{1.092164in}{2.523040in}}%
\pgfpathlineto{\pgfqpoint{1.095546in}{2.518585in}}%
\pgfpathlineto{\pgfqpoint{1.098364in}{2.513263in}}%
\pgfpathlineto{\pgfqpoint{1.102028in}{2.500571in}}%
\pgfpathlineto{\pgfqpoint{1.106538in}{2.475981in}}%
\pgfpathlineto{\pgfqpoint{1.111047in}{2.454788in}}%
\pgfpathlineto{\pgfqpoint{1.114430in}{2.449708in}}%
\pgfpathlineto{\pgfqpoint{1.116966in}{2.444244in}}%
\pgfpathlineto{\pgfqpoint{1.120912in}{2.428933in}}%
\pgfpathlineto{\pgfqpoint{1.125422in}{2.402049in}}%
\pgfpathlineto{\pgfqpoint{1.129649in}{2.381026in}}%
\pgfpathlineto{\pgfqpoint{1.138105in}{2.358061in}}%
\pgfpathlineto{\pgfqpoint{1.142332in}{2.333447in}}%
\pgfpathlineto{\pgfqpoint{1.147687in}{2.302249in}}%
\pgfpathlineto{\pgfqpoint{1.154452in}{2.282848in}}%
\pgfpathlineto{\pgfqpoint{1.158680in}{2.261056in}}%
\pgfpathlineto{\pgfqpoint{1.165444in}{2.218485in}}%
\pgfpathlineto{\pgfqpoint{1.168262in}{2.209195in}}%
\pgfpathlineto{\pgfqpoint{1.175872in}{2.171748in}}%
\pgfpathlineto{\pgfqpoint{1.179818in}{2.141686in}}%
\pgfpathlineto{\pgfqpoint{1.182636in}{2.127373in}}%
\pgfpathlineto{\pgfqpoint{1.185455in}{2.113891in}}%
\pgfpathlineto{\pgfqpoint{1.187710in}{2.111773in}}%
\pgfpathlineto{\pgfqpoint{1.189401in}{2.100830in}}%
\pgfpathlineto{\pgfqpoint{1.193628in}{2.072103in}}%
\pgfpathlineto{\pgfqpoint{1.202366in}{2.015337in}}%
\pgfpathlineto{\pgfqpoint{1.204057in}{2.013711in}}%
\pgfpathlineto{\pgfqpoint{1.204902in}{2.012113in}}%
\pgfpathlineto{\pgfqpoint{1.206312in}{2.001513in}}%
\pgfpathlineto{\pgfqpoint{1.209694in}{1.976701in}}%
\pgfpathlineto{\pgfqpoint{1.211385in}{1.962755in}}%
\pgfpathlineto{\pgfqpoint{1.214767in}{1.936079in}}%
\pgfpathlineto{\pgfqpoint{1.216458in}{1.931619in}}%
\pgfpathlineto{\pgfqpoint{1.221813in}{1.909918in}}%
\pgfpathlineto{\pgfqpoint{1.223504in}{1.897698in}}%
\pgfpathlineto{\pgfqpoint{1.228014in}{1.862897in}}%
\pgfpathlineto{\pgfqpoint{1.234496in}{1.824391in}}%
\pgfpathlineto{\pgfqpoint{1.235905in}{1.821615in}}%
\pgfpathlineto{\pgfqpoint{1.237878in}{1.810737in}}%
\pgfpathlineto{\pgfqpoint{1.250561in}{1.729325in}}%
\pgfpathlineto{\pgfqpoint{1.253098in}{1.722127in}}%
\pgfpathlineto{\pgfqpoint{1.259299in}{1.690779in}}%
\pgfpathlineto{\pgfqpoint{1.262681in}{1.665521in}}%
\pgfpathlineto{\pgfqpoint{1.267190in}{1.635671in}}%
\pgfpathlineto{\pgfqpoint{1.275364in}{1.606309in}}%
\pgfpathlineto{\pgfqpoint{1.281565in}{1.568123in}}%
\pgfpathlineto{\pgfqpoint{1.286074in}{1.540980in}}%
\pgfpathlineto{\pgfqpoint{1.289174in}{1.534814in}}%
\pgfpathlineto{\pgfqpoint{1.291429in}{1.528739in}}%
\pgfpathlineto{\pgfqpoint{1.295093in}{1.509734in}}%
\pgfpathlineto{\pgfqpoint{1.306931in}{1.447804in}}%
\pgfpathlineto{\pgfqpoint{1.311158in}{1.440051in}}%
\pgfpathlineto{\pgfqpoint{1.314259in}{1.423625in}}%
\pgfpathlineto{\pgfqpoint{1.322996in}{1.376549in}}%
\pgfpathlineto{\pgfqpoint{1.326378in}{1.364283in}}%
\pgfpathlineto{\pgfqpoint{1.332015in}{1.352964in}}%
\pgfpathlineto{\pgfqpoint{1.346389in}{1.286625in}}%
\pgfpathlineto{\pgfqpoint{1.352308in}{1.274818in}}%
\pgfpathlineto{\pgfqpoint{1.363018in}{1.226035in}}%
\pgfpathlineto{\pgfqpoint{1.367528in}{1.214911in}}%
\pgfpathlineto{\pgfqpoint{1.371192in}{1.209475in}}%
\pgfpathlineto{\pgfqpoint{1.373728in}{1.198082in}}%
\pgfpathlineto{\pgfqpoint{1.379647in}{1.169442in}}%
\pgfpathlineto{\pgfqpoint{1.387257in}{1.153981in}}%
\pgfpathlineto{\pgfqpoint{1.392048in}{1.145410in}}%
\pgfpathlineto{\pgfqpoint{1.395149in}{1.130424in}}%
\pgfpathlineto{\pgfqpoint{1.399376in}{1.112669in}}%
\pgfpathlineto{\pgfqpoint{1.402195in}{1.109321in}}%
\pgfpathlineto{\pgfqpoint{1.405859in}{1.105381in}}%
\pgfpathlineto{\pgfqpoint{1.412059in}{1.094760in}}%
\pgfpathlineto{\pgfqpoint{1.415442in}{1.081166in}}%
\pgfpathlineto{\pgfqpoint{1.419387in}{1.067521in}}%
\pgfpathlineto{\pgfqpoint{1.421360in}{1.066432in}}%
\pgfpathlineto{\pgfqpoint{1.426152in}{1.066801in}}%
\pgfpathlineto{\pgfqpoint{1.429534in}{1.062115in}}%
\pgfpathlineto{\pgfqpoint{1.433762in}{1.052983in}}%
\pgfpathlineto{\pgfqpoint{1.439962in}{1.037522in}}%
\pgfpathlineto{\pgfqpoint{1.441371in}{1.038470in}}%
\pgfpathlineto{\pgfqpoint{1.446726in}{1.044086in}}%
\pgfpathlineto{\pgfqpoint{1.449263in}{1.040851in}}%
\pgfpathlineto{\pgfqpoint{1.459128in}{1.026754in}}%
\pgfpathlineto{\pgfqpoint{1.460537in}{1.027968in}}%
\pgfpathlineto{\pgfqpoint{1.463074in}{1.034575in}}%
\pgfpathlineto{\pgfqpoint{1.466174in}{1.041028in}}%
\pgfpathlineto{\pgfqpoint{1.467865in}{1.040605in}}%
\pgfpathlineto{\pgfqpoint{1.473502in}{1.036453in}}%
\pgfpathlineto{\pgfqpoint{1.479139in}{1.038220in}}%
\pgfpathlineto{\pgfqpoint{1.481112in}{1.042957in}}%
\pgfpathlineto{\pgfqpoint{1.487594in}{1.061737in}}%
\pgfpathlineto{\pgfqpoint{1.492667in}{1.061913in}}%
\pgfpathlineto{\pgfqpoint{1.496613in}{1.070185in}}%
\pgfpathlineto{\pgfqpoint{1.500559in}{1.081256in}}%
\pgfpathlineto{\pgfqpoint{1.508733in}{1.111134in}}%
\pgfpathlineto{\pgfqpoint{1.511551in}{1.113770in}}%
\pgfpathlineto{\pgfqpoint{1.513806in}{1.120679in}}%
\pgfpathlineto{\pgfqpoint{1.521134in}{1.155341in}}%
\pgfpathlineto{\pgfqpoint{1.528744in}{1.193639in}}%
\pgfpathlineto{\pgfqpoint{1.532408in}{1.203956in}}%
\pgfpathlineto{\pgfqpoint{1.535790in}{1.224716in}}%
\pgfpathlineto{\pgfqpoint{1.549882in}{1.319519in}}%
\pgfpathlineto{\pgfqpoint{1.553264in}{1.338530in}}%
\pgfpathlineto{\pgfqpoint{1.557492in}{1.379298in}}%
\pgfpathlineto{\pgfqpoint{1.567920in}{1.477878in}}%
\pgfpathlineto{\pgfqpoint{1.574685in}{1.540604in}}%
\pgfpathlineto{\pgfqpoint{1.594696in}{1.826344in}}%
\pgfpathlineto{\pgfqpoint{1.602024in}{1.956306in}}%
\pgfpathlineto{\pgfqpoint{1.606252in}{2.020915in}}%
\pgfpathlineto{\pgfqpoint{1.615271in}{2.182051in}}%
\pgfpathlineto{\pgfqpoint{1.634718in}{2.447700in}}%
\pgfpathlineto{\pgfqpoint{1.640637in}{2.499772in}}%
\pgfpathlineto{\pgfqpoint{1.644864in}{2.539538in}}%
\pgfpathlineto{\pgfqpoint{1.651629in}{2.605786in}}%
\pgfpathlineto{\pgfqpoint{1.658393in}{2.644740in}}%
\pgfpathlineto{\pgfqpoint{1.664030in}{2.675248in}}%
\pgfpathlineto{\pgfqpoint{1.671358in}{2.720989in}}%
\pgfpathlineto{\pgfqpoint{1.675304in}{2.729905in}}%
\pgfpathlineto{\pgfqpoint{1.682914in}{2.745581in}}%
\pgfpathlineto{\pgfqpoint{1.691087in}{2.768542in}}%
\pgfpathlineto{\pgfqpoint{1.692778in}{2.766964in}}%
\pgfpathlineto{\pgfqpoint{1.701516in}{2.754661in}}%
\pgfpathlineto{\pgfqpoint{1.705743in}{2.754293in}}%
\pgfpathlineto{\pgfqpoint{1.708280in}{2.753295in}}%
\pgfpathlineto{\pgfqpoint{1.710253in}{2.749466in}}%
\pgfpathlineto{\pgfqpoint{1.712789in}{2.738722in}}%
\pgfpathlineto{\pgfqpoint{1.721527in}{2.696896in}}%
\pgfpathlineto{\pgfqpoint{1.727727in}{2.672419in}}%
\pgfpathlineto{\pgfqpoint{1.731391in}{2.649832in}}%
\pgfpathlineto{\pgfqpoint{1.750275in}{2.487128in}}%
\pgfpathlineto{\pgfqpoint{1.756194in}{2.404481in}}%
\pgfpathlineto{\pgfqpoint{1.770004in}{2.212943in}}%
\pgfpathlineto{\pgfqpoint{1.778742in}{2.049972in}}%
\pgfpathlineto{\pgfqpoint{1.786915in}{1.901964in}}%
\pgfpathlineto{\pgfqpoint{1.793961in}{1.773067in}}%
\pgfpathlineto{\pgfqpoint{1.812281in}{1.510609in}}%
\pgfpathlineto{\pgfqpoint{1.823273in}{1.392864in}}%
\pgfpathlineto{\pgfqpoint{1.828910in}{1.338845in}}%
\pgfpathlineto{\pgfqpoint{1.839338in}{1.270031in}}%
\pgfpathlineto{\pgfqpoint{1.849203in}{1.207182in}}%
\pgfpathlineto{\pgfqpoint{1.855967in}{1.187386in}}%
\pgfpathlineto{\pgfqpoint{1.862732in}{1.156938in}}%
\pgfpathlineto{\pgfqpoint{1.868369in}{1.140322in}}%
\pgfpathlineto{\pgfqpoint{1.870060in}{1.140523in}}%
\pgfpathlineto{\pgfqpoint{1.874287in}{1.143392in}}%
\pgfpathlineto{\pgfqpoint{1.876260in}{1.139248in}}%
\pgfpathlineto{\pgfqpoint{1.880770in}{1.128793in}}%
\pgfpathlineto{\pgfqpoint{1.883025in}{1.128474in}}%
\pgfpathlineto{\pgfqpoint{1.887816in}{1.130468in}}%
\pgfpathlineto{\pgfqpoint{1.889789in}{1.135122in}}%
\pgfpathlineto{\pgfqpoint{1.895708in}{1.151859in}}%
\pgfpathlineto{\pgfqpoint{1.897963in}{1.150959in}}%
\pgfpathlineto{\pgfqpoint{1.899654in}{1.151157in}}%
\pgfpathlineto{\pgfqpoint{1.901345in}{1.154209in}}%
\pgfpathlineto{\pgfqpoint{1.905009in}{1.167606in}}%
\pgfpathlineto{\pgfqpoint{1.910927in}{1.195144in}}%
\pgfpathlineto{\pgfqpoint{1.916283in}{1.221863in}}%
\pgfpathlineto{\pgfqpoint{1.921074in}{1.235975in}}%
\pgfpathlineto{\pgfqpoint{1.924738in}{1.260920in}}%
\pgfpathlineto{\pgfqpoint{1.939112in}{1.363243in}}%
\pgfpathlineto{\pgfqpoint{1.942212in}{1.386207in}}%
\pgfpathlineto{\pgfqpoint{1.951513in}{1.487321in}}%
\pgfpathlineto{\pgfqpoint{1.964760in}{1.641118in}}%
\pgfpathlineto{\pgfqpoint{1.974061in}{1.781308in}}%
\pgfpathlineto{\pgfqpoint{1.981107in}{1.897730in}}%
\pgfpathlineto{\pgfqpoint{1.988717in}{2.033348in}}%
\pgfpathlineto{\pgfqpoint{1.991254in}{2.075070in}}%
\pgfpathlineto{\pgfqpoint{2.000555in}{2.235297in}}%
\pgfpathlineto{\pgfqpoint{2.009010in}{2.349386in}}%
\pgfpathlineto{\pgfqpoint{2.013801in}{2.405618in}}%
\pgfpathlineto{\pgfqpoint{2.021129in}{2.497080in}}%
\pgfpathlineto{\pgfqpoint{2.027894in}{2.553416in}}%
\pgfpathlineto{\pgfqpoint{2.032685in}{2.591094in}}%
\pgfpathlineto{\pgfqpoint{2.040013in}{2.655303in}}%
\pgfpathlineto{\pgfqpoint{2.046496in}{2.686850in}}%
\pgfpathlineto{\pgfqpoint{2.052132in}{2.711810in}}%
\pgfpathlineto{\pgfqpoint{2.059460in}{2.751435in}}%
\pgfpathlineto{\pgfqpoint{2.062561in}{2.756215in}}%
\pgfpathlineto{\pgfqpoint{2.071580in}{2.766242in}}%
\pgfpathlineto{\pgfqpoint{2.078626in}{2.780217in}}%
\pgfpathlineto{\pgfqpoint{2.080317in}{2.777861in}}%
\pgfpathlineto{\pgfqpoint{2.083981in}{2.766903in}}%
\pgfpathlineto{\pgfqpoint{2.089054in}{2.754322in}}%
\pgfpathlineto{\pgfqpoint{2.098637in}{2.734809in}}%
\pgfpathlineto{\pgfqpoint{2.102019in}{2.713618in}}%
\pgfpathlineto{\pgfqpoint{2.109347in}{2.665925in}}%
\pgfpathlineto{\pgfqpoint{2.116393in}{2.623669in}}%
\pgfpathlineto{\pgfqpoint{2.120339in}{2.588062in}}%
\pgfpathlineto{\pgfqpoint{2.139505in}{2.357056in}}%
\pgfpathlineto{\pgfqpoint{2.160361in}{1.957603in}}%
\pgfpathlineto{\pgfqpoint{2.164307in}{1.882978in}}%
\pgfpathlineto{\pgfqpoint{2.167689in}{1.817940in}}%
\pgfpathlineto{\pgfqpoint{2.178400in}{1.625151in}}%
\pgfpathlineto{\pgfqpoint{2.195874in}{1.423725in}}%
\pgfpathlineto{\pgfqpoint{2.199538in}{1.401197in}}%
\pgfpathlineto{\pgfqpoint{2.204330in}{1.375747in}}%
\pgfpathlineto{\pgfqpoint{2.212785in}{1.326747in}}%
\pgfpathlineto{\pgfqpoint{2.216731in}{1.310877in}}%
\pgfpathlineto{\pgfqpoint{2.218422in}{1.310393in}}%
\pgfpathlineto{\pgfqpoint{2.223777in}{1.314482in}}%
\pgfpathlineto{\pgfqpoint{2.229132in}{1.308959in}}%
\pgfpathlineto{\pgfqpoint{2.231950in}{1.311892in}}%
\pgfpathlineto{\pgfqpoint{2.235614in}{1.317685in}}%
\pgfpathlineto{\pgfqpoint{2.237869in}{1.327380in}}%
\pgfpathlineto{\pgfqpoint{2.254780in}{1.436872in}}%
\pgfpathlineto{\pgfqpoint{2.259008in}{1.484126in}}%
\pgfpathlineto{\pgfqpoint{2.266336in}{1.571594in}}%
\pgfpathlineto{\pgfqpoint{2.270000in}{1.622281in}}%
\pgfpathlineto{\pgfqpoint{2.278737in}{1.790222in}}%
\pgfpathlineto{\pgfqpoint{2.283247in}{1.870510in}}%
\pgfpathlineto{\pgfqpoint{2.290575in}{2.035240in}}%
\pgfpathlineto{\pgfqpoint{2.294520in}{2.118837in}}%
\pgfpathlineto{\pgfqpoint{2.300439in}{2.234618in}}%
\pgfpathlineto{\pgfqpoint{2.304385in}{2.305260in}}%
\pgfpathlineto{\pgfqpoint{2.310586in}{2.404500in}}%
\pgfpathlineto{\pgfqpoint{2.316504in}{2.481738in}}%
\pgfpathlineto{\pgfqpoint{2.323269in}{2.542070in}}%
\pgfpathlineto{\pgfqpoint{2.328342in}{2.586751in}}%
\pgfpathlineto{\pgfqpoint{2.335952in}{2.631354in}}%
\pgfpathlineto{\pgfqpoint{2.338488in}{2.633857in}}%
\pgfpathlineto{\pgfqpoint{2.340179in}{2.636581in}}%
\pgfpathlineto{\pgfqpoint{2.342434in}{2.645770in}}%
\pgfpathlineto{\pgfqpoint{2.346662in}{2.662283in}}%
\pgfpathlineto{\pgfqpoint{2.348635in}{2.663025in}}%
\pgfpathlineto{\pgfqpoint{2.354835in}{2.659911in}}%
\pgfpathlineto{\pgfqpoint{2.357090in}{2.651771in}}%
\pgfpathlineto{\pgfqpoint{2.361882in}{2.633786in}}%
\pgfpathlineto{\pgfqpoint{2.365827in}{2.628813in}}%
\pgfpathlineto{\pgfqpoint{2.368082in}{2.617487in}}%
\pgfpathlineto{\pgfqpoint{2.376538in}{2.553917in}}%
\pgfpathlineto{\pgfqpoint{2.384148in}{2.487767in}}%
\pgfpathlineto{\pgfqpoint{2.387248in}{2.456765in}}%
\pgfpathlineto{\pgfqpoint{2.396549in}{2.318046in}}%
\pgfpathlineto{\pgfqpoint{2.416560in}{1.935335in}}%
\pgfpathlineto{\pgfqpoint{2.420224in}{1.872739in}}%
\pgfpathlineto{\pgfqpoint{2.423888in}{1.790313in}}%
\pgfpathlineto{\pgfqpoint{2.431780in}{1.645336in}}%
\pgfpathlineto{\pgfqpoint{2.436853in}{1.586408in}}%
\pgfpathlineto{\pgfqpoint{2.452072in}{1.407149in}}%
\pgfpathlineto{\pgfqpoint{2.459964in}{1.353333in}}%
\pgfpathlineto{\pgfqpoint{2.464474in}{1.322146in}}%
\pgfpathlineto{\pgfqpoint{2.471520in}{1.295295in}}%
\pgfpathlineto{\pgfqpoint{2.473493in}{1.293999in}}%
\pgfpathlineto{\pgfqpoint{2.476311in}{1.293322in}}%
\pgfpathlineto{\pgfqpoint{2.478284in}{1.288327in}}%
\pgfpathlineto{\pgfqpoint{2.483076in}{1.274112in}}%
\pgfpathlineto{\pgfqpoint{2.484485in}{1.274790in}}%
\pgfpathlineto{\pgfqpoint{2.487585in}{1.280953in}}%
\pgfpathlineto{\pgfqpoint{2.492095in}{1.292620in}}%
\pgfpathlineto{\pgfqpoint{2.500832in}{1.323136in}}%
\pgfpathlineto{\pgfqpoint{2.502805in}{1.329626in}}%
\pgfpathlineto{\pgfqpoint{2.505623in}{1.348759in}}%
\pgfpathlineto{\pgfqpoint{2.521970in}{1.497994in}}%
\pgfpathlineto{\pgfqpoint{2.525353in}{1.543045in}}%
\pgfpathlineto{\pgfqpoint{2.542263in}{1.839300in}}%
\pgfpathlineto{\pgfqpoint{2.551001in}{2.043697in}}%
\pgfpathlineto{\pgfqpoint{2.554101in}{2.106991in}}%
\pgfpathlineto{\pgfqpoint{2.563402in}{2.301979in}}%
\pgfpathlineto{\pgfqpoint{2.569884in}{2.402162in}}%
\pgfpathlineto{\pgfqpoint{2.582567in}{2.557113in}}%
\pgfpathlineto{\pgfqpoint{2.587077in}{2.588680in}}%
\pgfpathlineto{\pgfqpoint{2.594687in}{2.625177in}}%
\pgfpathlineto{\pgfqpoint{2.603142in}{2.661949in}}%
\pgfpathlineto{\pgfqpoint{2.604551in}{2.661383in}}%
\pgfpathlineto{\pgfqpoint{2.606806in}{2.656867in}}%
\pgfpathlineto{\pgfqpoint{2.621744in}{2.616326in}}%
\pgfpathlineto{\pgfqpoint{2.624844in}{2.591471in}}%
\pgfpathlineto{\pgfqpoint{2.640346in}{2.423376in}}%
\pgfpathlineto{\pgfqpoint{2.644574in}{2.347889in}}%
\pgfpathlineto{\pgfqpoint{2.682623in}{1.547431in}}%
\pgfpathlineto{\pgfqpoint{2.691642in}{1.453991in}}%
\pgfpathlineto{\pgfqpoint{2.700661in}{1.376473in}}%
\pgfpathlineto{\pgfqpoint{2.703198in}{1.368887in}}%
\pgfpathlineto{\pgfqpoint{2.706016in}{1.367720in}}%
\pgfpathlineto{\pgfqpoint{2.707989in}{1.365093in}}%
\pgfpathlineto{\pgfqpoint{2.713626in}{1.355187in}}%
\pgfpathlineto{\pgfqpoint{2.716444in}{1.356247in}}%
\pgfpathlineto{\pgfqpoint{2.719263in}{1.359090in}}%
\pgfpathlineto{\pgfqpoint{2.721236in}{1.365728in}}%
\pgfpathlineto{\pgfqpoint{2.724054in}{1.385875in}}%
\pgfpathlineto{\pgfqpoint{2.729973in}{1.428668in}}%
\pgfpathlineto{\pgfqpoint{2.733637in}{1.454901in}}%
\pgfpathlineto{\pgfqpoint{2.740119in}{1.526781in}}%
\pgfpathlineto{\pgfqpoint{2.744347in}{1.598866in}}%
\pgfpathlineto{\pgfqpoint{2.760412in}{1.932584in}}%
\pgfpathlineto{\pgfqpoint{2.766331in}{2.073087in}}%
\pgfpathlineto{\pgfqpoint{2.775068in}{2.258432in}}%
\pgfpathlineto{\pgfqpoint{2.783806in}{2.410701in}}%
\pgfpathlineto{\pgfqpoint{2.788033in}{2.463737in}}%
\pgfpathlineto{\pgfqpoint{2.793670in}{2.530009in}}%
\pgfpathlineto{\pgfqpoint{2.801280in}{2.592000in}}%
\pgfpathlineto{\pgfqpoint{2.804099in}{2.596184in}}%
\pgfpathlineto{\pgfqpoint{2.805790in}{2.600681in}}%
\pgfpathlineto{\pgfqpoint{2.813681in}{2.628988in}}%
\pgfpathlineto{\pgfqpoint{2.819600in}{2.627430in}}%
\pgfpathlineto{\pgfqpoint{2.821573in}{2.618900in}}%
\pgfpathlineto{\pgfqpoint{2.827774in}{2.587264in}}%
\pgfpathlineto{\pgfqpoint{2.830310in}{2.580732in}}%
\pgfpathlineto{\pgfqpoint{2.832565in}{2.566214in}}%
\pgfpathlineto{\pgfqpoint{2.840457in}{2.489545in}}%
\pgfpathlineto{\pgfqpoint{2.853422in}{2.301145in}}%
\pgfpathlineto{\pgfqpoint{2.870896in}{1.922564in}}%
\pgfpathlineto{\pgfqpoint{2.874560in}{1.840208in}}%
\pgfpathlineto{\pgfqpoint{2.886116in}{1.619903in}}%
\pgfpathlineto{\pgfqpoint{2.891189in}{1.544603in}}%
\pgfpathlineto{\pgfqpoint{2.898799in}{1.457318in}}%
\pgfpathlineto{\pgfqpoint{2.903590in}{1.426966in}}%
\pgfpathlineto{\pgfqpoint{2.909227in}{1.384086in}}%
\pgfpathlineto{\pgfqpoint{2.916837in}{1.358781in}}%
\pgfpathlineto{\pgfqpoint{2.917965in}{1.359864in}}%
\pgfpathlineto{\pgfqpoint{2.920783in}{1.368522in}}%
\pgfpathlineto{\pgfqpoint{2.923038in}{1.371713in}}%
\pgfpathlineto{\pgfqpoint{2.924729in}{1.369951in}}%
\pgfpathlineto{\pgfqpoint{2.926702in}{1.368744in}}%
\pgfpathlineto{\pgfqpoint{2.928111in}{1.371544in}}%
\pgfpathlineto{\pgfqpoint{2.930648in}{1.384673in}}%
\pgfpathlineto{\pgfqpoint{2.938539in}{1.441705in}}%
\pgfpathlineto{\pgfqpoint{2.946149in}{1.518255in}}%
\pgfpathlineto{\pgfqpoint{2.948968in}{1.554508in}}%
\pgfpathlineto{\pgfqpoint{2.957705in}{1.717212in}}%
\pgfpathlineto{\pgfqpoint{2.980253in}{2.248023in}}%
\pgfpathlineto{\pgfqpoint{2.984198in}{2.324649in}}%
\pgfpathlineto{\pgfqpoint{2.989272in}{2.418814in}}%
\pgfpathlineto{\pgfqpoint{2.997445in}{2.513967in}}%
\pgfpathlineto{\pgfqpoint{3.003364in}{2.559718in}}%
\pgfpathlineto{\pgfqpoint{3.007310in}{2.589375in}}%
\pgfpathlineto{\pgfqpoint{3.009846in}{2.594511in}}%
\pgfpathlineto{\pgfqpoint{3.014920in}{2.597302in}}%
\pgfpathlineto{\pgfqpoint{3.016329in}{2.595093in}}%
\pgfpathlineto{\pgfqpoint{3.019147in}{2.584912in}}%
\pgfpathlineto{\pgfqpoint{3.022530in}{2.575984in}}%
\pgfpathlineto{\pgfqpoint{3.025066in}{2.571419in}}%
\pgfpathlineto{\pgfqpoint{3.027039in}{2.560975in}}%
\pgfpathlineto{\pgfqpoint{3.030421in}{2.526522in}}%
\pgfpathlineto{\pgfqpoint{3.039722in}{2.405990in}}%
\pgfpathlineto{\pgfqpoint{3.047332in}{2.277833in}}%
\pgfpathlineto{\pgfqpoint{3.069316in}{1.734169in}}%
\pgfpathlineto{\pgfqpoint{3.076644in}{1.600952in}}%
\pgfpathlineto{\pgfqpoint{3.082845in}{1.506150in}}%
\pgfpathlineto{\pgfqpoint{3.087354in}{1.456262in}}%
\pgfpathlineto{\pgfqpoint{3.092991in}{1.417836in}}%
\pgfpathlineto{\pgfqpoint{3.096373in}{1.408602in}}%
\pgfpathlineto{\pgfqpoint{3.099192in}{1.399276in}}%
\pgfpathlineto{\pgfqpoint{3.103701in}{1.383655in}}%
\pgfpathlineto{\pgfqpoint{3.104829in}{1.383981in}}%
\pgfpathlineto{\pgfqpoint{3.106520in}{1.388045in}}%
\pgfpathlineto{\pgfqpoint{3.110466in}{1.405544in}}%
\pgfpathlineto{\pgfqpoint{3.114693in}{1.433723in}}%
\pgfpathlineto{\pgfqpoint{3.125122in}{1.527128in}}%
\pgfpathlineto{\pgfqpoint{3.131886in}{1.645497in}}%
\pgfpathlineto{\pgfqpoint{3.141469in}{1.844681in}}%
\pgfpathlineto{\pgfqpoint{3.156407in}{2.221351in}}%
\pgfpathlineto{\pgfqpoint{3.169090in}{2.453977in}}%
\pgfpathlineto{\pgfqpoint{3.177263in}{2.532693in}}%
\pgfpathlineto{\pgfqpoint{3.183746in}{2.584767in}}%
\pgfpathlineto{\pgfqpoint{3.186282in}{2.591308in}}%
\pgfpathlineto{\pgfqpoint{3.187973in}{2.591405in}}%
\pgfpathlineto{\pgfqpoint{3.191637in}{2.587631in}}%
\pgfpathlineto{\pgfqpoint{3.198965in}{2.578595in}}%
\pgfpathlineto{\pgfqpoint{3.201784in}{2.569643in}}%
\pgfpathlineto{\pgfqpoint{3.204602in}{2.551247in}}%
\pgfpathlineto{\pgfqpoint{3.208830in}{2.504062in}}%
\pgfpathlineto{\pgfqpoint{3.222077in}{2.328866in}}%
\pgfpathlineto{\pgfqpoint{3.226868in}{2.215733in}}%
\pgfpathlineto{\pgfqpoint{3.249134in}{1.699513in}}%
\pgfpathlineto{\pgfqpoint{3.259280in}{1.531981in}}%
\pgfpathlineto{\pgfqpoint{3.263790in}{1.478977in}}%
\pgfpathlineto{\pgfqpoint{3.269145in}{1.447629in}}%
\pgfpathlineto{\pgfqpoint{3.275346in}{1.417946in}}%
\pgfpathlineto{\pgfqpoint{3.280137in}{1.404734in}}%
\pgfpathlineto{\pgfqpoint{3.281546in}{1.404813in}}%
\pgfpathlineto{\pgfqpoint{3.282956in}{1.408371in}}%
\pgfpathlineto{\pgfqpoint{3.285492in}{1.422429in}}%
\pgfpathlineto{\pgfqpoint{3.297612in}{1.512112in}}%
\pgfpathlineto{\pgfqpoint{3.302121in}{1.570912in}}%
\pgfpathlineto{\pgfqpoint{3.310858in}{1.742343in}}%
\pgfpathlineto{\pgfqpoint{3.317905in}{1.913802in}}%
\pgfpathlineto{\pgfqpoint{3.327205in}{2.167564in}}%
\pgfpathlineto{\pgfqpoint{3.332279in}{2.284004in}}%
\pgfpathlineto{\pgfqpoint{3.339325in}{2.419852in}}%
\pgfpathlineto{\pgfqpoint{3.343271in}{2.471046in}}%
\pgfpathlineto{\pgfqpoint{3.354263in}{2.551553in}}%
\pgfpathlineto{\pgfqpoint{3.359054in}{2.564136in}}%
\pgfpathlineto{\pgfqpoint{3.360463in}{2.564306in}}%
\pgfpathlineto{\pgfqpoint{3.361873in}{2.560504in}}%
\pgfpathlineto{\pgfqpoint{3.364691in}{2.542826in}}%
\pgfpathlineto{\pgfqpoint{3.368919in}{2.520170in}}%
\pgfpathlineto{\pgfqpoint{3.371737in}{2.504894in}}%
\pgfpathlineto{\pgfqpoint{3.375119in}{2.468597in}}%
\pgfpathlineto{\pgfqpoint{3.381320in}{2.378465in}}%
\pgfpathlineto{\pgfqpoint{3.395130in}{2.071559in}}%
\pgfpathlineto{\pgfqpoint{3.403868in}{1.828376in}}%
\pgfpathlineto{\pgfqpoint{3.407814in}{1.741493in}}%
\pgfpathlineto{\pgfqpoint{3.413169in}{1.626200in}}%
\pgfpathlineto{\pgfqpoint{3.420215in}{1.516151in}}%
\pgfpathlineto{\pgfqpoint{3.433180in}{1.423509in}}%
\pgfpathlineto{\pgfqpoint{3.437126in}{1.414114in}}%
\pgfpathlineto{\pgfqpoint{3.438253in}{1.415682in}}%
\pgfpathlineto{\pgfqpoint{3.440226in}{1.424882in}}%
\pgfpathlineto{\pgfqpoint{3.445299in}{1.450434in}}%
\pgfpathlineto{\pgfqpoint{3.447836in}{1.459130in}}%
\pgfpathlineto{\pgfqpoint{3.450090in}{1.477024in}}%
\pgfpathlineto{\pgfqpoint{3.456855in}{1.560540in}}%
\pgfpathlineto{\pgfqpoint{3.460801in}{1.637509in}}%
\pgfpathlineto{\pgfqpoint{3.474329in}{1.952727in}}%
\pgfpathlineto{\pgfqpoint{3.477430in}{2.035580in}}%
\pgfpathlineto{\pgfqpoint{3.479684in}{2.098250in}}%
\pgfpathlineto{\pgfqpoint{3.483066in}{2.176068in}}%
\pgfpathlineto{\pgfqpoint{3.486167in}{2.249646in}}%
\pgfpathlineto{\pgfqpoint{3.491240in}{2.362038in}}%
\pgfpathlineto{\pgfqpoint{3.498568in}{2.469062in}}%
\pgfpathlineto{\pgfqpoint{3.504487in}{2.520031in}}%
\pgfpathlineto{\pgfqpoint{3.508433in}{2.554076in}}%
\pgfpathlineto{\pgfqpoint{3.511251in}{2.561225in}}%
\pgfpathlineto{\pgfqpoint{3.515197in}{2.565599in}}%
\pgfpathlineto{\pgfqpoint{3.516324in}{2.564078in}}%
\pgfpathlineto{\pgfqpoint{3.518297in}{2.555965in}}%
\pgfpathlineto{\pgfqpoint{3.523371in}{2.532499in}}%
\pgfpathlineto{\pgfqpoint{3.525907in}{2.526180in}}%
\pgfpathlineto{\pgfqpoint{3.527880in}{2.512794in}}%
\pgfpathlineto{\pgfqpoint{3.531826in}{2.464038in}}%
\pgfpathlineto{\pgfqpoint{3.537745in}{2.372087in}}%
\pgfpathlineto{\pgfqpoint{3.551273in}{2.079440in}}%
\pgfpathlineto{\pgfqpoint{3.558319in}{1.887876in}}%
\pgfpathlineto{\pgfqpoint{3.571284in}{1.602333in}}%
\pgfpathlineto{\pgfqpoint{3.575230in}{1.551911in}}%
\pgfpathlineto{\pgfqpoint{3.580585in}{1.501638in}}%
\pgfpathlineto{\pgfqpoint{3.585095in}{1.457897in}}%
\pgfpathlineto{\pgfqpoint{3.588759in}{1.443596in}}%
\pgfpathlineto{\pgfqpoint{3.591295in}{1.439500in}}%
\pgfpathlineto{\pgfqpoint{3.592423in}{1.441037in}}%
\pgfpathlineto{\pgfqpoint{3.594396in}{1.449662in}}%
\pgfpathlineto{\pgfqpoint{3.599469in}{1.473675in}}%
\pgfpathlineto{\pgfqpoint{3.602006in}{1.480435in}}%
\pgfpathlineto{\pgfqpoint{3.604260in}{1.496448in}}%
\pgfpathlineto{\pgfqpoint{3.608488in}{1.550403in}}%
\pgfpathlineto{\pgfqpoint{3.613280in}{1.635055in}}%
\pgfpathlineto{\pgfqpoint{3.627090in}{1.967112in}}%
\pgfpathlineto{\pgfqpoint{3.631600in}{2.095651in}}%
\pgfpathlineto{\pgfqpoint{3.638646in}{2.261865in}}%
\pgfpathlineto{\pgfqpoint{3.645974in}{2.418093in}}%
\pgfpathlineto{\pgfqpoint{3.650765in}{2.474941in}}%
\pgfpathlineto{\pgfqpoint{3.654429in}{2.492812in}}%
\pgfpathlineto{\pgfqpoint{3.657248in}{2.510088in}}%
\pgfpathlineto{\pgfqpoint{3.661757in}{2.536979in}}%
\pgfpathlineto{\pgfqpoint{3.663166in}{2.537864in}}%
\pgfpathlineto{\pgfqpoint{3.664576in}{2.535418in}}%
\pgfpathlineto{\pgfqpoint{3.667112in}{2.525327in}}%
\pgfpathlineto{\pgfqpoint{3.669931in}{2.504457in}}%
\pgfpathlineto{\pgfqpoint{3.680923in}{2.387573in}}%
\pgfpathlineto{\pgfqpoint{3.685714in}{2.287654in}}%
\pgfpathlineto{\pgfqpoint{3.700370in}{1.905477in}}%
\pgfpathlineto{\pgfqpoint{3.707980in}{1.700577in}}%
\pgfpathlineto{\pgfqpoint{3.716435in}{1.559459in}}%
\pgfpathlineto{\pgfqpoint{3.722354in}{1.483185in}}%
\pgfpathlineto{\pgfqpoint{3.725736in}{1.458436in}}%
\pgfpathlineto{\pgfqpoint{3.727709in}{1.455417in}}%
\pgfpathlineto{\pgfqpoint{3.731937in}{1.455863in}}%
\pgfpathlineto{\pgfqpoint{3.735883in}{1.450629in}}%
\pgfpathlineto{\pgfqpoint{3.736165in}{1.450864in}}%
\pgfpathlineto{\pgfqpoint{3.737574in}{1.453986in}}%
\pgfpathlineto{\pgfqpoint{3.740392in}{1.467668in}}%
\pgfpathlineto{\pgfqpoint{3.744056in}{1.496467in}}%
\pgfpathlineto{\pgfqpoint{3.747438in}{1.545464in}}%
\pgfpathlineto{\pgfqpoint{3.760403in}{1.789791in}}%
\pgfpathlineto{\pgfqpoint{3.764349in}{1.911377in}}%
\pgfpathlineto{\pgfqpoint{3.770268in}{2.071423in}}%
\pgfpathlineto{\pgfqpoint{3.776187in}{2.221471in}}%
\pgfpathlineto{\pgfqpoint{3.784078in}{2.398487in}}%
\pgfpathlineto{\pgfqpoint{3.787461in}{2.433837in}}%
\pgfpathlineto{\pgfqpoint{3.791125in}{2.470283in}}%
\pgfpathlineto{\pgfqpoint{3.796198in}{2.520664in}}%
\pgfpathlineto{\pgfqpoint{3.801271in}{2.543326in}}%
\pgfpathlineto{\pgfqpoint{3.802680in}{2.545092in}}%
\pgfpathlineto{\pgfqpoint{3.802962in}{2.544843in}}%
\pgfpathlineto{\pgfqpoint{3.804089in}{2.541595in}}%
\pgfpathlineto{\pgfqpoint{3.806626in}{2.523873in}}%
\pgfpathlineto{\pgfqpoint{3.810008in}{2.503814in}}%
\pgfpathlineto{\pgfqpoint{3.813672in}{2.490736in}}%
\pgfpathlineto{\pgfqpoint{3.816209in}{2.465096in}}%
\pgfpathlineto{\pgfqpoint{3.818182in}{2.439442in}}%
\pgfpathlineto{\pgfqpoint{3.818182in}{2.439442in}}%
\pgfusepath{stroke}%
\end{pgfscope}%
\begin{pgfscope}%
\pgfsetbuttcap%
\pgfsetroundjoin%
\definecolor{currentfill}{rgb}{0.000000,0.000000,0.000000}%
\pgfsetfillcolor{currentfill}%
\pgfsetlinewidth{0.501875pt}%
\definecolor{currentstroke}{rgb}{0.000000,0.000000,0.000000}%
\pgfsetstrokecolor{currentstroke}%
\pgfsetdash{}{0pt}%
\pgfsys@defobject{currentmarker}{\pgfqpoint{0.000000in}{0.000000in}}{\pgfqpoint{0.000000in}{0.055556in}}{%
\pgfpathmoveto{\pgfqpoint{0.000000in}{0.000000in}}%
\pgfpathlineto{\pgfqpoint{0.000000in}{0.055556in}}%
\pgfusepath{stroke,fill}%
}%
\begin{pgfscope}%
\pgfsys@transformshift{1.000000in}{0.600000in}%
\pgfsys@useobject{currentmarker}{}%
\end{pgfscope}%
\end{pgfscope}%
\begin{pgfscope}%
\pgfsetbuttcap%
\pgfsetroundjoin%
\definecolor{currentfill}{rgb}{0.000000,0.000000,0.000000}%
\pgfsetfillcolor{currentfill}%
\pgfsetlinewidth{0.501875pt}%
\definecolor{currentstroke}{rgb}{0.000000,0.000000,0.000000}%
\pgfsetstrokecolor{currentstroke}%
\pgfsetdash{}{0pt}%
\pgfsys@defobject{currentmarker}{\pgfqpoint{0.000000in}{-0.055556in}}{\pgfqpoint{0.000000in}{0.000000in}}{%
\pgfpathmoveto{\pgfqpoint{0.000000in}{0.000000in}}%
\pgfpathlineto{\pgfqpoint{0.000000in}{-0.055556in}}%
\pgfusepath{stroke,fill}%
}%
\begin{pgfscope}%
\pgfsys@transformshift{1.000000in}{2.781818in}%
\pgfsys@useobject{currentmarker}{}%
\end{pgfscope}%
\end{pgfscope}%
\begin{pgfscope}%
\pgftext[x=1.000000in,y=0.544444in,,top]{{\rmfamily\fontsize{12.000000}{14.400000}\selectfont 0}}%
\end{pgfscope}%
\begin{pgfscope}%
\pgfsetbuttcap%
\pgfsetroundjoin%
\definecolor{currentfill}{rgb}{0.000000,0.000000,0.000000}%
\pgfsetfillcolor{currentfill}%
\pgfsetlinewidth{0.501875pt}%
\definecolor{currentstroke}{rgb}{0.000000,0.000000,0.000000}%
\pgfsetstrokecolor{currentstroke}%
\pgfsetdash{}{0pt}%
\pgfsys@defobject{currentmarker}{\pgfqpoint{0.000000in}{0.000000in}}{\pgfqpoint{0.000000in}{0.055556in}}{%
\pgfpathmoveto{\pgfqpoint{0.000000in}{0.000000in}}%
\pgfpathlineto{\pgfqpoint{0.000000in}{0.055556in}}%
\pgfusepath{stroke,fill}%
}%
\begin{pgfscope}%
\pgfsys@transformshift{1.563636in}{0.600000in}%
\pgfsys@useobject{currentmarker}{}%
\end{pgfscope}%
\end{pgfscope}%
\begin{pgfscope}%
\pgfsetbuttcap%
\pgfsetroundjoin%
\definecolor{currentfill}{rgb}{0.000000,0.000000,0.000000}%
\pgfsetfillcolor{currentfill}%
\pgfsetlinewidth{0.501875pt}%
\definecolor{currentstroke}{rgb}{0.000000,0.000000,0.000000}%
\pgfsetstrokecolor{currentstroke}%
\pgfsetdash{}{0pt}%
\pgfsys@defobject{currentmarker}{\pgfqpoint{0.000000in}{-0.055556in}}{\pgfqpoint{0.000000in}{0.000000in}}{%
\pgfpathmoveto{\pgfqpoint{0.000000in}{0.000000in}}%
\pgfpathlineto{\pgfqpoint{0.000000in}{-0.055556in}}%
\pgfusepath{stroke,fill}%
}%
\begin{pgfscope}%
\pgfsys@transformshift{1.563636in}{2.781818in}%
\pgfsys@useobject{currentmarker}{}%
\end{pgfscope}%
\end{pgfscope}%
\begin{pgfscope}%
\pgftext[x=1.563636in,y=0.544444in,,top]{{\rmfamily\fontsize{12.000000}{14.400000}\selectfont 2}}%
\end{pgfscope}%
\begin{pgfscope}%
\pgfsetbuttcap%
\pgfsetroundjoin%
\definecolor{currentfill}{rgb}{0.000000,0.000000,0.000000}%
\pgfsetfillcolor{currentfill}%
\pgfsetlinewidth{0.501875pt}%
\definecolor{currentstroke}{rgb}{0.000000,0.000000,0.000000}%
\pgfsetstrokecolor{currentstroke}%
\pgfsetdash{}{0pt}%
\pgfsys@defobject{currentmarker}{\pgfqpoint{0.000000in}{0.000000in}}{\pgfqpoint{0.000000in}{0.055556in}}{%
\pgfpathmoveto{\pgfqpoint{0.000000in}{0.000000in}}%
\pgfpathlineto{\pgfqpoint{0.000000in}{0.055556in}}%
\pgfusepath{stroke,fill}%
}%
\begin{pgfscope}%
\pgfsys@transformshift{2.127273in}{0.600000in}%
\pgfsys@useobject{currentmarker}{}%
\end{pgfscope}%
\end{pgfscope}%
\begin{pgfscope}%
\pgfsetbuttcap%
\pgfsetroundjoin%
\definecolor{currentfill}{rgb}{0.000000,0.000000,0.000000}%
\pgfsetfillcolor{currentfill}%
\pgfsetlinewidth{0.501875pt}%
\definecolor{currentstroke}{rgb}{0.000000,0.000000,0.000000}%
\pgfsetstrokecolor{currentstroke}%
\pgfsetdash{}{0pt}%
\pgfsys@defobject{currentmarker}{\pgfqpoint{0.000000in}{-0.055556in}}{\pgfqpoint{0.000000in}{0.000000in}}{%
\pgfpathmoveto{\pgfqpoint{0.000000in}{0.000000in}}%
\pgfpathlineto{\pgfqpoint{0.000000in}{-0.055556in}}%
\pgfusepath{stroke,fill}%
}%
\begin{pgfscope}%
\pgfsys@transformshift{2.127273in}{2.781818in}%
\pgfsys@useobject{currentmarker}{}%
\end{pgfscope}%
\end{pgfscope}%
\begin{pgfscope}%
\pgftext[x=2.127273in,y=0.544444in,,top]{{\rmfamily\fontsize{12.000000}{14.400000}\selectfont 4}}%
\end{pgfscope}%
\begin{pgfscope}%
\pgfsetbuttcap%
\pgfsetroundjoin%
\definecolor{currentfill}{rgb}{0.000000,0.000000,0.000000}%
\pgfsetfillcolor{currentfill}%
\pgfsetlinewidth{0.501875pt}%
\definecolor{currentstroke}{rgb}{0.000000,0.000000,0.000000}%
\pgfsetstrokecolor{currentstroke}%
\pgfsetdash{}{0pt}%
\pgfsys@defobject{currentmarker}{\pgfqpoint{0.000000in}{0.000000in}}{\pgfqpoint{0.000000in}{0.055556in}}{%
\pgfpathmoveto{\pgfqpoint{0.000000in}{0.000000in}}%
\pgfpathlineto{\pgfqpoint{0.000000in}{0.055556in}}%
\pgfusepath{stroke,fill}%
}%
\begin{pgfscope}%
\pgfsys@transformshift{2.690909in}{0.600000in}%
\pgfsys@useobject{currentmarker}{}%
\end{pgfscope}%
\end{pgfscope}%
\begin{pgfscope}%
\pgfsetbuttcap%
\pgfsetroundjoin%
\definecolor{currentfill}{rgb}{0.000000,0.000000,0.000000}%
\pgfsetfillcolor{currentfill}%
\pgfsetlinewidth{0.501875pt}%
\definecolor{currentstroke}{rgb}{0.000000,0.000000,0.000000}%
\pgfsetstrokecolor{currentstroke}%
\pgfsetdash{}{0pt}%
\pgfsys@defobject{currentmarker}{\pgfqpoint{0.000000in}{-0.055556in}}{\pgfqpoint{0.000000in}{0.000000in}}{%
\pgfpathmoveto{\pgfqpoint{0.000000in}{0.000000in}}%
\pgfpathlineto{\pgfqpoint{0.000000in}{-0.055556in}}%
\pgfusepath{stroke,fill}%
}%
\begin{pgfscope}%
\pgfsys@transformshift{2.690909in}{2.781818in}%
\pgfsys@useobject{currentmarker}{}%
\end{pgfscope}%
\end{pgfscope}%
\begin{pgfscope}%
\pgftext[x=2.690909in,y=0.544444in,,top]{{\rmfamily\fontsize{12.000000}{14.400000}\selectfont 6}}%
\end{pgfscope}%
\begin{pgfscope}%
\pgfsetbuttcap%
\pgfsetroundjoin%
\definecolor{currentfill}{rgb}{0.000000,0.000000,0.000000}%
\pgfsetfillcolor{currentfill}%
\pgfsetlinewidth{0.501875pt}%
\definecolor{currentstroke}{rgb}{0.000000,0.000000,0.000000}%
\pgfsetstrokecolor{currentstroke}%
\pgfsetdash{}{0pt}%
\pgfsys@defobject{currentmarker}{\pgfqpoint{0.000000in}{0.000000in}}{\pgfqpoint{0.000000in}{0.055556in}}{%
\pgfpathmoveto{\pgfqpoint{0.000000in}{0.000000in}}%
\pgfpathlineto{\pgfqpoint{0.000000in}{0.055556in}}%
\pgfusepath{stroke,fill}%
}%
\begin{pgfscope}%
\pgfsys@transformshift{3.254545in}{0.600000in}%
\pgfsys@useobject{currentmarker}{}%
\end{pgfscope}%
\end{pgfscope}%
\begin{pgfscope}%
\pgfsetbuttcap%
\pgfsetroundjoin%
\definecolor{currentfill}{rgb}{0.000000,0.000000,0.000000}%
\pgfsetfillcolor{currentfill}%
\pgfsetlinewidth{0.501875pt}%
\definecolor{currentstroke}{rgb}{0.000000,0.000000,0.000000}%
\pgfsetstrokecolor{currentstroke}%
\pgfsetdash{}{0pt}%
\pgfsys@defobject{currentmarker}{\pgfqpoint{0.000000in}{-0.055556in}}{\pgfqpoint{0.000000in}{0.000000in}}{%
\pgfpathmoveto{\pgfqpoint{0.000000in}{0.000000in}}%
\pgfpathlineto{\pgfqpoint{0.000000in}{-0.055556in}}%
\pgfusepath{stroke,fill}%
}%
\begin{pgfscope}%
\pgfsys@transformshift{3.254545in}{2.781818in}%
\pgfsys@useobject{currentmarker}{}%
\end{pgfscope}%
\end{pgfscope}%
\begin{pgfscope}%
\pgftext[x=3.254545in,y=0.544444in,,top]{{\rmfamily\fontsize{12.000000}{14.400000}\selectfont 8}}%
\end{pgfscope}%
\begin{pgfscope}%
\pgfsetbuttcap%
\pgfsetroundjoin%
\definecolor{currentfill}{rgb}{0.000000,0.000000,0.000000}%
\pgfsetfillcolor{currentfill}%
\pgfsetlinewidth{0.501875pt}%
\definecolor{currentstroke}{rgb}{0.000000,0.000000,0.000000}%
\pgfsetstrokecolor{currentstroke}%
\pgfsetdash{}{0pt}%
\pgfsys@defobject{currentmarker}{\pgfqpoint{0.000000in}{0.000000in}}{\pgfqpoint{0.000000in}{0.055556in}}{%
\pgfpathmoveto{\pgfqpoint{0.000000in}{0.000000in}}%
\pgfpathlineto{\pgfqpoint{0.000000in}{0.055556in}}%
\pgfusepath{stroke,fill}%
}%
\begin{pgfscope}%
\pgfsys@transformshift{3.818182in}{0.600000in}%
\pgfsys@useobject{currentmarker}{}%
\end{pgfscope}%
\end{pgfscope}%
\begin{pgfscope}%
\pgfsetbuttcap%
\pgfsetroundjoin%
\definecolor{currentfill}{rgb}{0.000000,0.000000,0.000000}%
\pgfsetfillcolor{currentfill}%
\pgfsetlinewidth{0.501875pt}%
\definecolor{currentstroke}{rgb}{0.000000,0.000000,0.000000}%
\pgfsetstrokecolor{currentstroke}%
\pgfsetdash{}{0pt}%
\pgfsys@defobject{currentmarker}{\pgfqpoint{0.000000in}{-0.055556in}}{\pgfqpoint{0.000000in}{0.000000in}}{%
\pgfpathmoveto{\pgfqpoint{0.000000in}{0.000000in}}%
\pgfpathlineto{\pgfqpoint{0.000000in}{-0.055556in}}%
\pgfusepath{stroke,fill}%
}%
\begin{pgfscope}%
\pgfsys@transformshift{3.818182in}{2.781818in}%
\pgfsys@useobject{currentmarker}{}%
\end{pgfscope}%
\end{pgfscope}%
\begin{pgfscope}%
\pgftext[x=3.818182in,y=0.544444in,,top]{{\rmfamily\fontsize{12.000000}{14.400000}\selectfont 10}}%
\end{pgfscope}%
\begin{pgfscope}%
\pgfsetbuttcap%
\pgfsetroundjoin%
\definecolor{currentfill}{rgb}{0.000000,0.000000,0.000000}%
\pgfsetfillcolor{currentfill}%
\pgfsetlinewidth{0.501875pt}%
\definecolor{currentstroke}{rgb}{0.000000,0.000000,0.000000}%
\pgfsetstrokecolor{currentstroke}%
\pgfsetdash{}{0pt}%
\pgfsys@defobject{currentmarker}{\pgfqpoint{0.000000in}{0.000000in}}{\pgfqpoint{0.055556in}{0.000000in}}{%
\pgfpathmoveto{\pgfqpoint{0.000000in}{0.000000in}}%
\pgfpathlineto{\pgfqpoint{0.055556in}{0.000000in}}%
\pgfusepath{stroke,fill}%
}%
\begin{pgfscope}%
\pgfsys@transformshift{1.000000in}{0.600000in}%
\pgfsys@useobject{currentmarker}{}%
\end{pgfscope}%
\end{pgfscope}%
\begin{pgfscope}%
\pgfsetbuttcap%
\pgfsetroundjoin%
\definecolor{currentfill}{rgb}{0.000000,0.000000,0.000000}%
\pgfsetfillcolor{currentfill}%
\pgfsetlinewidth{0.501875pt}%
\definecolor{currentstroke}{rgb}{0.000000,0.000000,0.000000}%
\pgfsetstrokecolor{currentstroke}%
\pgfsetdash{}{0pt}%
\pgfsys@defobject{currentmarker}{\pgfqpoint{-0.055556in}{0.000000in}}{\pgfqpoint{0.000000in}{0.000000in}}{%
\pgfpathmoveto{\pgfqpoint{0.000000in}{0.000000in}}%
\pgfpathlineto{\pgfqpoint{-0.055556in}{0.000000in}}%
\pgfusepath{stroke,fill}%
}%
\begin{pgfscope}%
\pgfsys@transformshift{3.818182in}{0.600000in}%
\pgfsys@useobject{currentmarker}{}%
\end{pgfscope}%
\end{pgfscope}%
\begin{pgfscope}%
\pgftext[x=0.944444in,y=0.600000in,right,]{{\rmfamily\fontsize{12.000000}{14.400000}\selectfont -250}}%
\end{pgfscope}%
\begin{pgfscope}%
\pgfsetbuttcap%
\pgfsetroundjoin%
\definecolor{currentfill}{rgb}{0.000000,0.000000,0.000000}%
\pgfsetfillcolor{currentfill}%
\pgfsetlinewidth{0.501875pt}%
\definecolor{currentstroke}{rgb}{0.000000,0.000000,0.000000}%
\pgfsetstrokecolor{currentstroke}%
\pgfsetdash{}{0pt}%
\pgfsys@defobject{currentmarker}{\pgfqpoint{0.000000in}{0.000000in}}{\pgfqpoint{0.055556in}{0.000000in}}{%
\pgfpathmoveto{\pgfqpoint{0.000000in}{0.000000in}}%
\pgfpathlineto{\pgfqpoint{0.055556in}{0.000000in}}%
\pgfusepath{stroke,fill}%
}%
\begin{pgfscope}%
\pgfsys@transformshift{1.000000in}{1.036364in}%
\pgfsys@useobject{currentmarker}{}%
\end{pgfscope}%
\end{pgfscope}%
\begin{pgfscope}%
\pgfsetbuttcap%
\pgfsetroundjoin%
\definecolor{currentfill}{rgb}{0.000000,0.000000,0.000000}%
\pgfsetfillcolor{currentfill}%
\pgfsetlinewidth{0.501875pt}%
\definecolor{currentstroke}{rgb}{0.000000,0.000000,0.000000}%
\pgfsetstrokecolor{currentstroke}%
\pgfsetdash{}{0pt}%
\pgfsys@defobject{currentmarker}{\pgfqpoint{-0.055556in}{0.000000in}}{\pgfqpoint{0.000000in}{0.000000in}}{%
\pgfpathmoveto{\pgfqpoint{0.000000in}{0.000000in}}%
\pgfpathlineto{\pgfqpoint{-0.055556in}{0.000000in}}%
\pgfusepath{stroke,fill}%
}%
\begin{pgfscope}%
\pgfsys@transformshift{3.818182in}{1.036364in}%
\pgfsys@useobject{currentmarker}{}%
\end{pgfscope}%
\end{pgfscope}%
\begin{pgfscope}%
\pgftext[x=0.944444in,y=1.036364in,right,]{{\rmfamily\fontsize{12.000000}{14.400000}\selectfont -200}}%
\end{pgfscope}%
\begin{pgfscope}%
\pgfsetbuttcap%
\pgfsetroundjoin%
\definecolor{currentfill}{rgb}{0.000000,0.000000,0.000000}%
\pgfsetfillcolor{currentfill}%
\pgfsetlinewidth{0.501875pt}%
\definecolor{currentstroke}{rgb}{0.000000,0.000000,0.000000}%
\pgfsetstrokecolor{currentstroke}%
\pgfsetdash{}{0pt}%
\pgfsys@defobject{currentmarker}{\pgfqpoint{0.000000in}{0.000000in}}{\pgfqpoint{0.055556in}{0.000000in}}{%
\pgfpathmoveto{\pgfqpoint{0.000000in}{0.000000in}}%
\pgfpathlineto{\pgfqpoint{0.055556in}{0.000000in}}%
\pgfusepath{stroke,fill}%
}%
\begin{pgfscope}%
\pgfsys@transformshift{1.000000in}{1.472727in}%
\pgfsys@useobject{currentmarker}{}%
\end{pgfscope}%
\end{pgfscope}%
\begin{pgfscope}%
\pgfsetbuttcap%
\pgfsetroundjoin%
\definecolor{currentfill}{rgb}{0.000000,0.000000,0.000000}%
\pgfsetfillcolor{currentfill}%
\pgfsetlinewidth{0.501875pt}%
\definecolor{currentstroke}{rgb}{0.000000,0.000000,0.000000}%
\pgfsetstrokecolor{currentstroke}%
\pgfsetdash{}{0pt}%
\pgfsys@defobject{currentmarker}{\pgfqpoint{-0.055556in}{0.000000in}}{\pgfqpoint{0.000000in}{0.000000in}}{%
\pgfpathmoveto{\pgfqpoint{0.000000in}{0.000000in}}%
\pgfpathlineto{\pgfqpoint{-0.055556in}{0.000000in}}%
\pgfusepath{stroke,fill}%
}%
\begin{pgfscope}%
\pgfsys@transformshift{3.818182in}{1.472727in}%
\pgfsys@useobject{currentmarker}{}%
\end{pgfscope}%
\end{pgfscope}%
\begin{pgfscope}%
\pgftext[x=0.944444in,y=1.472727in,right,]{{\rmfamily\fontsize{12.000000}{14.400000}\selectfont -150}}%
\end{pgfscope}%
\begin{pgfscope}%
\pgfsetbuttcap%
\pgfsetroundjoin%
\definecolor{currentfill}{rgb}{0.000000,0.000000,0.000000}%
\pgfsetfillcolor{currentfill}%
\pgfsetlinewidth{0.501875pt}%
\definecolor{currentstroke}{rgb}{0.000000,0.000000,0.000000}%
\pgfsetstrokecolor{currentstroke}%
\pgfsetdash{}{0pt}%
\pgfsys@defobject{currentmarker}{\pgfqpoint{0.000000in}{0.000000in}}{\pgfqpoint{0.055556in}{0.000000in}}{%
\pgfpathmoveto{\pgfqpoint{0.000000in}{0.000000in}}%
\pgfpathlineto{\pgfqpoint{0.055556in}{0.000000in}}%
\pgfusepath{stroke,fill}%
}%
\begin{pgfscope}%
\pgfsys@transformshift{1.000000in}{1.909091in}%
\pgfsys@useobject{currentmarker}{}%
\end{pgfscope}%
\end{pgfscope}%
\begin{pgfscope}%
\pgfsetbuttcap%
\pgfsetroundjoin%
\definecolor{currentfill}{rgb}{0.000000,0.000000,0.000000}%
\pgfsetfillcolor{currentfill}%
\pgfsetlinewidth{0.501875pt}%
\definecolor{currentstroke}{rgb}{0.000000,0.000000,0.000000}%
\pgfsetstrokecolor{currentstroke}%
\pgfsetdash{}{0pt}%
\pgfsys@defobject{currentmarker}{\pgfqpoint{-0.055556in}{0.000000in}}{\pgfqpoint{0.000000in}{0.000000in}}{%
\pgfpathmoveto{\pgfqpoint{0.000000in}{0.000000in}}%
\pgfpathlineto{\pgfqpoint{-0.055556in}{0.000000in}}%
\pgfusepath{stroke,fill}%
}%
\begin{pgfscope}%
\pgfsys@transformshift{3.818182in}{1.909091in}%
\pgfsys@useobject{currentmarker}{}%
\end{pgfscope}%
\end{pgfscope}%
\begin{pgfscope}%
\pgftext[x=0.944444in,y=1.909091in,right,]{{\rmfamily\fontsize{12.000000}{14.400000}\selectfont -100}}%
\end{pgfscope}%
\begin{pgfscope}%
\pgfsetbuttcap%
\pgfsetroundjoin%
\definecolor{currentfill}{rgb}{0.000000,0.000000,0.000000}%
\pgfsetfillcolor{currentfill}%
\pgfsetlinewidth{0.501875pt}%
\definecolor{currentstroke}{rgb}{0.000000,0.000000,0.000000}%
\pgfsetstrokecolor{currentstroke}%
\pgfsetdash{}{0pt}%
\pgfsys@defobject{currentmarker}{\pgfqpoint{0.000000in}{0.000000in}}{\pgfqpoint{0.055556in}{0.000000in}}{%
\pgfpathmoveto{\pgfqpoint{0.000000in}{0.000000in}}%
\pgfpathlineto{\pgfqpoint{0.055556in}{0.000000in}}%
\pgfusepath{stroke,fill}%
}%
\begin{pgfscope}%
\pgfsys@transformshift{1.000000in}{2.345455in}%
\pgfsys@useobject{currentmarker}{}%
\end{pgfscope}%
\end{pgfscope}%
\begin{pgfscope}%
\pgfsetbuttcap%
\pgfsetroundjoin%
\definecolor{currentfill}{rgb}{0.000000,0.000000,0.000000}%
\pgfsetfillcolor{currentfill}%
\pgfsetlinewidth{0.501875pt}%
\definecolor{currentstroke}{rgb}{0.000000,0.000000,0.000000}%
\pgfsetstrokecolor{currentstroke}%
\pgfsetdash{}{0pt}%
\pgfsys@defobject{currentmarker}{\pgfqpoint{-0.055556in}{0.000000in}}{\pgfqpoint{0.000000in}{0.000000in}}{%
\pgfpathmoveto{\pgfqpoint{0.000000in}{0.000000in}}%
\pgfpathlineto{\pgfqpoint{-0.055556in}{0.000000in}}%
\pgfusepath{stroke,fill}%
}%
\begin{pgfscope}%
\pgfsys@transformshift{3.818182in}{2.345455in}%
\pgfsys@useobject{currentmarker}{}%
\end{pgfscope}%
\end{pgfscope}%
\begin{pgfscope}%
\pgftext[x=0.944444in,y=2.345455in,right,]{{\rmfamily\fontsize{12.000000}{14.400000}\selectfont -50}}%
\end{pgfscope}%
\begin{pgfscope}%
\pgfsetbuttcap%
\pgfsetroundjoin%
\definecolor{currentfill}{rgb}{0.000000,0.000000,0.000000}%
\pgfsetfillcolor{currentfill}%
\pgfsetlinewidth{0.501875pt}%
\definecolor{currentstroke}{rgb}{0.000000,0.000000,0.000000}%
\pgfsetstrokecolor{currentstroke}%
\pgfsetdash{}{0pt}%
\pgfsys@defobject{currentmarker}{\pgfqpoint{0.000000in}{0.000000in}}{\pgfqpoint{0.055556in}{0.000000in}}{%
\pgfpathmoveto{\pgfqpoint{0.000000in}{0.000000in}}%
\pgfpathlineto{\pgfqpoint{0.055556in}{0.000000in}}%
\pgfusepath{stroke,fill}%
}%
\begin{pgfscope}%
\pgfsys@transformshift{1.000000in}{2.781818in}%
\pgfsys@useobject{currentmarker}{}%
\end{pgfscope}%
\end{pgfscope}%
\begin{pgfscope}%
\pgfsetbuttcap%
\pgfsetroundjoin%
\definecolor{currentfill}{rgb}{0.000000,0.000000,0.000000}%
\pgfsetfillcolor{currentfill}%
\pgfsetlinewidth{0.501875pt}%
\definecolor{currentstroke}{rgb}{0.000000,0.000000,0.000000}%
\pgfsetstrokecolor{currentstroke}%
\pgfsetdash{}{0pt}%
\pgfsys@defobject{currentmarker}{\pgfqpoint{-0.055556in}{0.000000in}}{\pgfqpoint{0.000000in}{0.000000in}}{%
\pgfpathmoveto{\pgfqpoint{0.000000in}{0.000000in}}%
\pgfpathlineto{\pgfqpoint{-0.055556in}{0.000000in}}%
\pgfusepath{stroke,fill}%
}%
\begin{pgfscope}%
\pgfsys@transformshift{3.818182in}{2.781818in}%
\pgfsys@useobject{currentmarker}{}%
\end{pgfscope}%
\end{pgfscope}%
\begin{pgfscope}%
\pgftext[x=0.944444in,y=2.781818in,right,]{{\rmfamily\fontsize{12.000000}{14.400000}\selectfont 0}}%
\end{pgfscope}%
\begin{pgfscope}%
\pgfsetbuttcap%
\pgfsetroundjoin%
\pgfsetlinewidth{1.003750pt}%
\definecolor{currentstroke}{rgb}{0.000000,0.000000,0.000000}%
\pgfsetstrokecolor{currentstroke}%
\pgfsetdash{}{0pt}%
\pgfpathmoveto{\pgfqpoint{1.000000in}{2.781818in}}%
\pgfpathlineto{\pgfqpoint{3.818182in}{2.781818in}}%
\pgfusepath{stroke}%
\end{pgfscope}%
\begin{pgfscope}%
\pgfsetbuttcap%
\pgfsetroundjoin%
\pgfsetlinewidth{1.003750pt}%
\definecolor{currentstroke}{rgb}{0.000000,0.000000,0.000000}%
\pgfsetstrokecolor{currentstroke}%
\pgfsetdash{}{0pt}%
\pgfpathmoveto{\pgfqpoint{3.818182in}{0.600000in}}%
\pgfpathlineto{\pgfqpoint{3.818182in}{2.781818in}}%
\pgfusepath{stroke}%
\end{pgfscope}%
\begin{pgfscope}%
\pgfsetbuttcap%
\pgfsetroundjoin%
\pgfsetlinewidth{1.003750pt}%
\definecolor{currentstroke}{rgb}{0.000000,0.000000,0.000000}%
\pgfsetstrokecolor{currentstroke}%
\pgfsetdash{}{0pt}%
\pgfpathmoveto{\pgfqpoint{1.000000in}{0.600000in}}%
\pgfpathlineto{\pgfqpoint{3.818182in}{0.600000in}}%
\pgfusepath{stroke}%
\end{pgfscope}%
\begin{pgfscope}%
\pgfsetbuttcap%
\pgfsetroundjoin%
\pgfsetlinewidth{1.003750pt}%
\definecolor{currentstroke}{rgb}{0.000000,0.000000,0.000000}%
\pgfsetstrokecolor{currentstroke}%
\pgfsetdash{}{0pt}%
\pgfpathmoveto{\pgfqpoint{1.000000in}{0.600000in}}%
\pgfpathlineto{\pgfqpoint{1.000000in}{2.781818in}}%
\pgfusepath{stroke}%
\end{pgfscope}%
\begin{pgfscope}%
\pgftext[x=2.409091in,y=2.851263in,,base]{{\rmfamily\fontsize{14.400000}{17.280000}\selectfont e2}}%
\end{pgfscope}%
\begin{pgfscope}%
\pgfsetbuttcap%
\pgfsetroundjoin%
\definecolor{currentfill}{rgb}{1.000000,1.000000,1.000000}%
\pgfsetfillcolor{currentfill}%
\pgfsetlinewidth{0.000000pt}%
\definecolor{currentstroke}{rgb}{0.000000,0.000000,0.000000}%
\pgfsetstrokecolor{currentstroke}%
\pgfsetstrokeopacity{0.000000}%
\pgfsetdash{}{0pt}%
\pgfpathmoveto{\pgfqpoint{4.381818in}{0.600000in}}%
\pgfpathlineto{\pgfqpoint{7.200000in}{0.600000in}}%
\pgfpathlineto{\pgfqpoint{7.200000in}{2.781818in}}%
\pgfpathlineto{\pgfqpoint{4.381818in}{2.781818in}}%
\pgfpathclose%
\pgfusepath{fill}%
\end{pgfscope}%
\begin{pgfscope}%
\pgfpathrectangle{\pgfqpoint{4.381818in}{0.600000in}}{\pgfqpoint{2.818182in}{2.181818in}} %
\pgfusepath{clip}%
\pgfsetrectcap%
\pgfsetroundjoin%
\pgfsetlinewidth{1.003750pt}%
\definecolor{currentstroke}{rgb}{0.000000,0.000000,1.000000}%
\pgfsetstrokecolor{currentstroke}%
\pgfsetdash{}{0pt}%
\pgfpathmoveto{\pgfqpoint{4.381818in}{1.690909in}}%
\pgfpathlineto{\pgfqpoint{4.384355in}{1.661044in}}%
\pgfpathlineto{\pgfqpoint{4.389992in}{1.595715in}}%
\pgfpathlineto{\pgfqpoint{4.391119in}{1.592319in}}%
\pgfpathlineto{\pgfqpoint{4.391683in}{1.592889in}}%
\pgfpathlineto{\pgfqpoint{4.392810in}{1.599098in}}%
\pgfpathlineto{\pgfqpoint{4.395065in}{1.629612in}}%
\pgfpathlineto{\pgfqpoint{4.399293in}{1.685875in}}%
\pgfpathlineto{\pgfqpoint{4.400138in}{1.687128in}}%
\pgfpathlineto{\pgfqpoint{4.400420in}{1.686661in}}%
\pgfpathlineto{\pgfqpoint{4.401547in}{1.680886in}}%
\pgfpathlineto{\pgfqpoint{4.404084in}{1.652438in}}%
\pgfpathlineto{\pgfqpoint{4.410003in}{1.583280in}}%
\pgfpathlineto{\pgfqpoint{4.410567in}{1.582290in}}%
\pgfpathlineto{\pgfqpoint{4.411130in}{1.583169in}}%
\pgfpathlineto{\pgfqpoint{4.412258in}{1.590756in}}%
\pgfpathlineto{\pgfqpoint{4.414794in}{1.629842in}}%
\pgfpathlineto{\pgfqpoint{4.418458in}{1.677257in}}%
\pgfpathlineto{\pgfqpoint{4.419304in}{1.678791in}}%
\pgfpathlineto{\pgfqpoint{4.419586in}{1.678484in}}%
\pgfpathlineto{\pgfqpoint{4.420713in}{1.673772in}}%
\pgfpathlineto{\pgfqpoint{4.423250in}{1.649599in}}%
\pgfpathlineto{\pgfqpoint{4.430014in}{1.572290in}}%
\pgfpathlineto{\pgfqpoint{4.430578in}{1.572986in}}%
\pgfpathlineto{\pgfqpoint{4.431705in}{1.581321in}}%
\pgfpathlineto{\pgfqpoint{4.434242in}{1.625485in}}%
\pgfpathlineto{\pgfqpoint{4.437624in}{1.671565in}}%
\pgfpathlineto{\pgfqpoint{4.438187in}{1.672743in}}%
\pgfpathlineto{\pgfqpoint{4.438751in}{1.672111in}}%
\pgfpathlineto{\pgfqpoint{4.440160in}{1.664479in}}%
\pgfpathlineto{\pgfqpoint{4.444106in}{1.624201in}}%
\pgfpathlineto{\pgfqpoint{4.449743in}{1.562884in}}%
\pgfpathlineto{\pgfqpoint{4.450025in}{1.563407in}}%
\pgfpathlineto{\pgfqpoint{4.451152in}{1.572235in}}%
\pgfpathlineto{\pgfqpoint{4.453689in}{1.621918in}}%
\pgfpathlineto{\pgfqpoint{4.456789in}{1.668189in}}%
\pgfpathlineto{\pgfqpoint{4.457353in}{1.669143in}}%
\pgfpathlineto{\pgfqpoint{4.457635in}{1.668771in}}%
\pgfpathlineto{\pgfqpoint{4.458762in}{1.662729in}}%
\pgfpathlineto{\pgfqpoint{4.463554in}{1.613545in}}%
\pgfpathlineto{\pgfqpoint{4.469191in}{1.554090in}}%
\pgfpathlineto{\pgfqpoint{4.469754in}{1.556288in}}%
\pgfpathlineto{\pgfqpoint{4.471163in}{1.574843in}}%
\pgfpathlineto{\pgfqpoint{4.476237in}{1.666296in}}%
\pgfpathlineto{\pgfqpoint{4.476519in}{1.666177in}}%
\pgfpathlineto{\pgfqpoint{4.477364in}{1.662088in}}%
\pgfpathlineto{\pgfqpoint{4.480183in}{1.629156in}}%
\pgfpathlineto{\pgfqpoint{4.488074in}{1.544789in}}%
\pgfpathlineto{\pgfqpoint{4.488356in}{1.544931in}}%
\pgfpathlineto{\pgfqpoint{4.489202in}{1.550275in}}%
\pgfpathlineto{\pgfqpoint{4.490893in}{1.582381in}}%
\pgfpathlineto{\pgfqpoint{4.494839in}{1.662118in}}%
\pgfpathlineto{\pgfqpoint{4.495120in}{1.662598in}}%
\pgfpathlineto{\pgfqpoint{4.495402in}{1.662250in}}%
\pgfpathlineto{\pgfqpoint{4.496530in}{1.653901in}}%
\pgfpathlineto{\pgfqpoint{4.506958in}{1.533128in}}%
\pgfpathlineto{\pgfqpoint{4.507240in}{1.533873in}}%
\pgfpathlineto{\pgfqpoint{4.508367in}{1.546558in}}%
\pgfpathlineto{\pgfqpoint{4.511186in}{1.623721in}}%
\pgfpathlineto{\pgfqpoint{4.513440in}{1.656800in}}%
\pgfpathlineto{\pgfqpoint{4.514004in}{1.655189in}}%
\pgfpathlineto{\pgfqpoint{4.515413in}{1.638892in}}%
\pgfpathlineto{\pgfqpoint{4.519923in}{1.584130in}}%
\pgfpathlineto{\pgfqpoint{4.522178in}{1.553135in}}%
\pgfpathlineto{\pgfqpoint{4.524996in}{1.515345in}}%
\pgfpathlineto{\pgfqpoint{4.525278in}{1.515845in}}%
\pgfpathlineto{\pgfqpoint{4.526124in}{1.525047in}}%
\pgfpathlineto{\pgfqpoint{4.528096in}{1.584273in}}%
\pgfpathlineto{\pgfqpoint{4.530915in}{1.648017in}}%
\pgfpathlineto{\pgfqpoint{4.531479in}{1.646357in}}%
\pgfpathlineto{\pgfqpoint{4.533170in}{1.622892in}}%
\pgfpathlineto{\pgfqpoint{4.535706in}{1.595942in}}%
\pgfpathlineto{\pgfqpoint{4.537397in}{1.589935in}}%
\pgfpathlineto{\pgfqpoint{4.538807in}{1.568572in}}%
\pgfpathlineto{\pgfqpoint{4.542471in}{1.485739in}}%
\pgfpathlineto{\pgfqpoint{4.542752in}{1.487841in}}%
\pgfpathlineto{\pgfqpoint{4.543880in}{1.517701in}}%
\pgfpathlineto{\pgfqpoint{4.547262in}{1.633992in}}%
\pgfpathlineto{\pgfqpoint{4.547544in}{1.632995in}}%
\pgfpathlineto{\pgfqpoint{4.548953in}{1.608423in}}%
\pgfpathlineto{\pgfqpoint{4.550644in}{1.585527in}}%
\pgfpathlineto{\pgfqpoint{4.550926in}{1.586678in}}%
\pgfpathlineto{\pgfqpoint{4.552335in}{1.610958in}}%
\pgfpathlineto{\pgfqpoint{4.553463in}{1.626686in}}%
\pgfpathlineto{\pgfqpoint{4.554026in}{1.622897in}}%
\pgfpathlineto{\pgfqpoint{4.555154in}{1.585222in}}%
\pgfpathlineto{\pgfqpoint{4.558254in}{1.446957in}}%
\pgfpathlineto{\pgfqpoint{4.558536in}{1.450161in}}%
\pgfpathlineto{\pgfqpoint{4.559663in}{1.501732in}}%
\pgfpathlineto{\pgfqpoint{4.561918in}{1.604864in}}%
\pgfpathlineto{\pgfqpoint{4.562200in}{1.602018in}}%
\pgfpathlineto{\pgfqpoint{4.563891in}{1.540352in}}%
\pgfpathlineto{\pgfqpoint{4.564736in}{1.525004in}}%
\pgfpathlineto{\pgfqpoint{4.565018in}{1.528357in}}%
\pgfpathlineto{\pgfqpoint{4.566146in}{1.587622in}}%
\pgfpathlineto{\pgfqpoint{4.568119in}{1.689512in}}%
\pgfpathlineto{\pgfqpoint{4.568400in}{1.683797in}}%
\pgfpathlineto{\pgfqpoint{4.569810in}{1.585531in}}%
\pgfpathlineto{\pgfqpoint{4.571783in}{1.472654in}}%
\pgfpathlineto{\pgfqpoint{4.572064in}{1.477598in}}%
\pgfpathlineto{\pgfqpoint{4.574319in}{1.582181in}}%
\pgfpathlineto{\pgfqpoint{4.574883in}{1.567717in}}%
\pgfpathlineto{\pgfqpoint{4.577138in}{1.419382in}}%
\pgfpathlineto{\pgfqpoint{4.577701in}{1.433439in}}%
\pgfpathlineto{\pgfqpoint{4.580238in}{1.658927in}}%
\pgfpathlineto{\pgfqpoint{4.581084in}{1.623797in}}%
\pgfpathlineto{\pgfqpoint{4.582775in}{1.512484in}}%
\pgfpathlineto{\pgfqpoint{4.583338in}{1.529036in}}%
\pgfpathlineto{\pgfqpoint{4.585593in}{1.692252in}}%
\pgfpathlineto{\pgfqpoint{4.586157in}{1.662351in}}%
\pgfpathlineto{\pgfqpoint{4.588412in}{1.439785in}}%
\pgfpathlineto{\pgfqpoint{4.588975in}{1.453778in}}%
\pgfpathlineto{\pgfqpoint{4.590948in}{1.590396in}}%
\pgfpathlineto{\pgfqpoint{4.591512in}{1.573924in}}%
\pgfpathlineto{\pgfqpoint{4.593767in}{1.410366in}}%
\pgfpathlineto{\pgfqpoint{4.594330in}{1.436395in}}%
\pgfpathlineto{\pgfqpoint{4.596867in}{1.650176in}}%
\pgfpathlineto{\pgfqpoint{4.597431in}{1.637350in}}%
\pgfpathlineto{\pgfqpoint{4.599404in}{1.526591in}}%
\pgfpathlineto{\pgfqpoint{4.599967in}{1.535438in}}%
\pgfpathlineto{\pgfqpoint{4.602504in}{1.665863in}}%
\pgfpathlineto{\pgfqpoint{4.603068in}{1.645942in}}%
\pgfpathlineto{\pgfqpoint{4.605886in}{1.449851in}}%
\pgfpathlineto{\pgfqpoint{4.606450in}{1.463327in}}%
\pgfpathlineto{\pgfqpoint{4.608705in}{1.557381in}}%
\pgfpathlineto{\pgfqpoint{4.609268in}{1.552642in}}%
\pgfpathlineto{\pgfqpoint{4.611805in}{1.470502in}}%
\pgfpathlineto{\pgfqpoint{4.612369in}{1.481181in}}%
\pgfpathlineto{\pgfqpoint{4.614341in}{1.600547in}}%
\pgfpathlineto{\pgfqpoint{4.616033in}{1.668435in}}%
\pgfpathlineto{\pgfqpoint{4.616314in}{1.667200in}}%
\pgfpathlineto{\pgfqpoint{4.617442in}{1.626790in}}%
\pgfpathlineto{\pgfqpoint{4.619978in}{1.527118in}}%
\pgfpathlineto{\pgfqpoint{4.620260in}{1.528088in}}%
\pgfpathlineto{\pgfqpoint{4.621669in}{1.555825in}}%
\pgfpathlineto{\pgfqpoint{4.622797in}{1.573588in}}%
\pgfpathlineto{\pgfqpoint{4.623361in}{1.571243in}}%
\pgfpathlineto{\pgfqpoint{4.624488in}{1.539819in}}%
\pgfpathlineto{\pgfqpoint{4.627025in}{1.459954in}}%
\pgfpathlineto{\pgfqpoint{4.627306in}{1.461363in}}%
\pgfpathlineto{\pgfqpoint{4.628434in}{1.491420in}}%
\pgfpathlineto{\pgfqpoint{4.632380in}{1.642495in}}%
\pgfpathlineto{\pgfqpoint{4.632943in}{1.640082in}}%
\pgfpathlineto{\pgfqpoint{4.634353in}{1.608854in}}%
\pgfpathlineto{\pgfqpoint{4.636889in}{1.564790in}}%
\pgfpathlineto{\pgfqpoint{4.637453in}{1.566010in}}%
\pgfpathlineto{\pgfqpoint{4.639426in}{1.577277in}}%
\pgfpathlineto{\pgfqpoint{4.639989in}{1.575609in}}%
\pgfpathlineto{\pgfqpoint{4.641117in}{1.559203in}}%
\pgfpathlineto{\pgfqpoint{4.644499in}{1.484250in}}%
\pgfpathlineto{\pgfqpoint{4.645063in}{1.486721in}}%
\pgfpathlineto{\pgfqpoint{4.646472in}{1.518869in}}%
\pgfpathlineto{\pgfqpoint{4.650981in}{1.642939in}}%
\pgfpathlineto{\pgfqpoint{4.651827in}{1.636302in}}%
\pgfpathlineto{\pgfqpoint{4.656900in}{1.567996in}}%
\pgfpathlineto{\pgfqpoint{4.657182in}{1.568085in}}%
\pgfpathlineto{\pgfqpoint{4.658309in}{1.567916in}}%
\pgfpathlineto{\pgfqpoint{4.659437in}{1.562118in}}%
\pgfpathlineto{\pgfqpoint{4.661692in}{1.528800in}}%
\pgfpathlineto{\pgfqpoint{4.663383in}{1.510893in}}%
\pgfpathlineto{\pgfqpoint{4.663665in}{1.510984in}}%
\pgfpathlineto{\pgfqpoint{4.664510in}{1.518015in}}%
\pgfpathlineto{\pgfqpoint{4.666201in}{1.559221in}}%
\pgfpathlineto{\pgfqpoint{4.670147in}{1.654873in}}%
\pgfpathlineto{\pgfqpoint{4.670711in}{1.652867in}}%
\pgfpathlineto{\pgfqpoint{4.672120in}{1.628937in}}%
\pgfpathlineto{\pgfqpoint{4.676066in}{1.557036in}}%
\pgfpathlineto{\pgfqpoint{4.676629in}{1.556561in}}%
\pgfpathlineto{\pgfqpoint{4.676911in}{1.557010in}}%
\pgfpathlineto{\pgfqpoint{4.679166in}{1.562980in}}%
\pgfpathlineto{\pgfqpoint{4.679730in}{1.561784in}}%
\pgfpathlineto{\pgfqpoint{4.681139in}{1.551625in}}%
\pgfpathlineto{\pgfqpoint{4.683394in}{1.536644in}}%
\pgfpathlineto{\pgfqpoint{4.683676in}{1.537419in}}%
\pgfpathlineto{\pgfqpoint{4.684803in}{1.549742in}}%
\pgfpathlineto{\pgfqpoint{4.687340in}{1.618255in}}%
\pgfpathlineto{\pgfqpoint{4.689876in}{1.665226in}}%
\pgfpathlineto{\pgfqpoint{4.690440in}{1.663180in}}%
\pgfpathlineto{\pgfqpoint{4.691849in}{1.637014in}}%
\pgfpathlineto{\pgfqpoint{4.696077in}{1.546459in}}%
\pgfpathlineto{\pgfqpoint{4.696641in}{1.547550in}}%
\pgfpathlineto{\pgfqpoint{4.698332in}{1.562549in}}%
\pgfpathlineto{\pgfqpoint{4.700305in}{1.575305in}}%
\pgfpathlineto{\pgfqpoint{4.700586in}{1.575218in}}%
\pgfpathlineto{\pgfqpoint{4.701714in}{1.570551in}}%
\pgfpathlineto{\pgfqpoint{4.703687in}{1.561946in}}%
\pgfpathlineto{\pgfqpoint{4.703969in}{1.562644in}}%
\pgfpathlineto{\pgfqpoint{4.705096in}{1.573506in}}%
\pgfpathlineto{\pgfqpoint{4.707633in}{1.634470in}}%
\pgfpathlineto{\pgfqpoint{4.709887in}{1.670734in}}%
\pgfpathlineto{\pgfqpoint{4.710451in}{1.668656in}}%
\pgfpathlineto{\pgfqpoint{4.711860in}{1.641668in}}%
\pgfpathlineto{\pgfqpoint{4.716088in}{1.540387in}}%
\pgfpathlineto{\pgfqpoint{4.716370in}{1.540600in}}%
\pgfpathlineto{\pgfqpoint{4.717215in}{1.546807in}}%
\pgfpathlineto{\pgfqpoint{4.721443in}{1.595900in}}%
\pgfpathlineto{\pgfqpoint{4.722007in}{1.594918in}}%
\pgfpathlineto{\pgfqpoint{4.724262in}{1.587351in}}%
\pgfpathlineto{\pgfqpoint{4.724543in}{1.587998in}}%
\pgfpathlineto{\pgfqpoint{4.725671in}{1.597599in}}%
\pgfpathlineto{\pgfqpoint{4.730180in}{1.671347in}}%
\pgfpathlineto{\pgfqpoint{4.731026in}{1.664764in}}%
\pgfpathlineto{\pgfqpoint{4.732717in}{1.622451in}}%
\pgfpathlineto{\pgfqpoint{4.736099in}{1.540301in}}%
\pgfpathlineto{\pgfqpoint{4.736381in}{1.539789in}}%
\pgfpathlineto{\pgfqpoint{4.736663in}{1.540437in}}%
\pgfpathlineto{\pgfqpoint{4.737790in}{1.552959in}}%
\pgfpathlineto{\pgfqpoint{4.742300in}{1.619566in}}%
\pgfpathlineto{\pgfqpoint{4.742582in}{1.619470in}}%
\pgfpathlineto{\pgfqpoint{4.743991in}{1.614902in}}%
\pgfpathlineto{\pgfqpoint{4.744836in}{1.613103in}}%
\pgfpathlineto{\pgfqpoint{4.745400in}{1.614007in}}%
\pgfpathlineto{\pgfqpoint{4.746527in}{1.622523in}}%
\pgfpathlineto{\pgfqpoint{4.750191in}{1.668157in}}%
\pgfpathlineto{\pgfqpoint{4.751037in}{1.664594in}}%
\pgfpathlineto{\pgfqpoint{4.752446in}{1.638409in}}%
\pgfpathlineto{\pgfqpoint{4.756674in}{1.545924in}}%
\pgfpathlineto{\pgfqpoint{4.757238in}{1.548559in}}%
\pgfpathlineto{\pgfqpoint{4.758647in}{1.571494in}}%
\pgfpathlineto{\pgfqpoint{4.762874in}{1.645009in}}%
\pgfpathlineto{\pgfqpoint{4.763438in}{1.645465in}}%
\pgfpathlineto{\pgfqpoint{4.763720in}{1.645025in}}%
\pgfpathlineto{\pgfqpoint{4.765975in}{1.638464in}}%
\pgfpathlineto{\pgfqpoint{4.766538in}{1.639091in}}%
\pgfpathlineto{\pgfqpoint{4.767948in}{1.647139in}}%
\pgfpathlineto{\pgfqpoint{4.770484in}{1.662821in}}%
\pgfpathlineto{\pgfqpoint{4.770766in}{1.662290in}}%
\pgfpathlineto{\pgfqpoint{4.771894in}{1.652438in}}%
\pgfpathlineto{\pgfqpoint{4.774430in}{1.595316in}}%
\pgfpathlineto{\pgfqpoint{4.776967in}{1.559566in}}%
\pgfpathlineto{\pgfqpoint{4.777812in}{1.565019in}}%
\pgfpathlineto{\pgfqpoint{4.779785in}{1.605615in}}%
\pgfpathlineto{\pgfqpoint{4.783449in}{1.672527in}}%
\pgfpathlineto{\pgfqpoint{4.784013in}{1.673625in}}%
\pgfpathlineto{\pgfqpoint{4.784577in}{1.672577in}}%
\pgfpathlineto{\pgfqpoint{4.788522in}{1.656178in}}%
\pgfpathlineto{\pgfqpoint{4.789368in}{1.656873in}}%
\pgfpathlineto{\pgfqpoint{4.790495in}{1.657322in}}%
\pgfpathlineto{\pgfqpoint{4.790777in}{1.656877in}}%
\pgfpathlineto{\pgfqpoint{4.791905in}{1.651059in}}%
\pgfpathlineto{\pgfqpoint{4.793878in}{1.623307in}}%
\pgfpathlineto{\pgfqpoint{4.797260in}{1.581207in}}%
\pgfpathlineto{\pgfqpoint{4.798105in}{1.586444in}}%
\pgfpathlineto{\pgfqpoint{4.799796in}{1.618308in}}%
\pgfpathlineto{\pgfqpoint{4.804024in}{1.702445in}}%
\pgfpathlineto{\pgfqpoint{4.804588in}{1.703088in}}%
\pgfpathlineto{\pgfqpoint{4.804870in}{1.702339in}}%
\pgfpathlineto{\pgfqpoint{4.806279in}{1.690576in}}%
\pgfpathlineto{\pgfqpoint{4.810225in}{1.654701in}}%
\pgfpathlineto{\pgfqpoint{4.813325in}{1.643363in}}%
\pgfpathlineto{\pgfqpoint{4.817271in}{1.610733in}}%
\pgfpathlineto{\pgfqpoint{4.818116in}{1.613362in}}%
\pgfpathlineto{\pgfqpoint{4.819526in}{1.632825in}}%
\pgfpathlineto{\pgfqpoint{4.824599in}{1.733299in}}%
\pgfpathlineto{\pgfqpoint{4.825163in}{1.731640in}}%
\pgfpathlineto{\pgfqpoint{4.826572in}{1.713801in}}%
\pgfpathlineto{\pgfqpoint{4.830799in}{1.653581in}}%
\pgfpathlineto{\pgfqpoint{4.831363in}{1.653105in}}%
\pgfpathlineto{\pgfqpoint{4.831645in}{1.653407in}}%
\pgfpathlineto{\pgfqpoint{4.834182in}{1.658701in}}%
\pgfpathlineto{\pgfqpoint{4.834745in}{1.657928in}}%
\pgfpathlineto{\pgfqpoint{4.837564in}{1.647902in}}%
\pgfpathlineto{\pgfqpoint{4.838409in}{1.650216in}}%
\pgfpathlineto{\pgfqpoint{4.839819in}{1.667434in}}%
\pgfpathlineto{\pgfqpoint{4.844610in}{1.762853in}}%
\pgfpathlineto{\pgfqpoint{4.845455in}{1.758900in}}%
\pgfpathlineto{\pgfqpoint{4.847147in}{1.726464in}}%
\pgfpathlineto{\pgfqpoint{4.850811in}{1.660476in}}%
\pgfpathlineto{\pgfqpoint{4.851374in}{1.661107in}}%
\pgfpathlineto{\pgfqpoint{4.852783in}{1.672591in}}%
\pgfpathlineto{\pgfqpoint{4.855884in}{1.694994in}}%
\pgfpathlineto{\pgfqpoint{4.856729in}{1.694051in}}%
\pgfpathlineto{\pgfqpoint{4.858139in}{1.692112in}}%
\pgfpathlineto{\pgfqpoint{4.858420in}{1.692578in}}%
\pgfpathlineto{\pgfqpoint{4.859548in}{1.700103in}}%
\pgfpathlineto{\pgfqpoint{4.861521in}{1.737052in}}%
\pgfpathlineto{\pgfqpoint{4.864621in}{1.791029in}}%
\pgfpathlineto{\pgfqpoint{4.865185in}{1.789538in}}%
\pgfpathlineto{\pgfqpoint{4.866312in}{1.773083in}}%
\pgfpathlineto{\pgfqpoint{4.870822in}{1.676082in}}%
\pgfpathlineto{\pgfqpoint{4.871385in}{1.677625in}}%
\pgfpathlineto{\pgfqpoint{4.872795in}{1.695600in}}%
\pgfpathlineto{\pgfqpoint{4.876459in}{1.743512in}}%
\pgfpathlineto{\pgfqpoint{4.877022in}{1.744049in}}%
\pgfpathlineto{\pgfqpoint{4.877586in}{1.743471in}}%
\pgfpathlineto{\pgfqpoint{4.878713in}{1.742394in}}%
\pgfpathlineto{\pgfqpoint{4.878995in}{1.742773in}}%
\pgfpathlineto{\pgfqpoint{4.880123in}{1.749352in}}%
\pgfpathlineto{\pgfqpoint{4.882095in}{1.781228in}}%
\pgfpathlineto{\pgfqpoint{4.884914in}{1.817528in}}%
\pgfpathlineto{\pgfqpoint{4.885759in}{1.811363in}}%
\pgfpathlineto{\pgfqpoint{4.887451in}{1.773305in}}%
\pgfpathlineto{\pgfqpoint{4.890833in}{1.702173in}}%
\pgfpathlineto{\pgfqpoint{4.891396in}{1.703536in}}%
\pgfpathlineto{\pgfqpoint{4.892806in}{1.724121in}}%
\pgfpathlineto{\pgfqpoint{4.897315in}{1.799855in}}%
\pgfpathlineto{\pgfqpoint{4.897879in}{1.800185in}}%
\pgfpathlineto{\pgfqpoint{4.898161in}{1.799866in}}%
\pgfpathlineto{\pgfqpoint{4.899570in}{1.797707in}}%
\pgfpathlineto{\pgfqpoint{4.900134in}{1.798418in}}%
\pgfpathlineto{\pgfqpoint{4.901261in}{1.805172in}}%
\pgfpathlineto{\pgfqpoint{4.904925in}{1.843466in}}%
\pgfpathlineto{\pgfqpoint{4.905771in}{1.840272in}}%
\pgfpathlineto{\pgfqpoint{4.907180in}{1.817290in}}%
\pgfpathlineto{\pgfqpoint{4.911126in}{1.740099in}}%
\pgfpathlineto{\pgfqpoint{4.911408in}{1.740586in}}%
\pgfpathlineto{\pgfqpoint{4.912535in}{1.753380in}}%
\pgfpathlineto{\pgfqpoint{4.918454in}{1.863562in}}%
\pgfpathlineto{\pgfqpoint{4.919017in}{1.862491in}}%
\pgfpathlineto{\pgfqpoint{4.921272in}{1.855202in}}%
\pgfpathlineto{\pgfqpoint{4.921836in}{1.855724in}}%
\pgfpathlineto{\pgfqpoint{4.923245in}{1.862505in}}%
\pgfpathlineto{\pgfqpoint{4.925500in}{1.872523in}}%
\pgfpathlineto{\pgfqpoint{4.926345in}{1.868804in}}%
\pgfpathlineto{\pgfqpoint{4.928036in}{1.844215in}}%
\pgfpathlineto{\pgfqpoint{4.931419in}{1.792964in}}%
\pgfpathlineto{\pgfqpoint{4.931982in}{1.795230in}}%
\pgfpathlineto{\pgfqpoint{4.933392in}{1.818718in}}%
\pgfpathlineto{\pgfqpoint{4.938747in}{1.939502in}}%
\pgfpathlineto{\pgfqpoint{4.939028in}{1.939442in}}%
\pgfpathlineto{\pgfqpoint{4.939874in}{1.935322in}}%
\pgfpathlineto{\pgfqpoint{4.943538in}{1.909106in}}%
\pgfpathlineto{\pgfqpoint{4.943820in}{1.909198in}}%
\pgfpathlineto{\pgfqpoint{4.945229in}{1.912690in}}%
\pgfpathlineto{\pgfqpoint{4.946356in}{1.914411in}}%
\pgfpathlineto{\pgfqpoint{4.946638in}{1.914043in}}%
\pgfpathlineto{\pgfqpoint{4.947766in}{1.907918in}}%
\pgfpathlineto{\pgfqpoint{4.951430in}{1.868218in}}%
\pgfpathlineto{\pgfqpoint{4.952275in}{1.872742in}}%
\pgfpathlineto{\pgfqpoint{4.953684in}{1.902347in}}%
\pgfpathlineto{\pgfqpoint{4.958758in}{2.046631in}}%
\pgfpathlineto{\pgfqpoint{4.959040in}{2.045634in}}%
\pgfpathlineto{\pgfqpoint{4.960167in}{2.029429in}}%
\pgfpathlineto{\pgfqpoint{4.963549in}{1.972300in}}%
\pgfpathlineto{\pgfqpoint{4.964395in}{1.977649in}}%
\pgfpathlineto{\pgfqpoint{4.967213in}{2.010262in}}%
\pgfpathlineto{\pgfqpoint{4.967777in}{2.008193in}}%
\pgfpathlineto{\pgfqpoint{4.969186in}{1.985198in}}%
\pgfpathlineto{\pgfqpoint{4.970595in}{1.963786in}}%
\pgfpathlineto{\pgfqpoint{4.971159in}{1.966286in}}%
\pgfpathlineto{\pgfqpoint{4.972286in}{2.000455in}}%
\pgfpathlineto{\pgfqpoint{4.975950in}{2.199224in}}%
\pgfpathlineto{\pgfqpoint{4.976514in}{2.187272in}}%
\pgfpathlineto{\pgfqpoint{4.979051in}{2.027753in}}%
\pgfpathlineto{\pgfqpoint{4.979896in}{2.050636in}}%
\pgfpathlineto{\pgfqpoint{4.981869in}{2.194278in}}%
\pgfpathlineto{\pgfqpoint{4.982433in}{2.181986in}}%
\pgfpathlineto{\pgfqpoint{4.984688in}{1.918587in}}%
\pgfpathlineto{\pgfqpoint{4.985533in}{1.962410in}}%
\pgfpathlineto{\pgfqpoint{4.987506in}{2.203148in}}%
\pgfpathlineto{\pgfqpoint{4.988070in}{2.179647in}}%
\pgfpathlineto{\pgfqpoint{4.990043in}{1.994838in}}%
\pgfpathlineto{\pgfqpoint{4.990606in}{2.018028in}}%
\pgfpathlineto{\pgfqpoint{4.992861in}{2.247704in}}%
\pgfpathlineto{\pgfqpoint{4.993707in}{2.218960in}}%
\pgfpathlineto{\pgfqpoint{4.996807in}{2.026144in}}%
\pgfpathlineto{\pgfqpoint{4.997089in}{2.028335in}}%
\pgfpathlineto{\pgfqpoint{4.999344in}{2.076198in}}%
\pgfpathlineto{\pgfqpoint{5.000189in}{2.064510in}}%
\pgfpathlineto{\pgfqpoint{5.003853in}{1.942485in}}%
\pgfpathlineto{\pgfqpoint{5.004699in}{1.954929in}}%
\pgfpathlineto{\pgfqpoint{5.008645in}{2.082867in}}%
\pgfpathlineto{\pgfqpoint{5.009490in}{2.073207in}}%
\pgfpathlineto{\pgfqpoint{5.016254in}{1.945781in}}%
\pgfpathlineto{\pgfqpoint{5.017945in}{1.925415in}}%
\pgfpathlineto{\pgfqpoint{5.021891in}{1.864276in}}%
\pgfpathlineto{\pgfqpoint{5.022173in}{1.865146in}}%
\pgfpathlineto{\pgfqpoint{5.023301in}{1.879477in}}%
\pgfpathlineto{\pgfqpoint{5.027528in}{1.964945in}}%
\pgfpathlineto{\pgfqpoint{5.028092in}{1.962997in}}%
\pgfpathlineto{\pgfqpoint{5.029501in}{1.941264in}}%
\pgfpathlineto{\pgfqpoint{5.034856in}{1.839810in}}%
\pgfpathlineto{\pgfqpoint{5.041057in}{1.796654in}}%
\pgfpathlineto{\pgfqpoint{5.041902in}{1.798970in}}%
\pgfpathlineto{\pgfqpoint{5.043312in}{1.815078in}}%
\pgfpathlineto{\pgfqpoint{5.047539in}{1.873425in}}%
\pgfpathlineto{\pgfqpoint{5.047821in}{1.872577in}}%
\pgfpathlineto{\pgfqpoint{5.048949in}{1.860909in}}%
\pgfpathlineto{\pgfqpoint{5.052049in}{1.790043in}}%
\pgfpathlineto{\pgfqpoint{5.055149in}{1.741018in}}%
\pgfpathlineto{\pgfqpoint{5.057404in}{1.732896in}}%
\pgfpathlineto{\pgfqpoint{5.059659in}{1.731769in}}%
\pgfpathlineto{\pgfqpoint{5.060786in}{1.733220in}}%
\pgfpathlineto{\pgfqpoint{5.062195in}{1.739597in}}%
\pgfpathlineto{\pgfqpoint{5.064732in}{1.765424in}}%
\pgfpathlineto{\pgfqpoint{5.067269in}{1.784588in}}%
\pgfpathlineto{\pgfqpoint{5.067832in}{1.783543in}}%
\pgfpathlineto{\pgfqpoint{5.068960in}{1.773694in}}%
\pgfpathlineto{\pgfqpoint{5.071214in}{1.728078in}}%
\pgfpathlineto{\pgfqpoint{5.075442in}{1.645679in}}%
\pgfpathlineto{\pgfqpoint{5.076569in}{1.641629in}}%
\pgfpathlineto{\pgfqpoint{5.076851in}{1.641861in}}%
\pgfpathlineto{\pgfqpoint{5.077979in}{1.646566in}}%
\pgfpathlineto{\pgfqpoint{5.085589in}{1.690265in}}%
\pgfpathlineto{\pgfqpoint{5.086998in}{1.692074in}}%
\pgfpathlineto{\pgfqpoint{5.087280in}{1.691745in}}%
\pgfpathlineto{\pgfqpoint{5.088407in}{1.686868in}}%
\pgfpathlineto{\pgfqpoint{5.090098in}{1.665923in}}%
\pgfpathlineto{\pgfqpoint{5.096299in}{1.549250in}}%
\pgfpathlineto{\pgfqpoint{5.097144in}{1.551634in}}%
\pgfpathlineto{\pgfqpoint{5.098835in}{1.571668in}}%
\pgfpathlineto{\pgfqpoint{5.102499in}{1.612439in}}%
\pgfpathlineto{\pgfqpoint{5.103063in}{1.612889in}}%
\pgfpathlineto{\pgfqpoint{5.103345in}{1.612562in}}%
\pgfpathlineto{\pgfqpoint{5.104754in}{1.607230in}}%
\pgfpathlineto{\pgfqpoint{5.109827in}{1.575494in}}%
\pgfpathlineto{\pgfqpoint{5.112082in}{1.532422in}}%
\pgfpathlineto{\pgfqpoint{5.116310in}{1.453983in}}%
\pgfpathlineto{\pgfqpoint{5.116874in}{1.455943in}}%
\pgfpathlineto{\pgfqpoint{5.118283in}{1.476309in}}%
\pgfpathlineto{\pgfqpoint{5.122229in}{1.540112in}}%
\pgfpathlineto{\pgfqpoint{5.122792in}{1.538499in}}%
\pgfpathlineto{\pgfqpoint{5.124202in}{1.523467in}}%
\pgfpathlineto{\pgfqpoint{5.128429in}{1.476079in}}%
\pgfpathlineto{\pgfqpoint{5.130402in}{1.456983in}}%
\pgfpathlineto{\pgfqpoint{5.132657in}{1.402370in}}%
\pgfpathlineto{\pgfqpoint{5.135757in}{1.337247in}}%
\pgfpathlineto{\pgfqpoint{5.136321in}{1.340061in}}%
\pgfpathlineto{\pgfqpoint{5.137730in}{1.371825in}}%
\pgfpathlineto{\pgfqpoint{5.140830in}{1.451407in}}%
\pgfpathlineto{\pgfqpoint{5.141112in}{1.450997in}}%
\pgfpathlineto{\pgfqpoint{5.141958in}{1.440022in}}%
\pgfpathlineto{\pgfqpoint{5.146467in}{1.337409in}}%
\pgfpathlineto{\pgfqpoint{5.147313in}{1.341209in}}%
\pgfpathlineto{\pgfqpoint{5.147877in}{1.343553in}}%
\pgfpathlineto{\pgfqpoint{5.148440in}{1.342099in}}%
\pgfpathlineto{\pgfqpoint{5.149568in}{1.318219in}}%
\pgfpathlineto{\pgfqpoint{5.152950in}{1.162824in}}%
\pgfpathlineto{\pgfqpoint{5.153795in}{1.185175in}}%
\pgfpathlineto{\pgfqpoint{5.156332in}{1.371158in}}%
\pgfpathlineto{\pgfqpoint{5.157178in}{1.345709in}}%
\pgfpathlineto{\pgfqpoint{5.159432in}{1.190397in}}%
\pgfpathlineto{\pgfqpoint{5.159714in}{1.201368in}}%
\pgfpathlineto{\pgfqpoint{5.161969in}{1.419610in}}%
\pgfpathlineto{\pgfqpoint{5.162533in}{1.382842in}}%
\pgfpathlineto{\pgfqpoint{5.164787in}{1.096827in}}%
\pgfpathlineto{\pgfqpoint{5.165351in}{1.134648in}}%
\pgfpathlineto{\pgfqpoint{5.167324in}{1.318250in}}%
\pgfpathlineto{\pgfqpoint{5.167606in}{1.309666in}}%
\pgfpathlineto{\pgfqpoint{5.169861in}{1.116663in}}%
\pgfpathlineto{\pgfqpoint{5.170706in}{1.141126in}}%
\pgfpathlineto{\pgfqpoint{5.174370in}{1.379306in}}%
\pgfpathlineto{\pgfqpoint{5.174652in}{1.375748in}}%
\pgfpathlineto{\pgfqpoint{5.177470in}{1.296493in}}%
\pgfpathlineto{\pgfqpoint{5.178598in}{1.307914in}}%
\pgfpathlineto{\pgfqpoint{5.181416in}{1.361090in}}%
\pgfpathlineto{\pgfqpoint{5.181698in}{1.360184in}}%
\pgfpathlineto{\pgfqpoint{5.182826in}{1.341026in}}%
\pgfpathlineto{\pgfqpoint{5.185644in}{1.276168in}}%
\pgfpathlineto{\pgfqpoint{5.185926in}{1.276406in}}%
\pgfpathlineto{\pgfqpoint{5.186771in}{1.287812in}}%
\pgfpathlineto{\pgfqpoint{5.188744in}{1.360238in}}%
\pgfpathlineto{\pgfqpoint{5.192126in}{1.470599in}}%
\pgfpathlineto{\pgfqpoint{5.192690in}{1.473587in}}%
\pgfpathlineto{\pgfqpoint{5.193254in}{1.472134in}}%
\pgfpathlineto{\pgfqpoint{5.194945in}{1.452598in}}%
\pgfpathlineto{\pgfqpoint{5.197200in}{1.435190in}}%
\pgfpathlineto{\pgfqpoint{5.198045in}{1.437130in}}%
\pgfpathlineto{\pgfqpoint{5.200582in}{1.448757in}}%
\pgfpathlineto{\pgfqpoint{5.201146in}{1.447982in}}%
\pgfpathlineto{\pgfqpoint{5.202273in}{1.440339in}}%
\pgfpathlineto{\pgfqpoint{5.205091in}{1.414012in}}%
\pgfpathlineto{\pgfqpoint{5.205655in}{1.415151in}}%
\pgfpathlineto{\pgfqpoint{5.206782in}{1.429071in}}%
\pgfpathlineto{\pgfqpoint{5.209319in}{1.504235in}}%
\pgfpathlineto{\pgfqpoint{5.212419in}{1.572687in}}%
\pgfpathlineto{\pgfqpoint{5.212701in}{1.573120in}}%
\pgfpathlineto{\pgfqpoint{5.212983in}{1.572489in}}%
\pgfpathlineto{\pgfqpoint{5.214111in}{1.560866in}}%
\pgfpathlineto{\pgfqpoint{5.218338in}{1.506429in}}%
\pgfpathlineto{\pgfqpoint{5.219184in}{1.509086in}}%
\pgfpathlineto{\pgfqpoint{5.222284in}{1.525308in}}%
\pgfpathlineto{\pgfqpoint{5.222848in}{1.524568in}}%
\pgfpathlineto{\pgfqpoint{5.225666in}{1.514004in}}%
\pgfpathlineto{\pgfqpoint{5.226230in}{1.516008in}}%
\pgfpathlineto{\pgfqpoint{5.227639in}{1.534662in}}%
\pgfpathlineto{\pgfqpoint{5.232994in}{1.652438in}}%
\pgfpathlineto{\pgfqpoint{5.233840in}{1.645912in}}%
\pgfpathlineto{\pgfqpoint{5.236095in}{1.597130in}}%
\pgfpathlineto{\pgfqpoint{5.238631in}{1.562412in}}%
\pgfpathlineto{\pgfqpoint{5.239195in}{1.563759in}}%
\pgfpathlineto{\pgfqpoint{5.240604in}{1.577845in}}%
\pgfpathlineto{\pgfqpoint{5.243704in}{1.607395in}}%
\pgfpathlineto{\pgfqpoint{5.244268in}{1.607584in}}%
\pgfpathlineto{\pgfqpoint{5.244550in}{1.607214in}}%
\pgfpathlineto{\pgfqpoint{5.245959in}{1.604708in}}%
\pgfpathlineto{\pgfqpoint{5.246523in}{1.605509in}}%
\pgfpathlineto{\pgfqpoint{5.247650in}{1.614408in}}%
\pgfpathlineto{\pgfqpoint{5.249623in}{1.656299in}}%
\pgfpathlineto{\pgfqpoint{5.253005in}{1.722714in}}%
\pgfpathlineto{\pgfqpoint{5.253569in}{1.720865in}}%
\pgfpathlineto{\pgfqpoint{5.254978in}{1.696987in}}%
\pgfpathlineto{\pgfqpoint{5.258924in}{1.618021in}}%
\pgfpathlineto{\pgfqpoint{5.259488in}{1.620191in}}%
\pgfpathlineto{\pgfqpoint{5.260897in}{1.640221in}}%
\pgfpathlineto{\pgfqpoint{5.264561in}{1.694676in}}%
\pgfpathlineto{\pgfqpoint{5.265407in}{1.695738in}}%
\pgfpathlineto{\pgfqpoint{5.265688in}{1.695498in}}%
\pgfpathlineto{\pgfqpoint{5.266816in}{1.694627in}}%
\pgfpathlineto{\pgfqpoint{5.267098in}{1.695022in}}%
\pgfpathlineto{\pgfqpoint{5.268225in}{1.701874in}}%
\pgfpathlineto{\pgfqpoint{5.270198in}{1.736740in}}%
\pgfpathlineto{\pgfqpoint{5.273298in}{1.784907in}}%
\pgfpathlineto{\pgfqpoint{5.273862in}{1.782231in}}%
\pgfpathlineto{\pgfqpoint{5.275271in}{1.756931in}}%
\pgfpathlineto{\pgfqpoint{5.279217in}{1.676859in}}%
\pgfpathlineto{\pgfqpoint{5.280063in}{1.683250in}}%
\pgfpathlineto{\pgfqpoint{5.282317in}{1.733517in}}%
\pgfpathlineto{\pgfqpoint{5.285136in}{1.782004in}}%
\pgfpathlineto{\pgfqpoint{5.285981in}{1.784264in}}%
\pgfpathlineto{\pgfqpoint{5.286545in}{1.783766in}}%
\pgfpathlineto{\pgfqpoint{5.287954in}{1.782075in}}%
\pgfpathlineto{\pgfqpoint{5.288236in}{1.782525in}}%
\pgfpathlineto{\pgfqpoint{5.289363in}{1.789244in}}%
\pgfpathlineto{\pgfqpoint{5.293591in}{1.838553in}}%
\pgfpathlineto{\pgfqpoint{5.294437in}{1.833322in}}%
\pgfpathlineto{\pgfqpoint{5.296128in}{1.799634in}}%
\pgfpathlineto{\pgfqpoint{5.299228in}{1.739816in}}%
\pgfpathlineto{\pgfqpoint{5.299510in}{1.739762in}}%
\pgfpathlineto{\pgfqpoint{5.300355in}{1.746659in}}%
\pgfpathlineto{\pgfqpoint{5.302328in}{1.793786in}}%
\pgfpathlineto{\pgfqpoint{5.305711in}{1.868285in}}%
\pgfpathlineto{\pgfqpoint{5.306556in}{1.871446in}}%
\pgfpathlineto{\pgfqpoint{5.307120in}{1.870647in}}%
\pgfpathlineto{\pgfqpoint{5.309656in}{1.862452in}}%
\pgfpathlineto{\pgfqpoint{5.310220in}{1.863755in}}%
\pgfpathlineto{\pgfqpoint{5.311911in}{1.875982in}}%
\pgfpathlineto{\pgfqpoint{5.313884in}{1.887932in}}%
\pgfpathlineto{\pgfqpoint{5.314166in}{1.887663in}}%
\pgfpathlineto{\pgfqpoint{5.315012in}{1.882500in}}%
\pgfpathlineto{\pgfqpoint{5.316984in}{1.848048in}}%
\pgfpathlineto{\pgfqpoint{5.319521in}{1.809905in}}%
\pgfpathlineto{\pgfqpoint{5.319803in}{1.809998in}}%
\pgfpathlineto{\pgfqpoint{5.320648in}{1.817213in}}%
\pgfpathlineto{\pgfqpoint{5.322621in}{1.867698in}}%
\pgfpathlineto{\pgfqpoint{5.326567in}{1.964552in}}%
\pgfpathlineto{\pgfqpoint{5.327131in}{1.965407in}}%
\pgfpathlineto{\pgfqpoint{5.327413in}{1.964569in}}%
\pgfpathlineto{\pgfqpoint{5.328822in}{1.952079in}}%
\pgfpathlineto{\pgfqpoint{5.331359in}{1.933897in}}%
\pgfpathlineto{\pgfqpoint{5.332204in}{1.936007in}}%
\pgfpathlineto{\pgfqpoint{5.334741in}{1.947926in}}%
\pgfpathlineto{\pgfqpoint{5.335304in}{1.946759in}}%
\pgfpathlineto{\pgfqpoint{5.336432in}{1.936829in}}%
\pgfpathlineto{\pgfqpoint{5.339532in}{1.897476in}}%
\pgfpathlineto{\pgfqpoint{5.340096in}{1.899887in}}%
\pgfpathlineto{\pgfqpoint{5.341223in}{1.920796in}}%
\pgfpathlineto{\pgfqpoint{5.344324in}{2.046892in}}%
\pgfpathlineto{\pgfqpoint{5.346296in}{2.092078in}}%
\pgfpathlineto{\pgfqpoint{5.346578in}{2.091785in}}%
\pgfpathlineto{\pgfqpoint{5.347424in}{2.080517in}}%
\pgfpathlineto{\pgfqpoint{5.350806in}{2.007817in}}%
\pgfpathlineto{\pgfqpoint{5.351370in}{2.012355in}}%
\pgfpathlineto{\pgfqpoint{5.354470in}{2.067544in}}%
\pgfpathlineto{\pgfqpoint{5.355034in}{2.060917in}}%
\pgfpathlineto{\pgfqpoint{5.357007in}{1.987709in}}%
\pgfpathlineto{\pgfqpoint{5.357852in}{1.971394in}}%
\pgfpathlineto{\pgfqpoint{5.358134in}{1.973310in}}%
\pgfpathlineto{\pgfqpoint{5.358980in}{2.004757in}}%
\pgfpathlineto{\pgfqpoint{5.362080in}{2.227049in}}%
\pgfpathlineto{\pgfqpoint{5.362644in}{2.209337in}}%
\pgfpathlineto{\pgfqpoint{5.364898in}{2.043003in}}%
\pgfpathlineto{\pgfqpoint{5.365462in}{2.072324in}}%
\pgfpathlineto{\pgfqpoint{5.367435in}{2.254111in}}%
\pgfpathlineto{\pgfqpoint{5.367999in}{2.220114in}}%
\pgfpathlineto{\pgfqpoint{5.370253in}{1.947867in}}%
\pgfpathlineto{\pgfqpoint{5.370817in}{1.974594in}}%
\pgfpathlineto{\pgfqpoint{5.372790in}{2.157587in}}%
\pgfpathlineto{\pgfqpoint{5.373354in}{2.141387in}}%
\pgfpathlineto{\pgfqpoint{5.375608in}{1.982853in}}%
\pgfpathlineto{\pgfqpoint{5.376172in}{1.991958in}}%
\pgfpathlineto{\pgfqpoint{5.379272in}{2.190602in}}%
\pgfpathlineto{\pgfqpoint{5.380118in}{2.168088in}}%
\pgfpathlineto{\pgfqpoint{5.383782in}{2.033316in}}%
\pgfpathlineto{\pgfqpoint{5.384628in}{2.037807in}}%
\pgfpathlineto{\pgfqpoint{5.385755in}{2.044226in}}%
\pgfpathlineto{\pgfqpoint{5.386037in}{2.043491in}}%
\pgfpathlineto{\pgfqpoint{5.386882in}{2.032721in}}%
\pgfpathlineto{\pgfqpoint{5.389137in}{1.958723in}}%
\pgfpathlineto{\pgfqpoint{5.391110in}{1.917027in}}%
\pgfpathlineto{\pgfqpoint{5.391392in}{1.917574in}}%
\pgfpathlineto{\pgfqpoint{5.392237in}{1.930113in}}%
\pgfpathlineto{\pgfqpoint{5.396465in}{2.039631in}}%
\pgfpathlineto{\pgfqpoint{5.397029in}{2.035454in}}%
\pgfpathlineto{\pgfqpoint{5.398720in}{1.998828in}}%
\pgfpathlineto{\pgfqpoint{5.402384in}{1.927254in}}%
\pgfpathlineto{\pgfqpoint{5.406048in}{1.895686in}}%
\pgfpathlineto{\pgfqpoint{5.409994in}{1.841291in}}%
\pgfpathlineto{\pgfqpoint{5.410276in}{1.842199in}}%
\pgfpathlineto{\pgfqpoint{5.411403in}{1.855614in}}%
\pgfpathlineto{\pgfqpoint{5.415913in}{1.937875in}}%
\pgfpathlineto{\pgfqpoint{5.416476in}{1.934864in}}%
\pgfpathlineto{\pgfqpoint{5.418167in}{1.905977in}}%
\pgfpathlineto{\pgfqpoint{5.422677in}{1.821302in}}%
\pgfpathlineto{\pgfqpoint{5.429441in}{1.776365in}}%
\pgfpathlineto{\pgfqpoint{5.430005in}{1.777556in}}%
\pgfpathlineto{\pgfqpoint{5.431414in}{1.791018in}}%
\pgfpathlineto{\pgfqpoint{5.435642in}{1.850968in}}%
\pgfpathlineto{\pgfqpoint{5.436205in}{1.849715in}}%
\pgfpathlineto{\pgfqpoint{5.437333in}{1.837409in}}%
\pgfpathlineto{\pgfqpoint{5.440715in}{1.760035in}}%
\pgfpathlineto{\pgfqpoint{5.443815in}{1.717045in}}%
\pgfpathlineto{\pgfqpoint{5.446070in}{1.710987in}}%
\pgfpathlineto{\pgfqpoint{5.448607in}{1.710010in}}%
\pgfpathlineto{\pgfqpoint{5.449734in}{1.712346in}}%
\pgfpathlineto{\pgfqpoint{5.451425in}{1.723082in}}%
\pgfpathlineto{\pgfqpoint{5.455653in}{1.759233in}}%
\pgfpathlineto{\pgfqpoint{5.455935in}{1.758851in}}%
\pgfpathlineto{\pgfqpoint{5.457062in}{1.750825in}}%
\pgfpathlineto{\pgfqpoint{5.459035in}{1.714054in}}%
\pgfpathlineto{\pgfqpoint{5.464108in}{1.616090in}}%
\pgfpathlineto{\pgfqpoint{5.464954in}{1.614143in}}%
\pgfpathlineto{\pgfqpoint{5.465236in}{1.614442in}}%
\pgfpathlineto{\pgfqpoint{5.466363in}{1.619179in}}%
\pgfpathlineto{\pgfqpoint{5.472845in}{1.654206in}}%
\pgfpathlineto{\pgfqpoint{5.475100in}{1.658802in}}%
\pgfpathlineto{\pgfqpoint{5.475382in}{1.658621in}}%
\pgfpathlineto{\pgfqpoint{5.476228in}{1.656139in}}%
\pgfpathlineto{\pgfqpoint{5.477637in}{1.643385in}}%
\pgfpathlineto{\pgfqpoint{5.480173in}{1.590990in}}%
\pgfpathlineto{\pgfqpoint{5.484119in}{1.511968in}}%
\pgfpathlineto{\pgfqpoint{5.484683in}{1.510124in}}%
\pgfpathlineto{\pgfqpoint{5.485247in}{1.511287in}}%
\pgfpathlineto{\pgfqpoint{5.486656in}{1.524894in}}%
\pgfpathlineto{\pgfqpoint{5.490602in}{1.568056in}}%
\pgfpathlineto{\pgfqpoint{5.491165in}{1.568248in}}%
\pgfpathlineto{\pgfqpoint{5.491447in}{1.567764in}}%
\pgfpathlineto{\pgfqpoint{5.493138in}{1.559728in}}%
\pgfpathlineto{\pgfqpoint{5.498212in}{1.523135in}}%
\pgfpathlineto{\pgfqpoint{5.500748in}{1.465758in}}%
\pgfpathlineto{\pgfqpoint{5.504412in}{1.391877in}}%
\pgfpathlineto{\pgfqpoint{5.504976in}{1.393031in}}%
\pgfpathlineto{\pgfqpoint{5.506103in}{1.407369in}}%
\pgfpathlineto{\pgfqpoint{5.510049in}{1.473555in}}%
\pgfpathlineto{\pgfqpoint{5.510331in}{1.472992in}}%
\pgfpathlineto{\pgfqpoint{5.511458in}{1.462195in}}%
\pgfpathlineto{\pgfqpoint{5.517377in}{1.387234in}}%
\pgfpathlineto{\pgfqpoint{5.519068in}{1.353309in}}%
\pgfpathlineto{\pgfqpoint{5.523296in}{1.222534in}}%
\pgfpathlineto{\pgfqpoint{5.523860in}{1.227710in}}%
\pgfpathlineto{\pgfqpoint{5.525269in}{1.278826in}}%
\pgfpathlineto{\pgfqpoint{5.527524in}{1.358126in}}%
\pgfpathlineto{\pgfqpoint{5.527806in}{1.357009in}}%
\pgfpathlineto{\pgfqpoint{5.528933in}{1.323614in}}%
\pgfpathlineto{\pgfqpoint{5.531470in}{1.220675in}}%
\pgfpathlineto{\pgfqpoint{5.531751in}{1.222996in}}%
\pgfpathlineto{\pgfqpoint{5.532879in}{1.269375in}}%
\pgfpathlineto{\pgfqpoint{5.534006in}{1.316133in}}%
\pgfpathlineto{\pgfqpoint{5.534570in}{1.308397in}}%
\pgfpathlineto{\pgfqpoint{5.535697in}{1.210079in}}%
\pgfpathlineto{\pgfqpoint{5.537388in}{1.036353in}}%
\pgfpathlineto{\pgfqpoint{5.537952in}{1.057610in}}%
\pgfpathlineto{\pgfqpoint{5.539925in}{1.297638in}}%
\pgfpathlineto{\pgfqpoint{5.540770in}{1.246984in}}%
\pgfpathlineto{\pgfqpoint{5.542462in}{1.078523in}}%
\pgfpathlineto{\pgfqpoint{5.542743in}{1.089741in}}%
\pgfpathlineto{\pgfqpoint{5.545562in}{1.376222in}}%
\pgfpathlineto{\pgfqpoint{5.546407in}{1.333956in}}%
\pgfpathlineto{\pgfqpoint{5.548662in}{1.189533in}}%
\pgfpathlineto{\pgfqpoint{5.548944in}{1.193779in}}%
\pgfpathlineto{\pgfqpoint{5.552044in}{1.304302in}}%
\pgfpathlineto{\pgfqpoint{5.552890in}{1.287396in}}%
\pgfpathlineto{\pgfqpoint{5.555708in}{1.197041in}}%
\pgfpathlineto{\pgfqpoint{5.555990in}{1.199214in}}%
\pgfpathlineto{\pgfqpoint{5.557118in}{1.233105in}}%
\pgfpathlineto{\pgfqpoint{5.562191in}{1.436577in}}%
\pgfpathlineto{\pgfqpoint{5.562473in}{1.435897in}}%
\pgfpathlineto{\pgfqpoint{5.563882in}{1.419507in}}%
\pgfpathlineto{\pgfqpoint{5.565855in}{1.400529in}}%
\pgfpathlineto{\pgfqpoint{5.566137in}{1.400637in}}%
\pgfpathlineto{\pgfqpoint{5.566982in}{1.405200in}}%
\pgfpathlineto{\pgfqpoint{5.570082in}{1.432721in}}%
\pgfpathlineto{\pgfqpoint{5.570646in}{1.431816in}}%
\pgfpathlineto{\pgfqpoint{5.572055in}{1.418987in}}%
\pgfpathlineto{\pgfqpoint{5.574028in}{1.400269in}}%
\pgfpathlineto{\pgfqpoint{5.574310in}{1.400302in}}%
\pgfpathlineto{\pgfqpoint{5.575156in}{1.406995in}}%
\pgfpathlineto{\pgfqpoint{5.576847in}{1.450197in}}%
\pgfpathlineto{\pgfqpoint{5.581638in}{1.590063in}}%
\pgfpathlineto{\pgfqpoint{5.581920in}{1.590042in}}%
\pgfpathlineto{\pgfqpoint{5.582766in}{1.583695in}}%
\pgfpathlineto{\pgfqpoint{5.586711in}{1.533590in}}%
\pgfpathlineto{\pgfqpoint{5.587275in}{1.535054in}}%
\pgfpathlineto{\pgfqpoint{5.588966in}{1.550581in}}%
\pgfpathlineto{\pgfqpoint{5.591503in}{1.567996in}}%
\pgfpathlineto{\pgfqpoint{5.592348in}{1.566277in}}%
\pgfpathlineto{\pgfqpoint{5.594039in}{1.560263in}}%
\pgfpathlineto{\pgfqpoint{5.594603in}{1.561105in}}%
\pgfpathlineto{\pgfqpoint{5.595730in}{1.572231in}}%
\pgfpathlineto{\pgfqpoint{5.597703in}{1.624917in}}%
\pgfpathlineto{\pgfqpoint{5.601649in}{1.729756in}}%
\pgfpathlineto{\pgfqpoint{5.601931in}{1.730014in}}%
\pgfpathlineto{\pgfqpoint{5.601931in}{1.730014in}}%
\pgfpathlineto{\pgfqpoint{5.601931in}{1.730014in}}%
\pgfpathlineto{\pgfqpoint{5.602777in}{1.723448in}}%
\pgfpathlineto{\pgfqpoint{5.607004in}{1.649529in}}%
\pgfpathlineto{\pgfqpoint{5.608132in}{1.655772in}}%
\pgfpathlineto{\pgfqpoint{5.613205in}{1.719456in}}%
\pgfpathlineto{\pgfqpoint{5.614050in}{1.718900in}}%
\pgfpathlineto{\pgfqpoint{5.614896in}{1.719284in}}%
\pgfpathlineto{\pgfqpoint{5.615742in}{1.724211in}}%
\pgfpathlineto{\pgfqpoint{5.617151in}{1.748587in}}%
\pgfpathlineto{\pgfqpoint{5.621942in}{1.871408in}}%
\pgfpathlineto{\pgfqpoint{5.622506in}{1.869173in}}%
\pgfpathlineto{\pgfqpoint{5.623915in}{1.841983in}}%
\pgfpathlineto{\pgfqpoint{5.627297in}{1.771957in}}%
\pgfpathlineto{\pgfqpoint{5.627861in}{1.775047in}}%
\pgfpathlineto{\pgfqpoint{5.629270in}{1.802211in}}%
\pgfpathlineto{\pgfqpoint{5.633498in}{1.889791in}}%
\pgfpathlineto{\pgfqpoint{5.634062in}{1.890531in}}%
\pgfpathlineto{\pgfqpoint{5.634625in}{1.889987in}}%
\pgfpathlineto{\pgfqpoint{5.635189in}{1.889560in}}%
\pgfpathlineto{\pgfqpoint{5.635471in}{1.889843in}}%
\pgfpathlineto{\pgfqpoint{5.636316in}{1.894574in}}%
\pgfpathlineto{\pgfqpoint{5.637726in}{1.920519in}}%
\pgfpathlineto{\pgfqpoint{5.641953in}{2.030009in}}%
\pgfpathlineto{\pgfqpoint{5.642517in}{2.026566in}}%
\pgfpathlineto{\pgfqpoint{5.643926in}{1.990597in}}%
\pgfpathlineto{\pgfqpoint{5.646745in}{1.912998in}}%
\pgfpathlineto{\pgfqpoint{5.647027in}{1.913557in}}%
\pgfpathlineto{\pgfqpoint{5.647872in}{1.927423in}}%
\pgfpathlineto{\pgfqpoint{5.650127in}{2.024068in}}%
\pgfpathlineto{\pgfqpoint{5.652663in}{2.103795in}}%
\pgfpathlineto{\pgfqpoint{5.652945in}{2.104637in}}%
\pgfpathlineto{\pgfqpoint{5.653227in}{2.103936in}}%
\pgfpathlineto{\pgfqpoint{5.655200in}{2.084440in}}%
\pgfpathlineto{\pgfqpoint{5.656046in}{2.091074in}}%
\pgfpathlineto{\pgfqpoint{5.657455in}{2.144433in}}%
\pgfpathlineto{\pgfqpoint{5.659991in}{2.259321in}}%
\pgfpathlineto{\pgfqpoint{5.660273in}{2.257379in}}%
\pgfpathlineto{\pgfqpoint{5.661119in}{2.218041in}}%
\pgfpathlineto{\pgfqpoint{5.663655in}{2.028109in}}%
\pgfpathlineto{\pgfqpoint{5.664219in}{2.050373in}}%
\pgfpathlineto{\pgfqpoint{5.666756in}{2.300829in}}%
\pgfpathlineto{\pgfqpoint{5.667601in}{2.243305in}}%
\pgfpathlineto{\pgfqpoint{5.669292in}{2.032886in}}%
\pgfpathlineto{\pgfqpoint{5.669574in}{2.042543in}}%
\pgfpathlineto{\pgfqpoint{5.670702in}{2.234422in}}%
\pgfpathlineto{\pgfqpoint{5.672111in}{2.417174in}}%
\pgfpathlineto{\pgfqpoint{5.672393in}{2.396546in}}%
\pgfpathlineto{\pgfqpoint{5.674647in}{2.091647in}}%
\pgfpathlineto{\pgfqpoint{5.675493in}{2.134053in}}%
\pgfpathlineto{\pgfqpoint{5.677466in}{2.295879in}}%
\pgfpathlineto{\pgfqpoint{5.677748in}{2.288993in}}%
\pgfpathlineto{\pgfqpoint{5.679157in}{2.169549in}}%
\pgfpathlineto{\pgfqpoint{5.681694in}{2.004130in}}%
\pgfpathlineto{\pgfqpoint{5.682539in}{2.028469in}}%
\pgfpathlineto{\pgfqpoint{5.685358in}{2.159425in}}%
\pgfpathlineto{\pgfqpoint{5.685921in}{2.156143in}}%
\pgfpathlineto{\pgfqpoint{5.687331in}{2.111425in}}%
\pgfpathlineto{\pgfqpoint{5.690149in}{2.030995in}}%
\pgfpathlineto{\pgfqpoint{5.690431in}{2.030992in}}%
\pgfpathlineto{\pgfqpoint{5.691276in}{2.038952in}}%
\pgfpathlineto{\pgfqpoint{5.693249in}{2.062545in}}%
\pgfpathlineto{\pgfqpoint{5.693531in}{2.061693in}}%
\pgfpathlineto{\pgfqpoint{5.694377in}{2.048804in}}%
\pgfpathlineto{\pgfqpoint{5.696350in}{1.968319in}}%
\pgfpathlineto{\pgfqpoint{5.699732in}{1.852177in}}%
\pgfpathlineto{\pgfqpoint{5.700295in}{1.855479in}}%
\pgfpathlineto{\pgfqpoint{5.701705in}{1.890952in}}%
\pgfpathlineto{\pgfqpoint{5.704523in}{1.958372in}}%
\pgfpathlineto{\pgfqpoint{5.705087in}{1.956127in}}%
\pgfpathlineto{\pgfqpoint{5.706496in}{1.928321in}}%
\pgfpathlineto{\pgfqpoint{5.710442in}{1.847487in}}%
\pgfpathlineto{\pgfqpoint{5.711006in}{1.848889in}}%
\pgfpathlineto{\pgfqpoint{5.712979in}{1.859896in}}%
\pgfpathlineto{\pgfqpoint{5.713542in}{1.858271in}}%
\pgfpathlineto{\pgfqpoint{5.714670in}{1.841943in}}%
\pgfpathlineto{\pgfqpoint{5.717488in}{1.747019in}}%
\pgfpathlineto{\pgfqpoint{5.719743in}{1.705578in}}%
\pgfpathlineto{\pgfqpoint{5.720588in}{1.714447in}}%
\pgfpathlineto{\pgfqpoint{5.723125in}{1.788949in}}%
\pgfpathlineto{\pgfqpoint{5.724816in}{1.813431in}}%
\pgfpathlineto{\pgfqpoint{5.725098in}{1.812384in}}%
\pgfpathlineto{\pgfqpoint{5.726225in}{1.794289in}}%
\pgfpathlineto{\pgfqpoint{5.731299in}{1.680781in}}%
\pgfpathlineto{\pgfqpoint{5.731580in}{1.681464in}}%
\pgfpathlineto{\pgfqpoint{5.733553in}{1.690798in}}%
\pgfpathlineto{\pgfqpoint{5.734117in}{1.689478in}}%
\pgfpathlineto{\pgfqpoint{5.735244in}{1.675481in}}%
\pgfpathlineto{\pgfqpoint{5.739754in}{1.571812in}}%
\pgfpathlineto{\pgfqpoint{5.740600in}{1.577693in}}%
\pgfpathlineto{\pgfqpoint{5.742291in}{1.621789in}}%
\pgfpathlineto{\pgfqpoint{5.744827in}{1.678521in}}%
\pgfpathlineto{\pgfqpoint{5.745109in}{1.678281in}}%
\pgfpathlineto{\pgfqpoint{5.745955in}{1.667991in}}%
\pgfpathlineto{\pgfqpoint{5.748209in}{1.592828in}}%
\pgfpathlineto{\pgfqpoint{5.751028in}{1.519192in}}%
\pgfpathlineto{\pgfqpoint{5.751592in}{1.517289in}}%
\pgfpathlineto{\pgfqpoint{5.752155in}{1.519034in}}%
\pgfpathlineto{\pgfqpoint{5.754410in}{1.532488in}}%
\pgfpathlineto{\pgfqpoint{5.754692in}{1.531752in}}%
\pgfpathlineto{\pgfqpoint{5.755819in}{1.518988in}}%
\pgfpathlineto{\pgfqpoint{5.759765in}{1.437254in}}%
\pgfpathlineto{\pgfqpoint{5.760611in}{1.442239in}}%
\pgfpathlineto{\pgfqpoint{5.762302in}{1.486784in}}%
\pgfpathlineto{\pgfqpoint{5.764838in}{1.541064in}}%
\pgfpathlineto{\pgfqpoint{5.765684in}{1.529812in}}%
\pgfpathlineto{\pgfqpoint{5.767657in}{1.450781in}}%
\pgfpathlineto{\pgfqpoint{5.771039in}{1.335369in}}%
\pgfpathlineto{\pgfqpoint{5.771321in}{1.335219in}}%
\pgfpathlineto{\pgfqpoint{5.772166in}{1.342907in}}%
\pgfpathlineto{\pgfqpoint{5.774421in}{1.369708in}}%
\pgfpathlineto{\pgfqpoint{5.774703in}{1.368273in}}%
\pgfpathlineto{\pgfqpoint{5.775830in}{1.346947in}}%
\pgfpathlineto{\pgfqpoint{5.778931in}{1.260678in}}%
\pgfpathlineto{\pgfqpoint{5.779494in}{1.265689in}}%
\pgfpathlineto{\pgfqpoint{5.780904in}{1.323567in}}%
\pgfpathlineto{\pgfqpoint{5.782876in}{1.403763in}}%
\pgfpathlineto{\pgfqpoint{5.783158in}{1.402211in}}%
\pgfpathlineto{\pgfqpoint{5.784004in}{1.371258in}}%
\pgfpathlineto{\pgfqpoint{5.787386in}{1.126580in}}%
\pgfpathlineto{\pgfqpoint{5.788232in}{1.159609in}}%
\pgfpathlineto{\pgfqpoint{5.790204in}{1.294694in}}%
\pgfpathlineto{\pgfqpoint{5.790486in}{1.285494in}}%
\pgfpathlineto{\pgfqpoint{5.791896in}{1.113847in}}%
\pgfpathlineto{\pgfqpoint{5.793023in}{1.010661in}}%
\pgfpathlineto{\pgfqpoint{5.793305in}{1.023960in}}%
\pgfpathlineto{\pgfqpoint{5.795560in}{1.379681in}}%
\pgfpathlineto{\pgfqpoint{5.796405in}{1.314314in}}%
\pgfpathlineto{\pgfqpoint{5.798096in}{1.100471in}}%
\pgfpathlineto{\pgfqpoint{5.798378in}{1.110192in}}%
\pgfpathlineto{\pgfqpoint{5.800915in}{1.379846in}}%
\pgfpathlineto{\pgfqpoint{5.801760in}{1.335094in}}%
\pgfpathlineto{\pgfqpoint{5.804297in}{1.111549in}}%
\pgfpathlineto{\pgfqpoint{5.804579in}{1.117434in}}%
\pgfpathlineto{\pgfqpoint{5.806552in}{1.236413in}}%
\pgfpathlineto{\pgfqpoint{5.808243in}{1.284353in}}%
\pgfpathlineto{\pgfqpoint{5.809088in}{1.270593in}}%
\pgfpathlineto{\pgfqpoint{5.811061in}{1.235185in}}%
\pgfpathlineto{\pgfqpoint{5.811343in}{1.237714in}}%
\pgfpathlineto{\pgfqpoint{5.812470in}{1.272153in}}%
\pgfpathlineto{\pgfqpoint{5.816698in}{1.451665in}}%
\pgfpathlineto{\pgfqpoint{5.816980in}{1.450080in}}%
\pgfpathlineto{\pgfqpoint{5.818107in}{1.422265in}}%
\pgfpathlineto{\pgfqpoint{5.821489in}{1.326655in}}%
\pgfpathlineto{\pgfqpoint{5.822335in}{1.336329in}}%
\pgfpathlineto{\pgfqpoint{5.827126in}{1.439962in}}%
\pgfpathlineto{\pgfqpoint{5.827972in}{1.437404in}}%
\pgfpathlineto{\pgfqpoint{5.829381in}{1.432360in}}%
\pgfpathlineto{\pgfqpoint{5.829663in}{1.433078in}}%
\pgfpathlineto{\pgfqpoint{5.830790in}{1.446178in}}%
\pgfpathlineto{\pgfqpoint{5.833045in}{1.515723in}}%
\pgfpathlineto{\pgfqpoint{5.835864in}{1.580012in}}%
\pgfpathlineto{\pgfqpoint{5.836427in}{1.576884in}}%
\pgfpathlineto{\pgfqpoint{5.837837in}{1.545242in}}%
\pgfpathlineto{\pgfqpoint{5.841219in}{1.469925in}}%
\pgfpathlineto{\pgfqpoint{5.842064in}{1.477767in}}%
\pgfpathlineto{\pgfqpoint{5.844319in}{1.537317in}}%
\pgfpathlineto{\pgfqpoint{5.847137in}{1.590708in}}%
\pgfpathlineto{\pgfqpoint{5.847701in}{1.592010in}}%
\pgfpathlineto{\pgfqpoint{5.848265in}{1.591333in}}%
\pgfpathlineto{\pgfqpoint{5.849392in}{1.589015in}}%
\pgfpathlineto{\pgfqpoint{5.849956in}{1.589892in}}%
\pgfpathlineto{\pgfqpoint{5.851083in}{1.600167in}}%
\pgfpathlineto{\pgfqpoint{5.853338in}{1.652517in}}%
\pgfpathlineto{\pgfqpoint{5.855875in}{1.695421in}}%
\pgfpathlineto{\pgfqpoint{5.856438in}{1.693149in}}%
\pgfpathlineto{\pgfqpoint{5.857848in}{1.667052in}}%
\pgfpathlineto{\pgfqpoint{5.861230in}{1.595814in}}%
\pgfpathlineto{\pgfqpoint{5.861512in}{1.596423in}}%
\pgfpathlineto{\pgfqpoint{5.862639in}{1.611986in}}%
\pgfpathlineto{\pgfqpoint{5.868558in}{1.742639in}}%
\pgfpathlineto{\pgfqpoint{5.869121in}{1.741852in}}%
\pgfpathlineto{\pgfqpoint{5.870249in}{1.739671in}}%
\pgfpathlineto{\pgfqpoint{5.870813in}{1.740399in}}%
\pgfpathlineto{\pgfqpoint{5.871940in}{1.748811in}}%
\pgfpathlineto{\pgfqpoint{5.876168in}{1.809864in}}%
\pgfpathlineto{\pgfqpoint{5.877013in}{1.805914in}}%
\pgfpathlineto{\pgfqpoint{5.878704in}{1.774320in}}%
\pgfpathlineto{\pgfqpoint{5.881523in}{1.728703in}}%
\pgfpathlineto{\pgfqpoint{5.882368in}{1.735827in}}%
\pgfpathlineto{\pgfqpoint{5.884059in}{1.780872in}}%
\pgfpathlineto{\pgfqpoint{5.888287in}{1.900070in}}%
\pgfpathlineto{\pgfqpoint{5.889133in}{1.902428in}}%
\pgfpathlineto{\pgfqpoint{5.889414in}{1.901797in}}%
\pgfpathlineto{\pgfqpoint{5.891951in}{1.890990in}}%
\pgfpathlineto{\pgfqpoint{5.892515in}{1.892558in}}%
\pgfpathlineto{\pgfqpoint{5.893924in}{1.906009in}}%
\pgfpathlineto{\pgfqpoint{5.896742in}{1.935632in}}%
\pgfpathlineto{\pgfqpoint{5.897024in}{1.935561in}}%
\pgfpathlineto{\pgfqpoint{5.897870in}{1.930300in}}%
\pgfpathlineto{\pgfqpoint{5.901534in}{1.880891in}}%
\pgfpathlineto{\pgfqpoint{5.902379in}{1.888542in}}%
\pgfpathlineto{\pgfqpoint{5.904070in}{1.940036in}}%
\pgfpathlineto{\pgfqpoint{5.908580in}{2.098591in}}%
\pgfpathlineto{\pgfqpoint{5.908862in}{2.098947in}}%
\pgfpathlineto{\pgfqpoint{5.908862in}{2.098947in}}%
\pgfpathlineto{\pgfqpoint{5.908862in}{2.098947in}}%
\pgfpathlineto{\pgfqpoint{5.909707in}{2.092129in}}%
\pgfpathlineto{\pgfqpoint{5.912526in}{2.054674in}}%
\pgfpathlineto{\pgfqpoint{5.912808in}{2.055634in}}%
\pgfpathlineto{\pgfqpoint{5.913935in}{2.071553in}}%
\pgfpathlineto{\pgfqpoint{5.916753in}{2.128259in}}%
\pgfpathlineto{\pgfqpoint{5.917317in}{2.126082in}}%
\pgfpathlineto{\pgfqpoint{5.918726in}{2.091550in}}%
\pgfpathlineto{\pgfqpoint{5.920417in}{2.056425in}}%
\pgfpathlineto{\pgfqpoint{5.920699in}{2.059610in}}%
\pgfpathlineto{\pgfqpoint{5.921827in}{2.109553in}}%
\pgfpathlineto{\pgfqpoint{5.924927in}{2.333185in}}%
\pgfpathlineto{\pgfqpoint{5.925491in}{2.310356in}}%
\pgfpathlineto{\pgfqpoint{5.927464in}{2.135702in}}%
\pgfpathlineto{\pgfqpoint{5.928027in}{2.152984in}}%
\pgfpathlineto{\pgfqpoint{5.930000in}{2.374190in}}%
\pgfpathlineto{\pgfqpoint{5.930846in}{2.317308in}}%
\pgfpathlineto{\pgfqpoint{5.932819in}{2.043314in}}%
\pgfpathlineto{\pgfqpoint{5.933382in}{2.064938in}}%
\pgfpathlineto{\pgfqpoint{5.935355in}{2.264603in}}%
\pgfpathlineto{\pgfqpoint{5.935919in}{2.254518in}}%
\pgfpathlineto{\pgfqpoint{5.938456in}{2.088407in}}%
\pgfpathlineto{\pgfqpoint{5.939301in}{2.107542in}}%
\pgfpathlineto{\pgfqpoint{5.942120in}{2.263102in}}%
\pgfpathlineto{\pgfqpoint{5.942683in}{2.256973in}}%
\pgfpathlineto{\pgfqpoint{5.944374in}{2.179293in}}%
\pgfpathlineto{\pgfqpoint{5.947475in}{2.056062in}}%
\pgfpathlineto{\pgfqpoint{5.955085in}{1.948023in}}%
\pgfpathlineto{\pgfqpoint{5.955930in}{1.952574in}}%
\pgfpathlineto{\pgfqpoint{5.957903in}{1.990827in}}%
\pgfpathlineto{\pgfqpoint{5.959876in}{2.018968in}}%
\pgfpathlineto{\pgfqpoint{5.960158in}{2.018903in}}%
\pgfpathlineto{\pgfqpoint{5.961003in}{2.011361in}}%
\pgfpathlineto{\pgfqpoint{5.962976in}{1.959179in}}%
\pgfpathlineto{\pgfqpoint{5.967486in}{1.836223in}}%
\pgfpathlineto{\pgfqpoint{5.970586in}{1.813074in}}%
\pgfpathlineto{\pgfqpoint{5.974532in}{1.792750in}}%
\pgfpathlineto{\pgfqpoint{5.975378in}{1.795125in}}%
\pgfpathlineto{\pgfqpoint{5.977350in}{1.812149in}}%
\pgfpathlineto{\pgfqpoint{5.979323in}{1.826052in}}%
\pgfpathlineto{\pgfqpoint{5.979605in}{1.825989in}}%
\pgfpathlineto{\pgfqpoint{5.980451in}{1.821262in}}%
\pgfpathlineto{\pgfqpoint{5.982142in}{1.790465in}}%
\pgfpathlineto{\pgfqpoint{5.989188in}{1.629849in}}%
\pgfpathlineto{\pgfqpoint{5.989752in}{1.629710in}}%
\pgfpathlineto{\pgfqpoint{5.990034in}{1.630122in}}%
\pgfpathlineto{\pgfqpoint{5.994261in}{1.638667in}}%
\pgfpathlineto{\pgfqpoint{5.994543in}{1.638489in}}%
\pgfpathlineto{\pgfqpoint{5.999334in}{1.634391in}}%
\pgfpathlineto{\pgfqpoint{6.000744in}{1.622682in}}%
\pgfpathlineto{\pgfqpoint{6.002717in}{1.581204in}}%
\pgfpathlineto{\pgfqpoint{6.008635in}{1.430049in}}%
\pgfpathlineto{\pgfqpoint{6.008917in}{1.430143in}}%
\pgfpathlineto{\pgfqpoint{6.009763in}{1.435080in}}%
\pgfpathlineto{\pgfqpoint{6.013427in}{1.470633in}}%
\pgfpathlineto{\pgfqpoint{6.013990in}{1.469504in}}%
\pgfpathlineto{\pgfqpoint{6.015400in}{1.457453in}}%
\pgfpathlineto{\pgfqpoint{6.022164in}{1.361799in}}%
\pgfpathlineto{\pgfqpoint{6.027237in}{1.174903in}}%
\pgfpathlineto{\pgfqpoint{6.028083in}{1.184668in}}%
\pgfpathlineto{\pgfqpoint{6.031183in}{1.277362in}}%
\pgfpathlineto{\pgfqpoint{6.032029in}{1.264691in}}%
\pgfpathlineto{\pgfqpoint{6.034847in}{1.152078in}}%
\pgfpathlineto{\pgfqpoint{6.035693in}{1.167697in}}%
\pgfpathlineto{\pgfqpoint{6.037666in}{1.254792in}}%
\pgfpathlineto{\pgfqpoint{6.037947in}{1.248547in}}%
\pgfpathlineto{\pgfqpoint{6.039075in}{1.144624in}}%
\pgfpathlineto{\pgfqpoint{6.040766in}{0.966991in}}%
\pgfpathlineto{\pgfqpoint{6.041048in}{0.975381in}}%
\pgfpathlineto{\pgfqpoint{6.043021in}{1.199585in}}%
\pgfpathlineto{\pgfqpoint{6.043866in}{1.157311in}}%
\pgfpathlineto{\pgfqpoint{6.045557in}{0.999581in}}%
\pgfpathlineto{\pgfqpoint{6.045839in}{1.007988in}}%
\pgfpathlineto{\pgfqpoint{6.047530in}{1.191074in}}%
\pgfpathlineto{\pgfqpoint{6.048939in}{1.279085in}}%
\pgfpathlineto{\pgfqpoint{6.049221in}{1.277725in}}%
\pgfpathlineto{\pgfqpoint{6.050349in}{1.229388in}}%
\pgfpathlineto{\pgfqpoint{6.052040in}{1.167930in}}%
\pgfpathlineto{\pgfqpoint{6.052322in}{1.168617in}}%
\pgfpathlineto{\pgfqpoint{6.053449in}{1.195776in}}%
\pgfpathlineto{\pgfqpoint{6.055986in}{1.260099in}}%
\pgfpathlineto{\pgfqpoint{6.056267in}{1.259527in}}%
\pgfpathlineto{\pgfqpoint{6.057395in}{1.241891in}}%
\pgfpathlineto{\pgfqpoint{6.059368in}{1.206394in}}%
\pgfpathlineto{\pgfqpoint{6.059650in}{1.206569in}}%
\pgfpathlineto{\pgfqpoint{6.060495in}{1.218777in}}%
\pgfpathlineto{\pgfqpoint{6.062468in}{1.303201in}}%
\pgfpathlineto{\pgfqpoint{6.066414in}{1.462624in}}%
\pgfpathlineto{\pgfqpoint{6.066978in}{1.465503in}}%
\pgfpathlineto{\pgfqpoint{6.067541in}{1.463692in}}%
\pgfpathlineto{\pgfqpoint{6.070642in}{1.435754in}}%
\pgfpathlineto{\pgfqpoint{6.071487in}{1.438523in}}%
\pgfpathlineto{\pgfqpoint{6.073460in}{1.461786in}}%
\pgfpathlineto{\pgfqpoint{6.075997in}{1.480638in}}%
\pgfpathlineto{\pgfqpoint{6.076842in}{1.478411in}}%
\pgfpathlineto{\pgfqpoint{6.078251in}{1.473391in}}%
\pgfpathlineto{\pgfqpoint{6.078815in}{1.474261in}}%
\pgfpathlineto{\pgfqpoint{6.079943in}{1.486378in}}%
\pgfpathlineto{\pgfqpoint{6.081915in}{1.545598in}}%
\pgfpathlineto{\pgfqpoint{6.086425in}{1.686354in}}%
\pgfpathlineto{\pgfqpoint{6.086707in}{1.686688in}}%
\pgfpathlineto{\pgfqpoint{6.086707in}{1.686688in}}%
\pgfpathlineto{\pgfqpoint{6.086707in}{1.686688in}}%
\pgfpathlineto{\pgfqpoint{6.087552in}{1.680840in}}%
\pgfpathlineto{\pgfqpoint{6.091216in}{1.630705in}}%
\pgfpathlineto{\pgfqpoint{6.092062in}{1.634352in}}%
\pgfpathlineto{\pgfqpoint{6.094035in}{1.664964in}}%
\pgfpathlineto{\pgfqpoint{6.097135in}{1.702489in}}%
\pgfpathlineto{\pgfqpoint{6.099672in}{1.711387in}}%
\pgfpathlineto{\pgfqpoint{6.101081in}{1.737218in}}%
\pgfpathlineto{\pgfqpoint{6.106436in}{1.900973in}}%
\pgfpathlineto{\pgfqpoint{6.107563in}{1.892205in}}%
\pgfpathlineto{\pgfqpoint{6.111509in}{1.817429in}}%
\pgfpathlineto{\pgfqpoint{6.112073in}{1.821522in}}%
\pgfpathlineto{\pgfqpoint{6.113764in}{1.860100in}}%
\pgfpathlineto{\pgfqpoint{6.117428in}{1.945158in}}%
\pgfpathlineto{\pgfqpoint{6.120810in}{1.967915in}}%
\pgfpathlineto{\pgfqpoint{6.122783in}{2.039724in}}%
\pgfpathlineto{\pgfqpoint{6.126165in}{2.148884in}}%
\pgfpathlineto{\pgfqpoint{6.127011in}{2.134521in}}%
\pgfpathlineto{\pgfqpoint{6.130393in}{2.017900in}}%
\pgfpathlineto{\pgfqpoint{6.130957in}{2.026171in}}%
\pgfpathlineto{\pgfqpoint{6.132366in}{2.094987in}}%
\pgfpathlineto{\pgfqpoint{6.135184in}{2.248361in}}%
\pgfpathlineto{\pgfqpoint{6.135466in}{2.247739in}}%
\pgfpathlineto{\pgfqpoint{6.136312in}{2.221131in}}%
\pgfpathlineto{\pgfqpoint{6.138003in}{2.150226in}}%
\pgfpathlineto{\pgfqpoint{6.138285in}{2.153210in}}%
\pgfpathlineto{\pgfqpoint{6.139130in}{2.204873in}}%
\pgfpathlineto{\pgfqpoint{6.141385in}{2.459451in}}%
\pgfpathlineto{\pgfqpoint{6.141949in}{2.432265in}}%
\pgfpathlineto{\pgfqpoint{6.144204in}{2.077984in}}%
\pgfpathlineto{\pgfqpoint{6.144767in}{2.135653in}}%
\pgfpathlineto{\pgfqpoint{6.146458in}{2.408279in}}%
\pgfpathlineto{\pgfqpoint{6.147022in}{2.370671in}}%
\pgfpathlineto{\pgfqpoint{6.149277in}{2.033162in}}%
\pgfpathlineto{\pgfqpoint{6.149840in}{2.056983in}}%
\pgfpathlineto{\pgfqpoint{6.152377in}{2.339884in}}%
\pgfpathlineto{\pgfqpoint{6.152941in}{2.322284in}}%
\pgfpathlineto{\pgfqpoint{6.156041in}{2.136768in}}%
\pgfpathlineto{\pgfqpoint{6.156605in}{2.142937in}}%
\pgfpathlineto{\pgfqpoint{6.159141in}{2.224397in}}%
\pgfpathlineto{\pgfqpoint{6.159987in}{2.211613in}}%
\pgfpathlineto{\pgfqpoint{6.161678in}{2.114311in}}%
\pgfpathlineto{\pgfqpoint{6.165060in}{1.927280in}}%
\pgfpathlineto{\pgfqpoint{6.165342in}{1.925415in}}%
\pgfpathlineto{\pgfqpoint{6.165624in}{1.926209in}}%
\pgfpathlineto{\pgfqpoint{6.166751in}{1.951705in}}%
\pgfpathlineto{\pgfqpoint{6.169852in}{2.033464in}}%
\pgfpathlineto{\pgfqpoint{6.170697in}{2.023572in}}%
\pgfpathlineto{\pgfqpoint{6.175207in}{1.911431in}}%
\pgfpathlineto{\pgfqpoint{6.176334in}{1.916990in}}%
\pgfpathlineto{\pgfqpoint{6.178025in}{1.930125in}}%
\pgfpathlineto{\pgfqpoint{6.178307in}{1.929577in}}%
\pgfpathlineto{\pgfqpoint{6.179152in}{1.919603in}}%
\pgfpathlineto{\pgfqpoint{6.180844in}{1.861688in}}%
\pgfpathlineto{\pgfqpoint{6.184789in}{1.715954in}}%
\pgfpathlineto{\pgfqpoint{6.185071in}{1.716008in}}%
\pgfpathlineto{\pgfqpoint{6.185917in}{1.727477in}}%
\pgfpathlineto{\pgfqpoint{6.189863in}{1.829629in}}%
\pgfpathlineto{\pgfqpoint{6.190708in}{1.820783in}}%
\pgfpathlineto{\pgfqpoint{6.192681in}{1.759345in}}%
\pgfpathlineto{\pgfqpoint{6.195500in}{1.688775in}}%
\pgfpathlineto{\pgfqpoint{6.196063in}{1.687499in}}%
\pgfpathlineto{\pgfqpoint{6.196345in}{1.688263in}}%
\pgfpathlineto{\pgfqpoint{6.198318in}{1.700077in}}%
\pgfpathlineto{\pgfqpoint{6.198882in}{1.698510in}}%
\pgfpathlineto{\pgfqpoint{6.200009in}{1.680396in}}%
\pgfpathlineto{\pgfqpoint{6.202546in}{1.582633in}}%
\pgfpathlineto{\pgfqpoint{6.204800in}{1.524204in}}%
\pgfpathlineto{\pgfqpoint{6.205082in}{1.524367in}}%
\pgfpathlineto{\pgfqpoint{6.205928in}{1.536388in}}%
\pgfpathlineto{\pgfqpoint{6.209874in}{1.649472in}}%
\pgfpathlineto{\pgfqpoint{6.210719in}{1.639319in}}%
\pgfpathlineto{\pgfqpoint{6.212692in}{1.564578in}}%
\pgfpathlineto{\pgfqpoint{6.215792in}{1.466181in}}%
\pgfpathlineto{\pgfqpoint{6.216074in}{1.465148in}}%
\pgfpathlineto{\pgfqpoint{6.216356in}{1.465413in}}%
\pgfpathlineto{\pgfqpoint{6.216356in}{1.465413in}}%
\pgfpathlineto{\pgfqpoint{6.217484in}{1.474992in}}%
\pgfpathlineto{\pgfqpoint{6.218893in}{1.484194in}}%
\pgfpathlineto{\pgfqpoint{6.219175in}{1.482878in}}%
\pgfpathlineto{\pgfqpoint{6.220302in}{1.462423in}}%
\pgfpathlineto{\pgfqpoint{6.224530in}{1.318662in}}%
\pgfpathlineto{\pgfqpoint{6.225375in}{1.328736in}}%
\pgfpathlineto{\pgfqpoint{6.227348in}{1.417603in}}%
\pgfpathlineto{\pgfqpoint{6.229039in}{1.467892in}}%
\pgfpathlineto{\pgfqpoint{6.229321in}{1.466438in}}%
\pgfpathlineto{\pgfqpoint{6.230448in}{1.430108in}}%
\pgfpathlineto{\pgfqpoint{6.234676in}{1.198193in}}%
\pgfpathlineto{\pgfqpoint{6.235240in}{1.205361in}}%
\pgfpathlineto{\pgfqpoint{6.237495in}{1.279360in}}%
\pgfpathlineto{\pgfqpoint{6.238340in}{1.263307in}}%
\pgfpathlineto{\pgfqpoint{6.240031in}{1.125222in}}%
\pgfpathlineto{\pgfqpoint{6.241441in}{1.021587in}}%
\pgfpathlineto{\pgfqpoint{6.242004in}{1.034285in}}%
\pgfpathlineto{\pgfqpoint{6.243413in}{1.251081in}}%
\pgfpathlineto{\pgfqpoint{6.244541in}{1.384202in}}%
\pgfpathlineto{\pgfqpoint{6.245105in}{1.356408in}}%
\pgfpathlineto{\pgfqpoint{6.247359in}{0.983230in}}%
\pgfpathlineto{\pgfqpoint{6.248205in}{1.085658in}}%
\pgfpathlineto{\pgfqpoint{6.249896in}{1.399600in}}%
\pgfpathlineto{\pgfqpoint{6.250178in}{1.378026in}}%
\pgfpathlineto{\pgfqpoint{6.252433in}{0.908561in}}%
\pgfpathlineto{\pgfqpoint{6.253278in}{1.002460in}}%
\pgfpathlineto{\pgfqpoint{6.255251in}{1.276060in}}%
\pgfpathlineto{\pgfqpoint{6.255533in}{1.275820in}}%
\pgfpathlineto{\pgfqpoint{6.256378in}{1.212482in}}%
\pgfpathlineto{\pgfqpoint{6.258351in}{1.062686in}}%
\pgfpathlineto{\pgfqpoint{6.258633in}{1.069997in}}%
\pgfpathlineto{\pgfqpoint{6.260042in}{1.185440in}}%
\pgfpathlineto{\pgfqpoint{6.262861in}{1.393247in}}%
\pgfpathlineto{\pgfqpoint{6.263143in}{1.392878in}}%
\pgfpathlineto{\pgfqpoint{6.263988in}{1.366415in}}%
\pgfpathlineto{\pgfqpoint{6.267089in}{1.232406in}}%
\pgfpathlineto{\pgfqpoint{6.267370in}{1.235027in}}%
\pgfpathlineto{\pgfqpoint{6.268780in}{1.281431in}}%
\pgfpathlineto{\pgfqpoint{6.272162in}{1.390922in}}%
\pgfpathlineto{\pgfqpoint{6.272725in}{1.387610in}}%
\pgfpathlineto{\pgfqpoint{6.275262in}{1.353419in}}%
\pgfpathlineto{\pgfqpoint{6.275826in}{1.357966in}}%
\pgfpathlineto{\pgfqpoint{6.277235in}{1.401793in}}%
\pgfpathlineto{\pgfqpoint{6.281745in}{1.579643in}}%
\pgfpathlineto{\pgfqpoint{6.282026in}{1.577624in}}%
\pgfpathlineto{\pgfqpoint{6.283154in}{1.550543in}}%
\pgfpathlineto{\pgfqpoint{6.286254in}{1.462378in}}%
\pgfpathlineto{\pgfqpoint{6.286536in}{1.462882in}}%
\pgfpathlineto{\pgfqpoint{6.287381in}{1.475133in}}%
\pgfpathlineto{\pgfqpoint{6.290200in}{1.573972in}}%
\pgfpathlineto{\pgfqpoint{6.292455in}{1.610066in}}%
\pgfpathlineto{\pgfqpoint{6.293300in}{1.604675in}}%
\pgfpathlineto{\pgfqpoint{6.294991in}{1.590469in}}%
\pgfpathlineto{\pgfqpoint{6.295555in}{1.592027in}}%
\pgfpathlineto{\pgfqpoint{6.296682in}{1.611658in}}%
\pgfpathlineto{\pgfqpoint{6.299783in}{1.736395in}}%
\pgfpathlineto{\pgfqpoint{6.301474in}{1.772347in}}%
\pgfpathlineto{\pgfqpoint{6.301756in}{1.772129in}}%
\pgfpathlineto{\pgfqpoint{6.302601in}{1.760366in}}%
\pgfpathlineto{\pgfqpoint{6.306265in}{1.658375in}}%
\pgfpathlineto{\pgfqpoint{6.307111in}{1.664347in}}%
\pgfpathlineto{\pgfqpoint{6.308802in}{1.718740in}}%
\pgfpathlineto{\pgfqpoint{6.312466in}{1.839880in}}%
\pgfpathlineto{\pgfqpoint{6.313029in}{1.841353in}}%
\pgfpathlineto{\pgfqpoint{6.313311in}{1.840406in}}%
\pgfpathlineto{\pgfqpoint{6.315566in}{1.823284in}}%
\pgfpathlineto{\pgfqpoint{6.316130in}{1.825573in}}%
\pgfpathlineto{\pgfqpoint{6.317257in}{1.846402in}}%
\pgfpathlineto{\pgfqpoint{6.321767in}{1.984789in}}%
\pgfpathlineto{\pgfqpoint{6.322612in}{1.975208in}}%
\pgfpathlineto{\pgfqpoint{6.326276in}{1.869474in}}%
\pgfpathlineto{\pgfqpoint{6.327122in}{1.880570in}}%
\pgfpathlineto{\pgfqpoint{6.328813in}{1.957588in}}%
\pgfpathlineto{\pgfqpoint{6.332477in}{2.127053in}}%
\pgfpathlineto{\pgfqpoint{6.332759in}{2.127794in}}%
\pgfpathlineto{\pgfqpoint{6.332759in}{2.127794in}}%
\pgfpathlineto{\pgfqpoint{6.332759in}{2.127794in}}%
\pgfpathlineto{\pgfqpoint{6.333604in}{2.118371in}}%
\pgfpathlineto{\pgfqpoint{6.335577in}{2.083041in}}%
\pgfpathlineto{\pgfqpoint{6.336141in}{2.087020in}}%
\pgfpathlineto{\pgfqpoint{6.337268in}{2.125888in}}%
\pgfpathlineto{\pgfqpoint{6.340369in}{2.290460in}}%
\pgfpathlineto{\pgfqpoint{6.340932in}{2.283836in}}%
\pgfpathlineto{\pgfqpoint{6.342060in}{2.203801in}}%
\pgfpathlineto{\pgfqpoint{6.344033in}{2.057083in}}%
\pgfpathlineto{\pgfqpoint{6.344314in}{2.063656in}}%
\pgfpathlineto{\pgfqpoint{6.345442in}{2.174434in}}%
\pgfpathlineto{\pgfqpoint{6.347415in}{2.442772in}}%
\pgfpathlineto{\pgfqpoint{6.347978in}{2.420170in}}%
\pgfpathlineto{\pgfqpoint{6.349951in}{2.078531in}}%
\pgfpathlineto{\pgfqpoint{6.350515in}{2.120743in}}%
\pgfpathlineto{\pgfqpoint{6.352488in}{2.556913in}}%
\pgfpathlineto{\pgfqpoint{6.353052in}{2.510449in}}%
\pgfpathlineto{\pgfqpoint{6.355306in}{2.095306in}}%
\pgfpathlineto{\pgfqpoint{6.355870in}{2.118412in}}%
\pgfpathlineto{\pgfqpoint{6.358125in}{2.359043in}}%
\pgfpathlineto{\pgfqpoint{6.358689in}{2.335825in}}%
\pgfpathlineto{\pgfqpoint{6.362071in}{2.053942in}}%
\pgfpathlineto{\pgfqpoint{6.362916in}{2.072615in}}%
\pgfpathlineto{\pgfqpoint{6.366017in}{2.237287in}}%
\pgfpathlineto{\pgfqpoint{6.366580in}{2.229531in}}%
\pgfpathlineto{\pgfqpoint{6.368271in}{2.145397in}}%
\pgfpathlineto{\pgfqpoint{6.371654in}{1.999451in}}%
\pgfpathlineto{\pgfqpoint{6.372217in}{1.997741in}}%
\pgfpathlineto{\pgfqpoint{6.372499in}{1.998617in}}%
\pgfpathlineto{\pgfqpoint{6.374190in}{2.006939in}}%
\pgfpathlineto{\pgfqpoint{6.374472in}{2.005884in}}%
\pgfpathlineto{\pgfqpoint{6.375599in}{1.987858in}}%
\pgfpathlineto{\pgfqpoint{6.378418in}{1.876808in}}%
\pgfpathlineto{\pgfqpoint{6.380391in}{1.832156in}}%
\pgfpathlineto{\pgfqpoint{6.380673in}{1.832538in}}%
\pgfpathlineto{\pgfqpoint{6.381518in}{1.844126in}}%
\pgfpathlineto{\pgfqpoint{6.384900in}{1.926363in}}%
\pgfpathlineto{\pgfqpoint{6.385464in}{1.923471in}}%
\pgfpathlineto{\pgfqpoint{6.386873in}{1.886021in}}%
\pgfpathlineto{\pgfqpoint{6.392228in}{1.708086in}}%
\pgfpathlineto{\pgfqpoint{6.392510in}{1.707824in}}%
\pgfpathlineto{\pgfqpoint{6.393074in}{1.708795in}}%
\pgfpathlineto{\pgfqpoint{6.394201in}{1.710785in}}%
\pgfpathlineto{\pgfqpoint{6.394483in}{1.710151in}}%
\pgfpathlineto{\pgfqpoint{6.395610in}{1.699254in}}%
\pgfpathlineto{\pgfqpoint{6.397865in}{1.641594in}}%
\pgfpathlineto{\pgfqpoint{6.400402in}{1.594802in}}%
\pgfpathlineto{\pgfqpoint{6.400966in}{1.597035in}}%
\pgfpathlineto{\pgfqpoint{6.402375in}{1.621965in}}%
\pgfpathlineto{\pgfqpoint{6.404630in}{1.662269in}}%
\pgfpathlineto{\pgfqpoint{6.404911in}{1.662121in}}%
\pgfpathlineto{\pgfqpoint{6.405757in}{1.651416in}}%
\pgfpathlineto{\pgfqpoint{6.407448in}{1.588679in}}%
\pgfpathlineto{\pgfqpoint{6.411676in}{1.421088in}}%
\pgfpathlineto{\pgfqpoint{6.412239in}{1.418039in}}%
\pgfpathlineto{\pgfqpoint{6.412803in}{1.419230in}}%
\pgfpathlineto{\pgfqpoint{6.414776in}{1.430784in}}%
\pgfpathlineto{\pgfqpoint{6.415340in}{1.428840in}}%
\pgfpathlineto{\pgfqpoint{6.416467in}{1.412187in}}%
\pgfpathlineto{\pgfqpoint{6.420413in}{1.327248in}}%
\pgfpathlineto{\pgfqpoint{6.420977in}{1.331234in}}%
\pgfpathlineto{\pgfqpoint{6.423795in}{1.386154in}}%
\pgfpathlineto{\pgfqpoint{6.424641in}{1.378210in}}%
\pgfpathlineto{\pgfqpoint{6.426050in}{1.313331in}}%
\pgfpathlineto{\pgfqpoint{6.429996in}{1.068350in}}%
\pgfpathlineto{\pgfqpoint{6.430278in}{1.071352in}}%
\pgfpathlineto{\pgfqpoint{6.431405in}{1.121447in}}%
\pgfpathlineto{\pgfqpoint{6.432814in}{1.184390in}}%
\pgfpathlineto{\pgfqpoint{6.433378in}{1.176137in}}%
\pgfpathlineto{\pgfqpoint{6.434787in}{1.061835in}}%
\pgfpathlineto{\pgfqpoint{6.435915in}{0.976468in}}%
\pgfpathlineto{\pgfqpoint{6.436478in}{0.994315in}}%
\pgfpathlineto{\pgfqpoint{6.438733in}{1.305387in}}%
\pgfpathlineto{\pgfqpoint{6.439579in}{1.215816in}}%
\pgfpathlineto{\pgfqpoint{6.441270in}{0.971172in}}%
\pgfpathlineto{\pgfqpoint{6.441551in}{0.987505in}}%
\pgfpathlineto{\pgfqpoint{6.443524in}{1.243743in}}%
\pgfpathlineto{\pgfqpoint{6.444370in}{1.191032in}}%
\pgfpathlineto{\pgfqpoint{6.446343in}{0.952860in}}%
\pgfpathlineto{\pgfqpoint{6.446907in}{0.964992in}}%
\pgfpathlineto{\pgfqpoint{6.450571in}{1.197038in}}%
\pgfpathlineto{\pgfqpoint{6.451416in}{1.190417in}}%
\pgfpathlineto{\pgfqpoint{6.452543in}{1.178698in}}%
\pgfpathlineto{\pgfqpoint{6.452825in}{1.179434in}}%
\pgfpathlineto{\pgfqpoint{6.453671in}{1.195122in}}%
\pgfpathlineto{\pgfqpoint{6.455926in}{1.307143in}}%
\pgfpathlineto{\pgfqpoint{6.458462in}{1.385935in}}%
\pgfpathlineto{\pgfqpoint{6.459308in}{1.375510in}}%
\pgfpathlineto{\pgfqpoint{6.462690in}{1.298489in}}%
\pgfpathlineto{\pgfqpoint{6.463254in}{1.303786in}}%
\pgfpathlineto{\pgfqpoint{6.464945in}{1.355432in}}%
\pgfpathlineto{\pgfqpoint{6.469172in}{1.493480in}}%
\pgfpathlineto{\pgfqpoint{6.473400in}{1.553092in}}%
\pgfpathlineto{\pgfqpoint{6.477628in}{1.639824in}}%
\pgfpathlineto{\pgfqpoint{6.478473in}{1.633313in}}%
\pgfpathlineto{\pgfqpoint{6.481855in}{1.582461in}}%
\pgfpathlineto{\pgfqpoint{6.482701in}{1.587721in}}%
\pgfpathlineto{\pgfqpoint{6.484392in}{1.630992in}}%
\pgfpathlineto{\pgfqpoint{6.489747in}{1.791975in}}%
\pgfpathlineto{\pgfqpoint{6.493975in}{1.839025in}}%
\pgfpathlineto{\pgfqpoint{6.497357in}{1.887625in}}%
\pgfpathlineto{\pgfqpoint{6.497639in}{1.887793in}}%
\pgfpathlineto{\pgfqpoint{6.497921in}{1.887165in}}%
\pgfpathlineto{\pgfqpoint{6.499048in}{1.877874in}}%
\pgfpathlineto{\pgfqpoint{6.501585in}{1.854861in}}%
\pgfpathlineto{\pgfqpoint{6.501867in}{1.856000in}}%
\pgfpathlineto{\pgfqpoint{6.502994in}{1.873127in}}%
\pgfpathlineto{\pgfqpoint{6.505249in}{1.960399in}}%
\pgfpathlineto{\pgfqpoint{6.509195in}{2.105024in}}%
\pgfpathlineto{\pgfqpoint{6.510604in}{2.112475in}}%
\pgfpathlineto{\pgfqpoint{6.510886in}{2.112268in}}%
\pgfpathlineto{\pgfqpoint{6.511731in}{2.111575in}}%
\pgfpathlineto{\pgfqpoint{6.512013in}{2.111925in}}%
\pgfpathlineto{\pgfqpoint{6.512859in}{2.116654in}}%
\pgfpathlineto{\pgfqpoint{6.514550in}{2.145180in}}%
\pgfpathlineto{\pgfqpoint{6.517086in}{2.187335in}}%
\pgfpathlineto{\pgfqpoint{6.517368in}{2.187086in}}%
\pgfpathlineto{\pgfqpoint{6.518214in}{2.178311in}}%
\pgfpathlineto{\pgfqpoint{6.520187in}{2.147714in}}%
\pgfpathlineto{\pgfqpoint{6.520468in}{2.149155in}}%
\pgfpathlineto{\pgfqpoint{6.521314in}{2.171677in}}%
\pgfpathlineto{\pgfqpoint{6.523569in}{2.338752in}}%
\pgfpathlineto{\pgfqpoint{6.524696in}{2.384374in}}%
\pgfpathlineto{\pgfqpoint{6.524978in}{2.379132in}}%
\pgfpathlineto{\pgfqpoint{6.526951in}{2.253440in}}%
\pgfpathlineto{\pgfqpoint{6.527796in}{2.285440in}}%
\pgfpathlineto{\pgfqpoint{6.529487in}{2.421954in}}%
\pgfpathlineto{\pgfqpoint{6.529769in}{2.414669in}}%
\pgfpathlineto{\pgfqpoint{6.532588in}{2.181149in}}%
\pgfpathlineto{\pgfqpoint{6.533433in}{2.212315in}}%
\pgfpathlineto{\pgfqpoint{6.535124in}{2.267656in}}%
\pgfpathlineto{\pgfqpoint{6.535406in}{2.263618in}}%
\pgfpathlineto{\pgfqpoint{6.538507in}{2.160479in}}%
\pgfpathlineto{\pgfqpoint{6.539634in}{2.173540in}}%
\pgfpathlineto{\pgfqpoint{6.542171in}{2.231659in}}%
\pgfpathlineto{\pgfqpoint{6.542734in}{2.229191in}}%
\pgfpathlineto{\pgfqpoint{6.543862in}{2.202382in}}%
\pgfpathlineto{\pgfqpoint{6.552035in}{1.938034in}}%
\pgfpathlineto{\pgfqpoint{6.555699in}{1.904502in}}%
\pgfpathlineto{\pgfqpoint{6.556545in}{1.903260in}}%
\pgfpathlineto{\pgfqpoint{6.556827in}{1.903513in}}%
\pgfpathlineto{\pgfqpoint{6.558236in}{1.908428in}}%
\pgfpathlineto{\pgfqpoint{6.559927in}{1.914246in}}%
\pgfpathlineto{\pgfqpoint{6.560209in}{1.914137in}}%
\pgfpathlineto{\pgfqpoint{6.561054in}{1.910524in}}%
\pgfpathlineto{\pgfqpoint{6.562464in}{1.891031in}}%
\pgfpathlineto{\pgfqpoint{6.565000in}{1.816308in}}%
\pgfpathlineto{\pgfqpoint{6.569792in}{1.676996in}}%
\pgfpathlineto{\pgfqpoint{6.571764in}{1.664008in}}%
\pgfpathlineto{\pgfqpoint{6.572610in}{1.664500in}}%
\pgfpathlineto{\pgfqpoint{6.575710in}{1.669457in}}%
\pgfpathlineto{\pgfqpoint{6.575992in}{1.669152in}}%
\pgfpathlineto{\pgfqpoint{6.577401in}{1.665109in}}%
\pgfpathlineto{\pgfqpoint{6.579938in}{1.648801in}}%
\pgfpathlineto{\pgfqpoint{6.582193in}{1.618339in}}%
\pgfpathlineto{\pgfqpoint{6.584729in}{1.548687in}}%
\pgfpathlineto{\pgfqpoint{6.589803in}{1.402672in}}%
\pgfpathlineto{\pgfqpoint{6.590366in}{1.400989in}}%
\pgfpathlineto{\pgfqpoint{6.590648in}{1.401522in}}%
\pgfpathlineto{\pgfqpoint{6.591776in}{1.410767in}}%
\pgfpathlineto{\pgfqpoint{6.594594in}{1.436533in}}%
\pgfpathlineto{\pgfqpoint{6.594876in}{1.436018in}}%
\pgfpathlineto{\pgfqpoint{6.596003in}{1.426327in}}%
\pgfpathlineto{\pgfqpoint{6.600513in}{1.344639in}}%
\pgfpathlineto{\pgfqpoint{6.602768in}{1.293003in}}%
\pgfpathlineto{\pgfqpoint{6.605586in}{1.150921in}}%
\pgfpathlineto{\pgfqpoint{6.607559in}{1.083198in}}%
\pgfpathlineto{\pgfqpoint{6.607841in}{1.084831in}}%
\pgfpathlineto{\pgfqpoint{6.608968in}{1.125606in}}%
\pgfpathlineto{\pgfqpoint{6.610941in}{1.214161in}}%
\pgfpathlineto{\pgfqpoint{6.611223in}{1.212528in}}%
\pgfpathlineto{\pgfqpoint{6.612350in}{1.160286in}}%
\pgfpathlineto{\pgfqpoint{6.613760in}{1.081269in}}%
\pgfpathlineto{\pgfqpoint{6.614323in}{1.092054in}}%
\pgfpathlineto{\pgfqpoint{6.616296in}{1.265875in}}%
\pgfpathlineto{\pgfqpoint{6.617142in}{1.221877in}}%
\pgfpathlineto{\pgfqpoint{6.619115in}{0.987513in}}%
\pgfpathlineto{\pgfqpoint{6.619678in}{1.015238in}}%
\pgfpathlineto{\pgfqpoint{6.621369in}{1.162459in}}%
\pgfpathlineto{\pgfqpoint{6.621933in}{1.152456in}}%
\pgfpathlineto{\pgfqpoint{6.624188in}{1.007307in}}%
\pgfpathlineto{\pgfqpoint{6.624752in}{1.021599in}}%
\pgfpathlineto{\pgfqpoint{6.629261in}{1.270009in}}%
\pgfpathlineto{\pgfqpoint{6.630107in}{1.265572in}}%
\pgfpathlineto{\pgfqpoint{6.631516in}{1.255980in}}%
\pgfpathlineto{\pgfqpoint{6.631798in}{1.256449in}}%
\pgfpathlineto{\pgfqpoint{6.632925in}{1.268852in}}%
\pgfpathlineto{\pgfqpoint{6.636871in}{1.329582in}}%
\pgfpathlineto{\pgfqpoint{6.637153in}{1.328911in}}%
\pgfpathlineto{\pgfqpoint{6.639689in}{1.313703in}}%
\pgfpathlineto{\pgfqpoint{6.640253in}{1.316396in}}%
\pgfpathlineto{\pgfqpoint{6.641381in}{1.336181in}}%
\pgfpathlineto{\pgfqpoint{6.643917in}{1.438363in}}%
\pgfpathlineto{\pgfqpoint{6.647581in}{1.559777in}}%
\pgfpathlineto{\pgfqpoint{6.648709in}{1.564486in}}%
\pgfpathlineto{\pgfqpoint{6.648990in}{1.563871in}}%
\pgfpathlineto{\pgfqpoint{6.651527in}{1.554681in}}%
\pgfpathlineto{\pgfqpoint{6.652091in}{1.556388in}}%
\pgfpathlineto{\pgfqpoint{6.653782in}{1.572763in}}%
\pgfpathlineto{\pgfqpoint{6.657728in}{1.611779in}}%
\pgfpathlineto{\pgfqpoint{6.659701in}{1.620987in}}%
\pgfpathlineto{\pgfqpoint{6.661110in}{1.643211in}}%
\pgfpathlineto{\pgfqpoint{6.663646in}{1.732128in}}%
\pgfpathlineto{\pgfqpoint{6.667310in}{1.848928in}}%
\pgfpathlineto{\pgfqpoint{6.667874in}{1.852161in}}%
\pgfpathlineto{\pgfqpoint{6.668438in}{1.851085in}}%
\pgfpathlineto{\pgfqpoint{6.670129in}{1.832351in}}%
\pgfpathlineto{\pgfqpoint{6.671820in}{1.819071in}}%
\pgfpathlineto{\pgfqpoint{6.672102in}{1.819508in}}%
\pgfpathlineto{\pgfqpoint{6.673229in}{1.830309in}}%
\pgfpathlineto{\pgfqpoint{6.676611in}{1.905992in}}%
\pgfpathlineto{\pgfqpoint{6.679712in}{1.943252in}}%
\pgfpathlineto{\pgfqpoint{6.681403in}{1.977258in}}%
\pgfpathlineto{\pgfqpoint{6.683939in}{2.092309in}}%
\pgfpathlineto{\pgfqpoint{6.686758in}{2.196699in}}%
\pgfpathlineto{\pgfqpoint{6.687040in}{2.197380in}}%
\pgfpathlineto{\pgfqpoint{6.687040in}{2.197380in}}%
\pgfpathlineto{\pgfqpoint{6.687040in}{2.197380in}}%
\pgfpathlineto{\pgfqpoint{6.687603in}{2.191921in}}%
\pgfpathlineto{\pgfqpoint{6.689576in}{2.129350in}}%
\pgfpathlineto{\pgfqpoint{6.690704in}{2.110244in}}%
\pgfpathlineto{\pgfqpoint{6.690985in}{2.112070in}}%
\pgfpathlineto{\pgfqpoint{6.692113in}{2.149106in}}%
\pgfpathlineto{\pgfqpoint{6.695213in}{2.314059in}}%
\pgfpathlineto{\pgfqpoint{6.695777in}{2.302568in}}%
\pgfpathlineto{\pgfqpoint{6.697750in}{2.197941in}}%
\pgfpathlineto{\pgfqpoint{6.698313in}{2.213312in}}%
\pgfpathlineto{\pgfqpoint{6.699723in}{2.395245in}}%
\pgfpathlineto{\pgfqpoint{6.700850in}{2.513581in}}%
\pgfpathlineto{\pgfqpoint{6.701132in}{2.503608in}}%
\pgfpathlineto{\pgfqpoint{6.703387in}{2.193668in}}%
\pgfpathlineto{\pgfqpoint{6.704232in}{2.279005in}}%
\pgfpathlineto{\pgfqpoint{6.705641in}{2.439877in}}%
\pgfpathlineto{\pgfqpoint{6.705923in}{2.433755in}}%
\pgfpathlineto{\pgfqpoint{6.707333in}{2.271973in}}%
\pgfpathlineto{\pgfqpoint{6.709024in}{2.134453in}}%
\pgfpathlineto{\pgfqpoint{6.709305in}{2.135849in}}%
\pgfpathlineto{\pgfqpoint{6.710433in}{2.194994in}}%
\pgfpathlineto{\pgfqpoint{6.712406in}{2.285108in}}%
\pgfpathlineto{\pgfqpoint{6.712688in}{2.282929in}}%
\pgfpathlineto{\pgfqpoint{6.713815in}{2.244773in}}%
\pgfpathlineto{\pgfqpoint{6.717197in}{2.124904in}}%
\pgfpathlineto{\pgfqpoint{6.717479in}{2.124682in}}%
\pgfpathlineto{\pgfqpoint{6.717479in}{2.124682in}}%
\pgfpathlineto{\pgfqpoint{6.717479in}{2.124682in}}%
\pgfpathlineto{\pgfqpoint{6.718606in}{2.132183in}}%
\pgfpathlineto{\pgfqpoint{6.719452in}{2.135717in}}%
\pgfpathlineto{\pgfqpoint{6.719734in}{2.134571in}}%
\pgfpathlineto{\pgfqpoint{6.720861in}{2.112886in}}%
\pgfpathlineto{\pgfqpoint{6.723116in}{1.996307in}}%
\pgfpathlineto{\pgfqpoint{6.726498in}{1.849341in}}%
\pgfpathlineto{\pgfqpoint{6.726780in}{1.847981in}}%
\pgfpathlineto{\pgfqpoint{6.727062in}{1.848503in}}%
\pgfpathlineto{\pgfqpoint{6.728189in}{1.865639in}}%
\pgfpathlineto{\pgfqpoint{6.730726in}{1.911883in}}%
\pgfpathlineto{\pgfqpoint{6.731008in}{1.911545in}}%
\pgfpathlineto{\pgfqpoint{6.731853in}{1.901511in}}%
\pgfpathlineto{\pgfqpoint{6.734108in}{1.830156in}}%
\pgfpathlineto{\pgfqpoint{6.736926in}{1.761509in}}%
\pgfpathlineto{\pgfqpoint{6.740590in}{1.728988in}}%
\pgfpathlineto{\pgfqpoint{6.742845in}{1.634696in}}%
\pgfpathlineto{\pgfqpoint{6.745946in}{1.521648in}}%
\pgfpathlineto{\pgfqpoint{6.746227in}{1.520258in}}%
\pgfpathlineto{\pgfqpoint{6.746509in}{1.520744in}}%
\pgfpathlineto{\pgfqpoint{6.746509in}{1.520744in}}%
\pgfpathlineto{\pgfqpoint{6.747355in}{1.532427in}}%
\pgfpathlineto{\pgfqpoint{6.750455in}{1.601005in}}%
\pgfpathlineto{\pgfqpoint{6.751019in}{1.597528in}}%
\pgfpathlineto{\pgfqpoint{6.752428in}{1.558085in}}%
\pgfpathlineto{\pgfqpoint{6.757219in}{1.397443in}}%
\pgfpathlineto{\pgfqpoint{6.757783in}{1.397411in}}%
\pgfpathlineto{\pgfqpoint{6.758065in}{1.397841in}}%
\pgfpathlineto{\pgfqpoint{6.758629in}{1.397877in}}%
\pgfpathlineto{\pgfqpoint{6.759474in}{1.390909in}}%
\pgfpathlineto{\pgfqpoint{6.760883in}{1.346647in}}%
\pgfpathlineto{\pgfqpoint{6.764829in}{1.150864in}}%
\pgfpathlineto{\pgfqpoint{6.765393in}{1.155614in}}%
\pgfpathlineto{\pgfqpoint{6.766802in}{1.225273in}}%
\pgfpathlineto{\pgfqpoint{6.768493in}{1.305735in}}%
\pgfpathlineto{\pgfqpoint{6.768775in}{1.305074in}}%
\pgfpathlineto{\pgfqpoint{6.769621in}{1.270817in}}%
\pgfpathlineto{\pgfqpoint{6.772721in}{1.002077in}}%
\pgfpathlineto{\pgfqpoint{6.773566in}{1.048095in}}%
\pgfpathlineto{\pgfqpoint{6.775258in}{1.229900in}}%
\pgfpathlineto{\pgfqpoint{6.775821in}{1.202759in}}%
\pgfpathlineto{\pgfqpoint{6.778076in}{0.795827in}}%
\pgfpathlineto{\pgfqpoint{6.778922in}{0.922974in}}%
\pgfpathlineto{\pgfqpoint{6.780613in}{1.254402in}}%
\pgfpathlineto{\pgfqpoint{6.780894in}{1.232997in}}%
\pgfpathlineto{\pgfqpoint{6.782867in}{0.907473in}}%
\pgfpathlineto{\pgfqpoint{6.783431in}{0.944397in}}%
\pgfpathlineto{\pgfqpoint{6.786250in}{1.284248in}}%
\pgfpathlineto{\pgfqpoint{6.786813in}{1.260787in}}%
\pgfpathlineto{\pgfqpoint{6.789632in}{1.054676in}}%
\pgfpathlineto{\pgfqpoint{6.790195in}{1.070040in}}%
\pgfpathlineto{\pgfqpoint{6.794423in}{1.267329in}}%
\pgfpathlineto{\pgfqpoint{6.794987in}{1.262062in}}%
\pgfpathlineto{\pgfqpoint{6.796678in}{1.239566in}}%
\pgfpathlineto{\pgfqpoint{6.797242in}{1.243132in}}%
\pgfpathlineto{\pgfqpoint{6.798369in}{1.278149in}}%
\pgfpathlineto{\pgfqpoint{6.803442in}{1.533872in}}%
\pgfpathlineto{\pgfqpoint{6.804288in}{1.523248in}}%
\pgfpathlineto{\pgfqpoint{6.807670in}{1.437872in}}%
\pgfpathlineto{\pgfqpoint{6.808234in}{1.443265in}}%
\pgfpathlineto{\pgfqpoint{6.809925in}{1.498146in}}%
\pgfpathlineto{\pgfqpoint{6.813589in}{1.617521in}}%
\pgfpathlineto{\pgfqpoint{6.814152in}{1.619309in}}%
\pgfpathlineto{\pgfqpoint{6.814716in}{1.617959in}}%
\pgfpathlineto{\pgfqpoint{6.815843in}{1.614386in}}%
\pgfpathlineto{\pgfqpoint{6.816125in}{1.614954in}}%
\pgfpathlineto{\pgfqpoint{6.816971in}{1.623825in}}%
\pgfpathlineto{\pgfqpoint{6.818662in}{1.680318in}}%
\pgfpathlineto{\pgfqpoint{6.822890in}{1.848540in}}%
\pgfpathlineto{\pgfqpoint{6.823171in}{1.848098in}}%
\pgfpathlineto{\pgfqpoint{6.824017in}{1.834987in}}%
\pgfpathlineto{\pgfqpoint{6.827399in}{1.749495in}}%
\pgfpathlineto{\pgfqpoint{6.827963in}{1.755358in}}%
\pgfpathlineto{\pgfqpoint{6.829372in}{1.803436in}}%
\pgfpathlineto{\pgfqpoint{6.833882in}{1.980948in}}%
\pgfpathlineto{\pgfqpoint{6.834163in}{1.981506in}}%
\pgfpathlineto{\pgfqpoint{6.834445in}{1.980959in}}%
\pgfpathlineto{\pgfqpoint{6.836136in}{1.972755in}}%
\pgfpathlineto{\pgfqpoint{6.836418in}{1.973862in}}%
\pgfpathlineto{\pgfqpoint{6.837264in}{1.987021in}}%
\pgfpathlineto{\pgfqpoint{6.838955in}{2.060607in}}%
\pgfpathlineto{\pgfqpoint{6.842337in}{2.208869in}}%
\pgfpathlineto{\pgfqpoint{6.843183in}{2.190727in}}%
\pgfpathlineto{\pgfqpoint{6.846001in}{2.054352in}}%
\pgfpathlineto{\pgfqpoint{6.846847in}{2.072383in}}%
\pgfpathlineto{\pgfqpoint{6.848538in}{2.223398in}}%
\pgfpathlineto{\pgfqpoint{6.850792in}{2.413393in}}%
\pgfpathlineto{\pgfqpoint{6.851074in}{2.406477in}}%
\pgfpathlineto{\pgfqpoint{6.852202in}{2.284390in}}%
\pgfpathlineto{\pgfqpoint{6.853329in}{2.163030in}}%
\pgfpathlineto{\pgfqpoint{6.853893in}{2.180641in}}%
\pgfpathlineto{\pgfqpoint{6.855302in}{2.457644in}}%
\pgfpathlineto{\pgfqpoint{6.856429in}{2.608868in}}%
\pgfpathlineto{\pgfqpoint{6.856711in}{2.576739in}}%
\pgfpathlineto{\pgfqpoint{6.858966in}{2.080144in}}%
\pgfpathlineto{\pgfqpoint{6.859530in}{2.161204in}}%
\pgfpathlineto{\pgfqpoint{6.861221in}{2.480859in}}%
\pgfpathlineto{\pgfqpoint{6.861784in}{2.442172in}}%
\pgfpathlineto{\pgfqpoint{6.864603in}{2.084412in}}%
\pgfpathlineto{\pgfqpoint{6.864885in}{2.093092in}}%
\pgfpathlineto{\pgfqpoint{6.866576in}{2.279612in}}%
\pgfpathlineto{\pgfqpoint{6.867985in}{2.358656in}}%
\pgfpathlineto{\pgfqpoint{6.868267in}{2.353947in}}%
\pgfpathlineto{\pgfqpoint{6.869676in}{2.268322in}}%
\pgfpathlineto{\pgfqpoint{6.872776in}{2.092408in}}%
\pgfpathlineto{\pgfqpoint{6.873340in}{2.090107in}}%
\pgfpathlineto{\pgfqpoint{6.873622in}{2.091518in}}%
\pgfpathlineto{\pgfqpoint{6.875031in}{2.103698in}}%
\pgfpathlineto{\pgfqpoint{6.875595in}{2.101824in}}%
\pgfpathlineto{\pgfqpoint{6.876722in}{2.074087in}}%
\pgfpathlineto{\pgfqpoint{6.879823in}{1.896065in}}%
\pgfpathlineto{\pgfqpoint{6.881795in}{1.842306in}}%
\pgfpathlineto{\pgfqpoint{6.882641in}{1.855312in}}%
\pgfpathlineto{\pgfqpoint{6.886023in}{1.961901in}}%
\pgfpathlineto{\pgfqpoint{6.886869in}{1.953600in}}%
\pgfpathlineto{\pgfqpoint{6.888560in}{1.886516in}}%
\pgfpathlineto{\pgfqpoint{6.892506in}{1.730067in}}%
\pgfpathlineto{\pgfqpoint{6.893633in}{1.725982in}}%
\pgfpathlineto{\pgfqpoint{6.893915in}{1.726223in}}%
\pgfpathlineto{\pgfqpoint{6.894479in}{1.726317in}}%
\pgfpathlineto{\pgfqpoint{6.894760in}{1.725641in}}%
\pgfpathlineto{\pgfqpoint{6.895606in}{1.718129in}}%
\pgfpathlineto{\pgfqpoint{6.897297in}{1.671034in}}%
\pgfpathlineto{\pgfqpoint{6.901243in}{1.545340in}}%
\pgfpathlineto{\pgfqpoint{6.901807in}{1.549725in}}%
\pgfpathlineto{\pgfqpoint{6.903498in}{1.599145in}}%
\pgfpathlineto{\pgfqpoint{6.905752in}{1.649440in}}%
\pgfpathlineto{\pgfqpoint{6.906598in}{1.636001in}}%
\pgfpathlineto{\pgfqpoint{6.908571in}{1.540747in}}%
\pgfpathlineto{\pgfqpoint{6.912235in}{1.378347in}}%
\pgfpathlineto{\pgfqpoint{6.912799in}{1.375152in}}%
\pgfpathlineto{\pgfqpoint{6.913362in}{1.376223in}}%
\pgfpathlineto{\pgfqpoint{6.914490in}{1.380787in}}%
\pgfpathlineto{\pgfqpoint{6.914771in}{1.380438in}}%
\pgfpathlineto{\pgfqpoint{6.915617in}{1.371571in}}%
\pgfpathlineto{\pgfqpoint{6.917308in}{1.314563in}}%
\pgfpathlineto{\pgfqpoint{6.920127in}{1.215954in}}%
\pgfpathlineto{\pgfqpoint{6.920408in}{1.216548in}}%
\pgfpathlineto{\pgfqpoint{6.921254in}{1.235968in}}%
\pgfpathlineto{\pgfqpoint{6.924072in}{1.349570in}}%
\pgfpathlineto{\pgfqpoint{6.924636in}{1.337869in}}%
\pgfpathlineto{\pgfqpoint{6.926045in}{1.233511in}}%
\pgfpathlineto{\pgfqpoint{6.928864in}{0.999363in}}%
\pgfpathlineto{\pgfqpoint{6.929146in}{1.005997in}}%
\pgfpathlineto{\pgfqpoint{6.931400in}{1.174836in}}%
\pgfpathlineto{\pgfqpoint{6.931964in}{1.143890in}}%
\pgfpathlineto{\pgfqpoint{6.934219in}{0.841291in}}%
\pgfpathlineto{\pgfqpoint{6.934783in}{0.916047in}}%
\pgfpathlineto{\pgfqpoint{6.936755in}{1.282788in}}%
\pgfpathlineto{\pgfqpoint{6.937037in}{1.260800in}}%
\pgfpathlineto{\pgfqpoint{6.939010in}{0.977346in}}%
\pgfpathlineto{\pgfqpoint{6.939574in}{1.020561in}}%
\pgfpathlineto{\pgfqpoint{6.941829in}{1.286076in}}%
\pgfpathlineto{\pgfqpoint{6.942392in}{1.266213in}}%
\pgfpathlineto{\pgfqpoint{6.945211in}{1.013222in}}%
\pgfpathlineto{\pgfqpoint{6.945775in}{1.026897in}}%
\pgfpathlineto{\pgfqpoint{6.950566in}{1.233996in}}%
\pgfpathlineto{\pgfqpoint{6.950848in}{1.233740in}}%
\pgfpathlineto{\pgfqpoint{6.951412in}{1.234541in}}%
\pgfpathlineto{\pgfqpoint{6.952257in}{1.245393in}}%
\pgfpathlineto{\pgfqpoint{6.953948in}{1.318084in}}%
\pgfpathlineto{\pgfqpoint{6.958176in}{1.519451in}}%
\pgfpathlineto{\pgfqpoint{6.959021in}{1.506045in}}%
\pgfpathlineto{\pgfqpoint{6.962404in}{1.400058in}}%
\pgfpathlineto{\pgfqpoint{6.963249in}{1.407511in}}%
\pgfpathlineto{\pgfqpoint{6.964940in}{1.468005in}}%
\pgfpathlineto{\pgfqpoint{6.968886in}{1.605939in}}%
\pgfpathlineto{\pgfqpoint{6.971986in}{1.653483in}}%
\pgfpathlineto{\pgfqpoint{6.975087in}{1.789700in}}%
\pgfpathlineto{\pgfqpoint{6.977341in}{1.840460in}}%
\pgfpathlineto{\pgfqpoint{6.978187in}{1.828933in}}%
\pgfpathlineto{\pgfqpoint{6.981569in}{1.736348in}}%
\pgfpathlineto{\pgfqpoint{6.982415in}{1.743705in}}%
\pgfpathlineto{\pgfqpoint{6.984106in}{1.809546in}}%
\pgfpathlineto{\pgfqpoint{6.988897in}{2.011369in}}%
\pgfpathlineto{\pgfqpoint{6.991434in}{2.049144in}}%
\pgfpathlineto{\pgfqpoint{6.993688in}{2.147288in}}%
\pgfpathlineto{\pgfqpoint{6.995943in}{2.226495in}}%
\pgfpathlineto{\pgfqpoint{6.996225in}{2.226476in}}%
\pgfpathlineto{\pgfqpoint{6.997071in}{2.208136in}}%
\pgfpathlineto{\pgfqpoint{6.999889in}{2.089240in}}%
\pgfpathlineto{\pgfqpoint{7.000453in}{2.100341in}}%
\pgfpathlineto{\pgfqpoint{7.001862in}{2.207806in}}%
\pgfpathlineto{\pgfqpoint{7.004399in}{2.424813in}}%
\pgfpathlineto{\pgfqpoint{7.004680in}{2.417588in}}%
\pgfpathlineto{\pgfqpoint{7.006653in}{2.242381in}}%
\pgfpathlineto{\pgfqpoint{7.007499in}{2.293491in}}%
\pgfpathlineto{\pgfqpoint{7.009472in}{2.593267in}}%
\pgfpathlineto{\pgfqpoint{7.010036in}{2.527857in}}%
\pgfpathlineto{\pgfqpoint{7.012008in}{2.187194in}}%
\pgfpathlineto{\pgfqpoint{7.012572in}{2.218414in}}%
\pgfpathlineto{\pgfqpoint{7.014263in}{2.398845in}}%
\pgfpathlineto{\pgfqpoint{7.014827in}{2.385497in}}%
\pgfpathlineto{\pgfqpoint{7.017927in}{2.124632in}}%
\pgfpathlineto{\pgfqpoint{7.018773in}{2.146363in}}%
\pgfpathlineto{\pgfqpoint{7.021591in}{2.302788in}}%
\pgfpathlineto{\pgfqpoint{7.022155in}{2.297583in}}%
\pgfpathlineto{\pgfqpoint{7.023564in}{2.235626in}}%
\pgfpathlineto{\pgfqpoint{7.027792in}{2.046481in}}%
\pgfpathlineto{\pgfqpoint{7.030610in}{1.960987in}}%
\pgfpathlineto{\pgfqpoint{7.035402in}{1.784570in}}%
\pgfpathlineto{\pgfqpoint{7.036247in}{1.795078in}}%
\pgfpathlineto{\pgfqpoint{7.039629in}{1.875248in}}%
\pgfpathlineto{\pgfqpoint{7.040475in}{1.867812in}}%
\pgfpathlineto{\pgfqpoint{7.042166in}{1.809439in}}%
\pgfpathlineto{\pgfqpoint{7.047521in}{1.601099in}}%
\pgfpathlineto{\pgfqpoint{7.050903in}{1.496665in}}%
\pgfpathlineto{\pgfqpoint{7.054004in}{1.419840in}}%
\pgfpathlineto{\pgfqpoint{7.054567in}{1.423007in}}%
\pgfpathlineto{\pgfqpoint{7.055977in}{1.454331in}}%
\pgfpathlineto{\pgfqpoint{7.058231in}{1.502556in}}%
\pgfpathlineto{\pgfqpoint{7.058513in}{1.501881in}}%
\pgfpathlineto{\pgfqpoint{7.059359in}{1.487021in}}%
\pgfpathlineto{\pgfqpoint{7.061332in}{1.389860in}}%
\pgfpathlineto{\pgfqpoint{7.065559in}{1.182351in}}%
\pgfpathlineto{\pgfqpoint{7.068660in}{1.078258in}}%
\pgfpathlineto{\pgfqpoint{7.070633in}{1.014429in}}%
\pgfpathlineto{\pgfqpoint{7.070914in}{1.016973in}}%
\pgfpathlineto{\pgfqpoint{7.072042in}{1.072968in}}%
\pgfpathlineto{\pgfqpoint{7.074015in}{1.192391in}}%
\pgfpathlineto{\pgfqpoint{7.074297in}{1.187592in}}%
\pgfpathlineto{\pgfqpoint{7.075706in}{1.085338in}}%
\pgfpathlineto{\pgfqpoint{7.076551in}{1.044486in}}%
\pgfpathlineto{\pgfqpoint{7.076833in}{1.049436in}}%
\pgfpathlineto{\pgfqpoint{7.078806in}{1.182419in}}%
\pgfpathlineto{\pgfqpoint{7.079370in}{1.146248in}}%
\pgfpathlineto{\pgfqpoint{7.081343in}{0.914376in}}%
\pgfpathlineto{\pgfqpoint{7.081906in}{0.928811in}}%
\pgfpathlineto{\pgfqpoint{7.084161in}{1.057767in}}%
\pgfpathlineto{\pgfqpoint{7.084725in}{1.048730in}}%
\pgfpathlineto{\pgfqpoint{7.086416in}{1.011258in}}%
\pgfpathlineto{\pgfqpoint{7.086698in}{1.015238in}}%
\pgfpathlineto{\pgfqpoint{7.087825in}{1.068784in}}%
\pgfpathlineto{\pgfqpoint{7.092335in}{1.334936in}}%
\pgfpathlineto{\pgfqpoint{7.092898in}{1.330811in}}%
\pgfpathlineto{\pgfqpoint{7.095999in}{1.275570in}}%
\pgfpathlineto{\pgfqpoint{7.096844in}{1.282828in}}%
\pgfpathlineto{\pgfqpoint{7.099099in}{1.342940in}}%
\pgfpathlineto{\pgfqpoint{7.102481in}{1.412578in}}%
\pgfpathlineto{\pgfqpoint{7.104454in}{1.455656in}}%
\pgfpathlineto{\pgfqpoint{7.107273in}{1.591481in}}%
\pgfpathlineto{\pgfqpoint{7.110373in}{1.709272in}}%
\pgfpathlineto{\pgfqpoint{7.110655in}{1.710875in}}%
\pgfpathlineto{\pgfqpoint{7.110937in}{1.710742in}}%
\pgfpathlineto{\pgfqpoint{7.110937in}{1.710742in}}%
\pgfpathlineto{\pgfqpoint{7.111782in}{1.701154in}}%
\pgfpathlineto{\pgfqpoint{7.114882in}{1.643183in}}%
\pgfpathlineto{\pgfqpoint{7.115446in}{1.645420in}}%
\pgfpathlineto{\pgfqpoint{7.116855in}{1.675370in}}%
\pgfpathlineto{\pgfqpoint{7.123056in}{1.845115in}}%
\pgfpathlineto{\pgfqpoint{7.125029in}{1.904591in}}%
\pgfpathlineto{\pgfqpoint{7.129820in}{2.101646in}}%
\pgfpathlineto{\pgfqpoint{7.130384in}{2.097226in}}%
\pgfpathlineto{\pgfqpoint{7.132075in}{2.042980in}}%
\pgfpathlineto{\pgfqpoint{7.133766in}{1.999298in}}%
\pgfpathlineto{\pgfqpoint{7.134048in}{1.999805in}}%
\pgfpathlineto{\pgfqpoint{7.134893in}{2.018388in}}%
\pgfpathlineto{\pgfqpoint{7.136866in}{2.139653in}}%
\pgfpathlineto{\pgfqpoint{7.139685in}{2.281197in}}%
\pgfpathlineto{\pgfqpoint{7.139967in}{2.280712in}}%
\pgfpathlineto{\pgfqpoint{7.141658in}{2.253391in}}%
\pgfpathlineto{\pgfqpoint{7.142221in}{2.259423in}}%
\pgfpathlineto{\pgfqpoint{7.143349in}{2.331548in}}%
\pgfpathlineto{\pgfqpoint{7.145604in}{2.531464in}}%
\pgfpathlineto{\pgfqpoint{7.145885in}{2.515594in}}%
\pgfpathlineto{\pgfqpoint{7.148140in}{2.170678in}}%
\pgfpathlineto{\pgfqpoint{7.148986in}{2.260192in}}%
\pgfpathlineto{\pgfqpoint{7.150395in}{2.467973in}}%
\pgfpathlineto{\pgfqpoint{7.150677in}{2.458471in}}%
\pgfpathlineto{\pgfqpoint{7.153214in}{2.117144in}}%
\pgfpathlineto{\pgfqpoint{7.154059in}{2.195842in}}%
\pgfpathlineto{\pgfqpoint{7.156032in}{2.419357in}}%
\pgfpathlineto{\pgfqpoint{7.156314in}{2.417310in}}%
\pgfpathlineto{\pgfqpoint{7.157441in}{2.351239in}}%
\pgfpathlineto{\pgfqpoint{7.159978in}{2.221905in}}%
\pgfpathlineto{\pgfqpoint{7.160542in}{2.226798in}}%
\pgfpathlineto{\pgfqpoint{7.162233in}{2.257738in}}%
\pgfpathlineto{\pgfqpoint{7.162796in}{2.254331in}}%
\pgfpathlineto{\pgfqpoint{7.163924in}{2.211341in}}%
\pgfpathlineto{\pgfqpoint{7.169279in}{1.871675in}}%
\pgfpathlineto{\pgfqpoint{7.170124in}{1.885463in}}%
\pgfpathlineto{\pgfqpoint{7.173506in}{1.986849in}}%
\pgfpathlineto{\pgfqpoint{7.174070in}{1.980844in}}%
\pgfpathlineto{\pgfqpoint{7.175761in}{1.921444in}}%
\pgfpathlineto{\pgfqpoint{7.179425in}{1.800895in}}%
\pgfpathlineto{\pgfqpoint{7.182244in}{1.783008in}}%
\pgfpathlineto{\pgfqpoint{7.183935in}{1.713799in}}%
\pgfpathlineto{\pgfqpoint{7.188444in}{1.496806in}}%
\pgfpathlineto{\pgfqpoint{7.189290in}{1.512226in}}%
\pgfpathlineto{\pgfqpoint{7.192672in}{1.637936in}}%
\pgfpathlineto{\pgfqpoint{7.193518in}{1.627616in}}%
\pgfpathlineto{\pgfqpoint{7.195209in}{1.546204in}}%
\pgfpathlineto{\pgfqpoint{7.198873in}{1.382990in}}%
\pgfpathlineto{\pgfqpoint{7.199154in}{1.382397in}}%
\pgfpathlineto{\pgfqpoint{7.199436in}{1.383153in}}%
\pgfpathlineto{\pgfqpoint{7.200000in}{1.386941in}}%
\pgfpathlineto{\pgfqpoint{7.200000in}{1.386941in}}%
\pgfusepath{stroke}%
\end{pgfscope}%
\begin{pgfscope}%
\pgfsetbuttcap%
\pgfsetroundjoin%
\definecolor{currentfill}{rgb}{0.000000,0.000000,0.000000}%
\pgfsetfillcolor{currentfill}%
\pgfsetlinewidth{0.501875pt}%
\definecolor{currentstroke}{rgb}{0.000000,0.000000,0.000000}%
\pgfsetstrokecolor{currentstroke}%
\pgfsetdash{}{0pt}%
\pgfsys@defobject{currentmarker}{\pgfqpoint{0.000000in}{0.000000in}}{\pgfqpoint{0.000000in}{0.055556in}}{%
\pgfpathmoveto{\pgfqpoint{0.000000in}{0.000000in}}%
\pgfpathlineto{\pgfqpoint{0.000000in}{0.055556in}}%
\pgfusepath{stroke,fill}%
}%
\begin{pgfscope}%
\pgfsys@transformshift{4.381818in}{0.600000in}%
\pgfsys@useobject{currentmarker}{}%
\end{pgfscope}%
\end{pgfscope}%
\begin{pgfscope}%
\pgfsetbuttcap%
\pgfsetroundjoin%
\definecolor{currentfill}{rgb}{0.000000,0.000000,0.000000}%
\pgfsetfillcolor{currentfill}%
\pgfsetlinewidth{0.501875pt}%
\definecolor{currentstroke}{rgb}{0.000000,0.000000,0.000000}%
\pgfsetstrokecolor{currentstroke}%
\pgfsetdash{}{0pt}%
\pgfsys@defobject{currentmarker}{\pgfqpoint{0.000000in}{-0.055556in}}{\pgfqpoint{0.000000in}{0.000000in}}{%
\pgfpathmoveto{\pgfqpoint{0.000000in}{0.000000in}}%
\pgfpathlineto{\pgfqpoint{0.000000in}{-0.055556in}}%
\pgfusepath{stroke,fill}%
}%
\begin{pgfscope}%
\pgfsys@transformshift{4.381818in}{2.781818in}%
\pgfsys@useobject{currentmarker}{}%
\end{pgfscope}%
\end{pgfscope}%
\begin{pgfscope}%
\pgftext[x=4.381818in,y=0.544444in,,top]{{\rmfamily\fontsize{12.000000}{14.400000}\selectfont 0}}%
\end{pgfscope}%
\begin{pgfscope}%
\pgfsetbuttcap%
\pgfsetroundjoin%
\definecolor{currentfill}{rgb}{0.000000,0.000000,0.000000}%
\pgfsetfillcolor{currentfill}%
\pgfsetlinewidth{0.501875pt}%
\definecolor{currentstroke}{rgb}{0.000000,0.000000,0.000000}%
\pgfsetstrokecolor{currentstroke}%
\pgfsetdash{}{0pt}%
\pgfsys@defobject{currentmarker}{\pgfqpoint{0.000000in}{0.000000in}}{\pgfqpoint{0.000000in}{0.055556in}}{%
\pgfpathmoveto{\pgfqpoint{0.000000in}{0.000000in}}%
\pgfpathlineto{\pgfqpoint{0.000000in}{0.055556in}}%
\pgfusepath{stroke,fill}%
}%
\begin{pgfscope}%
\pgfsys@transformshift{4.945455in}{0.600000in}%
\pgfsys@useobject{currentmarker}{}%
\end{pgfscope}%
\end{pgfscope}%
\begin{pgfscope}%
\pgfsetbuttcap%
\pgfsetroundjoin%
\definecolor{currentfill}{rgb}{0.000000,0.000000,0.000000}%
\pgfsetfillcolor{currentfill}%
\pgfsetlinewidth{0.501875pt}%
\definecolor{currentstroke}{rgb}{0.000000,0.000000,0.000000}%
\pgfsetstrokecolor{currentstroke}%
\pgfsetdash{}{0pt}%
\pgfsys@defobject{currentmarker}{\pgfqpoint{0.000000in}{-0.055556in}}{\pgfqpoint{0.000000in}{0.000000in}}{%
\pgfpathmoveto{\pgfqpoint{0.000000in}{0.000000in}}%
\pgfpathlineto{\pgfqpoint{0.000000in}{-0.055556in}}%
\pgfusepath{stroke,fill}%
}%
\begin{pgfscope}%
\pgfsys@transformshift{4.945455in}{2.781818in}%
\pgfsys@useobject{currentmarker}{}%
\end{pgfscope}%
\end{pgfscope}%
\begin{pgfscope}%
\pgftext[x=4.945455in,y=0.544444in,,top]{{\rmfamily\fontsize{12.000000}{14.400000}\selectfont 2}}%
\end{pgfscope}%
\begin{pgfscope}%
\pgfsetbuttcap%
\pgfsetroundjoin%
\definecolor{currentfill}{rgb}{0.000000,0.000000,0.000000}%
\pgfsetfillcolor{currentfill}%
\pgfsetlinewidth{0.501875pt}%
\definecolor{currentstroke}{rgb}{0.000000,0.000000,0.000000}%
\pgfsetstrokecolor{currentstroke}%
\pgfsetdash{}{0pt}%
\pgfsys@defobject{currentmarker}{\pgfqpoint{0.000000in}{0.000000in}}{\pgfqpoint{0.000000in}{0.055556in}}{%
\pgfpathmoveto{\pgfqpoint{0.000000in}{0.000000in}}%
\pgfpathlineto{\pgfqpoint{0.000000in}{0.055556in}}%
\pgfusepath{stroke,fill}%
}%
\begin{pgfscope}%
\pgfsys@transformshift{5.509091in}{0.600000in}%
\pgfsys@useobject{currentmarker}{}%
\end{pgfscope}%
\end{pgfscope}%
\begin{pgfscope}%
\pgfsetbuttcap%
\pgfsetroundjoin%
\definecolor{currentfill}{rgb}{0.000000,0.000000,0.000000}%
\pgfsetfillcolor{currentfill}%
\pgfsetlinewidth{0.501875pt}%
\definecolor{currentstroke}{rgb}{0.000000,0.000000,0.000000}%
\pgfsetstrokecolor{currentstroke}%
\pgfsetdash{}{0pt}%
\pgfsys@defobject{currentmarker}{\pgfqpoint{0.000000in}{-0.055556in}}{\pgfqpoint{0.000000in}{0.000000in}}{%
\pgfpathmoveto{\pgfqpoint{0.000000in}{0.000000in}}%
\pgfpathlineto{\pgfqpoint{0.000000in}{-0.055556in}}%
\pgfusepath{stroke,fill}%
}%
\begin{pgfscope}%
\pgfsys@transformshift{5.509091in}{2.781818in}%
\pgfsys@useobject{currentmarker}{}%
\end{pgfscope}%
\end{pgfscope}%
\begin{pgfscope}%
\pgftext[x=5.509091in,y=0.544444in,,top]{{\rmfamily\fontsize{12.000000}{14.400000}\selectfont 4}}%
\end{pgfscope}%
\begin{pgfscope}%
\pgfsetbuttcap%
\pgfsetroundjoin%
\definecolor{currentfill}{rgb}{0.000000,0.000000,0.000000}%
\pgfsetfillcolor{currentfill}%
\pgfsetlinewidth{0.501875pt}%
\definecolor{currentstroke}{rgb}{0.000000,0.000000,0.000000}%
\pgfsetstrokecolor{currentstroke}%
\pgfsetdash{}{0pt}%
\pgfsys@defobject{currentmarker}{\pgfqpoint{0.000000in}{0.000000in}}{\pgfqpoint{0.000000in}{0.055556in}}{%
\pgfpathmoveto{\pgfqpoint{0.000000in}{0.000000in}}%
\pgfpathlineto{\pgfqpoint{0.000000in}{0.055556in}}%
\pgfusepath{stroke,fill}%
}%
\begin{pgfscope}%
\pgfsys@transformshift{6.072727in}{0.600000in}%
\pgfsys@useobject{currentmarker}{}%
\end{pgfscope}%
\end{pgfscope}%
\begin{pgfscope}%
\pgfsetbuttcap%
\pgfsetroundjoin%
\definecolor{currentfill}{rgb}{0.000000,0.000000,0.000000}%
\pgfsetfillcolor{currentfill}%
\pgfsetlinewidth{0.501875pt}%
\definecolor{currentstroke}{rgb}{0.000000,0.000000,0.000000}%
\pgfsetstrokecolor{currentstroke}%
\pgfsetdash{}{0pt}%
\pgfsys@defobject{currentmarker}{\pgfqpoint{0.000000in}{-0.055556in}}{\pgfqpoint{0.000000in}{0.000000in}}{%
\pgfpathmoveto{\pgfqpoint{0.000000in}{0.000000in}}%
\pgfpathlineto{\pgfqpoint{0.000000in}{-0.055556in}}%
\pgfusepath{stroke,fill}%
}%
\begin{pgfscope}%
\pgfsys@transformshift{6.072727in}{2.781818in}%
\pgfsys@useobject{currentmarker}{}%
\end{pgfscope}%
\end{pgfscope}%
\begin{pgfscope}%
\pgftext[x=6.072727in,y=0.544444in,,top]{{\rmfamily\fontsize{12.000000}{14.400000}\selectfont 6}}%
\end{pgfscope}%
\begin{pgfscope}%
\pgfsetbuttcap%
\pgfsetroundjoin%
\definecolor{currentfill}{rgb}{0.000000,0.000000,0.000000}%
\pgfsetfillcolor{currentfill}%
\pgfsetlinewidth{0.501875pt}%
\definecolor{currentstroke}{rgb}{0.000000,0.000000,0.000000}%
\pgfsetstrokecolor{currentstroke}%
\pgfsetdash{}{0pt}%
\pgfsys@defobject{currentmarker}{\pgfqpoint{0.000000in}{0.000000in}}{\pgfqpoint{0.000000in}{0.055556in}}{%
\pgfpathmoveto{\pgfqpoint{0.000000in}{0.000000in}}%
\pgfpathlineto{\pgfqpoint{0.000000in}{0.055556in}}%
\pgfusepath{stroke,fill}%
}%
\begin{pgfscope}%
\pgfsys@transformshift{6.636364in}{0.600000in}%
\pgfsys@useobject{currentmarker}{}%
\end{pgfscope}%
\end{pgfscope}%
\begin{pgfscope}%
\pgfsetbuttcap%
\pgfsetroundjoin%
\definecolor{currentfill}{rgb}{0.000000,0.000000,0.000000}%
\pgfsetfillcolor{currentfill}%
\pgfsetlinewidth{0.501875pt}%
\definecolor{currentstroke}{rgb}{0.000000,0.000000,0.000000}%
\pgfsetstrokecolor{currentstroke}%
\pgfsetdash{}{0pt}%
\pgfsys@defobject{currentmarker}{\pgfqpoint{0.000000in}{-0.055556in}}{\pgfqpoint{0.000000in}{0.000000in}}{%
\pgfpathmoveto{\pgfqpoint{0.000000in}{0.000000in}}%
\pgfpathlineto{\pgfqpoint{0.000000in}{-0.055556in}}%
\pgfusepath{stroke,fill}%
}%
\begin{pgfscope}%
\pgfsys@transformshift{6.636364in}{2.781818in}%
\pgfsys@useobject{currentmarker}{}%
\end{pgfscope}%
\end{pgfscope}%
\begin{pgfscope}%
\pgftext[x=6.636364in,y=0.544444in,,top]{{\rmfamily\fontsize{12.000000}{14.400000}\selectfont 8}}%
\end{pgfscope}%
\begin{pgfscope}%
\pgfsetbuttcap%
\pgfsetroundjoin%
\definecolor{currentfill}{rgb}{0.000000,0.000000,0.000000}%
\pgfsetfillcolor{currentfill}%
\pgfsetlinewidth{0.501875pt}%
\definecolor{currentstroke}{rgb}{0.000000,0.000000,0.000000}%
\pgfsetstrokecolor{currentstroke}%
\pgfsetdash{}{0pt}%
\pgfsys@defobject{currentmarker}{\pgfqpoint{0.000000in}{0.000000in}}{\pgfqpoint{0.000000in}{0.055556in}}{%
\pgfpathmoveto{\pgfqpoint{0.000000in}{0.000000in}}%
\pgfpathlineto{\pgfqpoint{0.000000in}{0.055556in}}%
\pgfusepath{stroke,fill}%
}%
\begin{pgfscope}%
\pgfsys@transformshift{7.200000in}{0.600000in}%
\pgfsys@useobject{currentmarker}{}%
\end{pgfscope}%
\end{pgfscope}%
\begin{pgfscope}%
\pgfsetbuttcap%
\pgfsetroundjoin%
\definecolor{currentfill}{rgb}{0.000000,0.000000,0.000000}%
\pgfsetfillcolor{currentfill}%
\pgfsetlinewidth{0.501875pt}%
\definecolor{currentstroke}{rgb}{0.000000,0.000000,0.000000}%
\pgfsetstrokecolor{currentstroke}%
\pgfsetdash{}{0pt}%
\pgfsys@defobject{currentmarker}{\pgfqpoint{0.000000in}{-0.055556in}}{\pgfqpoint{0.000000in}{0.000000in}}{%
\pgfpathmoveto{\pgfqpoint{0.000000in}{0.000000in}}%
\pgfpathlineto{\pgfqpoint{0.000000in}{-0.055556in}}%
\pgfusepath{stroke,fill}%
}%
\begin{pgfscope}%
\pgfsys@transformshift{7.200000in}{2.781818in}%
\pgfsys@useobject{currentmarker}{}%
\end{pgfscope}%
\end{pgfscope}%
\begin{pgfscope}%
\pgftext[x=7.200000in,y=0.544444in,,top]{{\rmfamily\fontsize{12.000000}{14.400000}\selectfont 10}}%
\end{pgfscope}%
\begin{pgfscope}%
\pgfsetbuttcap%
\pgfsetroundjoin%
\definecolor{currentfill}{rgb}{0.000000,0.000000,0.000000}%
\pgfsetfillcolor{currentfill}%
\pgfsetlinewidth{0.501875pt}%
\definecolor{currentstroke}{rgb}{0.000000,0.000000,0.000000}%
\pgfsetstrokecolor{currentstroke}%
\pgfsetdash{}{0pt}%
\pgfsys@defobject{currentmarker}{\pgfqpoint{0.000000in}{0.000000in}}{\pgfqpoint{0.055556in}{0.000000in}}{%
\pgfpathmoveto{\pgfqpoint{0.000000in}{0.000000in}}%
\pgfpathlineto{\pgfqpoint{0.055556in}{0.000000in}}%
\pgfusepath{stroke,fill}%
}%
\begin{pgfscope}%
\pgfsys@transformshift{4.381818in}{0.600000in}%
\pgfsys@useobject{currentmarker}{}%
\end{pgfscope}%
\end{pgfscope}%
\begin{pgfscope}%
\pgfsetbuttcap%
\pgfsetroundjoin%
\definecolor{currentfill}{rgb}{0.000000,0.000000,0.000000}%
\pgfsetfillcolor{currentfill}%
\pgfsetlinewidth{0.501875pt}%
\definecolor{currentstroke}{rgb}{0.000000,0.000000,0.000000}%
\pgfsetstrokecolor{currentstroke}%
\pgfsetdash{}{0pt}%
\pgfsys@defobject{currentmarker}{\pgfqpoint{-0.055556in}{0.000000in}}{\pgfqpoint{0.000000in}{0.000000in}}{%
\pgfpathmoveto{\pgfqpoint{0.000000in}{0.000000in}}%
\pgfpathlineto{\pgfqpoint{-0.055556in}{0.000000in}}%
\pgfusepath{stroke,fill}%
}%
\begin{pgfscope}%
\pgfsys@transformshift{7.200000in}{0.600000in}%
\pgfsys@useobject{currentmarker}{}%
\end{pgfscope}%
\end{pgfscope}%
\begin{pgfscope}%
\pgftext[x=4.326263in,y=0.600000in,right,]{{\rmfamily\fontsize{12.000000}{14.400000}\selectfont -1500}}%
\end{pgfscope}%
\begin{pgfscope}%
\pgfsetbuttcap%
\pgfsetroundjoin%
\definecolor{currentfill}{rgb}{0.000000,0.000000,0.000000}%
\pgfsetfillcolor{currentfill}%
\pgfsetlinewidth{0.501875pt}%
\definecolor{currentstroke}{rgb}{0.000000,0.000000,0.000000}%
\pgfsetstrokecolor{currentstroke}%
\pgfsetdash{}{0pt}%
\pgfsys@defobject{currentmarker}{\pgfqpoint{0.000000in}{0.000000in}}{\pgfqpoint{0.055556in}{0.000000in}}{%
\pgfpathmoveto{\pgfqpoint{0.000000in}{0.000000in}}%
\pgfpathlineto{\pgfqpoint{0.055556in}{0.000000in}}%
\pgfusepath{stroke,fill}%
}%
\begin{pgfscope}%
\pgfsys@transformshift{4.381818in}{0.963636in}%
\pgfsys@useobject{currentmarker}{}%
\end{pgfscope}%
\end{pgfscope}%
\begin{pgfscope}%
\pgfsetbuttcap%
\pgfsetroundjoin%
\definecolor{currentfill}{rgb}{0.000000,0.000000,0.000000}%
\pgfsetfillcolor{currentfill}%
\pgfsetlinewidth{0.501875pt}%
\definecolor{currentstroke}{rgb}{0.000000,0.000000,0.000000}%
\pgfsetstrokecolor{currentstroke}%
\pgfsetdash{}{0pt}%
\pgfsys@defobject{currentmarker}{\pgfqpoint{-0.055556in}{0.000000in}}{\pgfqpoint{0.000000in}{0.000000in}}{%
\pgfpathmoveto{\pgfqpoint{0.000000in}{0.000000in}}%
\pgfpathlineto{\pgfqpoint{-0.055556in}{0.000000in}}%
\pgfusepath{stroke,fill}%
}%
\begin{pgfscope}%
\pgfsys@transformshift{7.200000in}{0.963636in}%
\pgfsys@useobject{currentmarker}{}%
\end{pgfscope}%
\end{pgfscope}%
\begin{pgfscope}%
\pgftext[x=4.326263in,y=0.963636in,right,]{{\rmfamily\fontsize{12.000000}{14.400000}\selectfont -1000}}%
\end{pgfscope}%
\begin{pgfscope}%
\pgfsetbuttcap%
\pgfsetroundjoin%
\definecolor{currentfill}{rgb}{0.000000,0.000000,0.000000}%
\pgfsetfillcolor{currentfill}%
\pgfsetlinewidth{0.501875pt}%
\definecolor{currentstroke}{rgb}{0.000000,0.000000,0.000000}%
\pgfsetstrokecolor{currentstroke}%
\pgfsetdash{}{0pt}%
\pgfsys@defobject{currentmarker}{\pgfqpoint{0.000000in}{0.000000in}}{\pgfqpoint{0.055556in}{0.000000in}}{%
\pgfpathmoveto{\pgfqpoint{0.000000in}{0.000000in}}%
\pgfpathlineto{\pgfqpoint{0.055556in}{0.000000in}}%
\pgfusepath{stroke,fill}%
}%
\begin{pgfscope}%
\pgfsys@transformshift{4.381818in}{1.327273in}%
\pgfsys@useobject{currentmarker}{}%
\end{pgfscope}%
\end{pgfscope}%
\begin{pgfscope}%
\pgfsetbuttcap%
\pgfsetroundjoin%
\definecolor{currentfill}{rgb}{0.000000,0.000000,0.000000}%
\pgfsetfillcolor{currentfill}%
\pgfsetlinewidth{0.501875pt}%
\definecolor{currentstroke}{rgb}{0.000000,0.000000,0.000000}%
\pgfsetstrokecolor{currentstroke}%
\pgfsetdash{}{0pt}%
\pgfsys@defobject{currentmarker}{\pgfqpoint{-0.055556in}{0.000000in}}{\pgfqpoint{0.000000in}{0.000000in}}{%
\pgfpathmoveto{\pgfqpoint{0.000000in}{0.000000in}}%
\pgfpathlineto{\pgfqpoint{-0.055556in}{0.000000in}}%
\pgfusepath{stroke,fill}%
}%
\begin{pgfscope}%
\pgfsys@transformshift{7.200000in}{1.327273in}%
\pgfsys@useobject{currentmarker}{}%
\end{pgfscope}%
\end{pgfscope}%
\begin{pgfscope}%
\pgftext[x=4.326263in,y=1.327273in,right,]{{\rmfamily\fontsize{12.000000}{14.400000}\selectfont -500}}%
\end{pgfscope}%
\begin{pgfscope}%
\pgfsetbuttcap%
\pgfsetroundjoin%
\definecolor{currentfill}{rgb}{0.000000,0.000000,0.000000}%
\pgfsetfillcolor{currentfill}%
\pgfsetlinewidth{0.501875pt}%
\definecolor{currentstroke}{rgb}{0.000000,0.000000,0.000000}%
\pgfsetstrokecolor{currentstroke}%
\pgfsetdash{}{0pt}%
\pgfsys@defobject{currentmarker}{\pgfqpoint{0.000000in}{0.000000in}}{\pgfqpoint{0.055556in}{0.000000in}}{%
\pgfpathmoveto{\pgfqpoint{0.000000in}{0.000000in}}%
\pgfpathlineto{\pgfqpoint{0.055556in}{0.000000in}}%
\pgfusepath{stroke,fill}%
}%
\begin{pgfscope}%
\pgfsys@transformshift{4.381818in}{1.690909in}%
\pgfsys@useobject{currentmarker}{}%
\end{pgfscope}%
\end{pgfscope}%
\begin{pgfscope}%
\pgfsetbuttcap%
\pgfsetroundjoin%
\definecolor{currentfill}{rgb}{0.000000,0.000000,0.000000}%
\pgfsetfillcolor{currentfill}%
\pgfsetlinewidth{0.501875pt}%
\definecolor{currentstroke}{rgb}{0.000000,0.000000,0.000000}%
\pgfsetstrokecolor{currentstroke}%
\pgfsetdash{}{0pt}%
\pgfsys@defobject{currentmarker}{\pgfqpoint{-0.055556in}{0.000000in}}{\pgfqpoint{0.000000in}{0.000000in}}{%
\pgfpathmoveto{\pgfqpoint{0.000000in}{0.000000in}}%
\pgfpathlineto{\pgfqpoint{-0.055556in}{0.000000in}}%
\pgfusepath{stroke,fill}%
}%
\begin{pgfscope}%
\pgfsys@transformshift{7.200000in}{1.690909in}%
\pgfsys@useobject{currentmarker}{}%
\end{pgfscope}%
\end{pgfscope}%
\begin{pgfscope}%
\pgftext[x=4.326263in,y=1.690909in,right,]{{\rmfamily\fontsize{12.000000}{14.400000}\selectfont 0}}%
\end{pgfscope}%
\begin{pgfscope}%
\pgfsetbuttcap%
\pgfsetroundjoin%
\definecolor{currentfill}{rgb}{0.000000,0.000000,0.000000}%
\pgfsetfillcolor{currentfill}%
\pgfsetlinewidth{0.501875pt}%
\definecolor{currentstroke}{rgb}{0.000000,0.000000,0.000000}%
\pgfsetstrokecolor{currentstroke}%
\pgfsetdash{}{0pt}%
\pgfsys@defobject{currentmarker}{\pgfqpoint{0.000000in}{0.000000in}}{\pgfqpoint{0.055556in}{0.000000in}}{%
\pgfpathmoveto{\pgfqpoint{0.000000in}{0.000000in}}%
\pgfpathlineto{\pgfqpoint{0.055556in}{0.000000in}}%
\pgfusepath{stroke,fill}%
}%
\begin{pgfscope}%
\pgfsys@transformshift{4.381818in}{2.054545in}%
\pgfsys@useobject{currentmarker}{}%
\end{pgfscope}%
\end{pgfscope}%
\begin{pgfscope}%
\pgfsetbuttcap%
\pgfsetroundjoin%
\definecolor{currentfill}{rgb}{0.000000,0.000000,0.000000}%
\pgfsetfillcolor{currentfill}%
\pgfsetlinewidth{0.501875pt}%
\definecolor{currentstroke}{rgb}{0.000000,0.000000,0.000000}%
\pgfsetstrokecolor{currentstroke}%
\pgfsetdash{}{0pt}%
\pgfsys@defobject{currentmarker}{\pgfqpoint{-0.055556in}{0.000000in}}{\pgfqpoint{0.000000in}{0.000000in}}{%
\pgfpathmoveto{\pgfqpoint{0.000000in}{0.000000in}}%
\pgfpathlineto{\pgfqpoint{-0.055556in}{0.000000in}}%
\pgfusepath{stroke,fill}%
}%
\begin{pgfscope}%
\pgfsys@transformshift{7.200000in}{2.054545in}%
\pgfsys@useobject{currentmarker}{}%
\end{pgfscope}%
\end{pgfscope}%
\begin{pgfscope}%
\pgftext[x=4.326263in,y=2.054545in,right,]{{\rmfamily\fontsize{12.000000}{14.400000}\selectfont 500}}%
\end{pgfscope}%
\begin{pgfscope}%
\pgfsetbuttcap%
\pgfsetroundjoin%
\definecolor{currentfill}{rgb}{0.000000,0.000000,0.000000}%
\pgfsetfillcolor{currentfill}%
\pgfsetlinewidth{0.501875pt}%
\definecolor{currentstroke}{rgb}{0.000000,0.000000,0.000000}%
\pgfsetstrokecolor{currentstroke}%
\pgfsetdash{}{0pt}%
\pgfsys@defobject{currentmarker}{\pgfqpoint{0.000000in}{0.000000in}}{\pgfqpoint{0.055556in}{0.000000in}}{%
\pgfpathmoveto{\pgfqpoint{0.000000in}{0.000000in}}%
\pgfpathlineto{\pgfqpoint{0.055556in}{0.000000in}}%
\pgfusepath{stroke,fill}%
}%
\begin{pgfscope}%
\pgfsys@transformshift{4.381818in}{2.418182in}%
\pgfsys@useobject{currentmarker}{}%
\end{pgfscope}%
\end{pgfscope}%
\begin{pgfscope}%
\pgfsetbuttcap%
\pgfsetroundjoin%
\definecolor{currentfill}{rgb}{0.000000,0.000000,0.000000}%
\pgfsetfillcolor{currentfill}%
\pgfsetlinewidth{0.501875pt}%
\definecolor{currentstroke}{rgb}{0.000000,0.000000,0.000000}%
\pgfsetstrokecolor{currentstroke}%
\pgfsetdash{}{0pt}%
\pgfsys@defobject{currentmarker}{\pgfqpoint{-0.055556in}{0.000000in}}{\pgfqpoint{0.000000in}{0.000000in}}{%
\pgfpathmoveto{\pgfqpoint{0.000000in}{0.000000in}}%
\pgfpathlineto{\pgfqpoint{-0.055556in}{0.000000in}}%
\pgfusepath{stroke,fill}%
}%
\begin{pgfscope}%
\pgfsys@transformshift{7.200000in}{2.418182in}%
\pgfsys@useobject{currentmarker}{}%
\end{pgfscope}%
\end{pgfscope}%
\begin{pgfscope}%
\pgftext[x=4.326263in,y=2.418182in,right,]{{\rmfamily\fontsize{12.000000}{14.400000}\selectfont 1000}}%
\end{pgfscope}%
\begin{pgfscope}%
\pgfsetbuttcap%
\pgfsetroundjoin%
\definecolor{currentfill}{rgb}{0.000000,0.000000,0.000000}%
\pgfsetfillcolor{currentfill}%
\pgfsetlinewidth{0.501875pt}%
\definecolor{currentstroke}{rgb}{0.000000,0.000000,0.000000}%
\pgfsetstrokecolor{currentstroke}%
\pgfsetdash{}{0pt}%
\pgfsys@defobject{currentmarker}{\pgfqpoint{0.000000in}{0.000000in}}{\pgfqpoint{0.055556in}{0.000000in}}{%
\pgfpathmoveto{\pgfqpoint{0.000000in}{0.000000in}}%
\pgfpathlineto{\pgfqpoint{0.055556in}{0.000000in}}%
\pgfusepath{stroke,fill}%
}%
\begin{pgfscope}%
\pgfsys@transformshift{4.381818in}{2.781818in}%
\pgfsys@useobject{currentmarker}{}%
\end{pgfscope}%
\end{pgfscope}%
\begin{pgfscope}%
\pgfsetbuttcap%
\pgfsetroundjoin%
\definecolor{currentfill}{rgb}{0.000000,0.000000,0.000000}%
\pgfsetfillcolor{currentfill}%
\pgfsetlinewidth{0.501875pt}%
\definecolor{currentstroke}{rgb}{0.000000,0.000000,0.000000}%
\pgfsetstrokecolor{currentstroke}%
\pgfsetdash{}{0pt}%
\pgfsys@defobject{currentmarker}{\pgfqpoint{-0.055556in}{0.000000in}}{\pgfqpoint{0.000000in}{0.000000in}}{%
\pgfpathmoveto{\pgfqpoint{0.000000in}{0.000000in}}%
\pgfpathlineto{\pgfqpoint{-0.055556in}{0.000000in}}%
\pgfusepath{stroke,fill}%
}%
\begin{pgfscope}%
\pgfsys@transformshift{7.200000in}{2.781818in}%
\pgfsys@useobject{currentmarker}{}%
\end{pgfscope}%
\end{pgfscope}%
\begin{pgfscope}%
\pgftext[x=4.326263in,y=2.781818in,right,]{{\rmfamily\fontsize{12.000000}{14.400000}\selectfont 1500}}%
\end{pgfscope}%
\begin{pgfscope}%
\pgfsetbuttcap%
\pgfsetroundjoin%
\pgfsetlinewidth{1.003750pt}%
\definecolor{currentstroke}{rgb}{0.000000,0.000000,0.000000}%
\pgfsetstrokecolor{currentstroke}%
\pgfsetdash{}{0pt}%
\pgfpathmoveto{\pgfqpoint{4.381818in}{2.781818in}}%
\pgfpathlineto{\pgfqpoint{7.200000in}{2.781818in}}%
\pgfusepath{stroke}%
\end{pgfscope}%
\begin{pgfscope}%
\pgfsetbuttcap%
\pgfsetroundjoin%
\pgfsetlinewidth{1.003750pt}%
\definecolor{currentstroke}{rgb}{0.000000,0.000000,0.000000}%
\pgfsetstrokecolor{currentstroke}%
\pgfsetdash{}{0pt}%
\pgfpathmoveto{\pgfqpoint{7.200000in}{0.600000in}}%
\pgfpathlineto{\pgfqpoint{7.200000in}{2.781818in}}%
\pgfusepath{stroke}%
\end{pgfscope}%
\begin{pgfscope}%
\pgfsetbuttcap%
\pgfsetroundjoin%
\pgfsetlinewidth{1.003750pt}%
\definecolor{currentstroke}{rgb}{0.000000,0.000000,0.000000}%
\pgfsetstrokecolor{currentstroke}%
\pgfsetdash{}{0pt}%
\pgfpathmoveto{\pgfqpoint{4.381818in}{0.600000in}}%
\pgfpathlineto{\pgfqpoint{7.200000in}{0.600000in}}%
\pgfusepath{stroke}%
\end{pgfscope}%
\begin{pgfscope}%
\pgfsetbuttcap%
\pgfsetroundjoin%
\pgfsetlinewidth{1.003750pt}%
\definecolor{currentstroke}{rgb}{0.000000,0.000000,0.000000}%
\pgfsetstrokecolor{currentstroke}%
\pgfsetdash{}{0pt}%
\pgfpathmoveto{\pgfqpoint{4.381818in}{0.600000in}}%
\pgfpathlineto{\pgfqpoint{4.381818in}{2.781818in}}%
\pgfusepath{stroke}%
\end{pgfscope}%
\begin{pgfscope}%
\pgftext[x=5.790909in,y=2.851263in,,base]{{\rmfamily\fontsize{14.400000}{17.280000}\selectfont e2\_d}}%
\end{pgfscope}%
\end{pgfpicture}%
\makeatother%
\endgroup%

%\caption{Verlauf der Fehler}
%\end{figure}
\newpage
\section{Parameterabschätzung}

In der verallgemeinerten Herleitung der Bewegungsgleichungen sind die Modellparameter zunächst variabel. Für die Simulation und die weitere Betrachtung sind numerische Werte erforderlich, die das reale Verhalten beschreiben. Das Modell ist neben den geometrischen Eigenschaften, wie z.B. den Längen, den Positionen der Schwerpunkte und den daraus resultierenden Trägheitsmomenten noch von Parametern abhängig, welche die Dynamik beschreiben. Der folgende Abschnitt behandelt die Berechnung der Trägheitsmomente und Herleitung der Federsteifigkeiten und -dämfungen der passiven Gelenke.

\subsection{Federsteifigkeit und -dämpung}

Die Federsteifigkeiten $k_i$ und Federdämpfungen $c_i$ für $i=1,2$ werden separiert voneinander berechnet. 

\begin{figure}[h]
	\centering
	\input{Parameter_1.pdf_tex}
	\caption[Modellparameter]{Modellfreischnitt für Schwingungsbetrachtung}
	\label{fig:parameter_abschaetung}
\end{figure}

Dafür wird vereinbart, dass die nicht betrachteten Gelenke keine Auslenkungen aufweisen und keiner Dynamik unterliegen. 
Für das erste unaktuierte Gelenk kann die nichtlineare Schwingungsdifferentialgleichung (\ref{eq:parameter_nl_DGL}) aufgestellt werden. 

\begin{equation} \label{eq:parameter_nl_DGL}
J\ddot{\theta}_{12}+c_1\dot{\theta}_{12}+k_1\theta_{12}+mg\cos(\theta_{12})=\tau
\end{equation}

In der Gleichung (\ref{eq:parameter_nl_DGL}) beschreibt $c_1$ die Dämpfung im ersten Gelenk, $k_1$ die Federsteifigkeit im ersten Gelenk und $J$ das Trägheitsmoment des Balkens um die Drehachse. Die Gesamtlänge des Balkens ist $l$, dessen Masse $m$ und dessen Auslenkung um seine Ruhelage $\theta_{12}$. Die Erdbeschleunigung ist $g$ und die Stellgröße im ersten Gelenk ist $\tau$.

Für die weiteren Betrachtungen wird die Gleichung (\ref{eq:parameter_nl_DGL}) um $\theta_{12}^e=0$ linearisiert. Damit ist $\theta_{12}-\theta_{12}^e=\tilde{\theta}_{12}=\theta_{12}$. Der gleiche Zusammenhang gilt für die Zeitableitungen von $\theta_{12}$. Es ergibt sich die lineare homogene Differentialgleichung zweiter Ordnung (\ref{eq:parameter_l_DGL}).

\begin{equation} \label{eq:parameter_l_DGL}
J\ddot{\theta}_{12}+c_1\dot{\theta}_{12}+k_1\theta_{12}=0
\end{equation}

Nach trivialer Umstellung lässt sich ein Koeffizientenvergleich mit der allgemeinen Schwingungsdifferentialgleichung durchführen.

\begin{equation} \label{eq:parameter_koeffvergl}
\ddot{\theta}_{12}+\dfrac{c_1}{J}\dot{\theta}_{12}+\dfrac{k_1}{J}\theta_{12}\stackrel{!}{=}\ddot{\theta}_{12}+2\delta\dot{\theta}_{12}+\omega_0^2\theta_{12}=0
\end{equation}

Somit lassen sich die Koeffizienten 

\begin{equation} \label{eq:parameter_koeff}
\begin{aligned}
c_1&=2\delta J \mbox{ und}\\
k_1&=J\omega_0^2
\end{aligned}
\end{equation}

ablesen.

\subsection*{Vorgabe einer Dynamik}

Die Schwingung des Auslegers soll einer definierten Dynamik folgen. Als Abschätzung für die Dämpfung $\delta$ ist ein Abklingen der Schwingung innerhalb einer Zeit von $\Delta t=30\si{s}$ auf 10\% der Anfangsauslenkung gefordert.

\begin{figure}[h]
	\centering
	\def\svgscale{0.5}
	\input{Schwingung.pdf_tex}
	\caption[Schwingungszeitverlauf]{Lösung der Schwingungsdifferentialgleichung}
	\label{fig:dgl_lsg}
\end{figure}

In Abbildung \ref{fig:dgl_lsg} ist die Lösung der Differentialgleichung (\ref{eq:parameter_koeffvergl}) vorgegeben. Den abklingenden Teil beschreibt der Realteil der Lösung der Charakteristischen Gleichung der Ansatzmethode. Dieser wird durch $e^{-\delta t}$ beschrieben. Um die Vorgaben zu erfüllen muss die resultierende Gleichung (\ref{eq:delta}) gelöst werden.

\begin{equation} \label{eq:delta}
\delta=-\dfrac{1}{\Delta t}\ln\left(\dfrac{\theta_{12}(\Delta t)}{\theta_{12}(0)}\right)=-\dfrac{1}{30\si{s}}\ln(0,1)=0,0768\si{Hz}
\end{equation}

\chapter{Modellentwicklung}\label{ch:modellentwicklung}

In diesem Kapitel wird die systematische Herangehensweise behandelt, mit der die Simulationsumgebung sukzessive erweitert wurde. Dabei ist nachvollziehbar aufbereitet, welche Informationen der Regelung und Steuerung im jeweiligen Schritt zu Verfügung stehen.

\section{Modellübersicht}

Für die Simulation werden drei verschiedene Modelle betrachtet. Zum einen das Modell, welches das physikalische Verhalten der Betonpumpe abbilden soll und zum anderen die Modelle, die dem Entwurf der Regelung und der Vorsteuerung zugrunde liegen. Alle Modell können entweder vollständig oder unvollständig aktuierte Eigenschaft aufweisen. Die Möglichkeit einen Stelleingriff auf jedes betrachtete Gelenk ausüben zu können, bedeutet dabei, dass es sich um ein vollständig aktuiertes Modell handelt. Bei unvollständig aktuierten Modellen werden zusätzlich Gelenke integriert, die das dynamische Verhalten verursachen, welches bei realen Betonpumpen beobachtet werden kann.\\

\begin{table}[htbp]
	\centering
	\caption{Mögliche Modellkombinationen}
	\label{tab:Modellübersicht}
	\begin{tabular}{llll}
		Fall & Modell der Simulation & Entwurfsmodell des  & Entwurfsmodell der\\
		&						& Reglers			& Steuerung\\
		\toprule
		1.& vollst. akt. & vollst. akt. & vollst. akt.\\
		2.& unvollst. akt. & vollst. akt. & vollst. akt.\\
		3.& unvollst. akt. & unvollst. akt. & vollst. akt.\\
		4.& unvollst. akt. & unvollst. akt. & unvollst. akt.\\
		\bottomrule\\
	\end{tabular}
\end{table}

In Tabelle \ref{tab:Modellübersicht} sind alle möglichen Modellkombinationen aufgelistet. In der folgenden Betrachtung wird ein exemplarischer Ausleger betrachtet, welcher aus zwei Gliedern besteht. Der Implementierungsaufwand nimmt zu, je mehr unvollständig aktuierte Modelle verwendet werden.


\section{Alle Modelle vollständig aktuiert (1. Fall)}

Im ersten Fall gilt die Annahme, dass sehr massive steife Glieder verbaut wurden, wodurch von einem unelastischen Ausleger ausgegangen werden kann, bei dem Schwingungen vernachlässigt werden. Dadurch sind alle Gelenke vollständig aktuiert und der Regelungs- und Steuerungsentwurf vereinfacht sich.

\begin{figure}[h]
	\centering
	\input{Manipulatorvollakt.pdf_tex}
	\caption{Manipulator vollständig aktuiert}
	\label{fig:VollAkt}
\end{figure} 

In Abbildung \ref{fig:VollAkt} ist das physikalische Modell eines vollständig aktuierten Manipulators dargestellt. Der Winkel $\theta_{11}$ gibt die absolute Auslenkung des ersten Segmentes zur horizontalen Bezugsebene wider und der Winkel $\theta_{21}$ die Auslenkung des zweiten Segmentes relativ zum ersten.

\begin{equation} \label{eq:VollAkt}
\underbrace{M(\theta)\ddot{\theta}}_{\mbox{Trägheitsmoment}} + \underbrace{C(\theta,\dot{\theta})\dot{\theta}}_{\begin{matrix}
	\mbox{Zentrifugal-,} \\ \mbox{Coriolismoment} \end{matrix}}+\underbrace{g(\theta)}_{\begin{matrix}
	\mbox{Gravitations-} \\ \mbox{einfluss} \end{matrix}}=\underbrace{\tau}_{\begin{matrix}
	\mbox{Stell-} \\ \mbox{momente} \end{matrix}}
\end{equation}

Die Modellgleichung ist in (\ref{eq:VollAkt}) zusammengefasst. Die Gleichung ist für das Simulationsmodell, die Regelung und die Steuerung die selbe. Es ist sofort sichtbar, dass sich die Stellgrößen einfach aus einer geforderten Solltrajektorie von $\theta$ und ihren zeitlichen Ableitungen bis zur zweiten Ordnung berechnen lassen.

\section{Simulationsmodell unteraktuiert (2. Fall)}

Das Simulationsmodell bildet approximiert das reales Verhalten des Auslegers ab. Die Glieder sind elastisch, so dass ein schwingungsfähiges System vorliegt. Zu jedem Segment wird zusätzlich ein unaktuiertes Gelenk modelliert, welches die Dynamik verursacht.

\begin{figure}[h]
	\centering
	\input{Manipulatorunterakt.pdf_tex}
	\caption{Manipulator unvollständig aktuiert}
	\label{fig:UnterAkt}
\end{figure}   

In Abbildung \ref{fig:UnterAkt} ist der Manipulator aus Abbildung \ref{fig:VollAkt} mit zusätzlichen elastischen Gelenken modelliert. Die Winkel $\theta_{12}$ und $\theta_{22}$ geben die Durchbiegung der Segmente wider.

\begin{equation} \label{eq:UnterAkt}
\underbrace{M(\theta)\ddot{\theta}}_{\mbox{Trägheitsmomente}} + \underbrace{C(\theta,\dot{\theta})\dot{\theta}}_{\begin{matrix}
\mbox{Zentrifugal-,} \\ \mbox{Coriolismomente} \end{matrix}}+\underbrace{K(\theta,\dot{\theta})}_{\begin{matrix}
\mbox{Elastische} \\ \mbox{Fesselungsmomente} \end{matrix}}+\underbrace{g(\theta)}_{\begin{matrix}
\mbox{Gravitations-} \\ \mbox{einfluss} \end{matrix}}=\underbrace{\tau}_{\begin{matrix}
\mbox{Antriebs-} \\ \mbox{momente} \end{matrix}} 
\end{equation} 

Gleichung (\ref{eq:UnterAkt}) hat durch die elastischen Gelenke zusätzlich einen Term $K(\theta,\dot{\theta})$, welcher die geschwindigkeitsabhängige Dämpfung und auslenkungsabhängige Federmoment in den Gelenken $\theta_{12}$ und $\theta_{22}$ beschreibt. Die Dimension der Gleichungen ist um zwei erhöht worden. % Alle benannten Kräfte sind im rotatorischen System in ihrer Wirkung als Momente aufzufassen.
Aufgrund der Modellierung gibt es nur bei jedem zweiten Gelenk, in diesem Fall bei $\theta_{11}$ und $\theta_{21}$, einen Stelleingriff. Die übrigen Gelenke $\theta_{12}$ und $\theta_{22}$ sind passiv.

Das Modell für die Regelung und Steuerung ist vollständig aktuiert. Daraus folgt eine Einzelgelenkregelung. Alle Einflüsse, die durch die unaktuierten Gelenke und die Verkopplung resultieren, werden als Störgröße behandelt.

\section*{Implementierung} 

Bevor die ersten Simulationsergebnisse diskutiert werden, wird an dieser Stelle die Struktur der Implementierung erläutert.

\begin{figure}[h]
	\centering
	\input{Implementierung.pdf_tex}
	\caption{Implementierung des Regelkreises}
	\label{fig:Implementierung}
\end{figure}

In Abbildung \ref{fig:Implementierung} ist die Anordnung der einzelnen Teile des Regelkreises dargestellt. Das unvollständig aktuierte Modell des Auslegers ist in \textquotedblleft Modell\textquotedblright \,hinterlegt. Die Gleichungen (\ref{eq:UnterAkt}) sind in der Form eines Zustandsraumes zu lösen. Die Zustandsgröße ist $x=(\theta,\dot{\theta})^T$, mit den Winkeln und den Winkelgeschwindigkeiten als Komponenten.

\begin{equation} \label{eq:ModZR}
\dot{x}	=	\begin{pmatrix}
				x_2 \\
				M^{-1}(x_{1})(\tau - C(x_1,x_2)x_2 - g(x_1) - K(x_1,x_2))
			\end{pmatrix}
\end{equation}

In Gleichung (\ref{eq:ModZR}) ist der Zustandsraum des zu simulierenden gewöhnlichen Differentialgleichungssystem erster Ordnung notiert. In der Ruhelage $x^e=(x_1^e,0)^T$ kompensiert die Vorsteuerung den Gravitationseinfluss (Gleichung (\ref{eq:tauVor})) 

\begin{equation} \label{eq:tauVor}
\tau_{\mathrm{Vorsteuerung}} = g(x_1^e)
\end{equation}

Der PD-Regler verstärkt den Regelfehler des Winkels und der Winkelgeschwindigkeit und gibt die Stellgröße auf das Modell als Eingang, so dass der tatsächliche Ausgang dem Sollverlauf entspricht. Stationäre Regelabweichungen sind bei diesem Ansatz möglich.

\section*{Beispiel}

Im Folgenden werden einige Simulationsergebnisse behandelt. Für die Ruhelage wurden die Winkel:


\begin{align*}
\theta_{11}^e&=60^\circ\\
\theta_{21}^e&=-90^\circ\\
\theta_{12}^e&=0\\
\theta_{22}^e&=0\\
\end{align*}

gewählt. In Abbildung \ref{fig:animation}, in Kapitel \ref{ch:Visualisierung}, ist die Ruhelage skizziert. Die Werte von $\theta^e=x_1^e$ entsprechen den Anfangswerten $x_1(0)=\theta^e$ zum Zeitpunkt $t=0$.

\begin{figure}[h]
\centering
\includegraphics[width=1\linewidth]{RuhelageUnaktuierteGelenke.png}
\caption{Simulationsergebnisse unaktuierte Geleneke unvollständig aktuiertes Modell}
\label{fig:RuhelageUnaktuierteGelenke}
\end{figure}

In Abbildung \ref{fig:RuhelageUnaktuierteGelenke} ist der Winkelverlauf der unaktuierten Gelenke zu sehen. Zum Zeitpunkt $t=0$ haben diese Winkel keine Auslenkung. Innerhalb von zirka $3\,\si{s}$ schwingt sich der Ausleger in seine neue Ruhelage ein. Die Winkel sind negativ. Bei Betrachtung der Anordnung ist eine Durchbiegung, welche durch die zusätzlichen Gelenke modelliert ist, in mathematisch negative Richtung zu erwarten, da durch die Gravitationskraft ein Moment in der selben Richtung auf die Gelenke wirkt. Der Winkel $\theta_{12}$ hat eine größere Auslenkung, da der längere Teil des Auslegers durch seine höhere Masse ein stärkeres Moment ausübt.

\begin{figure}[h]
\centering
\includegraphics[width=1\linewidth]{RuhelageGelenkQ1Q2.png}
\caption{Simulationsergebnisse aktuierte Gelenke mit Einzelgelenkregelung}
\label{fig:RuhelageGelenkQ1Q2}
\end{figure}

Die Winkelverläufe der aktuierten Gelenke sind in Abbildung \ref{fig:RuhelageGelenkQ1Q2} gezeigt. Die Vorsteuerung kompensiert lediglich den Gravitationseinfluss, das heißt das Moment, welches dabei berechnet wird, ist über die gesamte Simulationsdauer konstant. Die Simulation startet nicht in der Ruhelage, für welche die Kompensation berechnet wurde. Damit entspricht dieser Wert auch nicht exakt dem notwendigen. Bei Beginn lenken die unaktuierten Gelenke im Uhrzeigersinn aus, wodurch ein leichter mathematisch positiver Ausschlag der aktuierten Gelenke verzeichnet wird. Die Regelung der einzelnen Gelenke wirkt dagegen. Nach dem Einschwingen verbleiben geringe stationäre Abweichungen, da nur ein PD-Regler verwendet wird, dessen Verstärkungen nicht optimal gewählt wurden. Die Simulationsergebnisse deuten an, dass die Einzelgelenkregelung in den Ruhelagen ausreichend ist, da nur geringe Abweichungen auftreten.
 
\section{Modell der Regelung unvollständig aktuiert (3. Fall)}

In diesem Fall kann durch die Kenntnis des unvollständig aktuierten Modells eine Ruhelage exakt berechnet werden. 

In unserem Beispiel ergeben sich folgende Werte:

\begin{align*}
\theta_{11}^e&=60^\circ\\
\theta_{21}^e&=-90^\circ\\
\theta_{12}^e&=-0,679^\circ\\
\theta_{22}^e&=-0,124^\circ\\
\end{align*}

Die tatsächliche Ruhelage der unaktuierten Gelenke befinden sich dadurch in der berechneten, wodurch daraus keine Störmomente auf die aktuierten Gelenke ausgeübt werden.

\begin{figure}[h]
\centering
\includegraphics[width=1\linewidth]{lsgunvollvoll_unakt.png}
\caption{Simulationsergebisse unaktuierter Gelenke mit unvollständig aktuiertem Modell der Regelung}
\label{fig:unvollMod}
\end{figure}

Die Winkel $\theta_{12}$ und $\theta_{22}$ der unaktuierten Gelenke befinden sich sofort an der berechneten Position und sind konstant (Abbildung \ref{fig:unvollMod}). Dadurch treten sie nicht in Wechselwirkung mit den Winkeln der anderen Gelenken.

\begin{figure}[h]
\centering
\includegraphics[width=1\linewidth]{lsgunvollvoll_akt.png}
\caption{Simulationsergebnisse der aktuierten Gelenke mit exakt berechneter Ruhelage}
\label{fig:lsgunvollvoll_unakt}
\end{figure}

Die Winkel $\theta_{11}$ und $\theta_{21}$ der aktuierten Gelenke erreichen ihre Sollposition nicht exakt, da die Steuerung noch das vollständig aktuierte Modell annimmt und die Stellgröße für die Ruhelage ungenau berechnet. Die stationäre Abweichung ist sehr gering und beträgt in diesem Beispiel zirka $0,0035^\circ$ (vgl. Abbildung \ref{fig:lsgunvollvoll_unakt}). Dieser Wert resultiert aus einer Simulation und ist in der  Praxis approximiert Null.

\newpage
\section{Alle Modelle unvollständig aktuiert (4. Fall)}

Bei der zusätzlichen Berücksichtigung eines unvollständigen Modells der Vorsteuerung ist die Kenntnis der Trajektorien aller Zustandskomponenten notwendig. Für eine Ruhelage sind diese mit akzeptablen Aufwand zu berechnen. Sofern ein Arbeitspunktübergang zwischen zwei Ruhelagen vollzogen werden soll, muss ein Randwertproblem gelöst werden. Die Lösung besteht aus Trajektorien, welche den Systemdifferentialgleichungen genügen und in gewünschtem Verhalten lösen. %Daraus ergibt sich beispielsweise, dass alle Winkel in der Ruhelage exakt dieser entsprechen.
\chapter{Modellierung der Last}

Im allgemeinen Betrieb einer mobilen Betonpumpe kann davon ausgegangen werden, dass durch den Vorgang des Pumpens von Beton eine Last wirkt. In diesem Kapitel wird eine Beschreibungsform der Last hergeleitet, damit eine Simulation berechnet werden kann.

\section{Ansatz}

\begin{figure}[h]
	\centering
	\input{Lastansatz.pdf_tex}
	\caption[]{Modellierungsannahme der Last}
	\label{fig:Manipulator_Last}
\end{figure}

Eine Betonpumpe pumpt den Beton impulsweise. Nach hinreichend langer Zeit ist die Rohrleitung vollständig mit Beton gefüllt. An jeder Stelle der Rohrleitung befindet sich über die gesamte Zeit eine konstante Masse. Zusätzlich kann man sich vorstellen, dass die Masse pro Bogenelement der Leitung auch als konstant betrachtet werden kann, da der Beton einer Art des Schiebens unterliegt. Dadurch ändern sich durch das Pumpen nur die Massenträgheitsmomente der Segmente des Auslegers. Am Ende des Auslegers befindet sich ein senkrecht hängender Schlauch, durch welchen sich der Beton zielgenau platzieren lässt. Durch die Reibung an der Schlauchwand ergibt sich ein Auswurfverhalten mit Tiefpasscharakter. Approximiert lässt sich eine sprungförmige Belastung annehmen, welche die schlimmste Form einer Belastung darstellt. 

In den Modellgleichungen wird die Last als Punktmasse modelliert. Dafür wird bei der Herleitung der Bewegungsgleichungen am Ende des letzten Segmentes eine Masse berücksichtigt, die einen Einfluss auf dessen Trägheitsmoment ausübt. Im Vergleich zu der Modellierung ohne Last kommt ein Term hinzu, der multiplikativ mit der Masse der Last verknüpft ist. In der Simulation kann dieser Wert in Abhängigkeit von der Zeit beeinflusst werden.  

\begin{figure}[h]
	\centering
	\input{Manipulatorunterakt_Last.pdf_tex}
	\caption[]{Modellierungsansatz der Last}
	\label{fig:Manipulator_Last}
\end{figure} 

\chapter{Visualisierungen}

Diverse Daten - beispielsweise Simulationsergebnisse - erfordern eine geeignete Aufbereitung und ansprechende Darstellung. In diesen Darstellungen können Zusammenhänge zwischen verschiedenen abhängigen Größen gezeigt werden, oder mit anderen verglichen werden.
Bei dynamischen Systemen sind oft die Zeitverläufe der Systemgrößen von Interesse, die für Analysen in Diagrammen gezeigt werden können. Für einen Überblick über das Verhalten eines Systems bieten sich darüber hinaus Animationen an.

In dem folgenden Abschnitt werden einige Erläuterungen zu der Animation der mobilen Betonpumpe gegeben. Bei sämtlichen abgebildeten Diagrammen stehen alle notwendigen Anmerkungen an den entsprechenden Stellen, so dass hier nicht weiter darauf eingegangen wird.

\section{Animation}

\begin{figure}[h]
\centering
\includegraphics[width=0.7\linewidth]{animation01}
\caption[Animation Betonpumpe]{Animationsgrafik von 2 Gliedern (Auslegern) einer mobilen Betonpumpe}
\label{fig:animation}
\end{figure}

In der Abbildung \ref{fig:animation} sind zwei Glieder einer mobilen Betonpumpe mit zwei Auslegern abgebildet. Beide Träger wurden in der Modellbildung durch ein elastisches Knickgelenk in ihrer Mitte den realen Gegebenheiten angenähert. An dieser Stelle treten die unerwünschten Effekte auf, die es im späteren gilt zu verringern. 
\chapter{Zustandsstabilisierung}
\label{Zustandsstabilisierung}

In diesem Kapitel ist die Stabilisierung einer Ruhelage mithilfe der Zustandsrückführung beschrieben. \newline
Bisher wurde nur die Einzelgelenkregelung besprochen. Dabei hat die Regelabweichung eines Gelenkes nur Auswirkungen auf die Stellgröße des Selbigen. Die Matrix $\vect{K_{Regler}}$ des Reglers hat nur Einträge auf der Diagonalen. Bei der im folgenden beschriebenen Zustandsrückführung kann jeder Zustand, bzw. jede Regelabweichung, die Stellgröße beeinflussen. Dabei ist die Matrix des Reglers theoretisch vollständig besetzt.  

\section{Berechnung der Zustandsrückführung}

Die Zustandsrückführung $\vect{F}$  kann auch als Regler aufgefasst werden, der die Strecke, d.h. das Modell der Betonpumpe, stabilisieren soll. Die Eingänge der Rückführung sind die Zustandsgrößen $\vect{\theta}$, $\dot{\vect{\theta}}$, die Winkel und Winkelgeschwindigkeiten der Gelenke und die Parameter der Ruhelage $\bar{\vect{\theta}}$. Die Stellgröße der Zustandsrückführung wird mit der Stellgröße der Vorsteuerung, wie in Abbildung \ref{fig:Blockdiagramm_Zustandsruckfuhrung} dargestellt, addiert und fungiert als Eingang der Strecke. Die Vorsteuerung berechnet aus dem Arbeitspunkt $\bar{\vect{\theta}}$, $\bar{\dot{\vect{\theta}}}$, $\bar{\ddot{\vect{\theta}}}$ die konstante Stellgröße zum Halten der Ruhelage. 

	\begin{figure}[h!]
		\centering
		\includegraphics[scale=0.6]{Bilder/Zustansrueckfuehrung.pdf}
		\caption{Blockdiagramm der Zustandsrückführung}
		\label{fig:Blockdiagramm_Zustandsruckfuhrung}
	\end{figure}
	
Die Zustandsrückführung lässt sich aus der Systemmatrix $\vect{A}$, der Eingangsmatrix $\vect{B}$ und den gewünschten Eigenwerten  $\vect{\lambda}$ des Gesamtsystems, das bedeutet Strecke mit Rückführung, berechnen. Dies kann beispielsweise mit der Ackermannformel für Mehrgrößensysteme geschehen. Da das System dafür aufwendig in die Regelungsnormalform überführt werden muss, wird in dieser Arbeit die Rückführung lediglich mit einer bereits existierenden Matlab-Funktion "`place()"' berechnet. 
	
	\begin{equation}
		\vect{F} = \text{place}(\vect{A},\vect{B},\vect{\lambda})
	\end{equation}   	 

Dabei wird die Zustandsrückführung einmalig in Matlab berechnet und nach Python übertragen. Bei sich ständig ändernden Ruhelagen ist diese Vorgehensweise jedoch keine Lösung. Dafür sollte eine Berechnung in Python selbst bevorzugt werden. Zusätzlich kann nicht immer gewährleistet werden, dass die existierenden Algorithmen unter Python oder Matlab eine Lösung liefern. Sie sind abhängig von den gewünschten Eigenwerten, die im nächsten Abschnitt genauer erläutert werden. Es bietet sich daher an, bei veränderlichen Ruhelagen einen eigenen Algorithmus für die Berechnung zu entwickeln.

\section{Polvorgabe}
\label{abs:Polvorgabe}

Aus dem vorangegangen Abschnitt ist ersichtlich, dass die Pole des Gesamtsystems berechnet werden müssen. Sie beschreiben das Verhalten des Systems, welches sich aus der Strecke und der Zustandsrückführung zusammensetzt.\newline
Für den Vergleich werden zuerst die Pole des Systems mit der Einzelgelenkregelung berechnet. Die Eigenwerte des Gesamtsystems lassen sich aus $\vect{A} \in \mathbb{R}^{8\times8}$, $\vect{B} \in \mathbb{R}^{8\times2}$ und $\vect{K_{Regler}} \in \mathbb{R}^{2\times8}$  mit der Python-Funktion "`np.linalg.eigvals()"' wie folgt berechnen:

	\begin{equation}
		\vect{\lambda} = \text{eig}(\vect{A} + \vect{B} \cdot \vect{K_{Regler}})
	\end{equation} 

mit:

	\begin{equation*}
		\vect{K_{Regler}} = \begin{pmatrix}
								3\cdot10^7 & 0 & 0 & 0 & 2\cdot10^7 & 0 & 0 & 0 \\ 
								0 & 0 & 6\cdot10^7 & 0 & 0 & 0 & 3\cdot10^7 & 0
							\end{pmatrix}. 
	\end{equation*} \newline

Es ergeben sich acht Pole, von denen sechs in dem Polplan in Abbildung \ref{fig:Polplan} als blaue Markierungen dargestellt sind. Alle Pole liegen auf der negativen reellen Halbebene. Das System ist somit stabil.  Der eine sichtbare Pol auf der reellen Achse hat eine Vielfachheit von zwei. Zwei der acht Pole sind rein reell mit einem sehr großen Betrag und daher in dem Diagramm nicht abgebildet. \newline
Ausgehend von der Einzelgelenkregelung werden, wie in Abbildung \ref{fig:Polplan} als grüne Markierungen dargestellt, die Pole der "`Polplatzierung 1"' gewählt. Die Beträge der Pole wurden verkleinert, um kleinere Stellsignale zu erhalten. Zusätzlich  wurde der Realteil der Pole, welche nahe an der imaginären Achse lagen, vergrößert. Bei Parameterschwankungen oder einer Regelung geringfügig außerhalb der Ruhelage können sich die Pole des Gesamtsystems verschieben. Liegen die Pole weiter links auf der imaginären Achse kann die Stabilität somit auch bei größeren Schwankungen gewährleistet werden.\newline
Bei der "`Polplatzierung 2"', siehe Abbildung \ref{fig:Polplan} rote Markierungen, wurden sechs statt wie bisher vier der acht Pole konjugiert-komplex gewählt. Des Weiteren sind sie auf einem Kreisbogen um den Ursprung und mit betragsmäßig größerem Realteil angeordnet. Durch das zusätzliche konjugiert-komplexe Polpaar sollte sich die Dynamik des geschlossenen Systems erhöhen. \newline
	
	\begin{figure}[h!]
		\centering
		\includegraphics[scale=0.5]{Bilder/Pole.png}
		\caption{Polplan des geschlossenen Kreises}
		\label{fig:Polplan}
	\end{figure}
	
Da sich bei der Polvorgabe in diesem Abschnitt keine eindeutige Lösung finden lässt, sollen im Folgenden die einzelnen Regelungen bzw. Zustandsrückführungen miteinander verglichen werden.

\section{Simulationsergebnisse}

Für den Vergleich der einzelnen Rückführungen belastet man das nichtlineare Modell der Betonpumpe mit einer impulsweise konstanten Last. Diese Art der Belastung entspricht näherungsweise einem realen Pumpvorgang und wurde in Kapitel \ref{abs:Lastmodellierung} beschrieben. Dabei wird die Strecke alle zwei Sekunden für $0,5\text{ s}$ mit einer zusätzlichen Masse von $100\text{ kg}$ belastet. Dabei bewegen sich die Gelenke aus der Ruhelage. Die Regelung hat die Aufgabe das System  möglichst schnell, ohne bleibe Regelabweichung und mit geringen Überschwingen in die Ruhelage zu überführen. \newline
Je nach Anwendungsart der Betonpumpe ist die Einhaltung der drei Bedingungen unterschiedlich wichtig. Soll sie beispielsweise in engen und/ oder geschlossenen Räumen manövrieren, darf kein großes Überschwingen auftreten. Wird sie im Freien eingesetzt, spielt das Überschwingen keine große Rolle, hier sollte die Abweichung möglichst schnell ausgeregelt werden. \newline
Die Ergebnisse der Einzelgelenkregelung und der beiden Polplatzierungen aus \mbox{Abschnitt \ref{abs:Polvorgabe}} sind in den folgenden Diagrammen in Abbildung \ref{fig:Ergebnis_Zustandsruckfuhrung} dargestellt. Es sind jeweils die Winkel der aktuierten Gelenke  $\varphi_1 = \theta_{11}$, $\varphi_3 = \theta_{21}$ und die unaktuierten Gelenke $\varphi_2 = \theta_{12}$, $\varphi_4 = \theta_{22}$ über der Zeit $t$ abgebildet. Die Ruhelagen der aktuierten Gelenke betragen $\bar{\varphi_1} = 60 \text{ \degree}$, $\bar{\varphi_3} = -90\text{ \degree}$. Der Arbeitspunkt der unaktuierten Gelenke wird bei $\bar{\varphi_2} = \bar{\varphi_4} = 0 \text{ \degree}$ angenommen. Aufgrund der Durchbiegung weicht die tatsächliche von der angenommen Ruhelage ab.\newline
Aufgrund dieser Annahme berechnet die Vorsteuerung eine nicht ganz korrekte Stellgröße. Die Regelung versucht am Anfang der Simulation, zwischen $t\in[0;0,5]\text{ s}$, die  Differenz zu kompensieren. Man erkennt, dass die Zustandsrückführung mit den sechs konjugiert-komplexen Eigenwerten (Polplatzierung 2) sehr schnell mit relativ großen Überschwingen reagiert. Der Stelleingriff an den aktuierten Gelenken hat auch einen Einfluss auf die Unaktuierten.   
    
	\begin{figure}[h!]
		\centering
		\includegraphics[scale=0.65]{Bilder/Ergebnissse_Zustandsrueckfuehrung.pdf}
		\caption{Verhalten der Zustandsrückführungen bei impulsweiser Belastung}
		\label{fig:Ergebnis_Zustandsruckfuhrung}
	\end{figure}

Die Einzelgelenkregelung schafft es nicht den Fehler der impulsweisen Belastung bei $\varphi_1$ innerhalb der 0,5 s auszuregeln. Auch nach der Entlastung wird der Winkel nur sehr langsam in die Ruhelage gebracht. Ein ähnliches Verhalten ist auch bei dem Gelenk  $\varphi_3$ zu beobachten. Die träge Regelung spiegelt sich auch bei den Winkelverläufen der unaktuierten Gelenke wieder. Bei dem Verlauf von $\varphi_4$ kann man die gewünschten unterschiedlichen Systemfrequenzen aus dem Abschnitt \ref{abs:Federparameter} beobachten.\newline
Die Zustandsrückführung der "`Polplatzierung 1"' in Abbildung \ref{fig:Ergebnis_Zustandsruckfuhrung} weist, wie zu erwarten, eine ähnliche Dynamik wie die Einzelgelenkregelung auf. Lediglich das Überschwingen ist bei $\varphi_1$ verringert. Die Regelung von Gelenk $\varphi_3$ ist dagegen um einiges schlechter. Hier tritt ein größeres Überschwingen bei bleibender Regelabweichung auf. \newline
Wie erwartet weist die Zustandsrückführung der "`Polplatzierung 2"' die größte Dynamik auf. Man erkennt in Abbildung \ref{fig:Ergebnis_Zustandsruckfuhrung}, bei dem Verlauf von $\varphi_1$, dass das System sehr schnell in einen Ruhepunkt überführt wird. Es tritt kaum Überschwingen auf. Während der Belastung wird das System jedoch um eine andere Ruhelage stabilisiert,d.h. es tritt eine bleibende Regelabweichung auf. Ein sehr ähnliches Verhalten zeigt sich auch bei der Regelung von Gelenk $\varphi_3$.\newline
Abschließend lässt sich sagen, dass die Zustandsrückführung der "`Polplatzierung 2"' die besten Ergebnisse liefert. Es würde sich anbieten die Realteile der Pole noch weiter zu erhöhen, um die bleibende Regelabweichung zu verringern. Auch die "`Polplatzierung 1"' liefert für das Gelenk $\varphi_1$ gute Ergebnisse. Es zeigt sich somit, dass bei gut gewählten Polen das System um einiges besser geregelt werden kann als mit der Einzelgelenkregelung. Lediglich das Einstellen der Pole benötigt Erfahrung um das Verhalten des Systems vorauszusagen. 

\chapter{Trajektorien-Folgeregelung}
Bis zu dem jetzigen Stand wurde die Betonpumpe lediglich um eine Ruhelage stabilisiert. Es ist jedoch zusätzlich notwendig den Ausleger der Betonpumpe von einem Arbeitspunkt in einen Anderen zu manövrieren. Die Planung, Steuerung und Regelung des Arbeitspunktwechsel soll im Folgenden näher beschrieben werden. 
\section{Trajektoriengenerierung}
Für den Arbeitspunktwechsel muss eine Trajektorie zwischen den Ruhelagen, im Gelenk\-raum, geplant werden. Diese soll für die Winkel, Winkelgeschwindigkeiten und -beschleuni\-gungen differentiell stetig und glatt sein. \newline
	\begin{figure}[h!]
		\centering
		\includegraphics[scale=0.47]{Bilder/Trajektoriengenerierung.png}
		\caption{Beispieltrajektorie für den Arbeitspunktwechsel}
		\label{fig:Trajektoriengenerierung}
	\end{figure}\newline
Es wird ein Polynom vom Grad fünf für die Trajektorie angenommen. Durch Differentiation ergibt sich ein Polynomgrad von drei für die Winkelbeschleunigung. Ein solches Polynom ist stetig differenzierbar und differentiell glatt an den Anfangs- und Endwerten. Die sich daraus ergebenden sechs Unbekannten lassen sich aus den sechs Anfangs- und Endbedingungen und deren Zeiten berechnen. Die Winkelgeschwindigkeiten und -beschleunigungen sind in den Arbeitspunkten null. Die Winkel selbst sind von der Ausgangs- $(\varphi_a)$ und der Ziel- Ruhelage $(\varphi_e)$ abhängig.\newline
Für $(\varphi_a)= 180\text{ \degree}$, $(\varphi_e)= 45\text{ \degree}$ und einer Zeit $\Delta t=22\text{ s}$ ergibt sich der in Abbildung \ref{fig:Trajektoriengenerierung} dargestellte Trajektorienverlauf.\newline	 
Da der Arbeitspunktwechsel im Gelenkraum geplant wird, kann wie oben beschrieben, eine separate Trajektorie für jedes Gelenk berechnet werden.\newline
In diesem Projekt wird dafür die bereits vorhandene Funktion "`trans\_poly"' der Bibliotheksklasse "`symb\_tools"' verwendet. Dabei werden wie oben beschrieben, die Anfangs- und Endbedingungen inkl. ihrer Zeitpunkte übergeben.\newline
Die beschriebene Planung der Trajektorie wird in den folgenden Abschnitten als Trajektoriengenerator zusammengefasst. Dieser berechnet zu diskreten Zeitpunkten für alle aktuierten Gelenke ihre Sollwinkel,-geschwindigkeiten und -beschleunigungen .  	
\section{Allgemeiner Aufbau}
Fügt man den neuen Block Trajektoriengenerator in den bereits existierenden Regelkreis mit Vorsteuerung und Einzelgelenkregelung, erhält man das Blockschaltbild aus Abbildung \ref{fig:Blockdiagramm_Trajektorien_Folgeregelung}. \newline
	\begin{figure}[h!]
		\centering
		\includegraphics[scale=0.7]{Bilder/Blockdiagramm_Trajektorien_Folgeregelung.pdf}
		\caption{Blockdiagramm der Trajektorien-Folgeregelung}
		\label{fig:Blockdiagramm_Trajektorien_Folgeregelung}
	\end{figure}\newline 
Dabei wird aus dem aktuellen und dem neuen Arbeitspunkt eine Trajektorie berechnet. Zu den, für die Simulation notwendigen, Zeitpunkten werden diskrete Sollwerte für den Regelkreis generiert. Für ein besseres Führungsverhalten berechnet die Vorsteuerung aus den Sollwerten ($\varphi$, $\dot{\varphi}$, $\ddot{\varphi}$) Sollmomente für die Gelenke.\newline
Um den Einfluss von Störungen oder Modellunsicherheiten zu minimieren, wird eine PD-Einzelgelenkreglung hinzugefügt. Diese minimiert den Fehler zwischen dem Soll- und den Istverlauf der Strecke. \newline
Das Modell ist in der Realität die Betonpumpe selbst oder während der Simulation ein mathematisches Modell davon. 
\section{Reglereinstellung}
Bei einer exakten Vorsteuerung ist eine Regelung nicht notwendig, da jedoch das exakte Modell der Betonpumpe nicht abgebildet werden kann, kann auch keine exakte Vorsteuerung berechnet werden. So wird beispielsweise angenommen, dass die unaktuierten Gelenke einen konstanten Winkel von $\varphi = 0$ haben, in Wahrheit weichen sie um bis zu einigen Grad davon ab. Eine Regelung ist somit zwingend notwendig. \newline
Zusätzlich handelt es sich um ein hochgradig nichtlineares Modell. Bei der Verwendung der Zustandsstabilisierung aus Kapitel \ref{Zustandsstabilisierung} wäre die Linearisierung und somit die Systemmatrix abhängig von der Zeit, man erhält ein zeitvariantes System. Die Verwendung von nichtlinearen Regelungsansätzen ist ebenfalls kompliziert, da es sich um ein Mehrgrößensystem handelt und kein flacher Ausgang für das System gefunden wurde.\newline
Demzufolge soll in dieser Arbeit die Trajektorie mit einer einfachen PD-Reglung stabilisiert werden. Ein I-Anteil ist nicht notwendig, da die Stellgröße als Moment vorgegeben wird und die Sollgröße ein Winkel ist. Der offener Kreis besitzt somit bereits zwei Integratoren.\newline
	\begin{figure}[h!]
		\centering
		\includegraphics[scale=0.7]{Bilder/PD_Regler.pdf}
		\caption{Blockdiagramm der PD-Regelung}
		\label{fig:Blockdiagramm_PD_Regelung}
	\end{figure}\newline
Die PD-Regelung aus Abbildung \ref{fig:Blockdiagramm_PD_Regelung} wird als Einzelgelenkregler verwendet. In dieser Arbeit werden der Einfachheit halber nur Betonpumpen mit zwei aktuierten Gelenken untersucht. Es müssen somit vier Reglerparameter einzeln eingestellt werden.\newline
Die Parameter werden gleichmäßig erhöht, bis sich ein stabiles Verhalten mit hinreichend geringer Regelabweichung einstellt. Abhängig davon stellt man nun die Verstärkungen unabhängig voneinander ein, bis das gewünschte Verhalten erzielt wird. Dieses Vorgehen kann nur simulativ durchgeführt werden, um die Betonpumpe im Falle von instabilen Verhalten nicht zu beschädigen. Als Qualitätskriterium wird die Regelabweichung nach ca. zehn Sekunden verwendet, welche ungefähr der bleibenden Regelabweichung entspricht. Tabelle \ref{tab:Reglerparameter} zeigt die bleibende Regelabweichung für verschiedene Reglerparameter.\newline
	\begin{table}[h!]
	\caption{PD-Reglerparametrierung}
	\label{tab:Reglerparameter}
	\begin{tabular}{|c|c|c|c|c|c|c|}
		\hline \rule[-2ex]{0pt}{5.5ex}  Regler & $K_{p,1}\text{ }[\frac{\text{Nm}}{\text{rad}}]$ & $K_{d,1}\text{ }[\frac{\text{Nm}\cdot\text{s}}{\text{rad}}]$ & $K_{p,2}\text{ }[\frac{\text{Nm}}{\text{rad}}]$ & $K_{d,2}\text{ }[\frac{\text{Nm}\cdot\text{s}}{\text{rad}}]$ & $e_{\varphi_1,\infty}\text{ }[\text{\degree}]$ & $e_{\varphi_3,\infty}\text{ }[\text{\degree}]$ \\ 
		\hline \rule[-2ex]{0pt}{5.5ex} $0$ & $1\cdot10^5$  & $1\cdot10^5$ & $1\cdot10^5$ & $1\cdot10^5$ & $-216,8$ & $-60,34$ \\
		\hline \rule[-2ex]{0pt}{5.5ex} $1$ & $1\cdot10^6$  & $1\cdot10^6$ & $1\cdot10^6$ & $1\cdot10^6$ & $6,854\cdot10^{-2}$ & $-1,226\cdot10^{-1}$ \\
		\hline \rule[-2ex]{0pt}{5.5ex} $2$ & $1\cdot10^7$  & $1\cdot10^7$ & $1\cdot10^7$ & $1\cdot10^7$ & $5,467\cdot10^{-3}$ & $-1,248\cdot10^{-2}$ \\
		\hline \rule[-2ex]{0pt}{5.5ex} $3$ & $3\cdot10^7$  & $2\cdot10^7$ & $6\cdot10^7$ & $3\cdot10^7$ & $1,783\cdot10^{-3}$ & $-2,087\cdot10^{-3}$ \\
		\hline 
	\end{tabular} 
	\end{table}
Man erkennt, dass bei zu geringen Verstärkungen (Regler 0) das System instabiles Verhalten aufweist. Erhöht man die Parameter um eine Potenz (Regler 1) stabilisiert der Regler die Strecke. Nun werden die Parameter solange angepasst, bis sich eine Genauigkeit von einigen tausendstel Grad einstellt. Es ergibt sich die Reglerparameterkonstellation drei.\newline
Es ist ersichtlich, dass trotz der zwei Integratoren der Strecke die bleibe Regelabweichung nicht auf Null geregelt werden kann. Eine Genauigkeit von wenigen tausendstel Grad ist jedoch für den Betrieb einer Betonpumpe mehr als ausreichend.
\section{Simulationsergebnisse}
Eine typische Bewegung der Betonpumpe ist in Abbildung \ref{fig:Verlauf_Trajektorienfolgeregelung} dargestellt. Dabei werden die Glieder aus einer Transportposition (alle Glieder sind auf dem Transportfahrzeug eingeklappt, siehe linker Teil) in eine Arbeitsposition (siehe rechter Teil) verfahren. Die in dem Diagramm in Abbildung \ref{fig:Verlauf_Trajektorienfolgeregelung} gezeigten Gelenkverläufe sind die simulierten Verläufe. 
	\begin{figure}[h!]
		\centering
		\includegraphics[scale=0.7]{Bilder/Verlauf_Trajektorienfolgeregelung.pdf}
		\caption{Gelenktrajektorien mit Folgeregelung}
		\label{fig:Verlauf_Trajektorienfolgeregelung}
	\end{figure}\newline
Stellt man die Regelabweichungen während der Bewegung der Gelenke $\varphi_1$  und $\varphi_3$ über die Zeit dar, erhält man das Diagramm in Abbildung \ref{fig:Regelabweichung_Folgeregelung}. Die bleibenden Abweichungen aus Tabelle \ref{tab:Reglerparameter} entsprechen den Regelabweichungen bei $t=30\text{ s}$. Man erkennt, dass die Differenz aus Soll- und Istwinkeln in der Transportposition ($t=3\text{ s}$) bedeutend kleiner als in der Arbeitsposition ($t=30\text{ s}$) ist.
\newline 
	\begin{figure}[h!]
		\centering
		\includegraphics[scale=0.55]{Bilder/Regelabweichung_Folgeregelung.png}
		\caption{Regelabweichung der Trajektorien-Folgeregelung für $\varphi_1$, $\varphi_3$}
		\label{fig:Regelabweichung_Folgeregelung}
	\end{figure}\newline
	
\section{Stellgrößenabschätzung}
Theoretisch wäre es möglich die bleibende Regelabweichungen durch sehr große Verstärkungen immer weiter zu minimieren. Die Stellgrößen hängen jedoch direkt von den Reglerparametern und der Vorsteuerung ab. Daher bietet es sich an im Folgenden eine maximale Schranke für die Stellgrößen zu schätzen. \newline
Für die obere Grenze soll die Betonpumpe vollständig ausgestreckt mit ihrer kompletten Masse am äußersten Punkt belastet werden. Abbildung \ref{fig:Stellgroessenabschaetzung} zeigt diese Anordnung für die in der Arbeit diskutierten Betonpumpe mit jeweils zwei aktuierten und unaktuierten Gelenken.   
\newline
	\begin{figure}[h!]
		\centering
		\includegraphics[scale=0.6]{Bilder/Stelgroessenabschaetzung.pdf}
		\caption{Abschätzung der maximalen Stellgrößen}
		\label{fig:Stellgroessenabschaetzung}
	\end{figure}\newline
Durch eine Belastung von $m_{gesamt} = 4250\text{ kg}$ ergibt sich ein Moment $\tau_g$ von:
	\begin{equation}
		\tau_g = l_{gesamt}\cdot g\cdot m_{gesmat}\approx 700 \text{ kNm}.
	\end{equation}  
Bei der Verwendung eines trivial angeordneten Hydraulikzylinders, wie in Abbildung \ref{fig:Stellgroessenabschaetzung}, muss eine Kraft $F_k$ aufgebracht werden, um das Moment $\tau_g$ zu kompensieren.  Es gilt:
	\begin{equation}
		\tau = \tau_k-\tau_g = 0 \text{ Nm}.
	\end{equation}
Somit ergibt sich $F_k$ zu
	\begin{equation}
		F_k = \sqrt{2}\cdot \frac{\tau_k}{2\text{ m}} \approx 500\text{ kN}.
	\end{equation}

Recherchen zeigen, dass Kräfte von mehr als $500\text{ kN}$ mit Hydraulikzylindern realisiert werden können. Aus Sicherheitsgründen wird die Kraft etwas erhöht und eine maximale Stellgröße von $\tau_{k,max} = 1 \text{ MNm}$ verwendet. Während der Simulation werden alle Stellgrößen auf diesen Wert beschränkt.\newline
\newline
	\begin{figure}[h!]
		\centering
		\includegraphics[scale=0.5]{Bilder/Stellgroessen_Trajektorien_Folgeregelung.png}
		\caption{Stellgrößenverlauf der Trajektorien-Folgeregelung}
		\label{fig:Stellgroessen_Trajektorien_Folgeregelung}
	\end{figure}\newline
Der Stellgrößenverlauf der Trajektorien-Folgeregelung ist in Abbildung \ref{fig:Stellgroessen_Trajektorien_Folgeregelung} dargestellt. Man erkennt, dass die maximale Stellgröße von $\tau_{k,max} = 1 \text{ MNm}$ weit unterschritten wird. Die Bewegung kann somit von einer realen Betonpumpe durchgeführt werden. 

\section{Einführung von Modellungenauigkeiten}
Aus Abbildung \ref{fig:Stellgroessen_Trajektorien_Folgeregelung} ist ersichtlich, dass die Regelung nach der Einschwingphase, ab $t=5\text{\,s}$ kaum dynamische Stellgrößen generiert. Es zeigt sich, dass die Vorsteuerung sehr gut dem inversen Streckenmodell entspricht. Bei einer realen Betonpumpe wird dies nicht mehr der Fall sein. Um dennoch Stabilität und ein gutes Führungsverhalten gewährleisten zu können, werden nachfolgend Modellungenauigkeiten eingefügt, die nicht in der Vorsteuerung betrachtet werden.\newline
	\begin{figure}[h!]
		\centering
		\includegraphics[scale=0.5]{Bilder/Modellungenauigkeiten.png}
		\caption{Regelabweichung mit Modellungenauigkeiten}
		\label{fig:Regelabweichung_Modellungenauigkeiten}
	\end{figure}\newline
Dabei wird der Wert aller Modellparameter für die Simulation um $\pm20 \,\%$ variiert. Abbildung \ref{fig:Regelabweichung_Modellungenauigkeiten} zeigt den Verlauf der Regelabweichung während der Bewegung. Es ist ersichtlich, dass das System trotz der Ungenauigkeiten stabilisiert werden kann. Lediglich die Regelabweichungen sind um einiges größer. Da es sich bei den Ungenauigkeiten um eine zufällige Verteilung handelt werden fünf Messungen durchgeführt und die stationären Regelabweichungen gemittelt. Die ermittelten Werte von $e_{\varphi_1,\infty}=4,508\cdot10^{-2}\,\degree$ und $e_{\varphi_3,\infty} = 5,368\cdot10^{-3}\,\degree$  liegen jedoch immer noch weit unter den für den Betrieb einer Betonpumpe notwendigen Abweichungen. Diese ersten Versuche zeigen, dass die Trajektorien-Folgeregelung durchaus auch bei einer realen Betonpumpe mit zwei Armsegmenten angewendet werden kann. 

\chapter{Zusammenfassung und Ausblick}
Die hier angefertigte Arbeit ist Teil des Oberseminars der Regelungs- und Steuerungstheorie. Ziel des Seminars war die Untersuchung, Steuerung und Regelung eines Auslegers einer mobilen Betonpumpe. Die Arbeit wurde in einer Gruppe von drei Studenten durchgeführt. Im folgenden Abschnitt werden die Ergebnisse noch einmal zusammengefasst dargestellt und ein Ausblick für das mögliche weitere Vorgehen gegeben
  
\section{Zusammenfassung der Ergebnisse}
Einen großen zeitlichen Aufwand stellte die Darstellung des Verhaltens der Ausleger der Betonpumpe als mathematische Gleichungen dar. Diese Modellierung wurde mit konzentrierten Parametern durchgeführt. Um das dynamische Verhalten und die Durchbiegung gut darstellen zu können, wurde pro Armsegment ein passives unaktuiertes Zusatzgelenk betrachtet. Diese Modellierung war die Voraussetzung für alle weiteren Schritte.\newline
Um das Systemverhalten der Betonpumpe abzubilden, wurden sämtliche Parameter der Modellierung unter Verwendung von realen Daten berechnet. \newline
Zusätzlich wurde das erhaltene Systemmodell linearisiert und in die Zustandsdarstellung überführt. Dadurch war es möglich einen Nachweis der Stabilität in der Ruhelage zu erbringen und eine Zustandsrückführung zu berechnen.\newline
Es wurden verschiedene Varianten diskutiert, wie viele Modellinformationen für die Regelung und Steuerung einbezogen werden sollen. Dadurch konnten ihre Auswirkungen auf das Verhalten der Betonpumpe in mehreren Simulationen untersucht werden.\newline
Es wurde eine impulsweise Belastung eingeführt, die die reale Belastung gut approximiert. Dadurch war es möglich verschiedene Regelungskonzepte miteinander zu vergleichen.\newline
Diese Regelungskonzepte sind u.a. die Einzelgelenkregelung, wo jedes Gelenk unabhängig von den anderen mit einem PD-Regler stabilisiert wird. Als eine weitere Möglichkeit für die Ruhelagenstabilisierung wurde die Zustandsrückführung vorgestellt. Dafür wurden die gewünschten Eigenwerte des Gesamtsystems vorgegeben, um eine Rückführung zu berechnen.\newline
Des Weiteren wurden Trajektorien im Gelenkraum geplant, um den Ausleger der Betonpumpe in eine neue Ruhelage überführen zu können. Die Stabilisierung während der Bewegung übernimmt eine Trajektorien-Folgeregelung. Für ein gutes Verhalten wurden die Parameter der Regelung gezielt eingestellt.\newline 
Letztendlich können durch die Einführung von Modellungenauigkeiten die genannten Konzepte der Regelung und Steuerung auch mit fehlerhaften Systemmodellen getestet werden. Dadurch ist es möglich ihr Verhalten bei einer realen Anwendung besser abschätzen zu können.
\section{Ausblick für weiteres Vorgehen}

Die Zusammenfassung in dem vorherigen Abschnitt zeigt, dass viele verschiedene Möglichkeiten der Modellierung, Regelung und Steuerung des Auslegers einer mobilen Betonpumpe untersucht wurden. Trotzdem konnten im Umfang des Oberseminars nicht alle möglichen Konzepte behandelt werden. \newline
So wurden bisher nur lineare Regelungskonzepte implementiert. Die Kenntnisse über Nichtlinearitäten konnten dabei leider nicht verwendet werden. Eine andere Möglichkeit ist die Eingang-Ausgang-Linearisierung. Dabei wird das vorhandene nichtlineare Modell partiell linearisiert, indem ein neuer Eingang eingeführt wird. Die neue Stellgröße ist die Winkelbeschleunigung. Bisher wurde das Moment als Eingang benutzt. Das daraus entstandene partiell lineare System kann mit den Methoden der linearen Regelungstechnik stabilisiert werden. \newline
Des Weiteren kann eine Trajektorienplanung für die unaktuierten Gelenke entwickelt werden. Da diese Gelenke nicht aktiv gesteuert werden können, beschreibt die Trajektorie nur die Bewegung, die durch die Steuerung der aktuierten Gelenke verursacht wird. Dadurch kann der gesamte Ausleger auf einer gezielten Trajektorie in Gelenk- und Arbeitsraum (Raumkoordinaten und Orientierung) bewegt werden. Bei dieser Aufgabe wird  eine Randwertaufgabe gelöst.\newline
In dieser Arbeit wurden bisher alle Untersuchungen lediglich an einem Ausleger mit zwei aktuierten Gelenken vorgenommen. In der Einleitung erkennt man, dass eine reale Betonpumpe mehr als fünf aktive Gelenke besitzen kann. Es sollten daher die in der Arbeit vorgestellten Konzepte der Regelung und Steuerung auf eine solche Anzahl übertragen werden. Dabei entstehen sehr große Gleichungssysteme, die ein symbolisches Lösen nahezu unmöglich machen.\newline
Trotz der genannten unbearbeiteten Aufgaben bieten die in dieser Arbeit gezeigten Ergebnisse eine gute Grundlage für die tatsächliche Regelung und Steuerung einer mobilen Betonpumpe.

%
%\input{DokArbeitRst}
%\chapter[kurzer Titel]{Ausführlicher Kapiteltitel, der wirklich viel zu lang für das Inhaltsverzeichnis in dieser Dokumentvorlage ist}
%\blindtext[6]
%\chapter{Beispielkapitel}
%\blindtext[4]
%\section{Etwas Mathematik}
%\LaTeX{} eignet sich in besonderem Maße zum Setzen von mathematischen Formeln. Eine einzelne Formel erhalten Sie mit der \texttt{equation}-Umgebung:
%\begin{equation}\label{eq:exp}
%1 + \mathrm{e}^{i\pi} = 0.
%\end{equation}
%Bitte beachten Sie, dass Formeln Teil des Satzes sind und somit mit den entsprechenden Satzzeichen versehen werden müssen.
%In der Regel genügt es für eine Gleichung nur dann eine Nummer zu vergeben, wenn Sie später auch auf diese verweisen. Um auf die Nummer einer Gleichung zugreifen zu können verwenden Sie den Befehl \texttt{eqref}: 
%\begin{center}
%\ldots wie in Gl.\ \eqref{eq:exp} gezeigt\ldots	.
%\end{center}
%Möchten Sie verhindern, dass eine Gleichung nummiert wird, verwenden Sie die \texttt{equation*}-Umgebung:
%\begin{equation*}
%E + F - K = 2.
%\end{equation*}
%Für Gleichungssysteme bietet sich die \texttt{align}- bzw. \texttt{align*}-Umgebung an, wobei bei letzterer keine Gleichungsnummern ausgegeben wird:
%\begin{align}\label{eq:align}
%\partiell{u}{x} &= \partiell{v}{y}\\
%\partiell{u}{y} &= -\partiell{v}{x}.
%\end{align}
%Alternativ können Sie auch eine $\texttt{aligned}$-Umgebung verwenden:
%\begin{equation}
%\begin{aligned}
%\partiell{u}{x} &= \partiell{v}{y}\\
%\partiell{u}{y} &= -\partiell{v}{x}.
%\end{aligned}
%\end{equation}
%Mit Hilfe der \texttt{subequations}-Umgebung lassen sich die Nummern der einzelnen Gleichungen eines Systems vereinheitlichen:
%\begin{subequations}
%\begin{align}
%\partiell{h}{t} + \partiell{(vh)}{x} =& 0\\
%\partiell{v}{t} + \partiell{}{x}\left(gh + \frac{u^2}{2}\right) &= 0.
%\end{align}
%\end{subequations}
%Die \texttt{subequations}-Umgebung funktioniert auch zusammen mit mehreren einzelnen Gleichungen:
%\begin{subequations}
%\begin{equation}
%\partiell{h}{t} + \partiell{(vh)}{x} = 0
%\end{equation}
%und
%\begin{equation}
%\partiell{v}{t} + \partiell{}{x}\left(gh + \frac{u^2}{2}\right) = 0.
%\end{equation}
%\end{subequations}
%\nocite{FLMR95ijc,Mik57de}
\bibliography{Arbeit}
\bibliographystyle{gerabbrv}
\end{document}